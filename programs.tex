\section{Program Descriptions} \label{sec:programs}

COS \tacq{IMAGE} has four commonly used combinations of two Science Apertures (SAs), the Primary Science Aperture (PSA) and the Bright Object Aperture (BOA), and two mirror modes, MIRRORA and MIRRORB.
During the 2009 servicing mission orbital verification (SMOV) phase, a series of Cycle~17 calibration programs in NUV imaging mode (P11469, P11473, \& P11471) carefully determined the two-dimensional offset from the COS WCA to the center of the PSA when observed with MIRRORA.
These X and Y offsets were loaded in the FSW TA parameters\footnote{In the COS FSW, these WCA-to-SA offsets are stored as patchable constants in the \textsc{pcta\_XImCalTargetOffset} (XD) and \textsc{pcta\_YImCalTargetOffset} (AD)}.}
A target was then centered using a PSA+MIRRORA \texttt{ACQ/IMAGE}, then a target image was taken along with a MIRRORB image
of the WCA image. These images were used to determine the AD (Y) and XD (X) offsets of the image target and WCA centroids.
These values were uploaded in the FSW paramaters. This bootstrapping procedure was repeated with the BOA+MIRRORA
and BOA+MIRRORB \texttt{ACQ/IMAGE} modes until all four \texttt{ACQ/IMAGE} modes were co-aligned.

In the COS TA Monitoring programs described in this ISR, we re-use this bootstrapping strategy to test the co-alignment of all four \texttt{ACQ/IMAGE} modes.\footnote{The underlying assumption of these programs is that that the PSA/MIRRORA \texttt{ACQ/IMAGE}~centering has not changed since SMOV.},
In addition to COS calibration programs listed above, and described in detail is \S~\ref{subsec:History}--\ref{subsec:elists},
COS \tacq{IMAGE} exposures obtained in numerous cycles of the "Focal Plane Calibration (SI-FGS Alignment)" series were used in the COS TA monitoring discussed in this ISR.
These programs were developed by the HST Telescope's division (PIs Cox and/or Lallo) for Fine Guidance Sensor (FGS) to Science Instrument (SI) alignment, and are described in \S~\ref{subsec:fgs2si}.

All data with a given program/cycle were intentionally taken contemporaneously to avoid any long-term detector or spacecraft effects from affecting our results.
Our requirement was that all data for a given program were taken within 45 days of each other.\\
