\subsection{COS TA Monitoring Program History}\label{subsec:History}
After the installation of COS into HST in 2009 (STS-125), and the
servicing mission orbital verification (SMOV) phase,
a series of calibration programs in NUV imaging mode carefully determined the two-dimensional offset from the COS WCA to the center of the PSA when observed with MIRRORA.
These X and Y offsets were loaded in the FSW TA parameters \textsc{XImCalTargetOffset} and \textsc{YImCalTargetOffset}.
A target was then centered using a PSA+MIRRORA \texttt{ACQ/IMAGE}, then a target image was taken along with a MIRRORB image
of the WCA image. These images were used to determine the AD (Y) and XD (X) offsets of the image target and WCA centroids.
These values were uploaded in the FSW paramaters. This bootstrapping procedure was repeated with the BOA+MIRRORA
and BOA+MIRRORB \texttt{ACQ/IMAGE} modes until all four \texttt{ACQ/IMAGE} modes were co-aligned.


In this program (13972) we use this bootstrapping strategy to test the co-alignment of all four \texttt{ACQ/IMAGE} modes.
\footnote{On November 6, 2014, the MIRRORB \texttt{ACQ/IMAGE} wavelength calibration lamp exposure was changed from a 30 second exposure
at LOW current (3mA) to a 12 second exposure at MEDIUM current. At this point the \textsc{pcta\_XImCalTargetOffset} and \textsc{pcta\_YImCalTargetOffset}
FSW parameters were also updated to reflect a small change in the WCA-to-SA imaging MIRRORB offsets. This program is the first to monitor the updated offsets.}
To accomplish this in only two orbits, this project leverages observations taken in FGS-to-SI alignment verification program (14035).

The FGS-to-SI program (P14035) performs a PSA/MIRRORA \texttt{ACQ/IMAGE} on a target that should be centered in the aperture.
The PSA+MIRRORA \texttt{ACQ/IMAGE} in Visit `A2' of P14035 can be used to verify the COS NUV PSA aperture position in the SIAF.
This exposure shows that the COS NUV PSA SIAF entry combined with the PSA+MIRRORA WCA-to-PSA offsets are
accurate to within [AD,XD] = [-0.020,0.105]\arcsec\ (this is the distance that the \texttt{ACQ/IMAGE} slewed to center the target).
The COS aperture is only repeatable in the XD direction to $\pm$ one motor step (0.05\arcsec). In addition, the WCA location
cannot be measured to better than 1/2 pixel as the pixel used to determine the median location in an integer.
On the NUV detector, 1 pixel is $\sim$ 0.023\arcsec. Based upon this information, the COS NUV PSA definition
in the SIAF file appears to meet our accuracy requirements for Cycle 22.
