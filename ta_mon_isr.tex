\documentclass[12pt]{reportj}
% RCS_ID: $Id: ta_mon_isr.tex,v 1.2 2018/02/27 18:02:38 penton Exp $
\usepackage{astron}
\usepackage{deluxetable}
\usepackage{times}
\usepackage[dvips,textures]{graphicx}
\usepackage{fancyheadings}
\usepackage{color}
\pagestyle{fancy}

\DeclareGraphicsRule{.ps}{eps}{.ps}{}
\setlength{\headheight}{5mm}
\setlength{\headsep}{10mm}
\setlength{\footskip}{10mm}
\setlength{\textheight}{220mm}
\setlength{\textwidth}{170mm}
\setlength{\topmargin}{-8.0mm}
\setlength{\oddsidemargin}{+6.0mm}
\setlength{\evensidemargin}{+6.0mm}
\setlength{\parskip}{1mm}
\setlength{\parsep}{100mm}
\setlength{\parindent}{10mm}
\def\arcsec{\hbox{$^{\prime\prime}$}}

%%%%%%%%%%%%%%%%
%For numbered sections use ssection/ssubsection/ssubsubsection. For sections without numbers user ssectionstar/ssubsectionstar/ssubsubsectionstar
%%%%%%%%%%%%%%%%
\def\ssection#1{\addtocounter{section}{1} \setcounter{subsection}{0} \S*{\hbox to \hsize{\large\bf \arabic{section}. #1\hfill }}}
\def\ssectionstar#1{\S*{\hbox to \hsize{\large\bf #1\hfill}}}
\def\ssubsection#1{\addtocounter{subsection}{1} \setcounter{subsubsection}{0} \subsection*{\hbox to \hsize{\normalsize\bfseries\itshape \arabic{section}.\arabic{subsection} #1\hfill}}}
\def\ssubsectionstar#1{\subsection*{\hbox to \hsize{\normalsize\bfseries\itshape #1\hfill}}}
\def\ssubsubsection#1{\addtocounter{subsubsection}{1} \subsection*{\hbox to \hsize{\normalsize\it \arabic{section}.\arabic{subsection}.\arabic{subsubsection} #1\hfill}}}
\def\ssubsubsectionstar#1{\subsection*{\hbox to \hsize{\normalsize\it #1\hfill}}}
%Set the footer on the first page
\lhead{}
\cfoot{\rm \footnotesize \hspace{-1.5cm}\it{
Operated by the Association of Universities for Research in Astronomy, Inc.,
for the National Aeronautics \newline and Space Administration.}\hspace{2.0 cm}}
\setlength{\headrulewidth}{0pt}
\setlength{\footrulewidth}{0pt}

\setlength{\textwidth}{147mm}

\begin{document}

~\\

\vspace{-2.4cm}
\noindent\includegraphics*[width=0.295\linewidth]{new_st_logo.eps}

\vspace{-0.4cm}

\begin{flushright}
% Put the instrument, year, and ISR number here (and also below)
{\bf Instrument Science Report COS2016-XX(v1)}

\vspace{1.1cm}

%Put ISR Title
{\bf\Huge Cycle 21-24 COS Target Acquisition Monitoring}

\rule{0.25\linewidth}{0.5pt}

\vspace{0.5cm}
%Put Authors
Steven V. Penton$^1$
\linebreak
\newline
%Put Author's affiliations
\footnotesize{$^1$ Space Telescope Science Institute, Baltimore, MD}
\vspace{0.5cm}

% Date here below
09-Feb-2018
\end{flushright}

\vspace{0.1cm}
%%%%%%%%
%Abstract
%%%%%%%%
\noindent\rule{\linewidth}{1.0pt}
\noindent{\bf A{\footnotesize BSTRACT}}

{\it \noindent
Beginning with HST Cycle 21, a series of annual HST programs were executed to verify that COS Target Acquisitions (TA) were performing
nominally.  This ISR documents these programs for HST Cycles 21-24. During this period, all FUV exposures were executed at Lifetime Position 3 (LP3),
and all NUV exposures were obtained at the nominal (LP1) position.
These programs were designed to monitor numerous aspects of both imaging and spectroscopic COS TAs, including
checking the TA subarrays, monitoring the required flashes of the internal PtNe lamps, and evaluating the accuracy of
numerous COS flight software (FSW) patchable constants required for TA.

There are three COS TA modes, FUV spectroscopic, NUV spectroscopic, and NUV imaging.
This project verified that all three modes were behaving nominally in Cycle 21-24, and determined that no SIAF or FSW parameter updates were required during this time.
There were, however, changed required for MIRRORB NUV ACQ/IMAGEs. These changes included a changing of the lamp current from LOW to MEDIUM, an adjustment of the LTACAL exposure time, and a modification of both the MIRRORB WCA and PSA/BOA TA subarrays. }

\vspace{-0.1cm}
\noindent\rule{\linewidth}{1.0pt}
%%%%%%%%
%Contents
%%%%%%%%
\vspace{-0.3cm}
\ssectionstar{Contents}
\vspace{-0.3cm}

%%%%%%%%%%%%%%%%%%%
%May need to compile twice for page references
%%%%%%%%%%%%%%%%%%%
\begin{itemize}
\item Introduction (page \pageref{sec:Introduction})
\item Program Descriptions \pageref{sec:programs}
\item SIAF Verification (page \pageref{sec:siaf})
\item ACQ/IMAGE Performance (page \pageref{sec:acqimage})
\item ACQ/PEAKXD Performance (page \pageref{sec:acqpeakxd})
\item TA Subarray Performance (page \pageref{sec:subarray})
\item Conclusion (page \pageref{sec:theend})
\end{itemize}
%%%%%%%%
%Body
%%%%%%%%
\clearpage
\ssection{Introduction}\label{sec:Introduction}

The basic results of the programs reviewed here were previously reported in the following COS ISRs:
\begin{itemize}
\item{COS ISR 2016-09 (Cycle 21 COS Target Acquisition Monitor Summary)}
\item{COS ISR 2017-09 (Cycle 22 COS Target Acquisition Monitor Summary)}
\item{COS ISR 2018-09 (Cycle 23 COS Target Acquisition Monitor Summary)}
\item{COS ISR 2018-09 (Cycle 24 COS Target Acquisition Monitor Summary)}
\end{itemize}
This ISR provides the full details of the following HST+COS Programs:
\begin{itemize}
\item{HST Program 13526 (COS Imaging TA and Spectroscopic WCA-PSA/BOA offset verifications, Cycle 21)}
\item{HST Program 13972 (COS Imaging TA and Spectroscopic WCA-PSA/BOA offset verifications, Cycle 22)}
\item{HST Program 14440 (COS Imaging TA and Spectroscopic WCA-PSA/BOA offset verifications, Cycle 23)}
\item{HST Program 14857 (COS Imaging TA and Spectroscopic WCA-PSA/BOA offset verifications, Cycle 24)}
\end{itemize}

There are three modes of COS Target Acquisition (TA); NUV imaging, NUV spectroscopic, and NUV imaging.
There are four COS TA (ACQ/) procedures; ACQ/IMAGE, PEAKD, PEAKXD, and SEARCH. The ACQ/PEAKD and SEARCH
step the telescope through dwell patterns on the sky. As long as the target light falls correctly within
the TA detector sub-arrays, ACQ/PEAKD and SEARCH will continue to nominally assist in TA (barring any
unforeseen anomalies). The ACQ/IMAGE and PEAKXD procedures rely on the sub-arrays and patchable constants
in the COS flight software (FSW) which assist in target centering. In both ACQ/IMAGE and PEAKXD, the internal
wavelength calibration lamp is flashed to locate the center of the wavelength calibration aperture (WCA). From this location, the center of the
science aperture (SA) in use can be predicted by applying the FSW constant that give the SA offset compared to the WCA center. For ACQ/IMAGE,
the offset is in both detector X (along-dispersion, AD) and Y (cross-dispersion, XD). For ACQ/PEAKXD, which
uses dispersed light, this offset is only in the Y (XD) direction. HST program 13972 also verifies that the TA subarrays
are proper and re-measures the actively used WCA-to-SA offsets to monitor the performance of COS TAs.

ACQ/IMAGE has four combinations of two SAs, the Primary Science Aperture (PSA) and the Bright
Object Aperture (BOA), and two mirror modes, MIRRORA and MIRRORB. Each combination is commonly used, and has a slightly different
WCA-to-SA offset in both X (AD) and Y(XD), which must be verified. Each COS grating has a different WCA-to-SA XD offset.
BOA spectroscopic TAs are not supported for COS, so 13972 only verifies the WCA-to-PSA offsets that are used in ACQ/PEAKXD.
Each COS grating has a different offset, as do each of the FUV lifetime positions (LP). This program only verifies the NUV LP1
and FUV LP3\footnote{The COS FUV channel was moved to LP3 on February 15, 2015.} position offsets.

The centering requirements are based upon the wavelength accuracy requirements in the AD, and for flux and resolution optimization
in the XD. The strictest NUV requirements are [AD,XD] = [0.041, 0.3]\arcsec, while for the FUV they are [AD,XD] = [0.106, 0.3]\arcsec. The XD
requirement is that all TAs center to within $\pm$ 0.3 \arcsec\ with a 1$\sigma$ goal of $\pm$ 0.1\arcsec.

In \S~\ref{sec:acqimage} we will discuss the verification of the FSW parameters associated with COS ACQ/IMAGEs.
Some data from visit 'A2' of program 14035 will be used to assist in this verification. This data will be used to verify
the COS NUV SIAF (Science Instrument Aperture File) entries. In ~\S~\ref{sec:peakxd},
we will discuss the verification of the FSW parameters associated with COS ACQ/PEAKXDs, and the verification of the
COS FUV SIAF entries.

All 13972 data were taken on October 6, 2015, while the data in 14035 used by this program (Visit A2) were taken on October 2, 2015.
These data were intentionally taken contemporaneously to avoid any long-term detector or spacecraft effects from affecting our results.\\

%!!!!!!!!!!!!!!!!!!!!!!!!!!!!!!!!!!!!!!!!!!!!!!!!!!!!!!!!!!!!!!!!!!!!!!!!!!!!!!!!!!!!!!!!!!!!!!!!!!!!!!!!!!!!!!!!!!!!!!!!!!!!!!!!!!!!!!!!!!!!!!!!!!!!!!!!!!!!!!!!!!!!!!!!!!!!!!!!!!!!!!!!!!!!!!!!!!!!!!!!!!!!!!!!!!!!!!!!!!!!!!!!!!!!!!!!!!!!!!!!
%THIS SECTION NEEDS TO GO AFTER THE END OF THE FIRST PAGE AND BEFORE THE END OF THE SECOND PAGE
%Fill in Instrument, Year, and ISR Number and delete "newpage" immediately after this message
%!!!!!!!!!!!!!!!!!!!!!!!!!!!!!!!!!!!!!!!!!!!!!!!!!!!!!!!!!!!!!!!!!!!!!!!!!!!!!!!!!!!!!!!!!!!!!!!!!!!!!!!!!!!!!!!!!!!!!!!!!!!!!!!!!!!!!!!!!!!!!!!!!!!!!!!!!!!!!!!!!!!!!!!!!!!!!!!!!!!!!!!!!!!!!!!!!!!!!!!!!!!!!!!!!!!!!!!!!!!!!!!!!!!!!!!!!!!!!!
\newpage

\lhead{}
\rhead{}
\cfoot{\rm {\hspace{-1.9cm} Instrument Science Report COS 2016-09(v1) Page \thepage}}

\clearpage
\ssection{Program Descriptions \label{sec:programs} }

In addition to 13972, COS exposures obtained in HST program 14035 (HST Cycle 22 Focal Plane Calibration (SI-FGS Alignment)) were used in this analysis.
This program builds upon the monitoring and calibration of the FGS-to-SI alignment program (14452 - HST Cycle 23- Focal Plane Calibration (SI-FGS Alignment)).  HST 14452 performs back-to-back PSA/MIRRORA and PSA/MIRRORB ACQ/IMAGEs, from which all the results herein are bootstrapped.

The FGS-to-SI program is repeated twice a year (every cycle) and we will use its COS exposures as the baseline for this TA co-alignment program. The historical list of FCS-to-SI proposals, and cycles, are:
11878 (C17)
12399 (C18)
12781 (C19)
13171 (C20)
13616 (C21)
14035 (C22)
14452 (C23)

13526 (C21)
13972 (C22)
14440 (C23)
14857 (C24)

The FGS-to-SI program (14452) performs a PSA/MIRRORA ACQ/IMAGE on a target that should be centered in the aperture.
This verifies the COS NUV PSA aperture position in the SIAF. After this PSA+MIRRORA ACQ/IMAGE, a PSA+MIRRORB ACQ/IMAGE is then performed.
This exposure bootstraps the PSA+MIRRORB centering to the PSA+MIRRORA SIAF verification.
This allows us to monitor the properties of the PSA+MIRRORB image in a controlled way on a centered target.
No spectra are taken in 14452 due to time constraints, but we are currently planning on adding in PSA/MIRRORA and PSA/MIRRORB lamp images.

Visits 01 and 02 of this program extend the COS SIAF/FGS-to-SI verification of Visit 02 of 14452 to the other two ACQ/IMAGE combinations (BOA+MIRRORA and BOA+MIRRORB) by bootstraping from the PSA+MIRRORB verification to co-align all the COS TA imaging modes.
The details of the observations are given is the observing section.

Visit 01 of this program bootstraps off Visit 02 of 14452  to co-align the PSA+MIRRORB ACQ/IMAGE mode to the BOA+MIRRORA.
We prefer that Visit 01 of this program executes within 45 days of Visit 02 of 14452, to ensure that no long term instrument or telescope focus changes impact our results.

Visit 02 of this program follows the style of Visit 01,  and bootstraps from the BOA+MIRRORA mode to the BOA+MIRRORB TA imaging mode.
Visit 02 should also occur within 45 days of visit 02 of 14452 and within 45 days of Visit 01 of this program.

Visit 3 of this program is an on-hold, contingency visit that would be used to replace the 14452 Visit 02 in case this program is, for whatever reason, not executed as planned.
In this case the 1st ACQ/IMAGE is PSA/MIRRORA and the 2nd ACQ/Image is PSA/MIRRORB.
This visit also takes several lamp images to measure the WCA-to-PSA imaging offset FSW patchable constants.

In all visits, lamp+target images are taken before and after the TA imaging mode that is being co-aligned (the second ACQ/IMAGE of the program.)

All visits in this program are single orbit visits, this program is very similar to the C22 version (13972).
Due to the change in OSM2 Home position, some NUV spectra have been re-ordered for efficiency AND some NUV cenwaves were changed to those that are known to have good stripe B WCA spectra.
\clearpage
\ssection{SIAF \label{sec:siaf} }

\ssection{Verifying the ACQ/IMAGE WCA-to-SA Offsets.\label{sec:acqimage} }
\vspace{-0.3cm}

In the servicing mision orbital verification (SMOV) phase, a series of programs in NUV imaging mode
carefully determined the two-dimensional offset from the WCA to the center of the PSA when observed with MIRRORA.
These X and Y offsets were loaded in the FSW parameters {\it pcta\_XImCalTargetOffset} and {\it pcta\_YImCalTargetOffset}.
A target was then centered using a PSA+MIRRORA ACQ/IMAGE, then a target image was taken along with a MIRRORB image
of the WCA image. These images were used to determine the X and Y offsets of the image target and WCA centroids.
These values were uploaded in the FSW paramaters. This bootstrapping procedure was repeated with the BOA+MIRRORA
and BOA+MIRRORB ACQ/IMAGE modes until all four ACQ/IMAGE modes were co-aligned.

In this program (13972) we use this bootstrapping strategy to test the co-alignment of all four ACQ/IMAGE modes.\footnote{On November 6, 2014,
the MIRRORB ACQ/IMAGE wavelength calibration lamp exposure was changed from a 30 second exposure
at LOW current (3mA) to a 12 second exposure at MEDIUM current. At this point the {\it pcta\_XImCalTargetOffset} and {\it pcta\_YImCalTargetOffset}
FSW parameters were also updated to reflect a small change in the WCA-to-SA imaging MIRRORB offsets. This program is the first
to monitor the updated offsets.}
To accomplish this in only two orbits, this project leverages observations taken in FGS-to-SI alignment verification
program (14035).
\clearpage
The FGS-to-SI program (14035) performs a PSA/MIRRORA ACQ/IMAGE on a target that should be centered in the aperture.
The PSA+MIRRORA ACQ/IMAGE in Visit 'A2' of 14035 can be used to verify the COS NUV PSA aperture position in the SIAF.
This exposure shows that the COS NUV PSA SIAF entry combined with the PSA+MIRRORA WCA-to-PSA offsets are
accurate to within [AD,XD] = [-0.020,0.105]\arcsec\ (this is the distance that the ACQ/IMAGE slewed to center the target).
The COS aperture is only repeatable in the XD direction to $\pm$ one motor step (0.05\arcsec). In addition, the WCA location
cannot be measured to better than 1/2 pixel as the pixel used to determine the median location in an integer.
On the NUV detector, 1 pixel is $\sim$ 0.023\arcsec. Based upon this information, the COS NUV PSA definition
in the SIAF file appears to meet our accuracy requirements for Cycle 22.

After the 14035 Visit 'A2' PSA+MIRRORA ACQ/IMAGE, a PSA+MIRRORB ACQ/IMAGE is performed.
This exposure bootstraps the PSA+MIRRORB centering to the PSA+MIRRORA + SIAF verification.
This allows us to monitor the properties of the PSA+MIRRORB image in a controlled way on a centered target. No spectra or images are taken in 14035 due to time constraints.
Visits 01 \& 02 of 13972 extend the COS SIAF/FGS-to-SI verification of Visit 02 of 14035 to the other two ACQ/IMAGE combinations (BOA+MIRRORA \& BOA+MIRRORB) by bootstraping from the PSA+MIRRORB verification to co-align all the COS TA imaging modes. The details of the observations are given is the observing section.
Visit 01 of 13972 bootstraps off Visit 02 of 14035 to co-align the PSA+MIRRORB ACQ/IMAGE mode to the BOA+MIRRORA. Visit 01 of 13972 executed within 45 days of Visit 02 of 14035, to ensure that no long term instrument or telescope focus changes impact our results.
Visit 02 of 13972 follows the style of Visit 01, and bootstraps from the BOA+MIRRORA mode to the BOA+MIRRORB TA imaging mode. Visit 02 should also occur within 45 days of visit 02 of 14035 and within 45 days of Visit 01 of 13972.

The results of 13972 \& 14035 show that, for ACQ/IMAGEs :
\footnotesize
\begin{itemize}
\item PSA+MIRRORA is aligned with PSA+MIRRORB to [AD, XD] $\le$ [0.022, 0.007] \arcsec\ (14035, Visit 'A2')
\item PSA+MIRRORB is aligned with BOA+MIRRORA to [AD, XD] $\le$ [0.023, 0.100] \arcsec\ (13972, Visit '01')
\item BOA+MIRRORA is aligned with BOA+MIRRORB to [AD, XD] $\le$ [0.022, 0.024] \arcsec\ (13972, Visit '02')
\end{itemize}
\normalsize

These results can be combined to show the measured offsets of PSA+MIRRORB, BOA+MIRRORA, \& BOA+MIRRORB when compared
to the initial PSA+MIRRORA ACQ/IMAGE of Visit 'A2' of 14035. These results are shown in Table~\ref{table:ai}.
Combined offsets from PSA+MIRRORA are provided in both NUV pixels (p) and in arcseconds (\arcsec).

\begin{deluxetable}{|r|r|r|r|r|r|}
\tabcolsep 10pt
\tabletypesize{\footnotesize}
\tablecolumns{6}
\tablewidth{0 pt}
\tablecaption{ACQ/IMAGE WCA-to-SA offsets from PSA+MIRRORA\label{table:ai}}
\tablehead{\colhead{Aperture}&\colhead{MIRROR}&AD Offset & XD Offset & AD Offset& XD Offset\\
&&(\arcsec) & (\arcsec) & (p) & (p)
}
\startdata
\hline
PSA & B & 0.021 &-0.049 & 0.298 & 0.893\\
BOA & A & 0.010	& 0.060 & 0.425 & 2.550\\
BOA & B & 0.036	& 0.070 &1.530 & 2.975 \\
\hline
\enddata

\end{deluxetable}
\clearpage
\ssection{Verifying the ACQ/PEAKXD WCA-to-PSA Offsets. \label{sec:acqpeakxd} }

After the series of ACQ/IMAGEs that start each visit of 13972, the target should be accurately
centered. We take advantage of this to test the WCA-to-PSA offsets required for the spectroscopic
TA sequence ACQ/PEAKXD (TA in the XD direction). In the COS FSW, these values are contained in the
patchable constant table {\it pcta\_CalTargetOffset}. The verification process is simple, take a normal
spectrum with a target signal-to-noise ratio of least 50 for the entire spectrum (2500 target counts)
used by ACQ/PEAKXD. For NUV exposures, this is the 'B' stripe only, and for the FUV, only events on the
'A' segment are used. TA subarrays are used to make out any detector hot-spots or Geocoronal light that
could interfere with the centering process.

In Visit 01, we take spectra that meet these requirements with the G130M/1309, G140L/1280, G285M/2676, \& G230L/3000, and in Visit 02,
we take spectra with the G160M/1600, G185M/1913, G225M/2306. Table ~\ref{table:peakxd} the results of these exposures are summarized.
The rightmost column gives the WCA-to-PSA offsets measured in 13972, in arcseconds (\arcsec).
All exposures, except {\bf lcri01h6q}, the G140L/1280 measurement, which showed an offset of 0.15\arcsec\ exceed our $\pm 0.1$\arcsec\ goal.
All exposures exceed our $\pm 0.33$\arcsec\ requirement. The XD profile of G140L spectra is wider that the medium
resolution gratings (G130M and G160M), making in more susceptible to detector 'Y-walk' (Penton \& Keyes, 2010).
No action is required at this time as the measured offset is 1/2 of our 0.3\arcsec\ requirement.

The final two exposures of the 02 visit intentionally offset the target by $\pm$ 0.7\arcsec\ to test the effects
of 'Y-walk' on G160M ACQ/PEAKXDs. All three G160M exposures in Visit 02 show offsets from the expected position
of $\le 0.05$\arcsec\ within our 0.1\arcsec\ goal. No action (e.g., updating the {\it pcta\_CalTargetOffset} in the FSW)
is required at this time.
%\input{WCA\_to_SA\_table.tex}
\clearpage
\ssection{Verifying the ACQ Subarrays \label{sec:subarray} }

The TA subarrays for all exposures were extracted from the "\_spt.fits" files. All values indicate that
the intended subarrays are being used for all TA and science exposures. All FUV spectra were visually
inspected to verify that the TA subarrays were successfully excluding all known detector hot-spots and the
bright Geocoronal emission lines that can negatively affect TAs.  No action is required based upon this
analysis of the TA and science sub-arrays used in HST Cycle 21.
\clearpage
\vspace{-0.3cm}
\ssection{WCA Lamp Images (aka, Lamp Family Portrait) \label{sec:family_portrait} }
\vspace{-0.3cm}

\begin{figure}
\noindent\includegraphics*[width=0.795\linewidth]{FP_13526.png}
\caption{Cycle 21 PtNe Lamp `Family Portrait'' \ref{fig:FG21}}
\end{figure}

\begin{figure}
\noindent\includegraphics*[width=0.795\linewidth]{FP_13972.png}
\caption{Cycle 22 PtNe Lamp `Family Portrait'' \ref{fig:FG22}}
\end{figure}
\begin{figure}
\noindent\includegraphics*[width=0.795\linewidth]{FP_14440_V01.png}
\caption{Cycle 23 PtNe Lamp `Family Portrait'' \ref{fig:FG23v1}}
\end{figure}
\begin{figure}
\noindent\includegraphics*[width=0.795\linewidth]{FP_14440_V02.png}
\caption{Cycle 23 PtNe Lamp `Family Portrait'' \ref{fig:FG23v2}}
\end{figure}

\begin{figure}
\noindent\includegraphics*[width=0.795\linewidth]{FP_14857.png}
\caption{Cycle 24 PtNe Lamp `Family Portrait'' \ref{fig:FG24}}
\end{figure}

\vspace{-0.3cm}
\ssection{Conclusions.\label{theend} }
\vspace{-0.3cm}

%%%%%%%%
%Acknowledgements
%%%%%%%%
\vspace{-0.3cm}
\ssectionstar{Acknowledgements}
\vspace{-0.3cm}
%%%%%%%%
%Change History
%%%%%%%%
\vspace{-0.3cm}
%Put instrument, year, and ISR number
\ssectionstar{Change History for COS ISR 2018-XX}\label{sec:History}
\vspace{-0.3cm}
%Put publication date
Version 1: 28-Feb-2018 Original Draft Document
%%%%%%%%
%References
%%%%%%%%
\vspace{-0.3cm}
\ssectionstar{References}\label{sec:References}
\vspace{-0.3cm}

\noindent
Penton, S., 2016, COS Instrument Science Report 2019-09
Penton, S., 2016, COS Instrument Science Report 2019-09, C22 Summary
Penton, S., 2017, COS Instrument Science Report 2019-09, C23 Summary
Penton, S., 2018, COS Instrument Science Report 2019-09, C24 Summary
\\
Penton \& Keys, COS TIR
\\
C22 IHB
C23 IHB
C24 IHB

\\
TAACOS1
TAACOS2
\\

\newpage
%%%%%%%%
%Appendix
%%%%%%%%
\vspace{-0.3cm}
\ssectionstar{Appendix A}\label{sec:Appendix}
\vspace{-0.3cm}
\end{document}
