\begin{deluxetable}{lclcccr}
%\tablewidth{0pt}
\tabcolsep 10pt
\tabletypesize{\footnotesize}
\tablecolumns{4}
\tablecaption{\tacq{PEAKXD} WCA-to-PSA offsets} \label{tab:wcatopsa}
\tablehead{
\colhead{OPT\_ELEM}&\colhead{LP1}&\colhead{LP2}&\colhead{LP3}\\
}

\startdata
\hline
\multicolumn{4}{c}{FUV\tablenotemark{f}}\\
\hline
G130M	&	 -898	&	-943	&	-892 \\
G140L	&	 -884	&	-950	&	-857 \\
G160M	&	 -898	&	-933	&	-901 \\
\hline
\multicolumn{4}{c}{NUV\tablenotemark{n}}\\
\hline
G185M	&	3742	&	\dots	&	\dots \\
G225M	&	3746	&	\dots	&	\dots \\
G230L	&	3734	&	\dots	&	\dots \\
G285M	&	3749	&	\dots	&	\dots \\
\hline
\enddata
\tablenotetext{f}{Divide the FUV numbers by -10 to get the number of XD rows between the PSA and WCA spectra for a target centered in the aperture.}
\tablenotetext{n}{Divide the NUV numbers by 10 to get the NUV WCA-to-PSA offset. }
\tablecomments{The FSW patchable constant \textsc{pcta\_CalTargetOffsetScale} determines the FSW scaling (currently set to 10).
FUV scalings are "negative" due to parity of HST slews relative to the COS coordinate system.}
\end{deluxetable}
