% $Id: pctaWCA2SA.tex,v 1.5 2018/03/30 20:22:12 penton Exp $
\begin{deluxetable}{lrrr}
%\tablewidth{0pt}
\tabcolsep 10 pt
%\tabletypesize{\footnotesize}
\tablecolumns{4}
\tablecaption{\texttt{ACQ/PEAKXD} WCA-to-PSA offsets \label{tab:wcatopsa}}
\tablehead{
\colhead{\texttt{OPT\_ELEM}}&\colhead{LP1}&\colhead{LP2}&\colhead{LP3}\\
}

\startdata
\hline
\multicolumn{4}{c}{FUV\tablenotemark{f}}\\
\hline
G130M	&	 -898	&	-943	&	-892 \\
G140L	&	 -884	&	-950	&	-857 \\
G160M	&	 -898	&	-933	&	-901 \\
\hline
\multicolumn{4}{c}{NUV\tablenotemark{n}}\\
\hline
G185M	&	3742	&	\dots	&	\dots \\
G225M	&	3746	&	\dots	&	\dots \\
G230L	&	3734	&	\dots	&	\dots \\
G285M	&	3749	&	\dots	&	\dots \\
\hline
\enddata
\tablenotetext{f}{Divide the FUV numbers by -10 to get the number of XD rows between the PSA and WCA spectra for a target centered in the aperture.}
\tablenotetext{n}{Divide the NUV numbers by 10 to get the NUV WCA-to-PSA offset. }
\tablecomments{The FSW patchable constant \textsc{pcta\_CalTargetOffsetScale} determines the FSW scaling (currently set to 10).
FUV scalings are "negative" due to parity of HST slews relative to the COS coordinate system. {\bf Note to reviewers: Do you think I should keep the numbers in their FSW
values (not scaled), or should I go ahead and scale them ?}}
\end{deluxetable}
