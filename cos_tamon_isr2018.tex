% $Id: cos_tamon_isr2018.tex,v 1.3 2018/03/30 21:22:16 penton Exp $
\documentclass{stsci_report}
\usepackage{graphicx}
\usepackage{threeparttable}
\usepackage{ulem}
%\usepackage{amsfonts}
\usepackage{latexsym}
%\usepackage{siunitx}
\usepackage{xcolor}
\usepackage{times}
\usepackage{deluxetable}
\usepackage{longtable}
\usepackage[colorlinks=true,linkcolor=blue]{hyperref}

%\bibliography{bibliography.bib}
\copyrighttext{Copyright\copyright\ \the\year\ The Association of Universities for Research in Astronomy, Inc. All Rights Reserved.}

\presubtitle{Instrument Science Report HST+COS 2018}
\title{\textbf{Cycle 21-24 HST+COS\\Target Acquisition Monitoring}}
\author{Steven V. Penton}
\date{\today}

\DeclareGraphicsRule{.ps}{eps}{.ps}{}
\definecolor{green}{rgb}{0, 1.0, 0}
\definecolor{red}{rgb}{1,0,0}
\definecolor{blue}{rgb}{0,0,1}
\definecolor{Lblue}{rgb}{0.8,0.85,1}
\definecolor{darkgreen}{rgb}{0.25,1.0,0.25}
\definecolor{Brown}{cmyk}{0, 0.8, 1, 0.6}
\definecolor{Yellow}{rgb}{1, 1, 0}
\definecolor{Light}{gray}{.80}
\definecolor{Dark}{gray}{.20}

\newcommand{\numpos}{\texttt{NUM$\_$POS}}
\newcommand{\numposone}{\texttt{NUM$\_$POS=1}}
\newcommand{\psiafdate}{2010.060}
\def\arcsec{\hbox{$^{\prime\prime}$}}
\def\degree{\hbox{$^{\circ}$}}
\newcommand{\nokmsno}{{\rm km~s}\ensuremath{^{-1}}}
\newcommand{\kmsno}{~\nokmsno}
\newcommand{\kms}{~\nokmsno\ }
\newcommand{\tacq}[1]{\texttt{ACQ/#1}}
\newcommand{\pid}[1]{{\bf P}#1}
\newcommand{\cenwaveno}{\textit{CENWAVE}}
\newcommand{\cenwave}{\cenwaveno\space}
\def\cenwaves{\textit{CENWAVE}s }
%\newcommand{\dotuline}{\bgroup \markoverwith{\lower .4ex\hbox{.}}\ULon}

\begin{document}
\maketitle

% $Id: abstract.tex,v 1.5 2018/04/18 04:10:05 penton Exp $
\abstract{
This ISR documents the  Cosmic Origins Spectrograph (COS) Target Acquisition (TA) annual monitoring programs for HST Cycles 20--24.
During this period, NUV exposures were obtained at the nominal (LP1) position, and FUV exposures were executed at Lifetime Positions LP2 and LP3.
These programs were designed to monitor numerous aspects of both imaging and spectroscopic TAs, including checking the TA subarrays, the NUV SIAF entries, the telescope slew distances, and evaluating the accuracy of numerous COS flight software (FSW) patchable constants required for TA.
This project verified that all three COS TA modes (FUV spectroscopic, NUV spectroscopic, and NUV imaging) were, on large, behaving nominally in Cycle 20--24, and determined that no SIAF or FSW parameter updates were required during this time, with the exception of changes to MIRRORB \texttt{ACQ/IMAGE} in 2014.
These changes included a changing of the lamp current from LOW to MEDIUM, an adjustment of the \textsc{LTACAL} exposure time, and a modification of both the MIRRORB WCA and PSA/BOA  \texttt{ACQ/IMAGE} TA subarrays.}

\clearpage
\tableofcontents
\listoffigures
\listoftables
\newpage
%% $Id: Contents.tex,v 1.2 2018/03/27 18:32:49 penton Exp $
%\vspace{-0.3cm}
\section*{Contents}
%\vspace{-0.3cm}

\begin{itemize}
\item Introduction (page \pageref{sec:Introduction})
\item COS TA Operations Summary (page \pageref{sec:TAoperations})
\item Program Descriptions (page \pageref{sec:programs})
\item SIAF Verification (page \pageref{sec:siaf})
\item Imaging TA Verifiation (page \pageref{sec:NimVER})
\item Spectroscopic TA Verification (page \pageref{sec:spVER})
\item TA Subarray Verification (page \pageref{sec:subarray})
\item Results (page \pageref{sec:results})
\item Conclusion (page \pageref{sec:theend})
\end{itemize}

% $Id: intro.tex,v 1.5 2018/03/30 20:22:12 penton Exp $
\section{Introduction}\label{sec:Introduction}

Preliminary results of the Hubble Space Telescopes' (HST) Cosmic Origins Spectrograph (COS) target acquisition (TA) programs reviewed here were previously reported in the following COS ISRs:
\small
\begin{itemize}
\item{COS ISR 2015-02 (Summary of the COS Cycle 20 Calibration Program)}
\item{COS ISR 2015-06 (Summary of the COS Cycle 21 Calibration Program)}
\item{COS ISR 2016-03 (Summary of the COS Cycle 22 Calibration Program)}
\item{COS ISR 2016-09 (Cycle 22 COS Target Acquisition Monitor Summary)}
\item{COS ISR 2017-18 (Cycle 23 COS Target Acquisition Monitor Summary)}
\item{COS ISR 2018-09 (Cycle 24 COS Target Acquisition Monitor Summary)}
\end{itemize}
\normalsize
The information in this ISR supercedes any previous preliminary results or conclusions.\\

This ISR provides the full details of the following HST+COS calibration {\bf P}rograms:
\small
\begin{itemize}
\item{\pid{13124} (COS Imaging TA and Spectroscopic WCA-PSA/BOA offset verifications, Cycle 20)}
\item{\pid{13526} (COS Imaging TA and Spectroscopic WCA-PSA/BOA offset verifications, Cycle 21)}
\item{\pid{13972} (COS Imaging TA and Spectroscopic WCA-PSA/BOA offset verifications, Cycle 22)}
\item{\pid{14440} (COS Imaging TA and Spectroscopic WCA-PSA/BOA offset verifications, Cycle 23)}
\item{\pid{14857} (COS Imaging TA and Spectroscopic WCA-PSA/BOA offset verifications, Cycle 24)}
\end{itemize}
\normalsize

\subsection{Introductory Notes and Conventions}\label{subsec:conventions}
%\vspace{-0.3cm}
There are a few COS conventions to be established before discussing the TA monitoring in detail.
\begin{enumerate}
	\item{COS TAs are performed in raw or ``detector'' coordinates, not the ``user'' coordinate system of calibrated
		COS files. To avoid confusion over the different coordinate systems, we will use along-dispersion (AD) and cross-dispersion (XD) whenever possible.
		\dotuline{All references to the coordinates ``X'' and ``Y'' are in the detector coordinate system unless otherwise specified.}
		In raw NUV coordinates, +X is -XD and +Y is -AD. In raw FUV coordinates, +X is -AD and +Y is +XD.
		The transformations between user and detector coordinates are :
		\begin{equation} NUV: X_{user} = 1023 - Y_{detector} \ ; Y_{user} = 1023 - X_{detector} \end{equation}
		\begin{equation} FUV: X_{user} = 16383 - X_{detector} \ ; Y_{user} = Y_{detector} \end{equation}
		}
	\item{When referencing NUV pixels, we will abbreviate pixel as p. For the FUV, we use DE (or rows/columens) to reference the FUV digital elements.}
	\item{When discussing the various subarrays used during COS TA, boxes will be specified by giving the lowest
		valued corner (C) and full size (S) for both X and Y. A box is fully specified by giving its XC, XS, YC, \& YS. In this TIR, these will always be given in detector coordinates.}
	\item{To clarify the names and locations of various TA parameters, the following convention will be used :
		\begin{itemize}
			\item{COS TA modes and their optional parameters will be in \texttt{Courier} (e.g., \tacq{IMAGE} and \numpos).}
			\item{Keywords in FITS headers will be in \textit{ITALICIZED ALL CAPITALS} (e.g., \textit{ACQSLEWY}).}
			\item{Flight SoftWare parameters (FSW) will be in \textsc{small capitals}.
All TA FSW patchable constants begin with ``\textsc{pcta\_}'' (e.g., \textsc{pcta\_CalTargetOffset}). In this ISR, this prefix is considered implied after the initial introduction of a paramater, and will not always be included.
FSW patchable constants relating to mechanism positions begin with \textsc{pcmech\_} and will always be included in references.}
			\item{Archived COS files are in FITS (.fits) format. FITS filenames, or portions of a filename, will be in {\sf sans-serif} (e.g., {\sf ld9mg2nrq\_rawtag.fits} or {\sf \_spt.fits}).
			COS filenames are in the form {\sf IPPPSSOOT\_{\it extension}.fits}.
			The HST naming convention breaks down for COS as I=Instrument=``L'', PPP=Program ID, SS=Visit ID, OO=Exposure ID,
			and T=``Q'' for nominally recorded observations. See the COS DHB for a full breakdown of the HST IPPPSSOOT naming conventions.
			COS TA files have the {\it extension} of {\sf rawacq}, and additional
			information useful for TA analysis is contained in the {\sf IPPPSSOOT\_{\it spt}.fits} file known as the support file,
			and in the {\sf IPPPSSOOT\_{\it jit/f}.fits} file known as the jitter files.}
		\end{itemize}
	}
%	\item{There are three centering options during \tacq{SEARCH} and \tacq{PEAKD}. In the Astronomers Proposal Tool (APT), these are
%		referred to as \texttt{CENTER}=\texttt{FLUX-WT}, \texttt{FLUX-WT-FLR}, and \texttt{BRIGHTEST}.
%		These parameters have slightly different names in the IHB, the FITS keywords, and the FSW.
%		In this ISR, we will refer to the centering options as \texttt{CENTER}=\texttt{Flux-Weighted (FW)}, \texttt{Flux-Weighted-Floor (FWF)}, and \texttt{Return-To-Brightest (RTB)}.
%	}
	\item{COS contains numerous FUV and NUV central wavelength settings, which are defined in the FSW by the OSM1 or OSM2 rotation positions.
	In this ISR, the term \cenwaveno, which is also the FITS keyword name, will be used to mean any of the pre-defined OSM1 + OSM2 rotation settings that uniquely define a central wavelength setting.}
	\item{COS \cenwaves are named for the (predicted) lowest wavelength that lands on the FUVA detector segment for \textit{FP-POS}=3. For convienence, when referring to
	a specific \cenwave we will either call out the grating and \cenwave is use as GRATING/\cenwave  (e.g. G130M/1222), or just use a leading ``C'' to identify a particular \cenwave (e.g., C1222) in the same manor as ``G" is used for GRATING (e.g., G130M).
	Note that the FITS header keyword equivalent of GRATING is \textit{OPT\_ELEM}.}
	\item{Unless specified, all spectroscopic exposures were taken at \textit{FP-POS}=3.}
	\item{When referring to an HST program number, we will use either ``HST PID" or a leading ``{\bf P}" in a similar fashion an ``C=\cenwave'' and ``G=GRATING'', but using a {\bf bold} font.}
	\item{The COS FUV detector has two independent segments, Segment-A and Segment-B. In this ISR, they will be referred to as FUVA \& FUVB.}
	\item{\tacq{IMAGE} can use either of two ``MIRROR'' modes, MIRRORA or MIRRORB. In this ISR, they will be referred to as MIRA \& MIRB.}
	\item{Following the conventions used in APT and the Phase~II Proposal Instructions (Rose et al., 2017), NUV \tacq{PEAKXD} exposures will specify which \texttt{STRIPE}\footnote{\texttt{STRIPE} is the optional parameter name in APT, therefore the \texttt{Courier} font is used.} is used during TA. In this ISR, we will always use
	the default (\texttt{STRIPE=DEF}) for a given \cenwave. This default is \texttt{STRIPE=MEDIUM} (or STRIPE=B) for all \cenwaves, except G230L/3360 where it is \texttt{STRIPE=SHORT} (STRIPE=A).}
	\item{When referring to a particular day, we will use YEAR.DAY. For example, day 60 of 2010 will be referred to as \psiafdate. We will also occasionally use decimal years. In these cases, there will only be a single digit in the fractional part (e.g., 2009.9).}
	\item{HST observations are grouped in approximately annual ``cycles''. `C\#\#' will be used as shorthand for ``HST Cycle \#\#'' (e.g., Cycle~19~=~C19).}
	\item{Unit abbreviations:
		\begin{itemize}
		\item{Milli-arcseconds (0.001\arcsec) will be abbreviated as mas.}
		\item{Milli-amperes (0.001A) will be abbreviated as mA.}
		\item{Counts per second will be abbreviated as cps.}
		\end{itemize}
	}
	\item{COS has two internal PtNe wavelength calibrations lamps that send light through the Wavelength Calibration Aperture (WCA) and onto the detectors.The two PtNe lamps are referred to in this ISR
	as P1 and P2. Each lamp has three current settings, LOW, MEDIUM (MED) or HIGH. The P1 lamp is used for spectroscopic lamp flashes during science exposures (``TAGFLASH''es), while the P2 lamp is used for all TA exposures.
	Both lamps have MED current settings of 10 mA, but the P1 lamp has LOW/HIGH current setting of 6/18~mA. The P2 lamp
	has LOW/HIGH current settings of 3/14 mA. COS Lamp output generally scales as current$^{2}$ ($P=I^2 R$).}
	\item{{\bf Note to reviewers: I often switch back and forth between the APT TA routine names (ACQ/) and the FSW equivalents (LTA..). If you find this confusing, I can put in a conversion table and establish a convention for
	when I use each flavor. Please advise.}}
\end{enumerate}

\subsection{ISR Organization}\label{subsec:org}
In \S~\ref{sec:TAoperations} we will discuss the concepts involved
in the TA monitoring strategy along with a basic review of COS TA operations and
centering requirements (\S~\ref{subsec:requirements}).
In \S~\ref{sec:structure} we will discuss the details of the individual
COS TA monitoring programs and, in \S~\ref{subsec:elists} list the individual exposures.
Also in this section, we will discuss the annual HST FGS-to-SI alignment programs and there
connection to the COS TA monitoring programs (\S~\ref{subsec:fgs2si}).

In \S~\ref{sec:subarray}, we discuss the numerous detector subarrays used in COS TA, and their verification by the programs in this ISR.

In \S~\ref{subsec:acqimage} we will discuss the verification of the FSW parameters, lamp operations, and subarrays associated with COS \tacq{IMAGE}s.

In ~\S~\ref{sec:spVER}, we will discuss the verification of the FSW parameters, lamp operations, and subarrays associated with COS spectroscopic TAs.
%As part of this process, we will verify the active COS NUV SIAF (Science Instrument Aperture File) entries.
%In the FUV sub-section we will discuss and the verification of COS FUV SIAF entries.


\clearpage
\section{COS TA Operations Summary}\label{sec:TAoperations}

There are three modes of Target Acquisition (TA) for the Cosmic Origins Spectograph (COS); NUV imaging, NUV spectroscopic, and FUV spectroscopic.
There are four COS TA (\tacq{}) procedures; \tacq{IMAGE}, \tacq{PEAKD}, \tacq{PEAKXD}, and \tacq{SEARCH}.
\tacq{PEAKD} and \tacq{SEARCH} step the telescope through dwell patterns on the sky.
As long as the target light falls correctly within the TA detector sub-arrays, \tacq{PEAKD} and \tacq{SEARCH} will continue to nominally assist in TA (barring any unforeseen anomalies, such as detector `hot-spots').
The \tacq{IMAGE} and \tacq{PEAKXD} procedures also rely on the sub-arrays, but also rely on numerous patchable (changeable) constants
in the COS flight software (FSW) which assist in target centering.

In both \tacq{IMAGE} and \tacq{PEAKXD}, the internal wavelength calibration lamp is flashed to locate the center of the wavelength calibration aperture (WCA).
From this location, the center of the science aperture (SA) in use, which could be the PSA or BOA, can be predicted by applying the FSW constants that give the SA offset compared to the WCA center. For \tacq{IMAGE},
the offset is in both detector `X' (along-dispersion, AD) and `Y' (cross-dispersion, XD).
For \tacq{PEAKXD}, which uses dispersed light, this offset is only in the Y (XD) direction.
All programs verify that the TA subarrays in use for the given cycle were proper for the \tacq modes tested, verify that the actively used WCA-to-SA offsets, and monitor, as much as possible, the performance of COS TAs.

BOA spectroscopic TAs were not supported for COS during Cycles 19-24, accordingly  the programs discussed here only verify PSA spectroscopic TAs.
WCA-to-PSA offsets are used in \tacq{PEAKXD}s, and each COS grating has a different XD offset. These offsets are both grating (\textsc{OPT\_ELEM})
and lifetime position (LP) dependent.\footnote{In the COS FSW, these WCA-to-SA XD offsets are stored in the \textsc{pcta\_CalTargetOffset} table.}
The programs listed here verify the NUV LP1 as well as the FUV LP2 and LP3\footnote{The COS FUV channel was moved to LP3 on February 15, 2015.} position offsets.
The FUV LP4 uses a different \tacq{PEAKXD} algorithm ({\numpos$>1$), and, like \tacq{PEAKD}, does not use the WCA-to-SA XD offsets\footnote{All NUV and FUV LP1-3 \tacq{PEAKXD} observations use the optional parameter, \numpos=1.}.

The initial HST/COS target pointing is based on definitions of the physical locations of the COS apertures in terms of [V2,V3] in the Science Instrument Aperture File (SIAF).
All of the actively used NUV (LP1) and FUV (LP2\footnote{The default COS FUV spectral location was moved to LP2 on February XX, YYY, for all CENWAVEs.}
and LP3\footnote{The default COS FUV spectral location was moved to LP3 on February 15, 2015, for all CENWAVEs except G130M/1055 and G130M/1096, which still operate at LP2. On October 2, 2017, the default location of COS FUV spectra were moved to LP4, with additional observing and TA constraints as outlined on the COS2025 website (http://www.stsci.edu/hst/cos/cos2025).})
SIAF entries used for TA during Cycles~21--24 are also verified in this program.\footnote{These entries are not really really tested that accurately, because ...\dots.}

The COS TA centering requirements are based upon the wavelength accuracy requirements in the AD, and for flux and resolution optimization
in the XD. The strictest NUV requirements are [AD,XD] = [0.041, 0.3]\arcsec, while for the FUV they are [AD,XD] = [0.106, 0.3]\arcsec.\footnote{While the XD requirement for all TAs is $\pm$ 0.3\arcsec, our 1$\sigma$ goal is $\pm$ 0.1\arcsec. This goal ensures that spectra fall on a consistent XD location on the the detector, which aids in extraction and calibration accuracy.}
COS TA requirements are discussed in detail in \S~\ref{subsec:requirements}.

This program does not attempt to monitor the AD accuracy of the COS spectroscopic TA modes.\footnote{For \tacq{PEAKD}, short-term fluctuations of the detector background rate due to environmental conditions remains the largest source of AD pointing error.}

\subsection{COS TA Centering Requirements}\label{subsec:requirements}

The COS TA centering requirements are based upon the wavelength accuracy requirements in the AD, and for flux and resolution optimization
in the XD.\footnote{The COS requirements are documented in the CEI (Contract End Item) Specification (Smith et. al., 2004).} The strictest NUV requirements are [AD,XD] = [0.041, 0.3]\arcsec, while for the FUV they are [AD,XD] = [0.106, 0.3]\arcsec.\footnote{While the XD requirement for all TAs is $\pm$ 0.3\arcsec, our 1$\sigma$ goal is $\pm$ 0.1\arcsec. This goal ensures that spectra fall on a consistent XD location on the the detector, which aids in extraction and calibration accuracy.}
Since the AD requirement is in units of \kmsno, it is detector, grating, and wavelength dependent as defined in equations~\ref{eq:TAcenter}--\ref{eq:TAcenterL}.
\small
\begin{eqnarray}\label{eq:TAcenter}
\Delta\ AD(G185M@1825\AA) = {{ 15\kms \times 1825\AA}\over{c \times 0 .037\AA/p\times 42.47 p/\arcsec}}  = 0.058\arcsec\\
\Delta\ AD(G225M@2250\AA) = {{ 15\kms \times 2250\AA}\over{c \times  0.035\AA/p\times 42.47 p/\arcsec}}  = 0.076\arcsec\\
\Delta\ AD(G285M@2850\AA) = {{ 15\kms \times 2850\AA}\over{c \times  0.040\AA/p\times 42.47 p/\arcsec}}  = 0.084\arcsec\\
\Delta\ AD(G230L@2450\AA) = {{175\kms \times 2450\AA}\over{c \times  0.390\AA/p\times 42.47 p/\arcsec}}  = 0.086\arcsec\\
\Delta\ AD(G130M@1300\AA) = {{ 15\kms \times 1300\AA}\over{c \times 0.00997\AA/ p\times 43.5 p/\arcsec}} = 0.150\arcsec\\
\Delta\ AD(G160M@1600\AA) = {{ 15\kms \times 1600\AA}\over{c \times 0.01223\AA/ p\times 42.9 p/\arcsec}} = 0.153\arcsec\\
\Delta\ AD(G140L@1800\AA) = {{150\kms \times 1800\AA}\over{c \times 0.08030\AA/ p\times 45.4 p/\arcsec}} = 0.247\arcsec\label{eq:TAcenterL}
\end{eqnarray}
\normalsize

Assuming that the wavelength error budget is split evenly between the COS TA and wavelength scale accuracies,
the error budgets for the COS gratings, in arc-seconds (\arcsec), are given in Table~\ref{tab:TAaccuracy}. By ``evenly'' we mean that when added in quadrature the total error budget is that given by the second column of Table~\ref{tab:TAaccuracy}.
Setting the TA error budget equal to the wavelength scale accuracy, the AD TA requirement given in the third column is the second column divided by $\sqrt{2}$.

% $Id: TAaccuracy.tex,v 1.6 2018/04/18 04:10:05 penton Exp $
\begin{deluxetable}{cccc}
\tabletypesize{\footnotesize}
\tablewidth{5.5 in}
%\tabcolsep 10pt
\tablecolumns{4}
\tablecaption{COS TA Centering Requirements}\label{tab:TAaccuracy}
\tablehead{\colhead{\textit{OPT\_ELEM}} & \colhead{Total AD Error Budget} & \colhead{AD TA Requirement\tablenotemark{a}} & \colhead{XD TA Requirement\tablenotemark{x}}\\
\colhead{(1)} & \colhead{(2)} & \colhead{(3)} & \colhead{(4)}
}
\startdata
	\toprule
	\multicolumn{4}{c}{NUV}\\
	\midrule
	G185M & 0.058\arcsec\ & 0.041\arcsec{} & 0.3 (0.1)\arcsec\\
	G225M & 0.076\arcsec\ & 0.054\arcsec{} & 0.3 (0.1)\arcsec\\
	G285M & 0.084\arcsec\ & 0.059\arcsec{} & 0.3 (0.1)\arcsec\\
	G230L & 0.086\arcsec\ & 0.061\arcsec{} & 0.3 (0.1)\arcsec\\
	\midrule
	\multicolumn{4}{c}{FUV}\\
	\midrule
	G130M & 0.150\arcsec\ & 0.106\arcsec{} & 0.3 (0.1)\arcsec\\
	G160M & 0.153\arcsec\ & 0.108\arcsec{} & 0.3 (0.1)\arcsec\\
	G140L & 0.247\arcsec\ & 0.175\arcsec{} & 0.3 (0.1)\arcsec\\
	\bottomrule
\enddata
\tablenotetext{a}{Assuming the total AD error budget (column~2) is split equally between TA centering and wavelength scale accuracy,
the AD TA requirements (column~3) are 1/$\sqrt{2}$ of the total AD error budget (equations~\ref{eq:TAcenter}--\ref{eq:TAcenterL}).}
\tablenotetext{x}{The XD requirement is 0.3\arcsec, but our 1$\sigma$ goal is 0.1\arcsec.}
\end{deluxetable}

\clearpage
% $Id: programs.tex,v 1.3 2018/03/29 19:16:24 penton Exp $
\section{Program Descriptions} \label{sec:programs}

COS \tacq{IMAGE} has four commonly used combinations of two Science Apertures (SAs), the Primary Science Aperture (PSA) and the Bright Object Aperture (BOA), and two mirror modes, MIRA and MIRB.
During the 2009 servicing mission orbital verification (SMOV) phase, a series of C17 calibration programs in NUV imaging mode (\pid{11469}, \pid{11473}, \& \pid{11471}) carefully determined the two-dimensional offset from the COS WCA to the center of the PSA when observed with MIRA.
These X and Y offsets were loaded in the FSW TA parameters\footnote{In the COS FSW, these WCA-to-SA offsets are stored as patchable constants in the \textsc{pcta\_XImCalTargetOffset} (XD) and \textsc{pcta\_YImCalTargetOffset} (AD)}.}
A target was then centered using a PSA+MIRA \texttt{ACQ/IMAGE}, then a target image was taken along with a MIRB image
of the WCA image. These images were used to determine the AD (Y) and XD (X) offsets of the image target and WCA centroids.
These values were uploaded in the FSW paramaters. This bootstrapping procedure was repeated with the BOA+MIRA
and BOA+MIRB \texttt{ACQ/IMAGE} modes until all four \texttt{ACQ/IMAGE} modes were co-aligned.

In the COS TA Monitoring programs described in this ISR, we re-use this bootstrapping strategy to test the co-alignment of all four \texttt{ACQ/IMAGE} modes\footnote{The underlying assumption of these programs is that that the PSA/MIRA \texttt{ACQ/IMAGE}~centering has not changed since SMOV.}.
In addition to COS calibration programs listed above, and described in detail is \S~\ref{subsec:History}--\ref{subsec:elists},
COS \tacq{IMAGE} exposures obtained in numerous cycles of the "Focal Plane Calibration (SI-FGS Alignment)" series were used in the COS TA monitoring discussed in this ISR.
These programs were developed by the HST Telescope's division (PIs Cox and/or Lallo) for Fine Guidance Sensor (FGS) to Science Instrument (SI) alignment, and are described in \S~\ref{subsec:fgs2si}.

All data for a given cycle were intentionally taken contemporaneously to avoid any long-term detector or spacecraft effects from affecting our results.
Our requirement was that all data for a given program were taken within 45 days of each other.
There were minor differences in the specific exposures in each cycles TA monitoring program, these are discussed in \S~\ref{subsec:differences}.\\

% $Id: fgs2si.tex,v 1.3 2018/03/29 19:16:24 penton Exp $
\subsection{FGS-to-SI Programs}\label{subsec:fgs2si}
From C17--C23, an FGS-to-SI program executed with COS visits twice a year. These programs contained COS exposures designed to assist in the monitoring of the COS NUV alignent to HST.
These programs used the same two target stars with COS in visits spaced six months apart. Both visits observed the astrometric open cluster M35, at orientations that were 180\degree~apart.
The two stars observed were 206W3 (in the Fall) and 427W3 (in the Spring). Due to time constraints, the exact content of the COS visits in these programs varied from year to year.

However, the COS portion of each program begins with a PSA$\times$MIRA \texttt{ACQ/IMAGE} on a target should be approximately centered due to observations with other instruments earlier in the visit.
Post-observation telemetry data, an the results of the \texttt{ACQ/IMAGE}, are used to refine this assumption.
This process verifies the COS NUV PSA aperture position\footnote{Specifically, the \textit{LFPSAA} SIAF entry.} in the SIAF to about 0.5 pixels or (~0.012\arcsec).

After this PSA$\times$MIRA \texttt{ACQ/IMAGE}, a PSA$\times$MIRB \texttt{ACQ/IMAGE} is then performed (together, a ``set'').
This bootstraps the PSA$\times$MIRB centering to the PSA$\times$MIRA and to the SIAF verification.
This allows us to monitor the properties of the PSA$\times$MIRB image in a controlled way on a centered target.

The historical list of FGS-to-SI proposals, HST cycles (C\#\#), and content are given in Table~\ref{tab:fgs2si}.
Where possible, time-tag (TT) images of the lamps and/or targets, along with NUV G230L spectra were acquired.

\begin{deluxetable}{rcl}
\tabletypesize{\footnotesize}
\tablecolumns{3}
\tablecaption{Historical List of FGS-to-SI proposals used for COS TA Monitoring.\label{tab:fgs2si}}
\tablehead{
\colhead{PID} & \colhead{Cycle} & \colhead{Summary of Contents}\\
}
\startdata
\pid{11878} & C17 & 2 sets of PSA \texttt{ACQ/IMAGE}s, Target+Lamp TT images, \& G230L Spectra \\
\pid{12399} & C18 & 2 sets of PSA \texttt{ACQ/IMAGE}s, 1 set of Target+Lamp TT images + G230L Spectrum (427W3) \\
\pid{12781} & C19 & 2 sets of PSA \texttt{ACQ/IMAGE}s \\
\pid{13171} & C20 & 2 sets of PSA \texttt{ACQ/IMAGE}s \\
\pid{13616} & C21 & 2 sets of PSA \texttt{ACQ/IMAGE}s \\
\pid{14035} & C22 & 2 sets of PSA \texttt{ACQ/IMAGE}s \\
\pid{14452} & C23 & 2 sets of PSA \texttt{ACQ/IMAGE}s,  with Lamp-Only TT images after each \texttt{ACQ/IMAGE} \\
\enddata
\end{deluxetable}

% $Id: structure.tex,v 1.5 2018/04/18 04:33:16 penton Exp penton $
\subsection{COS TA Monitoring Program Structure}\label{subsec:structure}

Each cycles TA monitoring program contains three single-orbit visits. The number of visits is mandated by the bootstrapping technique between the four different \tacq{IMAGE} SA$\times$MIR configurations.

Each visit begins with a comparison of the centering of two \tacq{IMAGE}~modes out of the possible four science apertures (SA, PSA or BOA) $\times$ (MIRA or MIRB).
This back-to-back process allows us to test that all \tacq{IMAGE} modes are centering the target to the same point in the aperture.
This comparison involves not only the \tacq{IMAGE}s, but NUV detector images of the PtNe lamp (WCA) image and, if possible, coeval target images.
These direct lamp+target comparisons are only available for the PSA modes. For the BOA modes, the WCA lamp images and target images are taken consecutively.
The lamp+target exposures are interleaved throughout the visit and are available to measure and verify the imaging WCA-to-SA offsets are still accurate for each HST Cycle.
Images will usually use the PtNe\#2 (\plamptwo{}) lamp, as it is the primary TA lamp, but some images will use PtNe\#1 (\plampone{}) to monitor both lamps in imaging mode.

In its generic format, the three, one-orbit, visits are configured as follows:
\begin{itemize}
	\item{The 1\ts{st} orbit on each program is designed to test the co-alignment of the PSA$\times$MIRA and PSA$\times$MIRB \tacq{IMAGE} configurations.
However, this exact configuration of \tacq{IMAGE}s occurs at the end of each semi-annual visit in the FGS-to-SI alignment programs (see \S~\ref{subsec:fgs2si}).
This visit was usually treated as an on-hold contingency visit in case, for whatever reason, the fall visit of the program did not execute in a given cycle.
%\footnote{This program was replaced with an improved process for aligning the FGSs. Accordingly,  we activated this contingency visit to obtain the necessary PSA$\times$MIRA and PSA$\times$MIRB exposures for C24.
The target for this contingency visit is 206W3, the same target as the Fall visit of the FGS-to-SI alignment program.
In one case, (C22, \pid{13972}), this visit was re-purposed to verify a change to the MIRB\tacq{IMAGE} configuration required due to the increasing background (see \pr{78749}).}
	\item{The 2\ts{nd} orbit of each program takes back-to-back PSA$\times$MIRB and BOA$\times$MIRA \tacq{IMAGE}s and target (WD1657+343) TIME-TAG images (with lamp flashes).
	A second PSA$\times$MIRB \tacq{IMAGE} is then performed to provide a second measurement of the offset.
	Additionally, NUV and FUV spectra are acquired to test their WCA-to-PSA offsets.}
	\item{The 3\ts{rd} orbit of each program takes back-to-back BOA$\times$MIRA and MIRB \tacq{IMAGE}s and target (HIP66578) TIME-TAG images (with lamp flashes).
	As in the 2\ts{nd} orbit, a second BOA$\times$MIRA \tacq{IMAGE} is then performed to provide a second measurement of the offset.
	Additional NUV and FUV spectra are acquired to the remaining WCA-to-PSA offsets not tested in the 2\ts{nd} orbit.}
	\item{All visits were executed in APT 3-Gyro mode (\texttt{3GOBAD}) with the \texttt{BASE1B3} guide star requirement set in APT.}
\end{itemize}
The exact configuration of which gratings and \cenwaves{} were spectroscopically tested varied with each cycle as the programs evolved.
Specifically, with the 2015 change in OSM2 home position\footnote{In May 2015, the ``home'' position of the COS Optic Select Mechanism \#2 (OSM2, the NUV grating wheel) was changed from G185M/1850 to the MIRA position to reduce wear on the OSM, increase observing efficiency, and reduce mechanism drift and position offsets during \tacq{IMAGE} TAs. (see \pr{80893} and \pr{80894}).}, NUV spectra were re-ordered for efficiency and some NUV \cenwaves{} were changed to those
that are known to have strong \textit{STRIPE=B} WCA spectra against the increasing detector background (Fix, 2018) and declining NUV sensitivity (Taylor, 2017).
In C21--C24, we took G160M/1600 exposures offset in XD by $\pm 0.7$\arcsec\footnote{Offsets set by using APT exposure level \texttt{POS$\_$TARG}s.} to test for the effects of gain sag induced`Ywalk` on FUV spectra.
In addition, one visit of each program, usually the second visit, performed an annual "family portrait"  of all the \plampone{}/\plamptwo{} MIRA/MIRB WCA lamp images to track any drifting of the centroids, or changes in the lamps with time.
The `Family Portrait` lamp images are discussed in \S~\ref{subsec:fportrait}.
Further details on the differences between the programs is provided in \S~\ref{subsec:differences}.

\input{differences.tex}
\subsection{Exposure Lists}\label{subsec:elists}

In Visit 01, we take spectra that meet these requirements with the G130M/1309, G140L/1280, G285M/2676, and G230L/3000, and in Visit 02,
we take spectra with the G160M/1600, G185M/1913, G225M/2306. Table ~\ref{tab:peakxd} the results of these exposures are summarized.
The rightmost column gives the WCA-to-PSA offsets measured in P13972, in arcseconds (\arcsec).
All exposures, except {\sf lcri01h6q}, the G140L/1280 measurement, which showed an offset of 0.15\arcsec\ exceed our $\pm 0.1$\arcsec\ goal.
All exposures exceed our $\pm 0.33$\arcsec\ requirement. The XD profile of G140L spectra is wider that the medium
resolution gratings (G130M and G160M), making in more susceptible to detector `Y-walk' (Penton \& Keyes, 2010).
No action is required at this time as the measured offset is 1/2 of our 0.3\arcsec\ requirement.

The final two exposures of the 02 visit intentionally offset the target by $\pm$ 0.7\arcsec\ to test the effects
of `Y-walk' on G160M \tacq{PEAKXD}s. All three G160M exposures in Visit 02 show offsets from the expected position
of $\le 0.05$\arcsec\ within our 0.1\arcsec\ goal. No action (e.g., updating the \textsc{pcta\_CalTargetOffset} in the FSW)
is required at this time.
% $Id: NUVimagetamonfiles.tex,v 1.4 2018/03/30 15:20:58 penton Exp $
\begin{deluxetable}{rrrrrrrrrrrrr}
\tabcolsep 4 pt
\tabletypesize{\tiny}
\tablecolumns{13}
%\tablewidth{0 pt}
\tablecaption{COS/NUV TA Monitoring Imaging Exposures\label{tab:NUVtamonimage}}
\tablehead{
\colhead{\textit{ROOTNAME}}&\colhead{\textit{PROPOSID}}&\colhead{\textit{TARGNAME}}&\colhead{\textit{OBSMODE\tablenotemark{t}}} &\colhead{\textit{EXPTYPE}}   &\colhead{\textit{EXPTIME}}  &\colhead{PtNe}&\colhead{Lamp}   &\colhead{\textit{APERTURE}}&\colhead{\textit{APERXPOS\tablenotemark{x}}}&\colhead{\textit{APERYPOS\tablenotemark{y}}}&\colhead{\textit{OPT\_ELEM}}&\colhead{\textit{DATE-OBS}}\\
\colhead{}                 &  &\colhead{}        &\colhead{}&\colhead{}&\colhead{(s)}&\colhead{Lamp \#}&\colhead{Current\tablenotemark{c}}&\colhead{}&\colhead{}&\colhead{ }&\colhead{}&\colhead{} \\
\colhead{(1)}&\colhead{(2)} & \colhead{(3)}&\colhead{(4)} &
\colhead{(5)}&\colhead{(6)} & \colhead{(7)}&\colhead{(8)} &
\colhead{(9)}&\colhead{(10)} & \colhead{(11)} &\colhead{(12)} & \colhead{(13)}
}
\startdata
\toprule
lc6ka1i1q	&	13171	&	427W3	&	ACCUM	&	ACQ/IMAGE	&	60	&	P2	&	Low	&	PSA	&	22.1	&	127.1	&	MIRA	&	2013-03-02	\\
lc6ka1i3q	&	13171	&	427W3	&	ACCUM	&	ACQ/IMAGE	&	300	&	P2	&	Low	&	PSA	&	22.1	&	127.1	&	MIRB	&	2013-03-02	\\
lc6ka2imq	&	13171	&	206W3	&	ACCUM	&	ACQ/IMAGE	&	60	&	P2	&	Low	&	PSA	&	22.1	&	127.1	&	MIRA	&	2013-09-01	\\
lc6ka2ioq	&	13171	&	206W3	&	ACCUM	&	ACQ/IMAGE	&	300	&	P2	&	Low	&	PSA	&	22.1	&	127.1	&	MIRB	&	2013-09-01	\\
lcgp01bpq	&	13523	&	WAVE	&	TT	&	WAVECAL	&	40	&	P2	&	Low	&	WCA	&	22.1	&	127.1	&	MIRB	&	2013-11-11	\\
lcgp01bsq	&	13523	&	WAVE	&	TT	&	WAVECAL	&	40	&	P1	&	Low	&	WCA	&	22.1	&	127.1	&	MIRB	&	2013-11-11	\\
lcgp01byq	&	13523	&	WAVE	&	TT	&	WAVECAL	&	20	&	P2	&	Low	&	WCA	&	22.1	&	127.1	&	MIRA	&	2013-11-11	\\
lcgp01c3q	&	13523	&	WAVE	&	TT	&	WAVECAL	&	20	&	P1	&	Low	&	WCA	&	22.1	&	127.1	&	MIRA	&	2013-11-11	\\
lci4a1dcq	&	13616	&	427W3	&	ACCUM	&	ACQ/IMAGE	&	60	&	P2	&	Low	&	PSA	&	22.1	&	127.1	&	MIRA	&	2014-04-03	\\
lci4a1deq	&	13616	&	427W3	&	ACCUM	&	ACQ/IMAGE	&	300	&	P2	&	Low	&	PSA	&	22.1	&	127.1	&	MIRB	&	2014-04-03	\\
lci4a2e3q	&	13616	&	206W3	&	ACCUM	&	ACQ/IMAGE	&	60	&	P2	&	Low	&	PSA	&	22.1	&	127.1	&	MIRA	&	2014-10-27	\\
lci4a2e5q	&	13616	&	206W3	&	ACCUM	&	ACQ/IMAGE	&	300	&	P2	&	Med	&	PSA	&	22.1	&	127.1	&	MIRB	&	2014-10-27	\\
lcgq01q5q	&	13526	&	WD-1657+343	&	ACCUM	&	ACQ/IMAGE	&	12	&	P2	&	Med	&	PSA	&	22.1	&	127.1	&	MIRB	&	2014-11-19	\\
lcgq01q7q	&	13526	&	WD-1657+343	&	TT	&	EXT/SCI	&	16	&	P2	&	Med	&	PSA	&	22.1	&	127.1	&	MIRB	&	2014-11-19	\\
lcgq01q9q	&	13526	&	WD-1657+343	&	TT	&	EXT/SCI	&	150	&	P2	&	Med	&	BOA	&	22.1	&	-153.1	&	MIRA	&	2014-11-19	\\
lcgq01qbq	&	13526	&	WAVE	&	TT	&	WAVECAL	&	7	&	P2	&	Low	&	WCA	&	22.1	&	126.1	&	MIRA	&	2014-11-19	\\
lcgq01qdq	&	13526	&	WD-1657+343	&	ACCUM	&	ACQ/IMAGE	&	150	&	P2	&	Low	&	BOA	&	22.1	&	-153.1	&	MIRA	&	2014-11-19	\\
lcgq01qfq	&	13526	&	WAVE	&	TT	&	WAVECAL	&	7	&	P2	&	Low	&	WCA	&	22.1	&	126.1	&	MIRA	&	2014-11-19	\\
lcgq01qhq	&	13526	&	WD-1657+343	&	TT	&	EXT/SCI	&	12	&	P2	&	Med	&	PSA	&	22.1	&	126.1	&	MIRB	&	2014-11-19	\\
lcgq01qjq	&	13526	&	WD-1657+343	&	ACCUM	&	ACQ/IMAGE	&	12	&	P2	&	Med	&	PSA	&	22.1	&	126.1	&	MIRB	&	2014-11-19	\\
lcgq02hmq	&	13526	&	HIP66578	&	ACCUM	&	ACQ/IMAGE	&	12	&	P2	&	Low	&	BOA	&	22.1	&	-153.1	&	MIRA	&	2014-11-17	\\
lcgq02hoq	&	13526	&	WAVE	&	TT	&	WAVECAL	&	7	&	P2	&	Low	&	WCA	&	22.1	&	126.1	&	MIRA	&	2014-11-17	\\
lcgq02hqq	&	13526	&	HIP66578	&	TT	&	EXT/SCI	&	181	&	P2	&	Low	&	BOA	&	22.1	&	-153.1	&	MIRB	&	2014-11-17	\\
lcgq02hsq	&	13526	&	WAVE	&	TT	&	WAVECAL	&	12	&	P2	&	Med	&	WCA	&	22.1	&	126.1	&	MIRB	&	2014-11-17	\\
lcgq02huq	&	13526	&	HIP66578	&	ACCUM	&	ACQ/IMAGE	&	181	&	P2	&	Med	&	BOA	&	22.1	&	-153.1	&	MIRB	&	2014-11-17	\\
lcgq02hwq	&	13526	&	WAVE	&	TT	&	WAVECAL	&	12	&	P2	&	Med	&	WCA	&	22.1	&	126.1	&	MIRB	&	2014-11-17	\\
lcgq02hyq	&	13526	&	WAVE	&	TT	&	WAVECAL	&	10	&	P2	&	Low	&	WCA	&	22.1	&	126.1	&	MIRA	&	2014-11-17	\\
lcgq02i0q	&	13526	&	HIP66578	&	ACCUM	&	ACQ/IMAGE	&	12	&	P2	&	Low	&	BOA	&	22.1	&	-153.1	&	MIRA	&	2014-11-17	\\
lcgq02icq	&	13526	&	WAVE	&	TT	&	WAVECAL	&	10	&	P1	&	Low	&	WCA	&	22.1	&	127.1	&	MIRA	&	2014-11-17	\\
lcgq02ieq	&	13526	&	WAVE	&	TT	&	WAVECAL	&	10	&	P2	&	Low	&	WCA	&	22.1	&	127.1	&	MIRA	&	2014-11-17	\\
lcgq02igq	&	13526	&	WAVE	&	TT	&	WAVECAL	&	30	&	P1	&	Low	&	WCA	&	22.1	&	127.1	&	MIRB	&	2014-11-17	\\
lcgq02iiq	&	13526	&	WAVE	&	TT	&	WAVECAL	&	20	&	P2	&	Med	&	WCA	&	22.1	&	127.1	&	MIRB	&	2014-11-17	\\
lcgq03dbq	&	13526	&	206W3	&	ACCUM	&	ACQ/IMAGE	&	15	&	P2	&	Low	&	PSA	&	22.1	&	127.1	&	MIRA	&	2014-10-06	\\
lcgq03ddq	&	13526	&	206W3	&	TT	&	EXT/SCI	&	15	&	P2	&	Low	&	PSA	&	22.1	&	127.1	&	MIRA	&	2014-10-06	\\
lcgq03dfq	&	13526	&	206W3	&	TT	&	EXT/SCI	&	160	&	P2	&	Low	&	PSA	&	22.1	&	127.1	&	MIRB	&	2014-10-06	\\
lcgq03dhq	&	13526	&	206W3	&	TT	&	EXT/SCI	&	180	&	P2	&	Low	&	PSA	&	22.1	&	127.1	&	MIRB	&	2014-10-06	\\
lcgq03djq	&	13526	&	206W3	&	TT	&	EXT/SCI	&	180	&	P2	&	Med	&	PSA	&	22.1	&	127.1	&	MIRB	&	2014-10-06	\\
lcgq03dlq	&	13526	&	206W3	&	ACCUM	&	ACQ/IMAGE	&	160	&	P2	&	Med	&	PSA	&	22.1	&	127.1	&	MIRB	&	2014-10-06	\\
lcgq03dnq	&	13526	&	206W3	&	TT	&	EXT/SCI	&	180	&	P2	&	Med	&	PSA	&	22.1	&	127.1	&	MIRB	&	2014-10-06	\\
lcgq03dpq	&	13526	&	206W3	&	TT	&	EXT/SCI	&	160	&	P2	&	Low	&	PSA	&	22.1	&	127.1	&	MIRB	&	2014-10-06	\\
lcgq03drq	&	13526	&	206W3	&	TT	&	EXT/SCI	&	12	&	P2	&	Low	&	PSA	&	22.1	&	127.1	&	MIRA	&	2014-10-06	\\
lcgq03dtq	&	13526	&	206W3	&	ACCUM	&	ACQ/IMAGE	&	12	&	P2	&	Low	&	PSA	&	22.1	&	127.1	&	MIRA	&	2014-10-06	\\
lcri01fzq	&	13972	&	WD-1657+343	&	ACCUM	&	ACQ/IMAGE	&	12	&	P2	&	Med	&	PSA	&	22.1	&	125.1	&	MIRB	&	2015-10-06	\\
lcri01g1q	&	13972	&	WD-1657+343	&	TT	&	EXT/SCI	&	12	&	P2	&	Med	&	PSA	&	22.1	&	125.1	&	MIRB	&	2015-10-06	\\
lcri01g3q	&	13972	&	WD-1657+343	&	TT	&	EXT/SCI	&	150	&	P2	&	Med	&	BOA	&	22.1	&	-153.1	&	MIRA	&	2015-10-06	\\
lcri01g5q	&	13972	&	WAVE	&	TT	&	WAVECAL	&	10	&	P2	&	Low	&	WCA	&	22.1	&	126.1	&	MIRA	&	2015-10-06	\\
lcri01g7q	&	13972	&	WD-1657+343	&	ACCUM	&	ACQ/IMAGE	&	150	&	P2	&	Low	&	BOA	&	22.1	&	-153.1	&	MIRA	&	2015-10-06	\\
lcri01g9q	&	13972	&	WAVE	&	TT	&	WAVECAL	&	10	&	P2	&	Low	&	WCA	&	22.1	&	126.1	&	MIRA	&	2015-10-06	\\
lcri01gcq	&	13972	&	WD-1657+343	&	TT	&	EXT/SCI	&	14	&	P2	&	Med	&	PSA	&	22.1	&	126.1	&	MIRB	&	2015-10-06	\\
lcri01geq	&	13972	&	WD-1657+343	&	ACCUM	&	ACQ/IMAGE	&	12	&	P2	&	Med	&	PSA	&	22.1	&	126.1	&	MIRB	&	2015-10-06	\\
lcri02h8q	&	13972	&	HIP66578	&	ACCUM	&	ACQ/IMAGE	&	12	&	P2	&	Low	&	BOA	&	22.1	&	-153.1	&	MIRA	&	2015-10-06	\\
lcri02haq	&	13972	&	WAVE	&	TT	&	WAVECAL	&	14	&	P2	&	Low	&	WCA	&	22.1	&	126.1	&	MIRA	&	2015-10-06	\\
lcri02hcq	&	13972	&	HIP66578	&	TT	&	EXT/SCI	&	181	&	P2	&	Low	&	BOA	&	22.1	&	-153.1	&	MIRB	&	2015-10-06	\\
lcri02heq	&	13972	&	WAVE	&	TT	&	WAVECAL	&	24	&	P2	&	Med	&	WCA	&	22.1	&	126.1	&	MIRB	&	2015-10-06	\\
lcri02hgq	&	13972	&	HIP66578	&	ACCUM	&	ACQ/IMAGE	&	181	&	P2	&	Med	&	BOA	&	22.1	&	-153.1	&	MIRB	&	2015-10-06	\\
lcri02hiq	&	13972	&	WAVE	&	TT	&	WAVECAL	&	24	&	P2	&	Med	&	WCA	&	22.1	&	126.1	&	MIRB	&	2015-10-06	\\
lcri02hkq	&	13972	&	WAVE	&	TT	&	WAVECAL	&	14	&	P2	&	Low	&	WCA	&	22.1	&	126.1	&	MIRA	&	2015-10-06	\\
lcri02hmq	&	13972	&	HIP66578	&	ACCUM	&	ACQ/IMAGE	&	12	&	P2	&	Low	&	BOA	&	22.1	&	-153.1	&	MIRA	&	2015-10-06	\\
lcri02hyq	&	13972	&	WAVE	&	TT	&	WAVECAL	&	14	&	P1	&	Low	&	WCA	&	22.1	&	125.1	&	MIRA	&	2015-10-06	\\
lcri02i0q	&	13972	&	WAVE	&	TT	&	WAVECAL	&	24	&	P2	&	Low	&	WCA	&	22.1	&	125.1	&	MIRA	&	2015-10-06	\\
lcri02i2q	&	13972	&	WAVE	&	TT	&	WAVECAL	&	30	&	P1	&	Low	&	WCA	&	22.1	&	125.1	&	MIRB	&	2015-10-06	\\
lcri02i4q	&	13972	&	WAVE	&	TT	&	WAVECAL	&	24	&	P2	&	Med	&	WCA	&	22.1	&	125.1	&	MIRB	&	2015-10-06	\\
lcsla1i4q	&	14035	&	427W3	&	ACCUM	&	ACQ/IMAGE	&	60	&	P2	&	Low	&	PSA	&	22.1	&	125.1	&	MIRA	&	2015-04-14	\\
lcsla1i6q	&	14035	&	427W3	&	ACCUM	&	ACQ/IMAGE	&	300	&	P2	&	Med	&	PSA	&	22.1	&	125.1	&	MIRB	&	2015-04-14	\\
lcsla2bhq	&	14035	&	206W3	&	ACCUM	&	ACQ/IMAGE	&	60	&	P2	&	Low	&	PSA	&	22.1	&	125.1	&	MIRA	&	2015-10-02	\\
lcsla2bjq	&	14035	&	206W3	&	ACCUM	&	ACQ/IMAGE	&	300	&	P2	&	Med	&	PSA	&	22.1	&	125.1	&	MIRB	&	2015-10-02	\\
\midrule
lcq	&	14452	&	427W3	&	ACCUM	&	ACQ/IMAGE	&	60	&	P2	&	Low	&	PSA	&	22.1	&	 	&	MIRA	&	2016	\\
lcq	&	14452	&	427W3	&	ACCUM	&	ACQ/IMAGE	&	300	&	P2	&	Med	&	PSA	&	22.1	&	 	&	MIRB	&	2016	\\
lcq	&	14452	&	206W3	&	ACCUM	&	ACQ/IMAGE	&	60	&	P2	&	Low	&	PSA	&	22.1	&	 	&	MIRA	&	2016	\\
lcq	&	14452	&	206W3	&	ACCUM	&	ACQ/IMAGE	&	300	&	P2	&	Med	&	PSA	&	22.1	&	 	&	MIRB	&	2016	\\
\midrule
ld3701gtq	&	14440	&	WD-1657+343	&	ACCUM	&	ACQ/IMAGE	&	13	&	P2	&	Med	&	PSA	&	22.1	&	125.1	&	MIRB	&	2016-10-18	\\
ld3701gvq	&	14440	&	WD-1657+343	&	TT	&	EXT/SCI	&	16	&	P2	&	Med	&	PSA	&	22.1	&	125.1	&	MIRB	&	2016-10-18	\\
ld3701gxq	&	14440	&	WD-1657+343	&	TT	&	EXT/SCI	&	150	&	P2	&	Med	&	BOA	&	22.1	&	-153.1	&	MIRA	&	2016-10-18	\\
ld3701gzq	&	14440	&	WAVE	&	TT	&	WAVECAL	&	9	&	P2	&	Low	&	WCA	&	22.1	&	126.1	&	MIRA	&	2016-10-18	\\
ld3701h1q	&	14440	&	WD-1657+343	&	ACCUM	&	ACQ/IMAGE	&	150	&	P2	&	Low	&	BOA	&	22.1	&	-153.1	&	MIRA	&	2016-10-18	\\
ld3701h3q	&	14440	&	WAVE	&	TT	&	WAVECAL	&	10	&	P2	&	Low	&	WCA	&	22.1	&	126.1	&	MIRA	&	2016-10-18	\\
ld3701h5q	&	14440	&	WD-1657+343	&	TT	&	EXT/SCI	&	16	&	P2	&	Med	&	PSA	&	22.1	&	126.1	&	MIRB	&	2016-10-18	\\
ld3701h7q	&	14440	&	WD-1657+343	&	ACCUM	&	ACQ/IMAGE	&	13	&	P2	&	Med	&	PSA	&	22.1	&	126.1	&	MIRB	&	2016-10-18	\\
ld3702mzq&	14440	&	HIP66578	&	ACCUM	&	ACQ/IMAGE	&	16	&	P2	&	Low	&	BOA	&	22.1	&	-153.1	&	MIRA	&	2016-10-19	\\
ld3702n1q	&	14440	&	WAVE	&	TT	&	WAVECAL	&	14	&	P2	&	Low	&	WCA	&	22.1	&	126.1	&	MIRA	&	2016-10-19	\\
ld3702n4q	&	14440	&	HIP66578	&	TT	&	EXT/SCI	&	183	&	P2	&	Low	&	BOA	&	22.1	&	-153.1	&	MIRB	&	2016-10-19	\\
ld3702n7q	&	14440	&	WAVE	&	TT	&	WAVECAL	&	24	&	P2	&	Med	&	WCA	&	22.1	&	126.1	&	MIRB	&	2016-10-19	\\
ld3702n9q	&	14440	&	HIP66578	&	ACCUM	&	ACQ/IMAGE	&	183	&	P2	&	Med	&	BOA	&	22.1	&	-153.1	&	MIRB	&	2016-10-19	\\
ld3702nbq	&	14440	&	WAVE	&	TT	&	WAVECAL	&	24	&	P2	&	Med	&	WCA	&	22.1	&	126.1	&	MIRB	&	2016-10-19	\\
ld3702neq	&	14440	&	WAVE	&	TT	&	WAVECAL	&	14	&	P2	&	Low	&	WCA	&	22.1	&	126.1	&	MIRA	&	2016-10-19	\\
ld3702nhq	&	14440	&	HIP66578	&	ACCUM	&	ACQ/IMAGE	&	16	&	P2	&	Low	&	BOA	&	22.1	&	-153.1	&	MIRA	&	2016-10-19	\\
ld3702o1q	&	14440	&	WAVE	&	TT	&	WAVECAL	&	14	&	P1	&	Low	&	WCA	&	22.1	&	125.1	&	MIRA	&	2016-10-19	\\
ld3702o3q	&	14440	&	WAVE	&	TT	&	WAVECAL	&	24	&	P2	&	Low	&	WCA	&	22.1	&	125.1	&	MIRA	&	2016-10-19	\\
ld3702o5q	&	14440	&	WAVE	&	TT	&	WAVECAL	&	30	&	P1	&	Low	&	WCA	&	22.1	&	125.1	&	MIRB	&	2016-10-19	\\
ld3702o7q	&	14440	&	WAVE	&	TT	&	WAVECAL	&	24	&	P2	&	Med	&	WCA	&	22.1	&	125.1	&	MIRB	&	2016-10-19	\\
ldozbadhq	&	14857	&	WD-1657+343	&	ACCUM	&	ACQ/IMAGE	&	13	&	P2	&	Med	&	PSA	&	22.1	&	125.1	&	MIRB	&	2017-09-04	\\
ldozbadjs	&	14857	&	WD-1657+343	&	TT	&	EXT/SCI	&	16	&	P2	&	Med	&	PSA	&	22.1	&	125.1	&	MIRB	&	2017-09-04	\\
ldozbadlq	&	14857	&	WD-1657+343	&	TT	&	EXT/SCI	&	150	&	P2	&	Med	&	BOA	&	22.1	&	-153.1	&	MIRA	&	2017-09-04	\\
ldozbadnq	&	14857	&	WAVE	&	TT	&	WAVECAL	&	9	&	P2	&	Low	&	WCA	&	22.1	&	126.1	&	MIRA	&	2017-09-04	\\
ldozbadpq	&	14857	&	WD-1657+343	&	ACCUM	&	ACQ/IMAGE	&	150	&	P2	&	Low	&	BOA	&	22.1	&	-153.1	&	MIRA	&	2017-09-04	\\
ldozbadrq	&	14857	&	WAVE	&	TT	&	WAVECAL	&	10	&	P2	&	Low	&	WCA	&	22.1	&	126.1	&	MIRA	&	2017-09-04	\\
ldozbadtq	&	14857	&	WD-1657+343	&	TT	&	EXT/SCI	&	16	&	P2	&	Med	&	PSA	&	22.1	&	126.1	&	MIRB	&	2017-09-04	\\
ldozbadvq	&	14857	&	WD-1657+343	&	ACCUM	&	ACQ/IMAGE	&	13	&	P2	&	Med	&	PSA	&	22.1	&	126.1	&	MIRB	&	2017-09-04	\\
ldozbbleq	&	14857	&	HIP66578	&	ACCUM	&	ACQ/IMAGE	&	16	&	P2	&	Low	&	BOA	&	22.1	&	-153.1	&	MIRA	&	2017-09-06	\\
ldozbblgq	&	14857	&	WAVE	&	TT	&	WAVECAL	&	14	&	P2	&	Low	&	WCA	&	22.1	&	126.1	&	MIRA	&	2017-09-06	\\
ldozbbliq	&	14857	&	HIP66578	&	TT	&	EXT/SCI	&	183	&	P2	&	Low	&	BOA	&	22.1	&	-153.1	&	MIRB	&	2017-09-06	\\
ldozbblkq	&	14857	&	WAVE	&	TT	&	WAVECAL	&	24	&	P2	&	Med	&	WCA	&	22.1	&	126.1	&	MIRB	&	2017-09-06	\\
ldozbblmq	&	14857	&	HIP66578	&	ACCUM	&	ACQ/IMAGE	&	183	&	P2	&	Med	&	BOA	&	22.1	&	-153.1	&	MIRB	&	2017-09-06	\\
ldozbbloq	&	14857	&	WAVE	&	TT	&	WAVECAL	&	24	&	P2	&	Med	&	WCA	&	22.1	&	126.1	&	MIRB	&	2017-09-06	\\
ldozbblqq	&	14857	&	WAVE	&	TT	&	WAVECAL	&	14	&	P2	&	Low	&	WCA	&	22.1	&	126.1	&	MIRA	&	2017-09-06	\\
ldozbblsq	&	14857	&	HIP66578	&	ACCUM	&	ACQ/IMAGE	&	16	&	P2	&	Low	&	BOA	&	22.1	&	-153.1	&	MIRA	&	2017-09-06	\\
ldozbbm4q&	14857	&	WAVE	&	TT	&	WAVECAL	&	16	&	P1	&	Low	&	WCA	&	22.1	&	125.1	&	MIRA	&	2017-09-06	\\
ldozbbm6q&	14857	&	WAVE	&	TT	&	WAVECAL	&	26	&	P2	&	Low	&	WCA	&	22.1	&	125.1	&	MIRA	&	2017-09-06	\\
ldozbbm8q&	14857	&	WAVE	&	TT	&	WAVECAL	&	32	&	P1	&	Low	&	WCA	&	22.1	&	125.1	&	MIRB	&	2017-09-06	\\
ldozbbmaq&	14857	&	WAVE	&	TT	&	WAVECAL	&	26	&	P2	&	Med	&	WCA	&	22.1	&	125.1	&	MIRB	&	2017-09-06	\\
ldozpbf5q	&	14857	&	206W3	&	ACCUM	&	ACQ/IMAGE	&	20	&	P2	&	Low	&	PSA	&	22.1	&	125.1	&	MIRA	&	2017-09-10	\\
ldozpbf7q	&	14857	&	206W3	&	TT	&	EXT/SCI	&	20	&	P2	&	Low	&	PSA	&	22.1	&	125.1	&	MIRA	&	2017-09-10	\\
ldozpbf9q	&	14857	&	206W3	&	TT	&	EXT/SCI	&	220	&	P2	&	Med	&	PSA	&	22.1	&	125.1	&	MIRB	&	2017-09-10	\\
ldozpbfbq	&	14857	&	206W3	&	ACCUM	&	ACQ/IMAGE	&	220	&	P2	&	Med	&	PSA	&	22.1	&	125.1	&	MIRB	&	2017-09-10	\\
ldozpbfdq	&	14857	&	206W3	&	TT	&	EXT/SCI	&	220	&	P2	&	Med	&	PSA	&	22.1	&	125.1	&	MIRB	&	2017-09-10	\\
ldozpbffq	&	14857	&	206W3	&	TT	&	EXT/SCI	&	20	&	P2	&	Low	&	PSA	&	22.1	&	125.1	&	MIRA	&	2017-09-10	\\
ldozpbfhq	&	14857	&	206W3	&	ACCUM	&	ACQ/IMAGE	&	20	&	P2	&	Low	&	PSA	&	22.1	&	125.1	&	MIRA	&	2017-09-10
\bottomrule
\enddata
%\end{center}
\tablenotetext{c}{For the P1 lamp, the three current settings are LOW (6mA), MED (10mA) and HIGH (18mA). For the P2 lamp, the current settings are LOW (3mA), MED (10mA) and HIGH (14mA).}
\tablenotetext{t}{TT = TIME-TAG.}
\tablecomments{Exposures listed as \textsc{EXPTYPE}=EXT/SCI contain coeval target and PtNe lamp (P1 or P2) images taken in time-tag (\textsc{OBSTYPE}=TT) mode.
Exposures listed as \textsc{EXPTYPE}=WAVECAL (target = WAVE) contain only TT PtNe lamp (WCA) images.  \tacq{IMAGE} exposures return before and after target images in \textsc{OBSTYPE}=ACCUM, but do not return  lamp images.}
\end{deluxetable}
%   740	const SHORT pcmech_ApMXDispPosition[TA_NUM_APERTURES][MIE_NUM_DETECTORS] =
%   741	{
%   742	   /*  FUV   NUV  */
%   743	   /*  ---   ---  */
%   744	      { 126,  126 }, /* PSA_LP1 */
%   745	      {-153, -153 }, /* BOA_LP1 */
%   746	      {-153, -153 }, /* FCA_LP1 */
%   747	      { 126,  126 }, /* WCA_LP1 */
%   748	      {  53,  126 }, /* PSA_LP2 */
%   749	      {-226, -153 }, /* BOA_LP2 */
%   750	      {-226, -153 }, /* FCA_LP2 */
%   751	      {  53,  126 }, /* WCA_LP2 */
%   752	      { 181,  126 }, /* PSA_LP3 */
%   753	      { -98, -153 }, /* BOA_LP3 */
%   754	      { -98, -153 }, /* FCA_LP3 */
%   755	      { 181,  126 }, /* WCA_LP3 */
%   756	      { 234,  126 }, /* PSA_LP4 */
%   757	      { -45, -153 }, /* BOA_LP4 */
%   758	      { -45, -153 }, /* FCA_LP4 */
%   759	      { 234,  126 }, /* WCA_LP4 */
%   760	      { 181,  126 }, /* PSA_LP5 */
%   761	      { -98, -153 }, /* BOA_LP5 */
%   762	      { -98, -153 }, /* FCA_LP5 */
%   763	      { 181,  126 }, /* WCA_LP5 */
%   764	      { 181,  126 }, /* PSA_LP6 */
%   765	      { -98, -153 }, /* BOA_LP6 */
%   766	      { -98, -153 }, /* FCA_LP6 */
%   767	      { 181,  126 }, /* WCA_LP6 */
%   768	      { 181,  126 }, /* PSA_LP7 */
%   769	      { -98, -153 }, /* BOA_LP7 */
%   770	      { -98, -153 }, /* FCA_LP7 */
%   771	      { 181,  126 }, /* WCA_LP7 */
%   772	      { 181,  126 }, /* PSA_LP8 */
%   773	      { -98, -153 }, /* BOA_LP8 */
%   774	      { -98, -153 }, /* FCA_LP8 */
%   775	      { 181,  126 }  /* WCA_LP8 */
%   776	};

\begin{deluxetable}{|r|r|r|r|r|r|r|r|r|r|r|}
\tabcolsep 2pt
\tabletypesize{\tiny}
\tablecolumns{11}
\tablewidth{0 pt}
\tablecaption{COS/NUV TA Spectroscopic Monitoring Exposures\label{table:NUVtamonspec}}
\tablehead{
\colhead{ROOTNAME}&\colhead{PROP}&\colhead{TARGNAME}&
\colhead{EXPTIME}&\colhead{LAMP}&\colhead{CEN}&
\colhead{LP}&\colhead{APER}&\colhead{APER}&\colhead{OPT}&\colhead{DATE}\\
\colhead{}&\colhead{ID}&\colhead{}&
\colhead{(s)}&\colhead{USED}&\colhead{WAVE}&
\colhead{}&\colhead{XPOS}&\colhead{YPOS}&\colhead{ELEM}&\colhead{OBS}\\

}
\startdata
lcgq01qlq	&	13526	&	WD-1657+343	&	20	&	P2	&	3000	&	1	&	22.1	&	126.1	&	G230L	&	2014-11-19	\\
lcgq01r6q	&	13526	&	WD-1657+343	&	151	&	P2	&	2850	&	1	&	22.1	&	126.1	&	G285M	&	2014-11-19	\\
lcgq02i2q	&	13526	&	HIP66578	&	40	&	P2	&	1890	&	1	&	22.1	&	126.1	&	G185M	&	2014-11-17	\\
lcgq02i4q	&	13526	&	HIP66578	&	52	&	P2	&	2306	&	1	&	22.1	&	126.1	&	G225M	&	2014-11-17	\\
lcri01ggq	&	13972	&	WD-1657+343	&	20	&	P2	&	3000	&	1	&	22.1	&	126.1	&	G230L	&	2015-10-06	\\
lcri01giq	&	13972	&	WD-1657+343	&	151	&	P2	&	2676	&	1	&	22.1	&	126.1	&	G285M	&	2015-10-06	\\
lcri02hoq	&	13972	&	HIP66578	&	52	&	P2	&	2306	&	1	&	22.1	&	126.1	&	G225M	&	2015-10-06	\\
lcri02hqq	&	13972	&	HIP66578	&	40	&	P2	&	1913	&	1	&	22.1	&	126.1	&	G185M	&	2015-10-06	\\
ld3701h9q	&	14440	&	WD-1657+343	&	21	&	P2	&	3000	&	1	&	22.1	&	126.1	&	G230L	&	2016-10-18	\\
ld3701hbq	&	14440	&	WD-1657+343	&	151	&	P2	&	2676	&	1	&	22.1	&	126.1	&	G285M	&	2016-10-18	\\
ld3702nmq	&	14440	&	HIP66578	&	53	&	P2	&	2306	&	1	&	22.1	&	126.1	&	G225M	&	2016-10-19	\\
ld3702noq	&	14440	&	HIP66578	&	40	&	P2	&	1913	&	1	&	22.1	&	126.1	&	G185M	&	2016-10-19	\\
ldozbadxq	&	14857	&	WD-1657+343	&	23	&	P2	&	3000	&	1	&	22.1	&	126.1	&	G230L	&	2017-09-04	\\
ldozbadzq	&	14857	&	WD-1657+343	&	151	&	P2	&	2676	&	1	&	22.1	&	126.1	&	G285M	&	2017-09-04	\\
ldozbbluq	&	14857	&	HIP66578	&	53	&	P2	&	2306	&	1	&	22.1	&	126.1	&	G225M	&	2017-09-06	\\
ldozbblwq	&	14857	&	HIP66578	&	40	&	P2	&	1913	&	1	&	22.1	&	126.1	&	G185M	&	2017-09-06	\\
\hline
\enddata
\tablenotetext{a}{All exposures were taken with the PSA at \texttt{FP-POS}=3.}
\end{deluxetable}

% $Id: FUVtamonfiles.tex,v 1.8 2018/04/17 18:38:43 penton Exp $
\begin{deluxetable}{ccrccccccrr}
\tabcolsep 4 pt
\tablewidth{5.7 in}
\tabletypesize{\scriptsize}
\tablecolumns{10}
\tablewidth{0pt}
\tablecaption{FUV TA Monitoring Exposures\label{tab:FUVtamon}}
\tablehead{
\colhead{\textit{PROPOSID}}&\colhead{\textit{ROOTNAME}}&\colhead{\textit{TARGNAME}}&
\colhead{\textit{EXPTIME}}&\colhead{\textit{OPT\_ELEM}}&\colhead{\cenwave{}}&
\colhead{LP}&\colhead{\textit{APER}}&\colhead{\textit{APERY}}&\colhead{\textit{DATE-OBS}}\\
\colhead{}&\colhead{}&\colhead{}&\colhead{(s)}&\colhead{}&\colhead{}&
\colhead{}&\colhead{\textit{XPOS}}&\colhead{\textit{YPOS}}&\colhead{}\\
\colhead{(1)}&\colhead{(2)} &
\colhead{(3)}&\colhead{(4)} &
\colhead{(5)}&\colhead{(6)} &
\colhead{(7)}&\colhead{(8)} &
\colhead{(9)}&\colhead{(10)}
}
\startdata
\toprule
13124	&	lc6601s7q	&	WD-1657+343	&	110	&	G130M	&1309	&	2	&	22.1	&	52.1	&	2013-10-24\\
13124	&	lc6601s9q	&	WD-1657+343	&	30	&	G140L	&1280	&	2	&	22.1	&	52.1	&	2013-10-24\\
13124	&	lc6602z3q	&	HIP66578	&	20	&	G160M	&1623	&	2	&	22.1	&	52.1	&	2013-11-01\\
13124	&	lc6602z9q\tablenotemark{a}	&	HIP66578	&	323	&	G160M	&	1623	&	2	&	22.1	&	-224.1	&	2013-11-01\\
13124	&	lc6602zbq\tablenotemark{b}	&	WAVE	&	12	&	G160M	&	1623	&	2	&	22.1	&	51.1	&	2013-11-01\\
13526	&lcgq01r8q	&	WD-1657+343	&	20	&	G130M	&	1309	&	2	&	22.1	&	52.1	&	2014-11-19	\\
13526	&lcgq01r8q	&	WD-1657+343	&	20	&	G130M	&	1309	&	2	&	22.1	&	52.1	&	2014-11-19	\\
13526	&lcgq01raq	&	WD-1657+343	&	7	&	G140L	&	1280	&	2	&	22.1	&	52.1	&	2014-11-19	\\
13526	&lcgq01raq	&	WD-1657+343	&	7	&	G140L	&	1280	&	2	&	22.1	&	52.1	&	2014-11-19	\\
13526	&lcgq02i6q	&	HIP66578	&	18	&	G160M	&	1600	&	2	&	22.1	&	52.1	&	2014-11-17	\\
13526	&lcgq02i8q	&	HIP66578	&	22	&	G160M	&	1600	&	2	&	22.1	&	52.1	&	2014-11-17	\\
13526	&lcgq02iaq	&	HIP66578	&	22	&	G160M	&	1600	&	2	&	22.1	&	52.1	&	2014-11-17	\\
\midrule
13972	&lcri01gkq	&	WD-1657+343	&	20	&	G130M	&	1309	&	3	&	22.1	&	182.1	&	2015-10-06	\\
13972	&lcri01gkq	&	WD-1657+343	&	20	&	G130M	&	1309	&	3	&	22.1	&	182.1	&	2015-10-06	\\
13972	&lcri01h6q	&	WD-1657+343	&	7	&	G140L	&	1280	&	3	&	22.1	&	182.1	&	2015-10-06	\\
13972	&lcri01h6q	&	WD-1657+343	&	7	&	G140L	&	1280	&	3	&	22.1	&	182.1	&	2015-10-06	\\
13972	&lcri02hsq	&	HIP66578	&	22	&	G160M	&	1600	&	3	&	22.1	&	182.1	&	2015-10-06	\\
13972	&lcri02huq	&	HIP66578	&	25	&	G160M	&	1600	&	3	&	22.1	&	182.1	&	2015-10-06	\\
13972	&lcri02hwq	&	HIP66578	&	25	&	G160M	&	1600	&	3	&	22.1	&	182.1	&	2015-10-06	\\
14440	&ld3701hdq	&	WD-1657+343	&	25	&	G130M	&	1309	&	3	&	22.1	&	182.1	&	2016-10-18	\\
14440	&ld3701hdq	&	WD-1657+343	&	25	&	G130M	&	1309	&	3	&	22.1	&	182.1	&	2016-10-18	\\
14440	&ld3701hfq	&	WD-1657+343	&	10	&	G140L	&	1280	&	3	&	22.1	&	182.1	&	2016-10-18	\\
14440	&ld3701hfq	&	WD-1657+343	&	10	&	G140L	&	1280	&	3	&	22.1	&	182.1	&	2016-10-18	\\
14440	&ld3702nqq	&	HIP66578	&	22	&	G160M	&	1600	&	3	&	22.1	&	182.1	&	2016-10-19	\\
14440	&ld3702nsq	&	HIP66578	&	25	&	G160M	&	1600	&	3	&	22.1	&	182.1	&	2016-10-19	\\
14440	&ld3702nuq	&	HIP66578	&	25	&	G160M	&	1600	&	3	&	22.1	&	182.1	&	2016-10-19	\\
14857	&ldozbae1q	&	WD-1657+343	&	25	&	G130M	&	1309	&	3	&	22.1	&	182.1	&	2017-09-04	\\
14857	&ldozbae1q	&	WD-1657+343	&	25	&	G130M	&	1309	&	3	&	22.1	&	182.1	&	2017-09-04	\\
14857	&ldozbae3q	&	WD-1657+343	&	10	&	G140L	&	1280	&	3	&	22.1	&	182.1	&	2017-09-04	\\
14857	&ldozbae3q	&	WD-1657+343	&	10	&	G140L	&	1280	&	3	&	22.1	&	182.1	&	2017-09-04	\\
14857	&ldozbblyq	&	HIP66578	&	22	&	G160M	&	1600	&	3	&	22.1	&	182.1	&	2017-09-06	\\
14857	&ldozbbm0q	&	HIP66578	&	27	&	G160M	&	1600	&	3	&	22.1	&	182.1	&	2017-09-06	\\
14857	&ldozbbm2q	&	HIP66578	&	27	&	G160M	&	1600	&	3	&	22.1	&	182.1	&	2017-09-06	\\
\bottomrule
\enddata
\tablenotetext{a}{For C20 only (\pid{13124}), a G160M BOA spectrum and WAVECAL were obtained to measure the WCA-to-BOA offset. The BOA was 2 steps off (0.105\arcsec) of its LP2 expected \textit{APERYPOS} position of -226 for this exposure.
This is similar to the $\pm1$ step offset often seen during \tacq{IMAGE}s.}
\tablenotetext{b}{This WAVECAL exposure was used to measure the WCA portion of the WCA-to-BOA offset for the proceeding BOA spectrum, and it off its nominal position of 52.1 by 1 \textit{APERYPOS} step.}
\tablecomments{All exposures taken at \textit{FP-POS=3}. All PSA spectra executed at the expected aperture positions (\textit{APERXPOS} \& \textit{APERYPOS}), while
the indicated BOA spectrum was off by 2 \textit{APERYPOS} steps.}
\end{deluxetable}


\clearpage
\section{Verifying the TA (\tacq{ }) Subarrays}\label{sec:subarray}

COS TA subarrays are loaded during the HST ground commanding uniquely for each TA exposure,
and are NUV stripe, FUV segment, \texttt{ACQ} mode, and CENWAVE dependent.
There are two stages to the TA verification, 1) ensuring that the intended subarrays were commanded and
2) that those subarrays are valid for the entire Cycle of usage.

Ideally, one would compare that commanded subarrays for all exposures to those
reported in the \textsc{\_rawacq.fits}. However, due to issues with the
COS TA subarrays\footnote{This issues should be addressed for Cycle~26 with the corrections outlined in PR\#XXXXX},
the subarrays were inferred from the telemetry reported in the \textsc{\_spt.fits} files.

Table~\ref{tab:NUVimsubs} gives the TA subarrays for all imaging modes,
Table~\ref{tab:NUVspecsubs} gives the TA subarrays for all NUV spectroscopic modes,
and Table~\ref{tab:FUVspecsubs} gives the TA all FUV CENWAVEs. Note that TA has not
been enabled for all CENWAVES, so only the TA subarrays that are in use are listed.
The FUV table includes subarrays for all four COS LPs even though only the LP2 and LP3
subarrys were used in this ISR.

All values indicate that the intended subarrays are being used for all TA and science exposures. All FUV spectra were visually
inspected to verify that the TA subarrays were successfully excluding all known detector hot-spots and the
bright Geocoronal emission lines that can negatively affect TAs.  No action is required based upon this
analysis of the TA and science sub-arrays used in HST Cycle 21.

COS TA subarrays are defined in detector coordinates, and are specified by giving the [X,Y] corners ([XC,YC]) and sizes ([XS,YS]).
Table~\ref{tab:NUVsubs} below gives the NUV spectroscopic TA subarrays used for \tacq{SEARCH} and \tacq{PEAKD}, which have not changed since SMOV.
Table~\ref{tab:NUVsubsXD} below gives the NUV spectroscopic TA subarrays used for \tacq{PEAKXD}, which include subarrays to measure the
calibration lamp XD location (Cal) as well as the target spectral location of the "B" (WAVELENGTH=MEDIUM) stripe.
These have not changed there updated in 2010 as STScI PR\#{}.

In this section, we describe the various subarrays used in COS TA.
These subarrays are defined by giving the detector coordinate of the lowest valued corner (C) and the full size (S) for both X and Y.
A subarray is fully specified by giving its XC, XS, YC, and YS. Unless noted, coordinates are in detector coordinates as this is the system in which COS TAs are performed.

\subsection{NUV Imaging TA subarrays}\label{subsec:NUVimSUBS}
The NUV imaging TA subarrays are given in Table~\ref{tab:TAnuvIMAGEsmov}.
These subarrays are used for both the \tacq{IMAGE} and \tacq{SEARCH} TA phases.
This table includes entries for the WCA and PSA and both
MIRRORA and MIRRORB. The COS FSW uses the same subarrays for the PSA
and BOA as the offset between the apertures is small ($\Delta$~[AD,XD]$\sim$[11.0,0.4]p).
As discussed in \S~\ref{sec:TAback}, the rising NUV detector background necessitates a reduction
in the TA subarray size for WCA+MIRRORB (\texttt{LTAIMCAL}).
The OSM positions, and hence the WCA \texttt{LTAIMCAL} MIRRORA and MIRROB image placements (see Figure~\ref{LTAIMCALpos}), are fairly repeatable and
it is recommended that both WCA TA subarrays be reduced by 50p in XD and 100p in AD as outlined in Table~\ref{tab:TAnuvIMAGEupdate}.

During \tacq{IMAGE}, the region of the detector used to determine the source location is small,
and is given by the square of the TA parameter \textsc{pcta\_CheckboxSize}, which is currently set to 9p (81 total pixels).
However, during \tacq{SEARCH}, the counts in the full subarray are used (currently $345 \times 816$=19,376p).
NUV \tacq{SEARCH} TAs are therefore much more vulnerable (by a factor of 3500)
to contamination from background events and SAA passages as described in \S~\ref{sec:TAback}.

\begin{deluxetable}{|l|l|r|r|r|r|}
\tabcolsep 14pt
\tabletypesize{\footnotesize}
\tablecolumns{6}
\tablewidth{0 pt}
\tablecaption{2009-20XX Imaging NUV \tacq{IMAGE} and \tacq{SEARCH} TA Subarrays\label{tab:TAnuvIMAGEsmov}.}
\tablehead{\colhead{Aperture}&\colhead{MIRROR}&\colhead{XC}&\colhead{YC}&\colhead{XS}&\colhead{YS}}
\startdata
WCA & MIRRORA & CHECK THESE  & 324 & 50 & 300\\
WCA & MIRRORB & 184 & 539 & 50 & 300\\
PSA/BOA & MIRRORA & 630 & 284 & 220 & 470\\
PSA/BOA & MIRRORB & 469 & 499 & 220 & 470
\enddata
\tablecomments{Updated on SMOV-SMS201 with PR\#63095 and PR\#67139.}
\end{deluxetable}

\begin{deluxetable}{|l|l|r|r|r|r|}
\tabcolsep 14pt
\tabletypesize{\footnotesize}
\tablecolumns{6}
\tablewidth{0 pt}
\tablecaption{Cycle~24 Imaging NUV \tacq{IMAGE} and \tacq{SEARCH} TA Subarrays\label{tab:TAimage09}.}
\tablehead{\colhead{Aperture}&\colhead{MIRROR}&\colhead{XC}&\colhead{YC}&\colhead{XS}&\colhead{YS}}
\startdata
WCA & MIRRORA & 345 & 324 & 50 & 300\\
WCA & MIRRORB & 184 & 539 & 50 & 300\\
PSA/BOA & MIRRORA & 630 & 284 & 220 & 470\\
PSA/BOA & MIRRORB & 469 & 499 & 220 & 470
\enddata
\tablecomments{Updated on SMOV-SMS201 with PR\#63095 and PR\#67139.}
\end{deluxetable}

Need to insert new MIRRORB TABLE here.

\begin{deluxetable}{|l|l|r|r|r|r|}
\tabcolsep 14pt
\tabletypesize{\footnotesize}
\tablecolumns{6}
\tablewidth{0 pt}
\tablecaption{2017 Imaging NUV \tacq{IMAGE} and \tacq{SEARCH} TA Subarrays\label{tab:TAnuvIMAGEpost}.}
\tablehead{\colhead{Aperture}&\colhead{MIRROR}&\colhead{XC}&\colhead{YC}&\colhead{XS}&\colhead{YS}}
\startdata
%WCA &MIRRORA&268&95&200&660\\
%WCA &MIRRORB&108&200&200&660\\
WCA &MIRRORA&268&95&200&660\\
WCA &MIRRORB&103&361&200&660\\
PSA/BOA &MIRRORA&572&108&345&816\\
PSA/BOA &MIRRORB&411&200&345&816\\
\enddata
\tablecomments{Updated on SMOV-SMS201 with PR\#63095.}
\end{deluxetable}

\begin{deluxetable}{|l|l|r|r|r|r|}
\tabcolsep 14pt
\tabletypesize{\footnotesize}
\tablecolumns{6}
\tablewidth{0 pt}
\tablecaption{Imaging NUV \tacq{SEARCH} TA Subarrays\label{tab:TAnuvIMAGEupdate}.}
\tablehead{\colhead{Aperture}&\colhead{MIRROR}&\colhead{XC}&\colhead{YC}&\colhead{XS}&\colhead{YS}}
\startdata
%WCA &MIRRORA&\textbf{293}&\textbf{195}&\textbf{150}&\textbf{560}\\
%WCA &MIRRORB&\textbf{133}&\textbf{300}&\textbf{150}&\textbf{560}\\
%PSA &MIRRORA&570&60&345&816\\
%PSA &MIRRORB&410&200&345&816\\
%\hline
%WCA &MIRRORA&\textbf{320} & \textbf{274} & \textbf{100} & \textbf{400}\\
%WCA &MIRRORB&\textbf{159} & \textbf{486} & \textbf{100} & \textbf{400}\\
%PSA/BOA &MIRRORA&\textbf{605} & \textbf{234} & \textbf{270} & \textbf{570}\\
%PSA/BOA &MIRRORB&\textbf{444} & \textbf{446} & \textbf{270} & \textbf{570}\\
WCA &MIRRORA&\textbf{345} & \textbf{324} & \textbf{50} & \textbf{300}\\
WCA &MIRRORB&\textbf{184} & \textbf{539} & \textbf{50} & \textbf{300}\\
PSA/BOA &MIRRORA&\textbf{630} & \textbf{284} & \textbf{220} & \textbf{470}\\
PSA/BOA &MIRRORB&\textbf{469} & \textbf{499} & \textbf{220} & \textbf{470}\\
\enddata
%Updated April 14, 2010
\tablecomments{To be updated in PR\#67139 (2011.017). New values are in \textbf{bold}.
These subarrays are the dashed ones displayed in Figure~\ref{LTAIMCALpos}.
Before this PR, the NUV \tacq{IMAGE} subarrays were identical to the NUV \tacq{SEARCH} subarrays.}
\end{deluxetable}

\section{COS Spectroscopic TA subarrays}\label{sec:taSUBS}
\subsection{COS NUV Spectroscopic TA Subarrays}\label{subsec:NUVspSUBS}
%\vspace{-0.3cm}
The NUV spectroscopic TA subarrays for the \tacq{SEARCH} and \tacq{PEAKD} phases are identical, and are given in Table~\ref{tab:NUVspSUBSsad}.
These subarrays are not grating-specific and are large enough to capture the flux from all three stripes (two for G230L, stripe C is not used).
The COS FSW uses the same subarrays for the PSA and BOA as the offset between the NUV spectra is small ($\Delta$~XD$\sim$5p).

The NUV spectroscopic TA subarrays for the \tacq{PEAKXD} are given in Table~\ref{tab:NUVspSUBSxd}.
These subarrays are large enough to only capture the flux from a single stripe.
Stripe-specific subarrays are defined for both the WCA and PSA.
The COS FSW uses the same subarrays for the PSA
and BOA as the offset between the apertures is small.
If used with an extended source, these subarrays are vulnerable to
cross-contamination of stripe light. In this table, only the values
of XC are listed, for all NUV \tacq{PEAKXD} YC=0, YS=1024, and XS=81.

\begin{deluxetable}{|r|r|r|r|r|}
\tabcolsep 10pt
\tabletypesize{\footnotesize}
\tablecolumns{5}
\tablewidth{0 pt}
\tablecaption{NUV \tacq{SEARCH} and \tacq{PEAKD} Spectroscopic TA subarrays \label{tab:NUVspSUBSsad}}
\tablehead{\colhead{GRating}&\colhead{XC}&\colhead{YC}&\colhead{XS}&\colhead{YS}\\
}
\startdata
\hline
G185M&509&0&420&1024\\
G225M&512&0&420&1024\\
G285M&499&0&420&1024\\
G230L&659&0&275&1024\\
\hline
\enddata
\tablecomments{These NUV TA subarrays were installed in HST commanding in PR\#{}.}
\end{deluxetable}

NEED TO DETERMINE WHICH OF THESE NUV PEAKXD tables is CORRECT !
\begin{deluxetable}{|l|r|r|r|r|r|r|}
\tablewidth{0pt}
\tabcolsep 8pt
\tabletypesize{\footnotesize}
\tablecolumns{7}
\tablecaption{XC Values for NUV \tacq{PEAKXD} TA Subarrays\label{tab:TAnuvPEAKXDxc}.}
\tablehead{\colhead{Grating}&\colhead{WCA-A}& \colhead{WCA-B} & \colhead{WCA-C} & \colhead{PSA-A} &\colhead{PSA-B}& \colhead{PSA-C}
}
\startdata
G185M&418&327&192&794&740&565\\
G225M&440&327&186&804&743&560\\
G285M&417&333&180&782&728&545\\
G230L&443&344&194&807&747&564\\
\enddata
\tablecomments{Updated after SMOV with PR \#63095. For all NUV \tacq{PEAKXD} TA subarrays: YC=0, YS=1024, and XS=81.}
\end{deluxetable}

\begin{deluxetable}{|r|r|r|r|r|r|r|}
\tabcolsep 10pt
\tabletypesize{\footnotesize}
\tablecolumns{7}
\tablewidth{0 pt}
\tablecaption{NUV \tacq{PEAKXD} WCA and PSA/BOA TA Subarrays \label{tab:NUVspSUBSxd}}
\tablehead{\colhead{OPT\_ELEM}&\colhead{CAL-A}&\colhead{CAL-B}&\colhead{CAL-C} &\colhead{SCI-A}&\colhead{SCI-B}&\colhead{SCI-C}
}
\startdata
\hline
G185M	&	418	&	327	&	192	&	794	&	\bf{700}	&	565	\\
G225M	&	430	&	327	&	186	&	804	&	703	&	560	\\
G285M	&	407	&	313	&	180	&	782	&	688	&	555	\\
G230L	&	433	&	334	&	194	&	807	&	707	&	564	\\
\hline
\enddata
\tablecomments{Updated after SMOV with PR \#63095. For all NUV \tacq{PEAKXD} TA subarrays: YC=0, YS=1024, and XS=81.}
\end{deluxetable}

\subsection{COS FUV Spectroscopic TA Subarrays}\label{subsec:FUVsupSUBS}
The FUV spectroscopic TA subarrays for the WCA are the same for \tacq{SEARCH},  \tacq{PEAKD}, and \tacq{PEAKXD}
and are given in Table~\ref{tab:TAsubWCAfuv} for both FUVA and FUVB.
Only one subarray is used for each FUV segment, these are labeled `A1' and `B1'.
As the data are taken in ``detector'' coordinates, all FUV TA subarrays values are valid only for the normal operating temperature range of COS. FUVB is not used in G140L TAs.

The FUV spectroscopic subarrays used for all exposures at LP1, LP2, and LP3 for FUVA are given in Table~\ref{tab:FUVsubA} and for FUVB in Table~\ref{tab:FUVsubB}.
There are two subarrays used for each FUV segment, these are labeled `A1', `A2', `B1', and `B1'.
The COS FSW uses the same subarrays for the PSA and BOA as the offset between the FUV spectra is small ($\Delta$~XD$\sim$3p).
As with the other HST spectrographs, FUV TA is susceptible to contamination from geocoronal light as outlined in
Table~\ref{tab:GEO}, particularly Ly$\alpha$ 1216\AA, {\rm O}\textsc{I} 1302\AA, and {\rm Si}{\sc II}1304\AA.
The FUV TA subarrays outlined in tables~\ref{tab:FUVsubA} and \ref{tab:FUVsubB} have been tailored to remove regions
of the target spectrum that may contain Geocoronal light.
The Geocoronal light fills the aperture and has a very different XD profile which could cause problems with FUV TAs.

%The FUV spectroscopic TA subarrays for LP2 are given for FUVA in Table~\ref{tab:FUVsubA2} and for FUVB in Table~\ref{tab:FUVsubB2}.
%The initial FUV spectroscopic TA subarrays for LP3 are given for FUVA in Table~\ref{tab:FUVsubA3} and for FUVB in Table~\ref{tab:FUVsubB3}.
In 201X, several ``hot-spots'' appeared during solar maximum.
On XX,YY/201X (PR\#XXXXX) the FUV LP3 subarrays were adjusted to avoid these hotspots.
Details are given in \S~\ref{sec:hotspots}, and the adjusted FUVB subarrays are given in Table~\ref{tab:FUVsubB3hs}.

\begin{deluxetable}{|r|rrrr|rrrr|}
\tablewidth{0pt}
\tabcolsep 8pt
\tabletypesize{\footnotesize}
\tablecolumns{9}
\tablecaption{FUV WCA Subarrays\label{tab:TAsubWCAfuv}.}
\tablehead{
\colhead{} &
\colhead{XS} & \colhead{YS} & \colhead{XC} & \colhead{YC} &
\colhead{XS} & \colhead{YS} & \colhead{XC} & \colhead{YC}\\

\colhead{Grating} &
\colhead{A1} & \colhead{A1} & \colhead{A1} & \colhead{A1} &
\colhead{B1} & \colhead{B1} & \colhead{B1} & \colhead{B1}
}
\startdata
\multicolumn{9}{|c|}{LP1}\\ \hline
G130M & 13799 & 44 & 1201 & 541\tablenotemark{a} & 13799 & 44 & 1501 & 585 \\
G160M & 13799 & 44 & 1201 & 535\tablenotemark{a} & 13799 & 44 & 1501 & 579\tablenotemark{a} \\ \hline
G140L & 13799 & 44 & 1201 & 547\tablenotemark{a} & NA  & NA & NA & NA    \\ \hline
\multicolumn{9}{|c|}{LP2}\\ \hline
\multicolumn{9}{|c|}{LP3}\\ \hline
\enddata
\tablenotetext{a}{Updated for SMOV--SMS201 with PR\#63095.}
\end{deluxetable}

\begin{deluxetable}{|l|r||rrrr|rrrr||rrrr|rrrr||}
\tablewidth{0pt}
\tabletypesize{\footnotesize}
\tabcolsep 10pt
\tablecolumns{10}
\tablecaption{PSA/BOA FUVA TA Subarrays\label{tab:FUVsubA}}
\tablehead{
\colhead{Grating} & \colhead{Cenwave} &
\colhead{XS} & \colhead{YS} & \colhead{XC} & \colhead{YC} &
\colhead{XS} & \colhead{YS} & \colhead{XC} & \colhead{YC} \\

\colhead{} & \colhead{(\AA)} &
\colhead{A1} & \colhead{A1} & \colhead{A1} & \colhead{A1} &
\colhead{A2} & \colhead{A2} & \colhead{A2} & \colhead{A2}
}
\startdata
\multicolumn{10}{c}{LP1}\\ \hline
G130M & 1291 & 6555\tablenotemark{b}  & 76 & 1201 & 437\tablenotemark{a} & 4078 & 76 & 8896\tablenotemark{b} & 437\tablenotemark{a} \\
G130M & 1300 & 7559\tablenotemark{b}  & 76 & 1201 & 437\tablenotemark{a} & 4078 & 76 & 9900\tablenotemark{b} & 437\tablenotemark{a} \\
G130M & 1309 & 8562\tablenotemark{b}  & 76 & 1201 & 437\tablenotemark{a} & 4097\tablenotemark{b} & 76 & 10903\tablenotemark{b} & 437\tablenotemark{a}\\
G130M & 1318 & 9465\tablenotemark{b}  & 76 & 1201 & 437\tablenotemark{a} & 3194\tablenotemark{b} & 76 & 11806\tablenotemark{b} & 437\tablenotemark{a} \\
G130M & 1327 & 10489\tablenotemark{b} & 76 & 1201 & 437\tablenotemark{a} & 2170\tablenotemark{b} & 76 & 12830\tablenotemark{b} & 437\tablenotemark{a} \\ \hline
G160M & ALL     & 13799 & 76 & 1201 & 432\tablenotemark{a,b} & \dots & \dots & \dots & \dots \\ \hline
G140L & 1105 & 10458 & 76 & 1201 & 445\tablenotemark{a,b} & 457 & 76 & 14543 & 445\tablenotemark{a,b} \\
G140L & 1230 & 12216 & 76 & 1201 & 445\tablenotemark{a,b} & \dots & \dots & \dots & \dots \\ \hline
\multicolumn{10}{c}{LP2}\\ \hline
G130M & 1291 & 6555\tablenotemark{b}  & 76 & 1201 & 437\tablenotemark{a} & 4078 & 76 & 8896\tablenotemark{b} & 437\tablenotemark{a} \\
G130M & 1300 & 7559\tablenotemark{b}  & 76 & 1201 & 437\tablenotemark{a} & 4078 & 76 & 9900\tablenotemark{b} & 437\tablenotemark{a} \\
G130M & 1309 & 8562\tablenotemark{b}  & 76 & 1201 & 437\tablenotemark{a} & 4097\tablenotemark{b} & 76 & 10903\tablenotemark{b} & 437\tablenotemark{a}\\
G130M & 1318 & 9465\tablenotemark{b}  & 76 & 1201 & 437\tablenotemark{a} & 3194\tablenotemark{b} & 76 & 11806\tablenotemark{b} & 437\tablenotemark{a} \\
G130M & 1327 & 10489\tablenotemark{b} & 76 & 1201 & 437\tablenotemark{a} & 2170\tablenotemark{b} & 76 & 12830\tablenotemark{b} & 437\tablenotemark{a} \\ \hline
G160M & ALL     & 13799 & 76 & 1201 & 432\tablenotemark{a,b} & \dots & \dots & \dots & \dots \\ \hline
G140L & 1105 & 10458 & 76 & 1201 & 445\tablenotemark{a,b} & 457 & 76 & 14543 & 445\tablenotemark{a,b} \\
G140L & 1230 & 12216 & 76 & 1201 & 445\tablenotemark{a,b} & \dots & \dots & \dots & \dots \\ \hline
\multicolumn{10}{|c|}{LP3}\\
\enddata
\tablenotetext{a}{Updated during SMOV (2009.201) with PR\#63095.}
\tablenotetext{b}{Updated for LP2 operations (201X.215) with PR\#63378.}
\tablenotetext{c}{Updated for LP3 operations on 20xx with PR\#63095.}
\end{deluxetable}

\begin{deluxetable}{|l|r||rrrr|rrrr||}
\tablewidth{0pt}
\tabcolsep 10pt
\tablecolumns{10}
\tabletypesize{\footnotesize}
\tablecaption{PSA/BOA FUVB TA Subarrays\label{tab:FUVsubB}.}
\tablehead{
\colhead{Grating} & \colhead{Cenwave} &
\colhead{XS} & \colhead{YS} & \colhead{XC} & \colhead{YC} &
\colhead{XS} & \colhead{YS} & \colhead{XC} & \colhead{YC}\\

\colhead{} & \colhead{(\AA)} &
\colhead{B1} & \colhead{B1} & \colhead{B1} & \colhead{B1} &
\colhead{B2} & \colhead{B2} & \colhead{B2} & \colhead{B2}
}
\startdata
\multicolumn{9}{|c|}{LP1}\\
\hline
G130M & 1291 & 5036\tablenotemark{b} & 76 & 1501 & 483 & 7477\tablenotemark{b} & 76 & 7773\tablenotemark{b} & 483\tablenotemark{a,b} \\
G130M & 1300 & 6039\tablenotemark{b} & 76 & 1501 & 483 & 6474\tablenotemark{b} & 76 & 8776\tablenotemark{b} & 483\tablenotemark{a,b} \\
G130M & 1309 & 7023\tablenotemark{b} & 76 & 1501 & 483 & 5490\tablenotemark{a} & 76 & 9760\tablenotemark{a} & 483\tablenotemark{a,b} \\
G130M & 1318 & 7977\tablenotemark{b} & 76 & 1501 & 483 & 4536\tablenotemark{b} & 76 & 10714\tablenotemark{b} & 483\tablenotemark{a,b} \\
G130M & 1327 & 7629\tablenotemark{b} & 76 & 2792\tablenotemark{b} & 483 & 3593\tablenotemark{b} & 76 & 11657\tablenotemark{b} & 483\tablenotemark{a,b} \\ \hline
G160M & ALL  & 13749 & 76 & 1501 & 477\tablenotemark{a,b} & \dots & \dots & \dots & \dots \\ \hline
G140L & 1105 & \dots & \dots & \dots & \dots & \dots & \dots & \dots & \dots \\
G140L & 1230 & \dots & \dots & \dots & \dots & \dots & \dots & \dots & \dots \\ \hline
\hline
\multicolumn{9}{|c|}{LP2}\\
\hline
G130M & 1291 & 5036\tablenotemark{b} & 76 & 1501 & 483 & 7477\tablenotemark{b} & 76 & 7773\tablenotemark{b} & 483\tablenotemark{a,b} \\
G130M & 1300 & 6039\tablenotemark{b} & 76 & 1501 & 483 & 6474\tablenotemark{b} & 76 & 8776\tablenotemark{b} & 483\tablenotemark{a,b} \\
G130M & 1309 & 7023\tablenotemark{b} & 76 & 1501 & 483 & 5490\tablenotemark{a} & 76 & 9760\tablenotemark{a} & 483\tablenotemark{a,b} \\
G130M & 1318 & 7977\tablenotemark{b} & 76 & 1501 & 483 & 4536\tablenotemark{b} & 76 & 10714\tablenotemark{b} & 483\tablenotemark{a,b} \\
G130M & 1327 & 7629\tablenotemark{b} & 76 & 2792\tablenotemark{b} & 483 & 3593\tablenotemark{b} & 76 & 11657\tablenotemark{b} & 483\tablenotemark{a,b} \\ \hline
G160M & ALL  & 13749 & 76 & 1501 & 477\tablenotemark{a,b} & \dots & \dots & \dots & \dots \\ \hline
G140L & 1105 & \dots & \dots & \dots & \dots & \dots & \dots & \dots & \dots \\
G140L & 1230 & \dots & \dots & \dots & \dots & \dots & \dots & \dots & \dots \\ \hline
\hline
\multicolumn{9}{|c|}{LP3 (Pre Hot-Spot)}\\
\hline
G130M & 1291 & 5036 & 76 & 1501 & 460 & 5036 & 76 & 7773 & 460 \\
G130M & 1300 & 6039 & 76 & 1501 & 460 & 6039 & 76 & 8776 & 460 \\
G130M & 1309 & 7023 & 76 & 1501 & 460 & 7023 & 76 & 9760 & 460 \\
G130M & 1318 & 7977 & 76 & 1501 & 460 & 7977 & 76 & 10714 & 460 \\
G130M & 1327 & 7629 & 76 & 2792 & 460 & 7629 & 76 & 11657 & 460 \\ \hline
G160M & ALL  & 13749 & 76 & 1501 & 453 & \dots & \dots & \dots & \dots \\ \hline
G140L & 1105 & \dots & \dots & \dots & \dots & \dots & \dots & \dots & \dots \\
G140L & 1230 & \dots & \dots & \dots & \dots & \dots & \dots & \dots & \dots \\ \hline
\hline
\multicolumn{9}{|c|}{LP3 (Post Hot-Spot) -- CHECK THESE}\\
\hline
G130M & 1291 & 5036\tablenotemark{d} & 76 & 1501 & 483 & 7477\tablenotemark{d} & 76 & 7773\tablenotemark{d} & 483 \\
G130M & 1300 & 6039\tablenotemark{d} & 76 & 1501 & 483 & 6474\tablenotemark{d} & 76 & 8776\tablenotemark{d} & 483 \\
G130M & 1309 & 7023\tablenotemark{d} & 76 & 1501 & 483 & 5490\tablenotemark{d} & 76 & 9760\tablenotemark{d} & 483 \\
G130M & 1318 & 7977\tablenotemark{d} & 76 & 1501 & 483 & 4536\tablenotemark{d} & 76 & 10714\tablenotemark{d} & 483 \\
G130M & 1327 & 7629\tablenotemark{d} & 76 & 2792\tablenotemark{b} & 483 & 3593\tablenotemark{d} & 76 & 11657\tablenotemark{d} & 483\tablenotemark{a,b} \\ \hline
G160M & ALL  & 13332 & 76 & 1501 & 477\tablenotemark{a,b} & \dots & \dots & \dots & \dots \\ \hline
G140L & 1105 & \dots & \dots & \dots & \dots & \dots & \dots & \dots & \dots \\
G140L & 1230 & \dots & \dots & \dots & \dots & \dots & \dots & \dots & \dots \\ \hline
\enddata
\tablenotetext{a}{Updated during SMOV (2009.201) with PR\#63095.}
\tablenotetext{b}{Updated for LP2 operations (201X.215) with PR\#63378.}
\tablenotetext{c}{Updated for LP3 operations on 20xx.YYY with PR\#63095.}
\tablenotetext{d}{Updated for continuing LP3 operations on 20xx.YYY with PR\#80571.}
\end{deluxetable}

\subsection{Trimming of FUV-B TA subarrays due to FUVB ``Hot-Spot''.}\label{subsec:hotspot}
A ``Hot Spot'' appeared on the COS FUV-B segment coincident with the solar maximum of 20XX.
This spot produced enough counts that it could cause mis-centering during all phases of the FUV LP3 spectroscopic TAs.
This mis-centerings could be in significant in either the AD or XD. On 201X\footnote{STScI PR\#{}.}, all affected
LP3 FUV subarrays were changed.

In raw FUV-B detector coordinates, the approximate location of the `Hot Spot' is at [X,Y]=[14895,482].
As this is near the detector edge, we are able to avoid this hotspot by stopping the last subarray of the FUVB subarrays at X=14833.
For the COS FUV gratings and the FUVB TA subarrays, the impacts were
\begin{enumerate}
	\item[G140L:]{NOT affected as no FUVB TA subarrays are used for G140L }
	\item[G160M:]{One subarray is used, all have XC1=1501, XS1=13749. These will all change to XS1=13332. (no change in Y)}
	\item[G130M:]{Two subarrays are used to avoid Geocoronal Ly$\alpha$. The X-size of the second subarray (XS2) will be trimmed to avoid the hotspot (XC1, XS1, XC2 and all the Y definitions do not change).}
\end{enumerate}

%\section{COS Spectroscopic TA subarrays}\label{sec:specSUBS}
\subsection{COS NUV Spectroscopic TA Subarrays}\label{subsec:NUVspSUBS}
%\vspace{-0.3cm}
The NUV spectroscopic TA subarrays for the \tacq{SEARCH} and \tacq{PEAKD} phases are identical, and are given in Table~\ref{tab:NUVspSUBSsad}.
These subarrays are not grating-specific and are large enough to capture the flux from all three stripes (two for G230L; \textit{STRIPE=C (LONG)} is not used for G230L TA).
COS uses the same NUV TA subarrays for the PSA and BOA as the XD offset between the NUV spectra is small ($\Delta$XD$\sim$5p).
\begin{table}
\centering
	\begin{threeparttable}[tbc]
		\caption[NUV Spectroscopic \tacq{SEARCH} and \tacq{PEAKD} SA Subarrays\tnote{1}]{NUV Spectroscopic \tacq{SEARCH} and \tacq{PEAKD} SA Subarrays}
		\begin{tabular*}{.7\linewidth}{@{\extracolsep{\fill}}|c|rrrr|}
			\hline
			\textit{OPT\_ELEM}& XC & YC & XS & YS \\
			\hline
			G185M&509&0&420&1024\\
			G225M&512&0&420&1024\\
			G285M&499&0&420&1024\\
			G230L&659&0&275&1024\\
			\hline
		\end{tabular*}
		\footnotesize
		\begin{tablenotes}[para]
			\item [1] These are the NUV \tacq{SEARCH} and \tacq{PEAKD} external target (SA) subarrays.
			 The NUV \tacq{PEAKXD} lamp and SA subarrays are given in \ref{tab:NUVspSUBSxd}.
			\item [2] Installed by HST commanding on 2009.201 (PR\#63095).
		   \end{tablenotes}
		\label{tab:NUVspSUBSsad}
		\normalsize
	\end{threeparttable}
\end{table}
%\begin{center}
%\begin{deluxetable}{rrrrr}
%\tabcolsep 10 pt
%\tabletypesize{\footnotesize}
%\tablecolumns{5}
%\tablecaption{NUV \tacq{SEARCH} and \tacq{PEAKD} Spectroscopic SA Subarrays \label{tab:NUVspSUBSsad2}}
%\tablehead{
%\colhead{\textit{OPT\_ELEM}}&\colhead{XC}&\colhead{YC}&\colhead{XS}&\colhead{YS}
%}
%\startdata
%\hline
%G185M&509&0&420&1024\\
%G225M&512&0&420&1024\\
%G285M&499&0&420&1024\\
%G230L&659&0&275&1024\\
%\hline
%\enddata
%\tablecomments{These subarrays are used for NUV \tacq{SEARCH} and \tacq{PEAKD} TA only, and were installed in HST commanding on 2009.201 (PR\#63095).}
%\end{deluxetable}
%\end{center}

The NUV spectroscopic TA SA subarrays for the \tacq{PEAKXD} are given in Table~\ref{tab:NUVspSUBSxd}.
These subarrays are large enough to only capture the flux from a single NUV stripe.
Stripe-specific subarrays are defined for both the WCA and PSA.
If used with an extended source, these subarrays are vulnerable to cross-contamination of stripe light. In this table, only the values of XC are listed.
For all NUV \tacq{PEAKXD}s, YC=0, YS=1024, and XS=81.

\begin{table}
\centering
	\begin{threeparttable}[tbc]
		\caption{NUV \tacq{PEAKXD} WCA and PSA/BOA Subarray ``XC''s\tnote{1}}
			\begin{tabular*}{.75\linewidth}{@{\extracolsep{\fill}}ccccccc}
			\\
			\hline
			\textit{OPT\_ELEM}&WCA-A & WCA-B &WCA-C &SCI-A&SCI-B&SCI-C\\
			\hline
			G185M	&	418	&	327	&	192	&	794	&	700	&	565	\\
			G225M	&	430	&	327	&	186	&	804	&	703	&	560	\\
			G285M	&	407	&	313	&	180	&	782	&	688	&	555	\\
			G230L	&	433	&	334	&	194	&	807	&	707	&	564 \\
			\hline
		\end{tabular*}
		\footnotesize
			\begin{tablenotes}
				\item[1] XC = X-Corner. For all NUV \tacq{PEAKXD} TA subarrays: YC=0, YS=1024, and XS=81; where S=Size. Updated on July 19, 2009 (2009.200) with STScI PR\#63095. Some early calibration observations used slightly different values.
			\end{tablenotes}
			\label{tab:NUVspSUBSxd}
		\normalsize
	\end{threeparttable}
\end{table}
%\begin{center}
%\begin{deluxetable}{rrrrrrr}
%\tabcolsep 10 pt
%\tabletypesize{\footnotesize}
%\tablecolumns{7}
%\tablewidth{5.5 in}
%\tablecaption{NUV \tacq{PEAKXD} WCA and PSA/BOA Subarray ``XC''s\tablenotemark{a} \label{tab:NUVspSUBSxd}}
%\tablehead{
%\colhead{\textit{OPT\_ELEM}}&\colhead{WCA-A}&\colhead{WCA-B}&\colhead{WCA-C} &\colhead{SCI-A}&\colhead{SCI-B}&\colhead{SCI-C}
%}
%\startdata
%\hline
%G185M	&	418	&	327	&	192	&	794	&	700	&	565	\\
%G225M	&	430	&	327	&	186	&	804	&	703	&	560	\\
%G285M	&	407	&	313	&	180	&	782	&	688	&	555	\\
%G230L	&	433	&	334	&	194	&	807	&	707	&	564 \\
%\hline
%\enddata
%\tablenotetext{a}{Updated on July 19, 2009 (2009.200) with STScI PR\#63095. Some early calibration observations used slightly different values.}
%\tablecomments{These are the `XC' (X-Corner) values. For all NUV \tacq{PEAKXD} TA subarrays: YC=0, YS=1024, and XS=81; where S=Size.}
%\end{deluxetable}
%\end{center}

\subsection{COS FUV Spectroscopic TA Subarrays}\label{subsec:FUVspSUBS}
The FUV spectroscopic TA subarrays for the WCA are currently the same for \tacq{SEARCH},  \tacq{PEAKD}, and \tacq{PEAKXD}
and are given in Table~\ref{tab:TAsubWCAfuv} for both FUVA and FUVB.
Only one subarray is used for the WCA for each FUV segment, these are labeled `A1' and `B1'.
As the data are taken in ``detector'' coordinates, all FUV TA subarrays values are valid only for the normal operating temperature range of COS.
FUVB is not used in G140L TAs. Up to four TA subarrays are possible for each FUV \texttt{SEGMENT}\footnote{Increased from 2 per \texttt{SEGMENT} on 2015.258 with \pr{81263}}

The FUV spectroscopic subarrays used for all external targets at LP1--4 for FUVA are given in Table~\ref{tab:FUVsubA} and for FUVB in Table~\ref{tab:FUVsubB}.
There are two subarrays used for each FUV segment, these are labeled `A1', `A2', `B1', and `B2'.
The COS FSW uses the same subarrays for the PSA and BOA as the offset between the FUV spectra is small ($\Delta$~XD$\sim$3p).
As with the other HST spectrographs, FUV TAs are susceptible to contamination from geocoronal light, particularly Ly$\alpha$ 1216\AA, {\rm O}\textsc{I} 1302\AA, and {\rm Si}{\sc II}1304\AA\ (Penton \& Keyes, 2010).
The FUV TA subarrays outlined in Tables~\ref{tab:FUVsubA} and \ref{tab:FUVsubB} have been tailored to remove regions
of the target spectrum that may contain Geocoronal light. The Geocoronal light fills the aperture and has a very different XD profile which could cause problems with FUV TAs.

In 2014--5, several ``hot-spots'' appeared during solar maximum. The FUV LP3 subarrays were adjusted to avoid these hot-spots.
Details are given in \S~\ref{subsec:hotspots}, and the adjusted FUVB subarrays are also given in Table~\ref{tab:FUVsubB}.

\begin{deluxetable}{r|rrrr|rrrr}
\tablewidth{0pt}
\tabcolsep 11 pt
\tabletypesize{\scriptsize}
\tablecolumns{9}
\tablecaption{FUV WCA Subarrays for LP1--4\label{tab:TAsubWCAfuv}.}
\tablehead{
\colhead{} & \multicolumn{4}{c}{A1 Subarray} & \multicolumn{4}{c}{B1 Subarray} \\
\colhead{\textit{OPT\_ELEM}} & \colhead{XC} & \colhead{YC} & \colhead{XS} & \colhead{YS} & \colhead{XC} & \colhead{YC} & \colhead{XS} & \colhead{YS}\\
\colhead{(1)}&\colhead{(2)} & \colhead{(3)}&\colhead{(4)} &
\colhead{(5)}&\colhead{(6)} & \colhead{(7)}&\colhead{(8)} & \colhead{(9)}
}
\startdata
	\toprule
	\multicolumn{9}{c}{LP1}\\
	\midrule
	G130M & 1201 & 541\tablenotemark{a} & 13799                   & 44 & 1501  & 585                 & 13799 & 44\\
	G160M & 1201 & 535\tablenotemark{a} & 13799                   & 44 & 1501  & 579\tablenotemark{a}& 13799 & 44\\
	G140L & 1201 & 547\tablenotemark{a} & 13799                   & 44 &\dots&\dots&\dots&\dots\\
	G140L & 4701 & 547\tablenotemark{b} & 10299\tablenotemark{b}  & 44 &\dots&\dots&\dots&\dots\\
	\midrule
	\multicolumn{9}{c}{LP2\tablenotemark{c}}\\
	\midrule
	G130M & 1201 & 581 & 13799  & 44 & 1501 & 630 & 13799  & 44 \\
	G160M & 1201 & 568 & 13799  & 44 & 1501 & 617 & 13799  & 44 \\
	G140L & 4701 & 587 & 10299  & 44 &\dots&\dots&\dots&\dots\\
	\midrule
	\multicolumn{9}{c}{LP3\tablenotemark{d}}\\
	\midrule
	G130M & 1201  & 515 & 13799 & 44 & 1501  & 567 & 13799 & 44\\
	G160M & 1201  & 504 & 13799 & 44 & 1501  & 559 & 13799 & 44\\
	G140L & 4701  & 521 & 10299 & 44 &\dots&\dots&\dots&\dots\\
	\midrule
	\multicolumn{9}{c}{LP4\tablenotemark{e}}\\
	\midrule
	G130M & 1201 & 483 & 13799 & 52 & 1501 &  539 & 13799 & 52 \\
	G160M & 1201 & 475 & 13799 & 52 & 1501 &  534 & 13799 & 52 \\
	G140L & 4701 & 491 & 10299 & 52 &\dots&\dots&\dots&\dots\\
\enddata
\tablenotetext{a}{These values were updated on 2009.200 (July 19, 2009) with \pr{63095}, some very early COS calibration and ERO datasets used slightly different TA subarrays.}
\tablenotetext{b}{G140L updates were made on Dec. 4, 2012 (2012.339) with \pr{72193} to futher optimize the G140L subarrays.}
\tablenotetext{c}{Updated for LP2 operations on July 18, 2012 (2012.200) with \pr{70548}.}
\tablenotetext{d}{Updated for LP3 operations on Aug. 26, 2014 (2014.238) with \pr{78747}.}
\tablenotetext{e}{Updated for LP4 operations on Feb. 20, 2017 (2017.051) with \pr{86945}.}
\end{deluxetable}

\begin{deluxetable}{lc|rrrr|rrrr}
\tablewidth{0pt}
\tabletypesize{\scriptsize}
\tabcolsep 8 pt
\tablecolumns{10}
\tablecaption{FUVA SA Subarrays for LP1--4\label{tab:FUVsubA}}
\tablehead{
\colhead{\textit{OPT\_ELEM}} & \colhead{\cenwave{}} & \multicolumn{4}{c}{A1 Subarray} & \multicolumn{4}{c}{A2 Subarray}\\
\colhead{} & \colhead{(\AA)} &
\colhead{XC} & \colhead{YC} & \colhead{XS} & \colhead{YS} & \colhead{XC} & \colhead{YC} &\colhead{XS} & \colhead{YS}\\
\colhead{(1)}&\colhead{(2)} & \colhead{(3)}&\colhead{(4)} &
\colhead{(5)}&\colhead{(6)} & \colhead{(7)}&\colhead{(8)} &
\colhead{(9)}&\colhead{(10)}
}
\startdata
\toprule
\multicolumn{10}{c}{LP1}\\
\midrule
G130M & 1291 & 1201& 6555\tablenotemark{b}   & 437\tablenotemark{a}& 76  & 4078 & 8896\tablenotemark{b} & 437\tablenotemark{a} & 76\\
G130M & 1300 & 1201& 7559\tablenotemark{b}   & 437\tablenotemark{a}& 76  & 4078 & 9900\tablenotemark{b} & 437\tablenotemark{a} & 76\\
G130M & 1309 & 1201& 8562\tablenotemark{b}   & 437\tablenotemark{a}& 76  & 4097\tablenotemark{b}  & 10903\tablenotemark{b} & 437\tablenotemark{a} & 76 \\
G130M & 1318 & 1201& 9465\tablenotemark{b}   & 437\tablenotemark{a}& 76  & 3194\tablenotemark{b}  & 11806\tablenotemark{b} & 437\tablenotemark{a} & 76 \\
G130M & 1327 & 1201& 10489\tablenotemark{b}  & 437\tablenotemark{a}& 76  & 2170\tablenotemark{b}  & 12830\tablenotemark{b} & 437\tablenotemark{a} & 76 \\
G160M & ALL  & 1201& 13799 & 432\tablenotemark{a,b} & 76 &\dots&\dots&\dots&\dots\\
G140L & 1105 & 1201& 10458\tablenotemark{c} & 445\tablenotemark{a,b}& 76  & 457  & 14543 & 445\tablenotemark{a,b} & 76\\
G140L & 1230\tablenotemark{g} & 1201& 12216\tablenotemark{c} & 445\tablenotemark{a,b}& 76  &\dots&\dots&\dots&\dots\\
G140L & 1280 & 1201& 12216\tablenotemark{c} & 445\tablenotemark{a,b}& 76  &\dots&\dots&\dots&\dots\\
\midrule
G140L & 1105 & 4701\tablenotemark{c}& 6958\tablenotemark{c} & 445\tablenotemark{a,b} & 76  & 457 & 14543 & 445\tablenotemark{a,b} & 76\\
G140L & 1230\tablenotemark{g} & 4701\tablenotemark{c}&8716\tablenotemark{c}& 445\tablenotemark{a,b}   & 76  &\dots&\dots&\dots&\dots\\
G140L & 1280 & 6201\tablenotemark{c}&7400\tablenotemark{c} & 445\tablenotemark{a,b}  & 76  &\dots&\dots&\dots&\dots\\
\midrule
\multicolumn{10}{c}{LP2\tablenotemark{d}}\\
\midrule
G130M & 1291 & 1201 & 472 & 6555 & 76 & 8896 & 472 & 4078 & 76\\
G130M & 1300 & 1201 & 472 & 7559 & 76 & 9900 & 472 & 4078 & 76\\
G130M & 1309 & 1201 & 472 & 8562 & 76 & 10903 & 472 & 4097 & 76\\
G130M & 1318 & 1201 & 472 & 9465 & 76 & 11806 & 472 & 3194 & 76\\
G130M & 1327 & 1201 & 472 & 10489 & 76 & 12830 & 472 & 2170 & 76\\
G160M & ALL & 1201 & 466 & 13799 & 76 &\dots&\dots&\dots& \dots\\
G140L & 1105 & 4701 & 479 & 6958 & 76 & 14543 & 479 & 457 & 76\\
G140L & 1280 & 6201 & 479 & 7400 & 76 &\dots& 34 &\dots&\dots\\
G140L & 1105 & 4701 & 479 & 6958 & 76 & 14543 & 479 & 457 & 76\\
G140L & 1280 & 6201 & 479 & 7400 & 76 &\dots&\dots&\dots&\dots\\
\midrule
\multicolumn{10}{c}{LP3\tablenotemark{e}}\\
\midrule
G130M & 1291 & 1201 & 409 & 6555 & 76 & 8896 & 409 & 4078 & 76\\
G130M & 1300 & 1201 & 409 & 7559 & 76 & 9900 & 409 & 4078 & 76\\
G130M & 1309 & 1201 & 409 & 8562 & 76 & 10903 & 409 & 4097 & 76\\
G130M & 1318 & 1201 & 409 & 9465 & 76 & 11806 & 409 & 3194 & 76\\
G130M & 1327 & 1201 & 409 & 10489 & 76 & 12830 & 409 & 2170 & 76\\
G160M & ALL & 1201 & 403 & 13799 & 76 &\dots&\dots&\dots& \dots\\
G140L & 1105 & 4701 & 418 & 6958 & 76 & 14543 & 418 & 457 & 76\\
G140L & 1280 & 6201 & 418 & 7400 & 76 &\dots&\dots&\dots& \dots\\
\midrule
\multicolumn{10}{c}{LP4\tablenotemark{f}}\\
\midrule
G130M & 1291 & 1201 & 362 & 6555 & 112 & 8896 & 362 & 4078 & 112\\
G130M & 1300 & 1201 & 362 & 7559 & 112 & 9900 & 362 & 4078 & 112\\
G130M & 1309 & 1201 & 362 & 8562 & 112 & 10903 & 362 & 4097 & 112\\
G130M & 1318 & 1201 & 362 & 9465 & 112 & 11806 & 362 & 3194 & 112\\
G130M & 1327 & 1201 & 362 & 10489 & 112 & 12830 & 362 & 2170 & 112\\
G160M & ALL  & 1201 & 356 & 13799 & 112 &\dots&\dots&\dots&\dots\\
G140L & 1105 & 4701 & 372 & 6958 & 112 & 14543 & 372 & 457 & 112\\
G140L & 1280 & 6201 & 372 & 7400 & 112 &\dots&\dots&\dots& \dots\\
\bottomrule
\enddata
\tablenotetext{a}{Updated on 2009.200 with \pr{63095}, some early ERO and calibrations used slightly different subarrays.}
\tablenotetext{b}{Updated early in LP1 on Aug 27, 2009 (2009.2239) with \pr{63378}.}
\tablenotetext{c}{G140L updates were made on Dec. 4, 2012 (2012.339) with \pr{72193} to futher optimize the G140L subarrays.}
\tablenotetext{d}{Updated for LP2 July 18, 2012 (2012.200) with \pr{70548}.}
\tablenotetext{e}{Updated for LP3 operations on Aug. 26, 2014 (2014.238) with \pr{78747}.}
\tablenotetext{f}{Updated for LP4 operations on Feb. 20, 2017 (2017.051) with \pr{86945}.}
\tablenotetext{g}{Starting with C18, the C1230 \textit{CENWAVE} was replaced with C1280 due to first-order light issues (see Dixon et al., 2010, \pr{64041} and \pr{64659}.}
\end{deluxetable}

\begin{deluxetable}{lc|rrrr|rrrr}
\tablewidth{0pt}
\tabcolsep 9pt
\tablecolumns{10}
\tabletypesize{\scriptsize}
\tablecaption{FUVB PSA/BOA Subarrays for LP1--4\label{tab:FUVsubB}.}
\tablehead{
\colhead{\textit{OPT\_ELEM}} & \colhead{\cenwave{}} & \multicolumn{4}{c}{B1 Subarray} & \multicolumn{4}{c}{B2 Subarray} \\
\colhead{ } & \colhead{(\AA)} &
\colhead{XC} & \colhead{YC} & \colhead{XS} & \colhead{YS} & \colhead{XC} & \colhead{YC} & \colhead{XS} & \colhead{YS}\\
\colhead{(1)}&\colhead{(2)} & \colhead{(3)}&\colhead{(4)} &
\colhead{(5)}&\colhead{(6)} & \colhead{(7)}&\colhead{(8)} & \colhead{(9)}&\colhead{(10)}
}
\startdata
\multicolumn{10}{c}{LP1\tablenotemark{c}}\\
\midrule
G130M & 1291 & 5036\tablenotemark{b} & 76 & 1501 & 483 & 7477\tablenotemark{b} & 76 & 7773\tablenotemark{b} & 483\tablenotemark{a,b} \\
G130M & 1300 & 6039\tablenotemark{b} & 76 & 1501 & 483 & 6474\tablenotemark{b} & 76 & 8776\tablenotemark{b} & 483\tablenotemark{a,b} \\
G130M & 1309 & 7023\tablenotemark{b} & 76 & 1501 & 483 & 5490\tablenotemark{a} & 76 & 9760\tablenotemark{a} & 483\tablenotemark{a,b} \\
G130M & 1318 & 7977\tablenotemark{b} & 76 & 1501 & 483 & 4536\tablenotemark{b} & 76 & 10714\tablenotemark{b} & 483\tablenotemark{a,b} \\
G130M & 1327 & 7629\tablenotemark{b} & 76 & 2792\tablenotemark{b} & 483 & 3593\tablenotemark{b} & 76 & 11657\tablenotemark{b} & 483\tablenotemark{a,b} \\
G160M & ALL  & 13749 & 76 & 1501 & 477\tablenotemark{a,b} &\dots&\dots&\dots&\dots\\
G140L & ALL\tablenotemark{g} &\dots&\dots&\dots&\dots&\dots&\dots&\dots&\dots\\
\midrule
\multicolumn{10}{c}{LP2\tablenotemark{c}}\\
\midrule
G130M & 1291 & 1501 & 522 & 5036 & 76 & 7773 & 522 & 7477 & 76\\
G130M & 1300 & 1501 & 522 & 6039 & 76 & 8776 & 522 & 6474 & 76\\
G130M & 1309 & 1501 & 522 & 7023 & 76 & 9760 & 522 & 5490 & 76\\
G130M & 1318 & 1501 & 522 & 7977 & 76 & 10714 & 522 & 4536 & 76\\
G130M & 1327 & 2792 & 522 & 7629 & 76 & 11657 & 522 & 3593 & 76\\
G160M & ALL & 1501 & 515 & 13749 & 76        &\dots&\dots&\dots&\dots\\
G140L & ALL &\dots&\dots&\dots&\dots&\dots&\dots&\dots&\dots\\
\midrule
\multicolumn{10}{c}{LP3\tablenotemark{d}~~(Pre FUVB ``Hot-Spot'')}\\
\midrule
G130M & 1291 & 1501 & 460 & 5036 & 76 & 7773 & 460 & 7477 & 76\\
G130M & 1300 & 1501 & 460 & 6039 & 76 & 8776 & 460 & 6474 & 76\\
G130M & 1309 & 1501 & 460 & 7023 & 76 & 9760 & 460 & 5490 & 76\\
G130M & 1318 & 1501 & 460 & 7977 & 76 & 10714 & 460 & 4536 & 76\\
G130M & 1327 & 2792 & 460 & 7629 & 76 & 11657 & 460 & 3593 & 76\\
G160M & ALL & 1501 & 453 & 13749 & 76        &\dots&\dots&\dots&\dots\\
G140L & ALL &\dots&\dots&\dots&\dots&\dots&\dots&\dots&\dots\\
\midrule
\multicolumn{10}{c}{LP3\tablenotemark{e}~~(Post FUVB ``Hot-Spot'')\tablenotemark{d}}\\
\midrule
G130M & 1291 & 1501 & 460 & 5036 & 76 & 7773 & 460 & 7060\tablenotemark{e} & 76\\
G130M & 1300 & 1501 & 460 & 6039 & 76 & 8776 & 460 & 6057\tablenotemark{e} & 76\\
G130M & 1309 & 1501 & 460 & 7023 & 76 & 9760 & 460 & 5073\tablenotemark{e} & 76\\
G130M & 1318 & 1501 & 460 & 7977 & 76 & 10714 & 460 & 4119\tablenotemark{e} & 76\\
G130M & 1327 & 2792 & 460 & 7629 & 76 & 11657 & 460 & 3176\tablenotemark{e} & 76\\
G160M & ALL & 1501 & 453 & 13332 & 76        &\dots&\dots&\dots&\dots\\
G140L & ALL &\dots&\dots&\dots&\dots&\dots&\dots&\dots&\dots\\
\midrule
\multicolumn{10}{c}{LP4\tablenotemark{f}}\\
\midrule
G130M & 1291 & 1501 & 419 &  5036 & 112 &  7773 & 419 & 7060 & 112 \\
G130M & 1300 & 1501 & 419 &  6039 & 112 &  8776 & 419 & 6057 & 112 \\
G130M & 1309 & 1501 & 419 &  7023 & 112 &  9760 & 419 & 5073 & 112 \\
G130M & 1318 & 1501 & 419 &  7977 & 112 & 10714 & 419 & 4119 & 112 \\
G130M & 1327 & 2792 & 419 &  7629 & 112 & 11657 & 419 & 3176 & 112 \\
G160M &  ALL & 1501 & 416 & 13332 & 112      &\dots&\dots&\dots&\dots\\
G140L & 1105 &\dots&\dots&\dots&\dots&\dots&\dots&\dots&\dots\\
G140L & 1280 &\dots&\dots&\dots&\dots&\dots&\dots&\dots&\dots\\
\enddata
\tablenotetext{a}{Updated during SMOV (2009.201) with \pr{63095}.}
\tablenotetext{b}{Updated for LP2 operations on July 18, 2012 (2012.200) with \pr{70548}.}
\tablenotetext{c}{Due to gain sag induced 'Y-walk', FUVB usage for \tacq{PEAKXD} (\texttt{NUM\_POS=1})
TAs was deprecated on 2011.098 with \pr{67985}. FUVB is still used for \tacq{SEARCH} and \tacq{PEAKD} TA exposures.}
\tablenotetext{d}{Updated for LP3  on Aug. 26, 2014 (2014.238) with \pr{78747}.}
\tablenotetext{e}{Updated for post ``Hot-Spot'' LP3 TA operations on April 20, 2015 (2015.110) with \pr{80571}.}
\tablenotetext{f}{Updated for LP4  on Feb. 20, 2017 (2017.051) with \pr{86945}. Both FUVA and FUVB are used for all LP4 \tacq{PEAKXD}s (\textit{NUM\_POS$>1$}) TA exposures. These subarrays also avoid the FUVB ``Hot-Spot''.}
\tablenotetext{g}{Starting with C18, the C1230 \textit{CENWAVE} was replaced with C1280 due to first-order light issues.
(see Dixon et al., 2010, \pr{64041} and \pr{64659}.}
\end{deluxetable}

\subsection{Trimming of COS FUV TA subarrays due to FUVB ``Hot-Spot''.}\label{subsec:hotspots}
A ``hot-spot'' appeared on the COS FUVB segment coincident with increased solar activity in 2014--15.
This spot produced enough counts that it could cause mis-centering during all phases of the FUV LP3 (\& LP4) spectroscopic TAs.
These mis-centerings could be in significant in either the AD or XD. All affected
LP3 FUVB TA subarrays were adjusted on April 20, 2015 (2015.110). See \pr{80571} for futher details.

In FUVB detector coordinates, the approximate location of the hot-spot is at [X$_{DET}$,Y$_{DET}$]=[14895,482]. As this is near the detector edge, we are able to avoid this hotspot by stopping the last subarray of the FUVB subarrays at X$_{DET}$=14833.
For the COS FUV gratings and the FUVB TA subarrays, the impacts were:\\
\begin{description}
	\item[G140L:] Not affected as no FUVB TA subarrays are used for G140L
	\item[G160M:] One FUVB subarray is used for each \cenwave{} with XC1=1501, XS1=13749. These were all changed to XS=13332 (no change in Y).
	\item[G130M:] Two \cenwave{}-specific FUVB subarrays are used to avoid Geocoronal Ly$\alpha$. The X-size (XS) of the second subarray (XS2) will be trimmed to avoid the hotspot (XC1, XS1, XC2 and all the Y definitions do not change).
\end{description}

As of March 2018, no additional hot-spots have appeared on either FUVA or FUVB that required adjustment of the TA subarrays.
Due to the possibility of future hot-spots, the number of allow FUV TA subarrays per segment was increased from two to four on  Sept 21, 2015 (2015.264) with \pr{81263}.

\clearpage
\section{NUV Imaging TA verification}\label{sec:NimVER}
{\bf Note to Reviewers: I am still working with Colin Cox on some details of the
initial pointing offsets that are provided outside of the exposures of these programs (from the telemetry stream that creates the jitter files).

In order to streamline the review process of this ISR, I prefer
to hold back this entire section until this portion of the
analysis has been complete as the offsets from Colin directly
affect the conclusions about the validity of the COS SIAF entries, and all offsets trickle down through all of the bootstrapping
analysis. The current analysis is contained in file ``NimVer.tex''
and this is currently commented out "%% $Id: NimVer.tex,v 1.5 2018/03/30 20:22:12 penton Exp $

\subsection{Verifying the \tacq{IMAGE} WCA-to-SA Offsets.}\label{subsec:WCA2SAVER}

The verification of the \tacq{IMAGE} WCA-to-SA (PSA or BOA) offsets is a multi-stage bootstrap
process similar to the one used to measure the initial offsets in the SMOV enabling program (\pid{11471}, COS FUV Target Acquisition Algorithm Verification).

Each visit of each cycles monitoring program directly compares two \tacq{IMAGE} combinations.
We can bootstrap these back to PSA$\times$MIRA to test the co-alignment of all four combinations.
We call this the `baseline` bootstrapping, the results of which are shown in Table~\ref{tab:basebootstrap}
and discussed in \S~\ref{subsec:basebootstrap}

These measurements have certain limitations, so as the $\pm 0.5p$ measurement uncertainity in both directions
when measuring the WCA centroid. The WCA lamp exposures of each cycles program assist in removing
these limitations from the bootstrapping. The PSA \tacq{IMAGE} visits level the shutter open when
taking these WCA lamp images so that a co-eval target+lamp TT image is acquired. This allows a direct
calculation of the WCA-to-PSA offset using any desired centroiding algoritm. The BOA \tacq{IMAGE} visits
take sequential lamp and BOA target images to measure the WCA-to-BOA offsets, but these are not co-eval,
and the aperture has been moved between the exposures, which often causes a $\pm$ one step offset.
Fortunately, this offset can be tracked with the telemetry keywords and accounted for.


\subsection{Baseline Bootstrap of \tacq{IMAGE modes}}\label{subsec:basebootstrap}

The baseline bootstrapping data is given in Table~\ref{tab:tamonbasicnimverT}.

% tamon_basic_nimver
%
% RCSID="$Id$"
%
\begin{deluxetable}{llrccccccrrrrr}
\tablecolumns{14}
\tabcolsep 3pt
\tablecaption{C17--C20 PSA \tacq{IMAGE} Co-alignment Measurements\label{tab:basebootstrap}}
\tabletypesize{\scriptsize}
\tablehead{
\colhead{\textit{PROP}}&\colhead{\textit{ROOT}}&\colhead{Configuration}  &
\multicolumn{2}{c}{WCA-Msrd\tablenotemark{a}}  &
\multicolumn{2}{c}{PSA-Msrd}  &
\multicolumn{2}{c}{PSA-Centered}  &
\multicolumn{2}{c}{SA-to-WCA}     &
\multicolumn{2}{c}{TA Centering}  &
\colhead{\textit{DATE-}}\\

\colhead{\textit{OSID}}& \colhead{\textit{NAME}} & \colhead{\textit{APERTURE}}&
\colhead{\textit{MXCR}}&\colhead{\textit{MYCR}}&
\colhead{\textit{ACQ}}&\colhead{\textit{ACQ}}&
\colhead{\textit{ACQ}}&\colhead{\textit{ACQ}}&
\colhead{} &\colhead{}&
\colhead{\textit{ACQ}}&\colhead{\textit{ACQ}}&
\colhead{\textit{OBS}}\\

\colhead{}&\colhead{}&\colhead{$\times$}&
\colhead{\textit{LAMP}}&\colhead{\textit{LAMP}}&
\colhead{\textit{CENTX}}&\colhead{\textit{CENTY}}&
\colhead{\textit{PREFX}}&\colhead{\textit{PREFY}}&
\colhead{} & \colhead{} &
\colhead{\textit{SLEWX}}&\colhead{\textit{SLEWY}}&
\colhead{}\\

\colhead{(PID)}&\colhead{}&\colhead{\textit{OPT\_ELEM}}&
\colhead{AD} &\colhead{XD\tablenotemark{b}}&
\colhead{AD} &\colhead{XD\tablenotemark{b}}&
\colhead{AD} &\colhead{XD\tablenotemark{b}}&
\colhead{AD} &\colhead{XD\tablenotemark{b}}&
\colhead{AD} &\colhead{XD\tablenotemark{b}}&
\colhead{}\\

\colhead{(1)} & \colhead{(2)} & \colhead{(3)}&
\colhead{(4)} & \colhead{(5)} & \colhead{(6)} & \colhead{(7)} &
\colhead{(8)} & \colhead{(9)} & \colhead{(10)} & \colhead{(11)} &
\colhead{(12)} & \colhead{(13)} & \colhead{(14)}
}
\startdata
\toprule
\multicolumn{14}{c}{C17} \\
\midrule
\pid{11878} &lbcla3s3q&PSA$\times$MIRA&555&653&509.3&282.7&509.7&280.3& 45.3&372.7&  -0.010&   0.056&2010-11-05 \\
\pid{11878} &lbcla3s7q&PSA$\times$MIRB&342&813&296.4&439.8&297.0&438.9& 45.0&374.1&  -0.015&   0.021&2010-11-05 \\
\midrule
\multicolumn{14}{c}{C18} \\
\midrule
\pid{12399} &lbm7a2ahq&PSA$\times$MIRA&529&653&475.2&279.5&483.7&280.3& 45.3&372.7&  -0.200&  -0.019&2011-09-12 \\
\pid{12399} &lbm7a2ajq&PSA$\times$MIRB&317&813&271.2&439.3&272.0&438.9& 45.0&374.1&  -0.018&   0.008&2011-09-12 \\
\midrule
\multicolumn{14}{c}{C19} \\
\midrule
\pid{12781} &lbx1a2ffq&PSA$\times$MIRA&503&650&450.0&280.6&457.7&277.3& 45.3&372.7&  -0.183&   0.078&2012-09-24 \\
\pid{12781} &lbx1a2fhq&PSA$\times$MIRB&296&811&249.8&436.0&251.0&436.9& 45.0&374.1&  -0.028&  -0.021&2012-09-24 \\
\midrule
\multicolumn{14}{c}{C20} \\
\midrule
\pid{13171} &lc6ka2imq&PSA$\times$MIRA&508&650&459.7&284.0&462.7&277.3& 45.3&372.7&  -0.070&   0.158&2013-09-01 \\
\pid{13171} &lc6ka2ioq&PSA$\times$MIRB&304&811&258.2&436.5&259.0&436.9& 45.0&374.1&  -0.019&  -0.009&2013-09-01 \\
\midrule
\bottomrule
\enddata
\tablenotetext{a}{Non-repeatability of the OSM and aperture mechanisms, along with environmental factors, result in
lamp center offsets of up to 3~p in AD and to 52~p in XD in these exposures. }
%\tablenotetext{b}{BOA \tacq{IMAGE}s move the aperture in the XD direction to obtain the WCA lamp image. Occasionally, the aperture mechanism misses the desired location by $\pm 1 $ \textit{APERXPOS} step of $\sim0.05\arcsec$.}
%\tablenotetext{c}{These PSA$\times$MIRA \tacq{IMAGE}s were part of the FGS-to-SI programs and do {\bf NOT} have a proceeding TA. The TA centering adjustments presented here are to be compared to the FGS-to-SI post processing results presented in Table~\ref{tab:fgs2siInit}.}
\tablecomments{If the table caption is in the \texttt{Courier} font, this value was taken directly from the indicated \textsc{\_rawacq.fits} header keyword. In DETector coordinates, +AD is -Y$_{DET}$, +XD is -X$_{DET}$, in USER coordinates, +AD is +Y$_{USER}$, +XD is +X$_{USER}$.}
\end{deluxetable}

\begin{deluxetable}{llrccccccrrrrr}
\tablecolumns{14}
\tabcolsep 3pt
\tablecaption{C21--24 \tacq{IMAGE} Bootstrapping Results\label{tab:basebootstrap}}
\tabletypesize{\scriptsize}
\tablehead{
\colhead{\textit{PROP}}&\colhead{\textit{ROOT}}&\colhead{Configuration}  &
\multicolumn{2}{c}{WCA-Msrd\tablenotemark{a}}  &
\multicolumn{2}{c}{SA-Msrd\tablenotemark{b}}  &
\multicolumn{2}{c}{SA-Centered}  &
\multicolumn{2}{c}{SA-to-WCA}     &
\multicolumn{2}{c}{TA Centering}  &
\colhead{\textit{DATE-}}\\

\colhead{\textit{OSID}}& \colhead{\textit{NAME}} & \colhead{\textit{APERTURE}}&
\colhead{\textit{MXCR}}&\colhead{\textit{MYCR}}&
\colhead{\textit{ACQ}}&\colhead{\textit{ACQ}}&
\colhead{\textit{ACQ}}&\colhead{\textit{ACQ}}&
\colhead{} &\colhead{}&
\colhead{\textit{ACQ}}&\colhead{\textit{ACQ}}&
\colhead{\textit{OBS}}\\

\colhead{}&\colhead{}&\colhead{$\times$}&
\colhead{\textit{LAMP}}&\colhead{\textit{LAMP}}&
\colhead{\textit{CENTX}}&\colhead{\textit{CENTY}}&
\colhead{\textit{PREFX}}&\colhead{\textit{PREFY}}&
\colhead{} & \colhead{} &
\colhead{\textit{SLEWX}}&\colhead{\textit{SLEWY}}&
\colhead{}\\

\colhead{(PID)}&\colhead{}&\colhead{\textit{OPT\_ELEM}}&
\colhead{AD} &\colhead{XD}&
\colhead{AD} &\colhead{XD}&
\colhead{AD} &\colhead{XD}&
\colhead{AD} &\colhead{XD}&
\colhead{AD} &\colhead{XD}&
\colhead{}\\

\colhead{(1)} & \colhead{(2)} & \colhead{(3)}&
\colhead{(4)} & \colhead{(5)} & \colhead{(6)} & \colhead{(7)} &
\colhead{(8)} & \colhead{(9)} & \colhead{(10)} & \colhead{(11)} &
\colhead{(12)} & \colhead{(13)} & \colhead{(14)}
}
\startdata
\toprule
\multicolumn{14}{c}{C21} \\
\midrule
\pid{13616} &lci4a2e3q&PSA$\times$MIRA&517&650&471.7&282.9&471.7&277.3& 45.3&372.7&   0.001&   0.133&2014-10-27 \\
\pid{13616} &lci4a2e5q&PSA$\times$MIRB&305&809&259.7&436.1&259.0&435.0& 46.0&374.0&   0.016&   0.027&2014-10-27 \\
\midrule
\pid{13526} &lcgq03dbq&PSA$\times$MIRA&566&648&527.7&268.2&520.7&275.3& 45.3&372.7&   0.166&  -0.168&2014-10-06 \\
\pid{13526} &lcgq03dlq&PSA$\times$MIRB&357&808&310.8&434.2&312.0&433.9& 45.0&374.1&  -0.029&   0.006&2014-10-06 \\
\pid{13526} &lcgq03dtq&PSA$\times$MIRA&574&649&530.1&275.2&528.7&276.3& 45.3&372.7&   0.032&  -0.026&2014-10-06 \\
\pid{13526} &lcgq01q5q&PSA$\times$MIRB&306&809&222.4&447.7&260.0&435.0& 46.0&374.0&  -0.888&   0.298&2014-11-19 \\
\pid{13526} &lcgq01qdq&BOA$\times$MIRA&520&651&472.9&283.2&474.5&282.6& 45.5&368.4&  -0.038&   0.013&2014-11-19 \\
\pid{13526} &lcgq01qjq&PSA$\times$MIRB&305&811&260.2&433.7&259.0&437.0& 46.0&374.0&   0.027&  -0.078&2014-11-19 \\
\pid{13526} &lcgq02hmq&BOA$\times$MIRA&495&649&452.0&293.6&449.5&280.6& 45.5&368.4&   0.058&   0.305&2014-11-17 \\
\pid{13526} &lcgq02huq&BOA$\times$MIRB&285&811&237.9&440.3&238.5&444.8& 46.5&366.2&  -0.015&  -0.105&2014-11-17 \\
\pid{13526} &lcgq02i0q&BOA$\times$MIRA&501&651&455.7&285.3&455.5&282.6& 45.5&368.4&   0.006&   0.063&2014-11-17 \\
\midrule
\multicolumn{14}{c}{C22} \\
\midrule
\pid{14035} &lcsla2bhq&PSA$\times$MIRA&505&653&458.8&284.8&459.7&280.3& 45.3&372.7&  -0.020&   0.105&2015-10-02 \\
\pid{14035} &lcsla2bjq&PSA$\times$MIRB&293&813&247.9&439.3&247.0&439.0& 46.0&374.0&   0.022&   0.007&2015-10-02 \\
\pid{13972} &lcri01fzq&PSA$\times$MIRB&302&813&264.0&427.4&256.0&439.0& 46.0&374.0&   0.189&  -0.273&2015-10-06 \\
\pid{13972} &lcri01g7q&BOA$\times$MIRA&517&651&471.0&287.2&471.5&282.6& 45.5&368.4&  -0.011&   0.109&2015-10-06 \\
\pid{13972} &lcri01geq&PSA$\times$MIRB&300&812&255.5&434.1&254.0&438.0& 46.0&374.0&   0.036&  -0.092&2015-10-06 \\
\pid{13972} &lcri02h8q&BOA$\times$MIRA&499&654&462.0&272.2&453.5&285.6& 45.5&368.4&   0.201&  -0.315&2015-10-06 \\
\pid{13972} &lcri02hgq&BOA$\times$MIRB&286&810&240.6&444.2&239.5&443.8& 46.5&366.2&   0.026&   0.010&2015-10-06 \\
\pid{13972} &lcri02hmq&BOA$\times$MIRA&504&651&457.7&284.2&458.5&282.6& 45.5&368.4&  -0.018&   0.038&2015-10-06 \\
\midrule
\multicolumn{14}{c}{C23} \\
\midrule
\pid{14452} &ld3la2ojq&PSA$\times$MIRA&527&654&481.0&284.5&481.7&281.3& 45.3&372.7&  -0.017&   0.075&2016-10-02 \\
\pid{14452} &ld3la2onq&PSA$\times$MIRB&306&813&259.9&439.9&260.0&439.0& 46.0&374.0&  -0.002&   0.021&2016-10-02 \\
\pid{14440} &ld3701gtq&PSA$\times$MIRB&297&813&246.6&442.7&251.0&439.0& 46.0&374.0&  -0.104&   0.088&2016-10-18 \\
\pid{14440} &ld3701h1q&BOA$\times$MIRA&518&651&471.7&287.1&472.5&282.6& 45.5&368.4&  -0.018&   0.105&2016-10-18 \\
\pid{14440} &ld3701h7q&PSA$\times$MIRB&295&811&249.7&434.0&249.0&437.0& 46.0&374.0&   0.016&  -0.071&2016-10-18 \\
\pid{14440} &ld3702mzq&BOA$\times$MIRA&509&654&480.0&299.3&463.5&285.6& 45.5&368.4&   0.390&   0.322&2016-10-19 \\
\pid{14440} &ld3702n9q&BOA$\times$MIRB&295&811&248.5&446.1&248.5&444.8& 46.5&366.2&   0.001&   0.030&2016-10-19 \\
\pid{14440} &ld3702nhq&BOA$\times$MIRA&516&651&470.0&285.0&470.5&282.6& 45.5&368.4&  -0.012&   0.057&2016-10-19 \\
\midrule
\multicolumn{14}{c}{C24} \\
\midrule
\pid{14857} &ldozpbf5q&PSA$\times$MIRA&515&654&467.3&266.2&469.7&281.3& 45.3&372.7&  -0.058&  -0.355&2017-09-10 \\
\pid{14857} &ldozpbfbq&PSA$\times$MIRB&289&813&243.8&440.0&243.0&439.0& 46.0&374.0&   0.020&   0.023&2017-09-10 \\
\pid{14857} &ldozpbfhq&PSA$\times$MIRA&510&653&463.9&279.8&464.7&280.3& 45.3&372.7&  -0.020&  -0.011&2017-09-10 \\
\pid{14857} &ldozbadhq&PSA$\times$MIRB&299&813&237.5&397.9&253.0&439.0& 46.0&374.0&  -0.366&  -0.967&2017-09-04 \\
\pid{14857} &ldozbadpq&BOA$\times$MIRA&524&651&477.9&287.4&478.5&282.6& 45.5&368.4&  -0.013&   0.113&2017-09-04 \\
\pid{14857} &ldozbadvq&PSA$\times$MIRB&298&811&253.2&434.5&252.0&437.0& 46.0&374.0&   0.029&  -0.059&2017-09-04 \\
\pid{14857} &ldozbbleq&BOA$\times$MIRA&518&653&484.8&289.5&472.5&284.6& 45.5&368.4&   0.290&   0.114&2017-09-06 \\
\pid{14857} &ldozbblmq&BOA$\times$MIRB&293&811&246.7&444.3&246.5&444.8& 46.5&366.2&   0.006&  -0.011&2017-09-06 \\
\pid{14857} &ldozbblsq&BOA$\times$MIRA&514&651&467.9&285.1&468.5&282.6& 45.5&368.4&  -0.015&   0.060&2017-09-06 \\
\bottomrule
\enddata
\tablenotetext{a}{Non-repeatability of the OSM and aperture mechanisms, along with environmental factors, result in
lamp center offsets of up to 6~p in AD and $> 50$~p in XD in these exposures.}
\tablenotetext{b}{BOA \tacq{IMAGE}s move the aperture in the XD direction to obtain the WCA lamp image. Occasionally, the aperture mechanism misses the desired location by $\pm 1$ \textit{APERYPOS} step of $\sim0.05\arcsec$.}
%\tablenotetext{c}{These PSA$\times$MIRA \tacq{IMAGE}s were part of the FGS-to-SI programs and do not have a proceeding \tacq{IMAGE}. The TA centerings presented here are to be compared to the FGS-to-SI post processing results presented in Table~\ref{tab:fgs2siInit}.}
\tablecomments{If the table caption is in the \textit{ITALICS}, this value was taken directly from the indicated \textsc{\_rawacq.fits} header keyword. Columns 4-11 are in units of NUV pixels (p). Columns 12 \& 13 are in arcseconds (\arcsec). In DETector coordinates, +AD is -Y$_{DET}$, +XD is -X$_{DET}$, in USER coordinates, +AD is +Y$_{USER}$, +XD is +X$_{USER}$.}
\end{deluxetable}

\begin{deluxetable}{llrccccccrrrrr}
\tablecolumns{14}
\tabcolsep 4pt
\tablecaption{\tacq{IMAGE} Bootstrapping Measurements Sorted by Configuration\label{tab:bootstrapAligned}}
\tabletypesize{\scriptsize}
\tablehead{
\colhead{\textit{PROP}}&\colhead{\textit{ROOT}}&
\colhead{HST}  &
\multicolumn{2}{c}{WCA-Msrd\tablenotemark{a}}  &
\multicolumn{2}{c}{SA-Msrd\tablenotemark{b}}  &
\multicolumn{2}{c}{SA-Center}  &
\multicolumn{2}{c}{SA-to-WCA}     &
\multicolumn{2}{c}{TA Centering}&\colhead{\textit{DATE}}\\

\colhead{\textit{OSID}}& \colhead{\textit{NAME}} & \colhead{Cycle}&
\colhead{\textit{LAMP}}&\colhead{\textit{LAMP}}&
\colhead{\textit{ACQ}}&\colhead{\textit{ACQ}}&
\colhead{\textit{ACQ}}&\colhead{\textit{ACQ}}&
\colhead{}& \colhead{} &\colhead{\textit{ACQ}}&\colhead{\textit{ACQ}}& \colhead{\textit{OBS}} \\

\colhead{}& \colhead{} & \colhead{}&
\colhead{\textit{MXCR}}&\colhead{\textit{MYCR}}&
\colhead{\textit{MSRDX}}&\colhead{\textit{MSRDY}}&
\colhead{\textit{PREFX}}&\colhead{\textit{PREFY}}&
\colhead{}& \colhead{} &\colhead{\textit{SLEWX}}&\colhead{\textit{SLEWY}}& \\
\colhead{(PID)}  &     \colhead{}     & \colhead{} &
\colhead{AD} &\colhead{XD}&
\colhead{AD} &\colhead{XD}&
\colhead{AD} &\colhead{XD}&
\colhead{AD} &\colhead{XD}& \colhead{} \\
\colhead{(1)}  &\colhead{(2)} & \colhead{(3)}&\colhead{(4)} &
\colhead{(5)}  &\colhead{(6)} & \colhead{(7)}&\colhead{(8)} &
\colhead{(9)}  &\colhead{(10)} & \colhead{(11)} &\colhead{(12)} &
\colhead{(13)} &\colhead{(14)}
}
\startdata
\toprule
\multicolumn{14}{c}{PSA$\times$MIRRORA}\\
\midrule
\pid{11878} &lbcla3s3q&C17&555&653&509.3&282.7&509.7&280.3& 45.3&372.7&  -0.010&   0.056&2010-11-05 \\
\pid{12399} &lbm7a2ahq&C18&529&653&475.2&279.5&483.7&280.3& 45.3&372.7&  -0.200&  -0.019&2011-09-12 \\
\pid{12781} &lbx1a2ffq&C19&503&650&450.0&280.6&457.7&277.3& 45.3&372.7&  -0.183&   0.078&2012-09-24 \\
\pid{13171} &lc6ka2imq&C20&508&650&459.7&284.0&462.7&277.3& 45.3&372.7&  -0.070&   0.158&2013-09-01 \\
\pid{13616} &lci4a2e3q&C21&517&650&471.7&282.9&471.7&277.3& 45.3&372.7&   0.001&   0.133&2014-10-27 \\
\pid{14035} &lcsla2bhq&C22&505&653&458.8&284.8&459.7&280.3& 45.3&372.7&  -0.020&   0.105&2015-10-02 \\
\pid{14452} &ld3la2ojq&C23&527&654&481.0&284.5&481.7&281.3& 45.3&372.7&  -0.017&   0.075&2016-10-02 \\
\pid{14857} &ldozpbf5q&C24&515&654&467.3&266.2&469.7&281.3& 45.3&372.7&  -0.058&  -0.355&2017-09-10 \\
\pid{14857} &ldozpbfhq&C24&510&653&463.9&279.8&464.7&280.3& 45.3&372.7&  -0.020&  -0.011&2017-09-10 \\
\midrule
\multicolumn{14}{c}{PSA$\times$MIRRORB\tablenotemark{c}}\\
\midrule
\pid{11878} &lbcla3s7q&C17&342&813&296.4&439.8&297.0&438.9& 45.0&374.1&  -0.015&   0.021&2010-11-05 \\
\pid{12399} &lbm7a2ajq&C18&317&813&271.2&439.3&272.0&438.9& 45.0&374.1&  -0.018&   0.008&2011-09-12 \\
\pid{12781} &lbx1a2fhq&C19&296&811&249.8&436.0&251.0&436.9& 45.0&374.1&  -0.028&  -0.021&2012-09-24 \\
\pid{13171} &lc6ka2ioq&C20&304&811&258.2&436.5&259.0&436.9& 45.0&374.1&  -0.019&  -0.009&2013-09-01 \\
\hline
\pid{13616} &lci4a2e5q&C21&305&809&259.7&436.1&259.0&435.0& 46.0&374.0&   0.016&   0.027&2014-10-27 \\
\pid{14035} &lcsla2bjq&C22&293&813&247.9&439.3&247.0&439.0& 46.0&374.0&   0.022&   0.007&2015-10-02 \\
\pid{13972} &lcri01fzq&C22&302&813&264.0&427.4&256.0&439.0& 46.0&374.0&   0.189&  -0.273&2015-10-06 \\
\pid{13972} &lcri01geq&C22&300&812&255.5&434.1&254.0&438.0& 46.0&374.0&   0.036&  -0.092&2015-10-06 \\
\pid{14452} &ld3la2onq&C23&306&813&259.9&439.9&260.0&439.0& 46.0&374.0&  -0.002&   0.021&2016-10-02 \\
\pid{14440} &ld3701gtq&C23&297&813&246.6&442.7&251.0&439.0& 46.0&374.0&  -0.104&   0.088&2016-10-18 \\
\pid{14440} &ld3701h7q&C23&295&811&249.7&434.0&249.0&437.0& 46.0&374.0&   0.016&  -0.071&2016-10-18 \\
\pid{14857} &ldozbadhq&C24&299&813&237.5&397.9&253.0&439.0& 46.0&374.0&  -0.366&  -0.967&2017-09-04 \\
\pid{14857} &ldozbadvq&C24&298&811&253.2&434.5&252.0&437.0& 46.0&374.0&   0.029&  -0.059&2017-09-04 \\
\pid{14857} &ldozpbfbq&C24&289&813&243.8&440.0&243.0&439.0& 46.0&374.0&   0.020&   0.023&2017-09-10 \\
\midrule
\multicolumn{14}{c}{BOA$\times$MIRRORA}\\
\midrule
\pid{13526} &lcgq01qdq&C21&520&651&472.9&283.2&474.5&282.6& 45.5&368.4&  -0.038&   0.013&2014-11-19 \\
\pid{13972} &lcri01g7q&C22&517&651&471.0&287.2&471.5&282.6& 45.5&368.4&  -0.011&   0.109&2015-10-06 \\
\pid{13972} &lcri02h8q&C22&499&654&462.0&272.2&453.5&285.6& 45.5&368.4&   0.201&  -0.315&2015-10-06 \\
\pid{13972} &lcri02hmq&C22&504&651&457.7&284.2&458.5&282.6& 45.5&368.4&  -0.018&   0.038&2015-10-06 \\
\pid{14440} &ld3701h1q&C23&518&651&471.7&287.1&472.5&282.6& 45.5&368.4&  -0.018&   0.105&2016-10-18 \\
\pid{14440} &ld3702mzq&C23&509&654&480.0&299.3&463.5&285.6& 45.5&368.4&   0.390&   0.322&2016-10-19 \\
\pid{14440} &ld3702nhq&C23&516&651&470.0&285.0&470.5&282.6& 45.5&368.4&  -0.012&   0.057&2016-10-19 \\
\pid{14857} &ldozbadpq&C24&524&651&477.9&287.4&478.5&282.6& 45.5&368.4&  -0.013&   0.113&2017-09-04 \\
\pid{14857} &ldozbbleq&C24&518&653&484.8&289.5&472.5&284.6& 45.5&368.4&   0.290&   0.114&2017-09-06 \\
\pid{14857} &ldozbblsq&C24&514&651&467.9&285.1&468.5&282.6& 45.5&368.4&  -0.015&   0.060&2017-09-06 \\
\midrule
\multicolumn{14}{c}{BOA$\times$MIRRORB\tablenotemark{c}}\\
\midrule
\pid{13526} &lcgq02huq&C21&285&811&237.9&440.3&238.5&444.8& 46.5&366.2&  -0.015&  -0.105&2014-11-17 \\
\pid{13972} &lcri02hgq&C22&286&810&240.6&444.2&239.5&443.8& 46.5&366.2&   0.026&   0.010&2015-10-06 \\
\pid{14440} &ld3702n9q&C23&295&811&248.5&446.1&248.5&444.8& 46.5&366.2&   0.001&   0.030&2016-10-19 \\
\pid{14857} &ldozbblmq&C24&293&811&246.7&444.3&246.5&444.8& 46.5&366.2&   0.006&  -0.011&2017-09-06 \\
\bottomrule
\enddata
\tablenotetext{a}{Environmental factors, and non-repeatability of the OSM and aperture mechanisms, results in
lamp offsets of up to 6~p in AD and $ > 50$~p in XD in these exposures. }
\tablenotetext{b}{BOA \tacq{IMAGE}s move the aperture in the XD direction to obtain the WCA lamp image. Occasionally, the aperture mechanism misses the desired location by $\pm 1 $ \textit{APERXPOS} step of $\sim0.05\arcsec$.}
%\tablenotetext{c}{These PSA$\times$MIRA \tacq{IMAGE}s were part of the FGS-to-SI programs and do {\bf NOT} have a proceeding TA. The TA centering adjustments presented here are to be compared to the FGS-to-SI post processing results presented in Table~\ref{tab:fgs2siInit}.}
\tablenotetext{c}{On November 6, 2014, the MIRB \texttt{ACQ/IMAGE} lamp exposure was changed in duration and current. FSW
tables were also updated at this time (\pr{67139}).}
\tablecomments{If the table caption is in the \texttt{Courier} font, this value was taken directly from the indicated \textsc{\_rawacq.fits} header keyword. In DETector coordinates, +AD is -Y$_{DET}$, +XD is -X$_{DET}$, in USER coordinates, +AD is +Y$_{USER}$, +XD is +X$_{USER}$.}
%           FROM   TO
%   PSA_B   3741   3740
%   BOA_B   3663   3662
%           FROM   TO
%   PSA_B   450    460
%   BOA_B   455    465
\end{deluxetable}


%\begin{deluxetable}{rrrrrrr}
%\tablecaption{Basic \tacq{IMAGE} Bootstrapping Results\label{tab:tamonbasicnimverB}}
%\tablecolumns{7}
%\tablhead{
%\colhead{ROOTNAME} & \colhead{\texttt{APERTURE}}& \colhead{\texttt{OPT\_ELEM}} & \colhead{}  & \colhead{} & \colhead{}  & \colhead{}\\
%\colhead{} & \colhead{}\colhead{} & \colhead{AD (Y)}  & \colhead{XD (X)} & \colhead{AD (Y)}  & \colhead{XD (X)}
%}
%\startdata
%\hline
%\multicolumn{7}{c}{C21 (\pid{})}\\
%\hline
%l	&	PSA &	MIRB	&	&	&	&	& \\
%l	&	BOA &	MIRA	&	&	&	&	& \\
%l	&	BOA &	MIRB	&	&	&	&	& \\
%\hline
%\multicolumn{7}{c}{C22 (\pid{})}\\
%\hline
%l	&	PSA &	MIRB	&	&	&	&	& \\
%l	&	BOA &	MIRA	&	&	&	&	& \\
%l	&	BOA &	MIRB	&	&	&	&	& \\
%\hline
%\multicolumn{7}{c}{C23 (\pid{})}\\
%\hline
%l	&	PSA &	MIRB	&	&	&	&	& \\
%l	&	BOA &	MIRA	&	&	&	&	& \\
%l	&	BOA &	MIRB	&	&	&	&	& \\
%\hline
%\multicolumn{7}{c}{C24 (\pid{})}\\
%\hline
%l	&	PSA &	MIRB	&	&	&	&	& \\
%l	&	BOA &	MIRA	&	&	&	&	& \\
%l	&	BOA &	MIRB	&	&	&	&	& \\
%\hline
%\enddata
%\tablecomments{+AD is -Y detector, +XD is -X detector.}
%\end{deluxetable}


% $Id: tamon_output.tex,v 1.6 2018/03/30 20:22:12 penton Exp $

\begin{deluxetable}{rrrrrrrrrrrrrrrrrrr}
\tabcolsep 2pt
\tabletypesize{\tiny}
\tablecolumns{19}
\tablecaption{COS TA Monitor \texttt{ACQ/IMAGE} Data}\label{tab:Imagedata}
\tablehead{
\colhead{\textit{ROOTNAME}}&\colhead{\textit{EXPTYPE}}&\colhead{\textit{OPT\_ELEM}}&\colhead{LAMP}&\colhead{Current}&\colhead{Target ET}&\colhead{Lamp ET}&\colhead{WCA}&\colhead{WCA}&\colhead{SA}&\colhead{SA}&\colhead{WtP}&\colhead{WtP}&\colhead{Lamp}&\colhead{Lamp}&\colhead{WCA}&\colhead{Lamp}&\colhead{Lamp}&\colhead{Target}\\
\colhead{}&\colhead{}&\colhead{ }&\colhead{}&\colhead{}&\colhead{(s)}&\colhead{(s)}&\colhead{AD}&\colhead{XD}&\colhead{AD}&\colhead{XD}&\colhead{AD}&\colhead{XD}&\colhead{counts}&\colhead{cps}&\colhead{bck}&\colhead{CPS}&\colhead{BP}&\colhead{BP}\\
\colhead{(1)}&\colhead{(2)} &
\colhead{(3)}&\colhead{(4)} &
\colhead{(5)}&\colhead{(6)} &
\colhead{(7)}&\colhead{(8)} &
\colhead{(9)}&\colhead{(10)} &
\colhead{(11)} &\colhead{(12)} &
\colhead{(13)}&\colhead{(14)} &
\colhead{(15)}&\colhead{(16)} &
\colhead{(17)}&\colhead{(18)} &
\colhead{(19)}
}

\startdata
lcgq01q7q & EXT/SCI & MIRB & P2 & MED & 16 & 16 & 717.0 & 214.0 & 763.1 & 588.9 & 46.1 & 374.9 & 4890.0 & 305.6 & 167 & 305.6 & 4.4 & 26.7\\
lcgq01q9q & EXT/SCI & MIRA & P2 & MED & 150 & 150 & 479.0 & 370.0 & 550.3 & 739.9 & 71.3 & 369.9 & 1718.0 & \dots & \dots & \dots & \dots & 0.2\\

lcgq02hoq & WAVECAL & MIRA & P2 & LOW & 7 & \dots & 529.0 & 372.0 & 891.6 & 635.6 & 362.6 & 263.6 & 2827.0 & 403.9 & 97 & 403.9 & 9.9 & 0.3\\
lcgq02hqq & EXT/SCI & MIRB & P2 & LOW & 181 & \dots & 713.0 & 211.0 & 784.4 & 582.7 & 71.4 & 371.7 & 2383.0 & \dots & \dots & \dots & \dots & 0.2\\

lcri01g1q & EXT/SCI & MIRB & P2 & MED & 12 & 12 & 722.0 & 210.0 & 767.7 & 584.2 & 45.7 & 374.2 & 3016.0 & 251.3 & 166 & 251.3 & 4.2 & 30.0\\
lcri01g3q & EXT/SCI & MIRA & P2 & MED & 150 & \dots & 474.0 & 370.0 & 552.0 & 735.7 & 78.0 & 365.7 & 1964.0 & \dots & \dots & \dots & \dots & 0.2\\

lcri02hcq & EXT/SCI & MIRB & P2 & LOW & 181 & 181 & 715.0 & 211.0 & 782.3 & 578.6 & 67.3 & 367.6 & 2406.0 & \dots & \dots & \dots & \dots & 0.2\\

ld3701gvq & EXT/SCI & MIRB & P2 & MED & 16 & 16 & 727.0 & 210.0 & 772.8 & 584.3 & 45.8 & 374.3 & 4147.0 & 259.2 & 184 & 259.2 & 4.3 & 19.0\\
ld3701gxq & EXT/SCI & MIRA & P2 & MED & 150 & 150 & 479.0 & 371.0 & 551.2 & 735.8 & 72.2 & 364.8 & 1739.0 & \dots & \dots & \dots & \dots & 0.2\\

ld3702n1q & WAVECAL & MIRA & P2 & LOW & 14 & \dots & 515.0 & 371.0 & 886.6 & 659.4 & 371.6 & 288.4 & 5589.0 & 399.2 & 167 & 399.2 & 7.7 & 0.2\\
ld3702n4q & EXT/SCI & MIRB & P2 & LOW & 183 & 183 & 723.0 & 213.0 & 774.9 & 577.6 & 51.9 & 364.6 & 2081.0 & \dots & \dots & \dots & \dots & 0.2\\

ldozbadjs & EXT/SCI & MIRB & P2 & MED & 16 & 16  & 724.0 & 210.0 & 769.8 & 583.4 & 45.8 & 373.4 & 4005.0 & 250.3 & 138 & 250.3 & 4.4 & 20.2\\
ldozbadlq & EXT/SCI & MIRA & P2 & MED & 150 & 150 & 472.0 & 371.0 & 545.1 & 735.6 & 73.1 & 364.6 & 1462.0 & \dots & \dots & \dots & \dots & 0.2\\

ldozbblgq & WAVECAL & MIRA & P2 & LOW & 14 & \dots & 507.0 & 372.0 & 748.6 & 911.9 & 241.6 & 539.9 & 5721.0 & 408.6 & 155 & 408.6 & 8.4 & 0.1\\
ldozbbliq & EXT/SCI & MIRB & P2 & LOW & 183 & \dots & 713.0 & 213.0 & 776.2 & 578.7 & 63.2 & 365.7 & 2283.0 & \dots & \dots & \dots & \dots & 0.2\\
\enddata
\tablecomments{{\bf Note to reviewer: Some of the numbers in this table are odd, I am researching.}}
\end{deluxetable}




The basic steps in the verification process are:
\begin{enumerate}
\item{Step1: Perform a PSA$\times$MIRA \tacq{IMAGE} {\bf with} a separate WCA lamp image, preferably in TT mode.
	\begin{enumerate}
		\item{If the PSA$\times$MIRA \tacq{IMAGE} was taken as part of an FGS-to-SI alignment program, then use this information to estimate the
		accuracy of the NUV SIAF entry by comparing the slew from the \tacq{IMAGE} to the known offset inferred from evaluation of the FGS-to-SI program data (from Colin Cox).}
		\item{Measure the [AD,XD] median of the lamp image (as done in \texttt{LTAIMCAL}), and the center of the target (in the same image) using both the \texttt{LTAIMAGE}
		9$\times$9 checkbox + flux-weighted centroid algorithm,
		and a 2D Gaussian fitting profile.}
	\end{enumerate}
	}
\item {Step 2}
\item {Step 3}
\item {Step 4}
\item {Step 5}
\end{enumerate}

These results can be combined to show the measured offsets of PSA+MIRB, BOA+MIRA, and BOA+MIRB when compared to the initial PSA+MIRA \tacq{IMAGE} of Visit `A2' of \pid{14035}. These results are shown in Table~\ref{tab:ai}.
Combined offsets from PSA+MIRA are provided in both NUV pixels (p) and in arcseconds (\arcsec).
\clearpage
The results of \pid{13972} and \pid{14035} show that, for \tacq{IMAGE}s :
\footnotesize
\begin{itemize}
\item PSA+MIRA is aligned with PSA+MIRB to [AD, XD] $\le$ [0.022, 0.007]\arcsec\ (14035, Visit `A2')
\item PSA+MIRB is aligned with BOA+MIRA to [AD, XD] $\le$ [0.023, 0.100]\arcsec\ (13972, Visit `01')
\item BOA+MIRA is aligned with BOA+MIRB to [AD, XD] $\le$ [0.022, 0.024]\arcsec\ (13972, Visit `02')
\end{itemize}

Discuss PR\#81834 : COS ACQ/IMAGE WCA2SCI[X,Y] not calculated properly

\begin{deluxetable}{rrrrrr}
\tabcolsep 8 pt
%\tabletypesize{\footnotesize}
\tablecolumns{6}
%\tablewidth{0 pt}
\tablecaption{\tacq{IMAGE} WCA-to-SA offsets from PSA+MIRA\label{tab:ai}}
\tablehead{\colhead{Aperture}&\colhead{MIRROR}&\colhead{AD Offset} & \colhead{XD Offset} & \colhead{AD Offset}& \colhead{XD Offset}\\
\colhead{}&\colhead{}&\colhead{(\arcsec)} & \colhead{(\arcsec)} & \colhead{(p)} & \colhead{(p)}\\
}
\startdata
\hline
\multicolumn{6}{c}{C21}\\
\hline
\hline
\multicolumn{6}{c}{C22}\\
\hline
\hline
\multicolumn{6}{c}{C23}\\
\hline
\hline
\multicolumn{6}{c}{C24}\\
\hline
PSA & B & 0.021 &-0.049 & 0.298 & 0.893\\
BOA & A & 0.010 & 0.060 & 0.425 & 2.550\\
BOA & B & 0.036 & 0.070 & 1.530 & 2.975 \\
\hline
\enddata
\end{deluxetable}

\begin{deluxetable}{lclcccr}
%\tablewidth{0pt}
\tabcolsep 6pt
\tablecolumns{7}
%\tabletypesize{\footnotesize}
\tablecaption{COS TA \tacq{IMAGE} Monitoring Results Summary\label{tab:airesults}}
\tablehead{
\colhead{\tacq{}} & \colhead{COS} & \colhead{Optical} & \colhead{Direction} & \colhead{Measured Offset\tablenotemark{a}} & \colhead{Requirement} & \colhead{Goal}\\
\colhead{Mode} & \colhead{Channel} & \colhead{Configuration} & \colhead{AD or XD} & \colhead{(mas)} & \colhead{(mas)} & \colhead{(mas)}
}

\startdata
\hline
\multicolumn{7}{c}{C21}\\
\hline

\hline
\multicolumn{7}{c}{C22}\\
\hline

\hline
\multicolumn{7}{c}{C23}\\
\hline

\hline
\multicolumn{7}{c}{C24}\\
\hline
IMAGE	&	NUV	&	PSA+MIRA	&	AD	&	20$\pm$14	&	41--105	&	40\\
IMAGE	&	NUV	&	PSA+MIRB	&	AD	&	10$\pm$14	&	41--105	&	40\\
IMAGE	&	NUV	&	BOA+MIRA	&	AD	&	20$\pm$14	&	41--105	&	40\\
IMAGE	&	NUV	&	BOA+MIRB	&	AD	&	15$\pm$14	&	41--105	&	40\\
\hline
IMAGE	&	NUV	&	PSA+MIRA	&	XD	&	75$\pm$14	&	300	&	100\\
IMAGE	&	NUV	&	PSA+MIRB	&	XD	&	20$\pm$14	&	300	&	100\\
IMAGE	&	NUV	&	BOA+MIRA	&	XD	&	95$\pm$14	&	300	&	100\\
IMAGE	&	NUV	&	BOA+MIRB	&	XD	&	12$\pm$14	&	300	&	100\\
\hline
PEAKXD	&	NUV	&	G185M	&	XD	&	 70$\pm$17	&	300	&	100\\
PEAKXD	&	NUV	&	G225M	&	XD	&	 60$\pm$17	&	300	&	100\\
PEAKXD	&	NUV	&	G285M	&	XD	&	 20$\pm$17	&	300	&	100\\
PEAKXD	&	NUV	&	G230L	&	XD	&	 20$\pm$17	&	300	&	100\\
PEAKXD	&	FUVA	&	G130M	&	XD	&	-30$\pm$71	&	300	&	100\\
PEAKXD	&	FUVA	&	G160M	&	XD	&	-20$\pm$71	&	300	&	100\\
PEAKXD	&	FUVA	&	G140L	&	XD	&	-170$\pm$71	&	300	&	100\\
\hline
\enddata
\tablenotetext{a}{The quoted error bars are associated with a 0.5 uncertainty when measuring the integer WCA coordinate,
and 1/3 of an NUV pixel when using the \tacq{IMAGE}~checkbox centering algorithm. Added in quadrature, the approximate
\tacq{IMAGE}~measurement error is $\approx 0.6$ NUV pixels, or 14 (mas).
Each \tacq{PEAKXD}~ WCA-to-SA measurement contains an error estimate of $\sqrt2 * 0.5 $ times the plate scale of the detector in use
(one half pixel or digital-element uncertainty for each measurement of an integer quantity).
For the NUV channel, this is 23.5 (mas)/p or $\sqrt2 * 0.5 * 23.5 = 17$ (mas).
For the FUV channel, this is $\approx \sqrt2 * 0.5 * 100 \approx 71$ (mas).}
\end{deluxetable}
" directly below
this section in the main .tex file for this ISR (cos\_tamon\_isr2018.tex).
The tables that I think are complete are included in the file ``NimVerT.tex''.}
% $Id: NimVer.tex,v 1.5 2018/03/30 20:22:12 penton Exp $

\subsection{Verifying the \tacq{IMAGE} WCA-to-SA Offsets.}\label{subsec:WCA2SAVER}

The verification of the \tacq{IMAGE} WCA-to-SA (PSA or BOA) offsets is a multi-stage bootstrap
process similar to the one used to measure the initial offsets in the SMOV enabling program (\pid{11471}, COS FUV Target Acquisition Algorithm Verification).

Each visit of each cycles monitoring program directly compares two \tacq{IMAGE} combinations.
We can bootstrap these back to PSA$\times$MIRA to test the co-alignment of all four combinations.
We call this the `baseline` bootstrapping, the results of which are shown in Table~\ref{tab:basebootstrap}
and discussed in \S~\ref{subsec:basebootstrap}

These measurements have certain limitations, so as the $\pm 0.5p$ measurement uncertainity in both directions
when measuring the WCA centroid. The WCA lamp exposures of each cycles program assist in removing
these limitations from the bootstrapping. The PSA \tacq{IMAGE} visits level the shutter open when
taking these WCA lamp images so that a co-eval target+lamp TT image is acquired. This allows a direct
calculation of the WCA-to-PSA offset using any desired centroiding algoritm. The BOA \tacq{IMAGE} visits
take sequential lamp and BOA target images to measure the WCA-to-BOA offsets, but these are not co-eval,
and the aperture has been moved between the exposures, which often causes a $\pm$ one step offset.
Fortunately, this offset can be tracked with the telemetry keywords and accounted for.


\subsection{Baseline Bootstrap of \tacq{IMAGE modes}}\label{subsec:basebootstrap}

The baseline bootstrapping data is given in Table~\ref{tab:tamonbasicnimverT}.

% tamon_basic_nimver
%
% RCSID="$Id$"
%
\begin{deluxetable}{llrccccccrrrrr}
\tablecolumns{14}
\tabcolsep 3pt
\tablecaption{C17--C20 PSA \tacq{IMAGE} Co-alignment Measurements\label{tab:basebootstrap}}
\tabletypesize{\scriptsize}
\tablehead{
\colhead{\textit{PROP}}&\colhead{\textit{ROOT}}&\colhead{Configuration}  &
\multicolumn{2}{c}{WCA-Msrd\tablenotemark{a}}  &
\multicolumn{2}{c}{PSA-Msrd}  &
\multicolumn{2}{c}{PSA-Centered}  &
\multicolumn{2}{c}{SA-to-WCA}     &
\multicolumn{2}{c}{TA Centering}  &
\colhead{\textit{DATE-}}\\

\colhead{\textit{OSID}}& \colhead{\textit{NAME}} & \colhead{\textit{APERTURE}}&
\colhead{\textit{MXCR}}&\colhead{\textit{MYCR}}&
\colhead{\textit{ACQ}}&\colhead{\textit{ACQ}}&
\colhead{\textit{ACQ}}&\colhead{\textit{ACQ}}&
\colhead{} &\colhead{}&
\colhead{\textit{ACQ}}&\colhead{\textit{ACQ}}&
\colhead{\textit{OBS}}\\

\colhead{}&\colhead{}&\colhead{$\times$}&
\colhead{\textit{LAMP}}&\colhead{\textit{LAMP}}&
\colhead{\textit{CENTX}}&\colhead{\textit{CENTY}}&
\colhead{\textit{PREFX}}&\colhead{\textit{PREFY}}&
\colhead{} & \colhead{} &
\colhead{\textit{SLEWX}}&\colhead{\textit{SLEWY}}&
\colhead{}\\

\colhead{(PID)}&\colhead{}&\colhead{\textit{OPT\_ELEM}}&
\colhead{AD} &\colhead{XD\tablenotemark{b}}&
\colhead{AD} &\colhead{XD\tablenotemark{b}}&
\colhead{AD} &\colhead{XD\tablenotemark{b}}&
\colhead{AD} &\colhead{XD\tablenotemark{b}}&
\colhead{AD} &\colhead{XD\tablenotemark{b}}&
\colhead{}\\

\colhead{(1)} & \colhead{(2)} & \colhead{(3)}&
\colhead{(4)} & \colhead{(5)} & \colhead{(6)} & \colhead{(7)} &
\colhead{(8)} & \colhead{(9)} & \colhead{(10)} & \colhead{(11)} &
\colhead{(12)} & \colhead{(13)} & \colhead{(14)}
}
\startdata
\toprule
\multicolumn{14}{c}{C17} \\
\midrule
\pid{11878} &lbcla3s3q&PSA$\times$MIRA&555&653&509.3&282.7&509.7&280.3& 45.3&372.7&  -0.010&   0.056&2010-11-05 \\
\pid{11878} &lbcla3s7q&PSA$\times$MIRB&342&813&296.4&439.8&297.0&438.9& 45.0&374.1&  -0.015&   0.021&2010-11-05 \\
\midrule
\multicolumn{14}{c}{C18} \\
\midrule
\pid{12399} &lbm7a2ahq&PSA$\times$MIRA&529&653&475.2&279.5&483.7&280.3& 45.3&372.7&  -0.200&  -0.019&2011-09-12 \\
\pid{12399} &lbm7a2ajq&PSA$\times$MIRB&317&813&271.2&439.3&272.0&438.9& 45.0&374.1&  -0.018&   0.008&2011-09-12 \\
\midrule
\multicolumn{14}{c}{C19} \\
\midrule
\pid{12781} &lbx1a2ffq&PSA$\times$MIRA&503&650&450.0&280.6&457.7&277.3& 45.3&372.7&  -0.183&   0.078&2012-09-24 \\
\pid{12781} &lbx1a2fhq&PSA$\times$MIRB&296&811&249.8&436.0&251.0&436.9& 45.0&374.1&  -0.028&  -0.021&2012-09-24 \\
\midrule
\multicolumn{14}{c}{C20} \\
\midrule
\pid{13171} &lc6ka2imq&PSA$\times$MIRA&508&650&459.7&284.0&462.7&277.3& 45.3&372.7&  -0.070&   0.158&2013-09-01 \\
\pid{13171} &lc6ka2ioq&PSA$\times$MIRB&304&811&258.2&436.5&259.0&436.9& 45.0&374.1&  -0.019&  -0.009&2013-09-01 \\
\midrule
\bottomrule
\enddata
\tablenotetext{a}{Non-repeatability of the OSM and aperture mechanisms, along with environmental factors, result in
lamp center offsets of up to 3~p in AD and to 52~p in XD in these exposures. }
%\tablenotetext{b}{BOA \tacq{IMAGE}s move the aperture in the XD direction to obtain the WCA lamp image. Occasionally, the aperture mechanism misses the desired location by $\pm 1 $ \textit{APERXPOS} step of $\sim0.05\arcsec$.}
%\tablenotetext{c}{These PSA$\times$MIRA \tacq{IMAGE}s were part of the FGS-to-SI programs and do {\bf NOT} have a proceeding TA. The TA centering adjustments presented here are to be compared to the FGS-to-SI post processing results presented in Table~\ref{tab:fgs2siInit}.}
\tablecomments{If the table caption is in the \texttt{Courier} font, this value was taken directly from the indicated \textsc{\_rawacq.fits} header keyword. In DETector coordinates, +AD is -Y$_{DET}$, +XD is -X$_{DET}$, in USER coordinates, +AD is +Y$_{USER}$, +XD is +X$_{USER}$.}
\end{deluxetable}

\begin{deluxetable}{llrccccccrrrrr}
\tablecolumns{14}
\tabcolsep 3pt
\tablecaption{C21--24 \tacq{IMAGE} Bootstrapping Results\label{tab:basebootstrap}}
\tabletypesize{\scriptsize}
\tablehead{
\colhead{\textit{PROP}}&\colhead{\textit{ROOT}}&\colhead{Configuration}  &
\multicolumn{2}{c}{WCA-Msrd\tablenotemark{a}}  &
\multicolumn{2}{c}{SA-Msrd\tablenotemark{b}}  &
\multicolumn{2}{c}{SA-Centered}  &
\multicolumn{2}{c}{SA-to-WCA}     &
\multicolumn{2}{c}{TA Centering}  &
\colhead{\textit{DATE-}}\\

\colhead{\textit{OSID}}& \colhead{\textit{NAME}} & \colhead{\textit{APERTURE}}&
\colhead{\textit{MXCR}}&\colhead{\textit{MYCR}}&
\colhead{\textit{ACQ}}&\colhead{\textit{ACQ}}&
\colhead{\textit{ACQ}}&\colhead{\textit{ACQ}}&
\colhead{} &\colhead{}&
\colhead{\textit{ACQ}}&\colhead{\textit{ACQ}}&
\colhead{\textit{OBS}}\\

\colhead{}&\colhead{}&\colhead{$\times$}&
\colhead{\textit{LAMP}}&\colhead{\textit{LAMP}}&
\colhead{\textit{CENTX}}&\colhead{\textit{CENTY}}&
\colhead{\textit{PREFX}}&\colhead{\textit{PREFY}}&
\colhead{} & \colhead{} &
\colhead{\textit{SLEWX}}&\colhead{\textit{SLEWY}}&
\colhead{}\\

\colhead{(PID)}&\colhead{}&\colhead{\textit{OPT\_ELEM}}&
\colhead{AD} &\colhead{XD}&
\colhead{AD} &\colhead{XD}&
\colhead{AD} &\colhead{XD}&
\colhead{AD} &\colhead{XD}&
\colhead{AD} &\colhead{XD}&
\colhead{}\\

\colhead{(1)} & \colhead{(2)} & \colhead{(3)}&
\colhead{(4)} & \colhead{(5)} & \colhead{(6)} & \colhead{(7)} &
\colhead{(8)} & \colhead{(9)} & \colhead{(10)} & \colhead{(11)} &
\colhead{(12)} & \colhead{(13)} & \colhead{(14)}
}
\startdata
\toprule
\multicolumn{14}{c}{C21} \\
\midrule
\pid{13616} &lci4a2e3q&PSA$\times$MIRA&517&650&471.7&282.9&471.7&277.3& 45.3&372.7&   0.001&   0.133&2014-10-27 \\
\pid{13616} &lci4a2e5q&PSA$\times$MIRB&305&809&259.7&436.1&259.0&435.0& 46.0&374.0&   0.016&   0.027&2014-10-27 \\
\midrule
\pid{13526} &lcgq03dbq&PSA$\times$MIRA&566&648&527.7&268.2&520.7&275.3& 45.3&372.7&   0.166&  -0.168&2014-10-06 \\
\pid{13526} &lcgq03dlq&PSA$\times$MIRB&357&808&310.8&434.2&312.0&433.9& 45.0&374.1&  -0.029&   0.006&2014-10-06 \\
\pid{13526} &lcgq03dtq&PSA$\times$MIRA&574&649&530.1&275.2&528.7&276.3& 45.3&372.7&   0.032&  -0.026&2014-10-06 \\
\pid{13526} &lcgq01q5q&PSA$\times$MIRB&306&809&222.4&447.7&260.0&435.0& 46.0&374.0&  -0.888&   0.298&2014-11-19 \\
\pid{13526} &lcgq01qdq&BOA$\times$MIRA&520&651&472.9&283.2&474.5&282.6& 45.5&368.4&  -0.038&   0.013&2014-11-19 \\
\pid{13526} &lcgq01qjq&PSA$\times$MIRB&305&811&260.2&433.7&259.0&437.0& 46.0&374.0&   0.027&  -0.078&2014-11-19 \\
\pid{13526} &lcgq02hmq&BOA$\times$MIRA&495&649&452.0&293.6&449.5&280.6& 45.5&368.4&   0.058&   0.305&2014-11-17 \\
\pid{13526} &lcgq02huq&BOA$\times$MIRB&285&811&237.9&440.3&238.5&444.8& 46.5&366.2&  -0.015&  -0.105&2014-11-17 \\
\pid{13526} &lcgq02i0q&BOA$\times$MIRA&501&651&455.7&285.3&455.5&282.6& 45.5&368.4&   0.006&   0.063&2014-11-17 \\
\midrule
\multicolumn{14}{c}{C22} \\
\midrule
\pid{14035} &lcsla2bhq&PSA$\times$MIRA&505&653&458.8&284.8&459.7&280.3& 45.3&372.7&  -0.020&   0.105&2015-10-02 \\
\pid{14035} &lcsla2bjq&PSA$\times$MIRB&293&813&247.9&439.3&247.0&439.0& 46.0&374.0&   0.022&   0.007&2015-10-02 \\
\pid{13972} &lcri01fzq&PSA$\times$MIRB&302&813&264.0&427.4&256.0&439.0& 46.0&374.0&   0.189&  -0.273&2015-10-06 \\
\pid{13972} &lcri01g7q&BOA$\times$MIRA&517&651&471.0&287.2&471.5&282.6& 45.5&368.4&  -0.011&   0.109&2015-10-06 \\
\pid{13972} &lcri01geq&PSA$\times$MIRB&300&812&255.5&434.1&254.0&438.0& 46.0&374.0&   0.036&  -0.092&2015-10-06 \\
\pid{13972} &lcri02h8q&BOA$\times$MIRA&499&654&462.0&272.2&453.5&285.6& 45.5&368.4&   0.201&  -0.315&2015-10-06 \\
\pid{13972} &lcri02hgq&BOA$\times$MIRB&286&810&240.6&444.2&239.5&443.8& 46.5&366.2&   0.026&   0.010&2015-10-06 \\
\pid{13972} &lcri02hmq&BOA$\times$MIRA&504&651&457.7&284.2&458.5&282.6& 45.5&368.4&  -0.018&   0.038&2015-10-06 \\
\midrule
\multicolumn{14}{c}{C23} \\
\midrule
\pid{14452} &ld3la2ojq&PSA$\times$MIRA&527&654&481.0&284.5&481.7&281.3& 45.3&372.7&  -0.017&   0.075&2016-10-02 \\
\pid{14452} &ld3la2onq&PSA$\times$MIRB&306&813&259.9&439.9&260.0&439.0& 46.0&374.0&  -0.002&   0.021&2016-10-02 \\
\pid{14440} &ld3701gtq&PSA$\times$MIRB&297&813&246.6&442.7&251.0&439.0& 46.0&374.0&  -0.104&   0.088&2016-10-18 \\
\pid{14440} &ld3701h1q&BOA$\times$MIRA&518&651&471.7&287.1&472.5&282.6& 45.5&368.4&  -0.018&   0.105&2016-10-18 \\
\pid{14440} &ld3701h7q&PSA$\times$MIRB&295&811&249.7&434.0&249.0&437.0& 46.0&374.0&   0.016&  -0.071&2016-10-18 \\
\pid{14440} &ld3702mzq&BOA$\times$MIRA&509&654&480.0&299.3&463.5&285.6& 45.5&368.4&   0.390&   0.322&2016-10-19 \\
\pid{14440} &ld3702n9q&BOA$\times$MIRB&295&811&248.5&446.1&248.5&444.8& 46.5&366.2&   0.001&   0.030&2016-10-19 \\
\pid{14440} &ld3702nhq&BOA$\times$MIRA&516&651&470.0&285.0&470.5&282.6& 45.5&368.4&  -0.012&   0.057&2016-10-19 \\
\midrule
\multicolumn{14}{c}{C24} \\
\midrule
\pid{14857} &ldozpbf5q&PSA$\times$MIRA&515&654&467.3&266.2&469.7&281.3& 45.3&372.7&  -0.058&  -0.355&2017-09-10 \\
\pid{14857} &ldozpbfbq&PSA$\times$MIRB&289&813&243.8&440.0&243.0&439.0& 46.0&374.0&   0.020&   0.023&2017-09-10 \\
\pid{14857} &ldozpbfhq&PSA$\times$MIRA&510&653&463.9&279.8&464.7&280.3& 45.3&372.7&  -0.020&  -0.011&2017-09-10 \\
\pid{14857} &ldozbadhq&PSA$\times$MIRB&299&813&237.5&397.9&253.0&439.0& 46.0&374.0&  -0.366&  -0.967&2017-09-04 \\
\pid{14857} &ldozbadpq&BOA$\times$MIRA&524&651&477.9&287.4&478.5&282.6& 45.5&368.4&  -0.013&   0.113&2017-09-04 \\
\pid{14857} &ldozbadvq&PSA$\times$MIRB&298&811&253.2&434.5&252.0&437.0& 46.0&374.0&   0.029&  -0.059&2017-09-04 \\
\pid{14857} &ldozbbleq&BOA$\times$MIRA&518&653&484.8&289.5&472.5&284.6& 45.5&368.4&   0.290&   0.114&2017-09-06 \\
\pid{14857} &ldozbblmq&BOA$\times$MIRB&293&811&246.7&444.3&246.5&444.8& 46.5&366.2&   0.006&  -0.011&2017-09-06 \\
\pid{14857} &ldozbblsq&BOA$\times$MIRA&514&651&467.9&285.1&468.5&282.6& 45.5&368.4&  -0.015&   0.060&2017-09-06 \\
\bottomrule
\enddata
\tablenotetext{a}{Non-repeatability of the OSM and aperture mechanisms, along with environmental factors, result in
lamp center offsets of up to 6~p in AD and $> 50$~p in XD in these exposures.}
\tablenotetext{b}{BOA \tacq{IMAGE}s move the aperture in the XD direction to obtain the WCA lamp image. Occasionally, the aperture mechanism misses the desired location by $\pm 1$ \textit{APERYPOS} step of $\sim0.05\arcsec$.}
%\tablenotetext{c}{These PSA$\times$MIRA \tacq{IMAGE}s were part of the FGS-to-SI programs and do not have a proceeding \tacq{IMAGE}. The TA centerings presented here are to be compared to the FGS-to-SI post processing results presented in Table~\ref{tab:fgs2siInit}.}
\tablecomments{If the table caption is in the \textit{ITALICS}, this value was taken directly from the indicated \textsc{\_rawacq.fits} header keyword. Columns 4-11 are in units of NUV pixels (p). Columns 12 \& 13 are in arcseconds (\arcsec). In DETector coordinates, +AD is -Y$_{DET}$, +XD is -X$_{DET}$, in USER coordinates, +AD is +Y$_{USER}$, +XD is +X$_{USER}$.}
\end{deluxetable}

\begin{deluxetable}{llrccccccrrrrr}
\tablecolumns{14}
\tabcolsep 4pt
\tablecaption{\tacq{IMAGE} Bootstrapping Measurements Sorted by Configuration\label{tab:bootstrapAligned}}
\tabletypesize{\scriptsize}
\tablehead{
\colhead{\textit{PROP}}&\colhead{\textit{ROOT}}&
\colhead{HST}  &
\multicolumn{2}{c}{WCA-Msrd\tablenotemark{a}}  &
\multicolumn{2}{c}{SA-Msrd\tablenotemark{b}}  &
\multicolumn{2}{c}{SA-Center}  &
\multicolumn{2}{c}{SA-to-WCA}     &
\multicolumn{2}{c}{TA Centering}&\colhead{\textit{DATE}}\\

\colhead{\textit{OSID}}& \colhead{\textit{NAME}} & \colhead{Cycle}&
\colhead{\textit{LAMP}}&\colhead{\textit{LAMP}}&
\colhead{\textit{ACQ}}&\colhead{\textit{ACQ}}&
\colhead{\textit{ACQ}}&\colhead{\textit{ACQ}}&
\colhead{}& \colhead{} &\colhead{\textit{ACQ}}&\colhead{\textit{ACQ}}& \colhead{\textit{OBS}} \\

\colhead{}& \colhead{} & \colhead{}&
\colhead{\textit{MXCR}}&\colhead{\textit{MYCR}}&
\colhead{\textit{MSRDX}}&\colhead{\textit{MSRDY}}&
\colhead{\textit{PREFX}}&\colhead{\textit{PREFY}}&
\colhead{}& \colhead{} &\colhead{\textit{SLEWX}}&\colhead{\textit{SLEWY}}& \\
\colhead{(PID)}  &     \colhead{}     & \colhead{} &
\colhead{AD} &\colhead{XD}&
\colhead{AD} &\colhead{XD}&
\colhead{AD} &\colhead{XD}&
\colhead{AD} &\colhead{XD}& \colhead{} \\
\colhead{(1)}  &\colhead{(2)} & \colhead{(3)}&\colhead{(4)} &
\colhead{(5)}  &\colhead{(6)} & \colhead{(7)}&\colhead{(8)} &
\colhead{(9)}  &\colhead{(10)} & \colhead{(11)} &\colhead{(12)} &
\colhead{(13)} &\colhead{(14)}
}
\startdata
\toprule
\multicolumn{14}{c}{PSA$\times$MIRRORA}\\
\midrule
\pid{11878} &lbcla3s3q&C17&555&653&509.3&282.7&509.7&280.3& 45.3&372.7&  -0.010&   0.056&2010-11-05 \\
\pid{12399} &lbm7a2ahq&C18&529&653&475.2&279.5&483.7&280.3& 45.3&372.7&  -0.200&  -0.019&2011-09-12 \\
\pid{12781} &lbx1a2ffq&C19&503&650&450.0&280.6&457.7&277.3& 45.3&372.7&  -0.183&   0.078&2012-09-24 \\
\pid{13171} &lc6ka2imq&C20&508&650&459.7&284.0&462.7&277.3& 45.3&372.7&  -0.070&   0.158&2013-09-01 \\
\pid{13616} &lci4a2e3q&C21&517&650&471.7&282.9&471.7&277.3& 45.3&372.7&   0.001&   0.133&2014-10-27 \\
\pid{14035} &lcsla2bhq&C22&505&653&458.8&284.8&459.7&280.3& 45.3&372.7&  -0.020&   0.105&2015-10-02 \\
\pid{14452} &ld3la2ojq&C23&527&654&481.0&284.5&481.7&281.3& 45.3&372.7&  -0.017&   0.075&2016-10-02 \\
\pid{14857} &ldozpbf5q&C24&515&654&467.3&266.2&469.7&281.3& 45.3&372.7&  -0.058&  -0.355&2017-09-10 \\
\pid{14857} &ldozpbfhq&C24&510&653&463.9&279.8&464.7&280.3& 45.3&372.7&  -0.020&  -0.011&2017-09-10 \\
\midrule
\multicolumn{14}{c}{PSA$\times$MIRRORB\tablenotemark{c}}\\
\midrule
\pid{11878} &lbcla3s7q&C17&342&813&296.4&439.8&297.0&438.9& 45.0&374.1&  -0.015&   0.021&2010-11-05 \\
\pid{12399} &lbm7a2ajq&C18&317&813&271.2&439.3&272.0&438.9& 45.0&374.1&  -0.018&   0.008&2011-09-12 \\
\pid{12781} &lbx1a2fhq&C19&296&811&249.8&436.0&251.0&436.9& 45.0&374.1&  -0.028&  -0.021&2012-09-24 \\
\pid{13171} &lc6ka2ioq&C20&304&811&258.2&436.5&259.0&436.9& 45.0&374.1&  -0.019&  -0.009&2013-09-01 \\
\hline
\pid{13616} &lci4a2e5q&C21&305&809&259.7&436.1&259.0&435.0& 46.0&374.0&   0.016&   0.027&2014-10-27 \\
\pid{14035} &lcsla2bjq&C22&293&813&247.9&439.3&247.0&439.0& 46.0&374.0&   0.022&   0.007&2015-10-02 \\
\pid{13972} &lcri01fzq&C22&302&813&264.0&427.4&256.0&439.0& 46.0&374.0&   0.189&  -0.273&2015-10-06 \\
\pid{13972} &lcri01geq&C22&300&812&255.5&434.1&254.0&438.0& 46.0&374.0&   0.036&  -0.092&2015-10-06 \\
\pid{14452} &ld3la2onq&C23&306&813&259.9&439.9&260.0&439.0& 46.0&374.0&  -0.002&   0.021&2016-10-02 \\
\pid{14440} &ld3701gtq&C23&297&813&246.6&442.7&251.0&439.0& 46.0&374.0&  -0.104&   0.088&2016-10-18 \\
\pid{14440} &ld3701h7q&C23&295&811&249.7&434.0&249.0&437.0& 46.0&374.0&   0.016&  -0.071&2016-10-18 \\
\pid{14857} &ldozbadhq&C24&299&813&237.5&397.9&253.0&439.0& 46.0&374.0&  -0.366&  -0.967&2017-09-04 \\
\pid{14857} &ldozbadvq&C24&298&811&253.2&434.5&252.0&437.0& 46.0&374.0&   0.029&  -0.059&2017-09-04 \\
\pid{14857} &ldozpbfbq&C24&289&813&243.8&440.0&243.0&439.0& 46.0&374.0&   0.020&   0.023&2017-09-10 \\
\midrule
\multicolumn{14}{c}{BOA$\times$MIRRORA}\\
\midrule
\pid{13526} &lcgq01qdq&C21&520&651&472.9&283.2&474.5&282.6& 45.5&368.4&  -0.038&   0.013&2014-11-19 \\
\pid{13972} &lcri01g7q&C22&517&651&471.0&287.2&471.5&282.6& 45.5&368.4&  -0.011&   0.109&2015-10-06 \\
\pid{13972} &lcri02h8q&C22&499&654&462.0&272.2&453.5&285.6& 45.5&368.4&   0.201&  -0.315&2015-10-06 \\
\pid{13972} &lcri02hmq&C22&504&651&457.7&284.2&458.5&282.6& 45.5&368.4&  -0.018&   0.038&2015-10-06 \\
\pid{14440} &ld3701h1q&C23&518&651&471.7&287.1&472.5&282.6& 45.5&368.4&  -0.018&   0.105&2016-10-18 \\
\pid{14440} &ld3702mzq&C23&509&654&480.0&299.3&463.5&285.6& 45.5&368.4&   0.390&   0.322&2016-10-19 \\
\pid{14440} &ld3702nhq&C23&516&651&470.0&285.0&470.5&282.6& 45.5&368.4&  -0.012&   0.057&2016-10-19 \\
\pid{14857} &ldozbadpq&C24&524&651&477.9&287.4&478.5&282.6& 45.5&368.4&  -0.013&   0.113&2017-09-04 \\
\pid{14857} &ldozbbleq&C24&518&653&484.8&289.5&472.5&284.6& 45.5&368.4&   0.290&   0.114&2017-09-06 \\
\pid{14857} &ldozbblsq&C24&514&651&467.9&285.1&468.5&282.6& 45.5&368.4&  -0.015&   0.060&2017-09-06 \\
\midrule
\multicolumn{14}{c}{BOA$\times$MIRRORB\tablenotemark{c}}\\
\midrule
\pid{13526} &lcgq02huq&C21&285&811&237.9&440.3&238.5&444.8& 46.5&366.2&  -0.015&  -0.105&2014-11-17 \\
\pid{13972} &lcri02hgq&C22&286&810&240.6&444.2&239.5&443.8& 46.5&366.2&   0.026&   0.010&2015-10-06 \\
\pid{14440} &ld3702n9q&C23&295&811&248.5&446.1&248.5&444.8& 46.5&366.2&   0.001&   0.030&2016-10-19 \\
\pid{14857} &ldozbblmq&C24&293&811&246.7&444.3&246.5&444.8& 46.5&366.2&   0.006&  -0.011&2017-09-06 \\
\bottomrule
\enddata
\tablenotetext{a}{Environmental factors, and non-repeatability of the OSM and aperture mechanisms, results in
lamp offsets of up to 6~p in AD and $ > 50$~p in XD in these exposures. }
\tablenotetext{b}{BOA \tacq{IMAGE}s move the aperture in the XD direction to obtain the WCA lamp image. Occasionally, the aperture mechanism misses the desired location by $\pm 1 $ \textit{APERXPOS} step of $\sim0.05\arcsec$.}
%\tablenotetext{c}{These PSA$\times$MIRA \tacq{IMAGE}s were part of the FGS-to-SI programs and do {\bf NOT} have a proceeding TA. The TA centering adjustments presented here are to be compared to the FGS-to-SI post processing results presented in Table~\ref{tab:fgs2siInit}.}
\tablenotetext{c}{On November 6, 2014, the MIRB \texttt{ACQ/IMAGE} lamp exposure was changed in duration and current. FSW
tables were also updated at this time (\pr{67139}).}
\tablecomments{If the table caption is in the \texttt{Courier} font, this value was taken directly from the indicated \textsc{\_rawacq.fits} header keyword. In DETector coordinates, +AD is -Y$_{DET}$, +XD is -X$_{DET}$, in USER coordinates, +AD is +Y$_{USER}$, +XD is +X$_{USER}$.}
%           FROM   TO
%   PSA_B   3741   3740
%   BOA_B   3663   3662
%           FROM   TO
%   PSA_B   450    460
%   BOA_B   455    465
\end{deluxetable}


%\begin{deluxetable}{rrrrrrr}
%\tablecaption{Basic \tacq{IMAGE} Bootstrapping Results\label{tab:tamonbasicnimverB}}
%\tablecolumns{7}
%\tablhead{
%\colhead{ROOTNAME} & \colhead{\texttt{APERTURE}}& \colhead{\texttt{OPT\_ELEM}} & \colhead{}  & \colhead{} & \colhead{}  & \colhead{}\\
%\colhead{} & \colhead{}\colhead{} & \colhead{AD (Y)}  & \colhead{XD (X)} & \colhead{AD (Y)}  & \colhead{XD (X)}
%}
%\startdata
%\hline
%\multicolumn{7}{c}{C21 (\pid{})}\\
%\hline
%l	&	PSA &	MIRB	&	&	&	&	& \\
%l	&	BOA &	MIRA	&	&	&	&	& \\
%l	&	BOA &	MIRB	&	&	&	&	& \\
%\hline
%\multicolumn{7}{c}{C22 (\pid{})}\\
%\hline
%l	&	PSA &	MIRB	&	&	&	&	& \\
%l	&	BOA &	MIRA	&	&	&	&	& \\
%l	&	BOA &	MIRB	&	&	&	&	& \\
%\hline
%\multicolumn{7}{c}{C23 (\pid{})}\\
%\hline
%l	&	PSA &	MIRB	&	&	&	&	& \\
%l	&	BOA &	MIRA	&	&	&	&	& \\
%l	&	BOA &	MIRB	&	&	&	&	& \\
%\hline
%\multicolumn{7}{c}{C24 (\pid{})}\\
%\hline
%l	&	PSA &	MIRB	&	&	&	&	& \\
%l	&	BOA &	MIRA	&	&	&	&	& \\
%l	&	BOA &	MIRB	&	&	&	&	& \\
%\hline
%\enddata
%\tablecomments{+AD is -Y detector, +XD is -X detector.}
%\end{deluxetable}


% $Id: tamon_output.tex,v 1.6 2018/03/30 20:22:12 penton Exp $

\begin{deluxetable}{rrrrrrrrrrrrrrrrrrr}
\tabcolsep 2pt
\tabletypesize{\tiny}
\tablecolumns{19}
\tablecaption{COS TA Monitor \texttt{ACQ/IMAGE} Data}\label{tab:Imagedata}
\tablehead{
\colhead{\textit{ROOTNAME}}&\colhead{\textit{EXPTYPE}}&\colhead{\textit{OPT\_ELEM}}&\colhead{LAMP}&\colhead{Current}&\colhead{Target ET}&\colhead{Lamp ET}&\colhead{WCA}&\colhead{WCA}&\colhead{SA}&\colhead{SA}&\colhead{WtP}&\colhead{WtP}&\colhead{Lamp}&\colhead{Lamp}&\colhead{WCA}&\colhead{Lamp}&\colhead{Lamp}&\colhead{Target}\\
\colhead{}&\colhead{}&\colhead{ }&\colhead{}&\colhead{}&\colhead{(s)}&\colhead{(s)}&\colhead{AD}&\colhead{XD}&\colhead{AD}&\colhead{XD}&\colhead{AD}&\colhead{XD}&\colhead{counts}&\colhead{cps}&\colhead{bck}&\colhead{CPS}&\colhead{BP}&\colhead{BP}\\
\colhead{(1)}&\colhead{(2)} &
\colhead{(3)}&\colhead{(4)} &
\colhead{(5)}&\colhead{(6)} &
\colhead{(7)}&\colhead{(8)} &
\colhead{(9)}&\colhead{(10)} &
\colhead{(11)} &\colhead{(12)} &
\colhead{(13)}&\colhead{(14)} &
\colhead{(15)}&\colhead{(16)} &
\colhead{(17)}&\colhead{(18)} &
\colhead{(19)}
}

\startdata
lcgq01q7q & EXT/SCI & MIRB & P2 & MED & 16 & 16 & 717.0 & 214.0 & 763.1 & 588.9 & 46.1 & 374.9 & 4890.0 & 305.6 & 167 & 305.6 & 4.4 & 26.7\\
lcgq01q9q & EXT/SCI & MIRA & P2 & MED & 150 & 150 & 479.0 & 370.0 & 550.3 & 739.9 & 71.3 & 369.9 & 1718.0 & \dots & \dots & \dots & \dots & 0.2\\

lcgq02hoq & WAVECAL & MIRA & P2 & LOW & 7 & \dots & 529.0 & 372.0 & 891.6 & 635.6 & 362.6 & 263.6 & 2827.0 & 403.9 & 97 & 403.9 & 9.9 & 0.3\\
lcgq02hqq & EXT/SCI & MIRB & P2 & LOW & 181 & \dots & 713.0 & 211.0 & 784.4 & 582.7 & 71.4 & 371.7 & 2383.0 & \dots & \dots & \dots & \dots & 0.2\\

lcri01g1q & EXT/SCI & MIRB & P2 & MED & 12 & 12 & 722.0 & 210.0 & 767.7 & 584.2 & 45.7 & 374.2 & 3016.0 & 251.3 & 166 & 251.3 & 4.2 & 30.0\\
lcri01g3q & EXT/SCI & MIRA & P2 & MED & 150 & \dots & 474.0 & 370.0 & 552.0 & 735.7 & 78.0 & 365.7 & 1964.0 & \dots & \dots & \dots & \dots & 0.2\\

lcri02hcq & EXT/SCI & MIRB & P2 & LOW & 181 & 181 & 715.0 & 211.0 & 782.3 & 578.6 & 67.3 & 367.6 & 2406.0 & \dots & \dots & \dots & \dots & 0.2\\

ld3701gvq & EXT/SCI & MIRB & P2 & MED & 16 & 16 & 727.0 & 210.0 & 772.8 & 584.3 & 45.8 & 374.3 & 4147.0 & 259.2 & 184 & 259.2 & 4.3 & 19.0\\
ld3701gxq & EXT/SCI & MIRA & P2 & MED & 150 & 150 & 479.0 & 371.0 & 551.2 & 735.8 & 72.2 & 364.8 & 1739.0 & \dots & \dots & \dots & \dots & 0.2\\

ld3702n1q & WAVECAL & MIRA & P2 & LOW & 14 & \dots & 515.0 & 371.0 & 886.6 & 659.4 & 371.6 & 288.4 & 5589.0 & 399.2 & 167 & 399.2 & 7.7 & 0.2\\
ld3702n4q & EXT/SCI & MIRB & P2 & LOW & 183 & 183 & 723.0 & 213.0 & 774.9 & 577.6 & 51.9 & 364.6 & 2081.0 & \dots & \dots & \dots & \dots & 0.2\\

ldozbadjs & EXT/SCI & MIRB & P2 & MED & 16 & 16  & 724.0 & 210.0 & 769.8 & 583.4 & 45.8 & 373.4 & 4005.0 & 250.3 & 138 & 250.3 & 4.4 & 20.2\\
ldozbadlq & EXT/SCI & MIRA & P2 & MED & 150 & 150 & 472.0 & 371.0 & 545.1 & 735.6 & 73.1 & 364.6 & 1462.0 & \dots & \dots & \dots & \dots & 0.2\\

ldozbblgq & WAVECAL & MIRA & P2 & LOW & 14 & \dots & 507.0 & 372.0 & 748.6 & 911.9 & 241.6 & 539.9 & 5721.0 & 408.6 & 155 & 408.6 & 8.4 & 0.1\\
ldozbbliq & EXT/SCI & MIRB & P2 & LOW & 183 & \dots & 713.0 & 213.0 & 776.2 & 578.7 & 63.2 & 365.7 & 2283.0 & \dots & \dots & \dots & \dots & 0.2\\
\enddata
\tablecomments{{\bf Note to reviewer: Some of the numbers in this table are odd, I am researching.}}
\end{deluxetable}




The basic steps in the verification process are:
\begin{enumerate}
\item{Step1: Perform a PSA$\times$MIRA \tacq{IMAGE} {\bf with} a separate WCA lamp image, preferably in TT mode.
	\begin{enumerate}
		\item{If the PSA$\times$MIRA \tacq{IMAGE} was taken as part of an FGS-to-SI alignment program, then use this information to estimate the
		accuracy of the NUV SIAF entry by comparing the slew from the \tacq{IMAGE} to the known offset inferred from evaluation of the FGS-to-SI program data (from Colin Cox).}
		\item{Measure the [AD,XD] median of the lamp image (as done in \texttt{LTAIMCAL}), and the center of the target (in the same image) using both the \texttt{LTAIMAGE}
		9$\times$9 checkbox + flux-weighted centroid algorithm,
		and a 2D Gaussian fitting profile.}
	\end{enumerate}
	}
\item {Step 2}
\item {Step 3}
\item {Step 4}
\item {Step 5}
\end{enumerate}

These results can be combined to show the measured offsets of PSA+MIRB, BOA+MIRA, and BOA+MIRB when compared to the initial PSA+MIRA \tacq{IMAGE} of Visit `A2' of \pid{14035}. These results are shown in Table~\ref{tab:ai}.
Combined offsets from PSA+MIRA are provided in both NUV pixels (p) and in arcseconds (\arcsec).
\clearpage
The results of \pid{13972} and \pid{14035} show that, for \tacq{IMAGE}s :
\footnotesize
\begin{itemize}
\item PSA+MIRA is aligned with PSA+MIRB to [AD, XD] $\le$ [0.022, 0.007]\arcsec\ (14035, Visit `A2')
\item PSA+MIRB is aligned with BOA+MIRA to [AD, XD] $\le$ [0.023, 0.100]\arcsec\ (13972, Visit `01')
\item BOA+MIRA is aligned with BOA+MIRB to [AD, XD] $\le$ [0.022, 0.024]\arcsec\ (13972, Visit `02')
\end{itemize}

Discuss PR\#81834 : COS ACQ/IMAGE WCA2SCI[X,Y] not calculated properly

\begin{deluxetable}{rrrrrr}
\tabcolsep 8 pt
%\tabletypesize{\footnotesize}
\tablecolumns{6}
%\tablewidth{0 pt}
\tablecaption{\tacq{IMAGE} WCA-to-SA offsets from PSA+MIRA\label{tab:ai}}
\tablehead{\colhead{Aperture}&\colhead{MIRROR}&\colhead{AD Offset} & \colhead{XD Offset} & \colhead{AD Offset}& \colhead{XD Offset}\\
\colhead{}&\colhead{}&\colhead{(\arcsec)} & \colhead{(\arcsec)} & \colhead{(p)} & \colhead{(p)}\\
}
\startdata
\hline
\multicolumn{6}{c}{C21}\\
\hline
\hline
\multicolumn{6}{c}{C22}\\
\hline
\hline
\multicolumn{6}{c}{C23}\\
\hline
\hline
\multicolumn{6}{c}{C24}\\
\hline
PSA & B & 0.021 &-0.049 & 0.298 & 0.893\\
BOA & A & 0.010 & 0.060 & 0.425 & 2.550\\
BOA & B & 0.036 & 0.070 & 1.530 & 2.975 \\
\hline
\enddata
\end{deluxetable}

\begin{deluxetable}{lclcccr}
%\tablewidth{0pt}
\tabcolsep 6pt
\tablecolumns{7}
%\tabletypesize{\footnotesize}
\tablecaption{COS TA \tacq{IMAGE} Monitoring Results Summary\label{tab:airesults}}
\tablehead{
\colhead{\tacq{}} & \colhead{COS} & \colhead{Optical} & \colhead{Direction} & \colhead{Measured Offset\tablenotemark{a}} & \colhead{Requirement} & \colhead{Goal}\\
\colhead{Mode} & \colhead{Channel} & \colhead{Configuration} & \colhead{AD or XD} & \colhead{(mas)} & \colhead{(mas)} & \colhead{(mas)}
}

\startdata
\hline
\multicolumn{7}{c}{C21}\\
\hline

\hline
\multicolumn{7}{c}{C22}\\
\hline

\hline
\multicolumn{7}{c}{C23}\\
\hline

\hline
\multicolumn{7}{c}{C24}\\
\hline
IMAGE	&	NUV	&	PSA+MIRA	&	AD	&	20$\pm$14	&	41--105	&	40\\
IMAGE	&	NUV	&	PSA+MIRB	&	AD	&	10$\pm$14	&	41--105	&	40\\
IMAGE	&	NUV	&	BOA+MIRA	&	AD	&	20$\pm$14	&	41--105	&	40\\
IMAGE	&	NUV	&	BOA+MIRB	&	AD	&	15$\pm$14	&	41--105	&	40\\
\hline
IMAGE	&	NUV	&	PSA+MIRA	&	XD	&	75$\pm$14	&	300	&	100\\
IMAGE	&	NUV	&	PSA+MIRB	&	XD	&	20$\pm$14	&	300	&	100\\
IMAGE	&	NUV	&	BOA+MIRA	&	XD	&	95$\pm$14	&	300	&	100\\
IMAGE	&	NUV	&	BOA+MIRB	&	XD	&	12$\pm$14	&	300	&	100\\
\hline
PEAKXD	&	NUV	&	G185M	&	XD	&	 70$\pm$17	&	300	&	100\\
PEAKXD	&	NUV	&	G225M	&	XD	&	 60$\pm$17	&	300	&	100\\
PEAKXD	&	NUV	&	G285M	&	XD	&	 20$\pm$17	&	300	&	100\\
PEAKXD	&	NUV	&	G230L	&	XD	&	 20$\pm$17	&	300	&	100\\
PEAKXD	&	FUVA	&	G130M	&	XD	&	-30$\pm$71	&	300	&	100\\
PEAKXD	&	FUVA	&	G160M	&	XD	&	-20$\pm$71	&	300	&	100\\
PEAKXD	&	FUVA	&	G140L	&	XD	&	-170$\pm$71	&	300	&	100\\
\hline
\enddata
\tablenotetext{a}{The quoted error bars are associated with a 0.5 uncertainty when measuring the integer WCA coordinate,
and 1/3 of an NUV pixel when using the \tacq{IMAGE}~checkbox centering algorithm. Added in quadrature, the approximate
\tacq{IMAGE}~measurement error is $\approx 0.6$ NUV pixels, or 14 (mas).
Each \tacq{PEAKXD}~ WCA-to-SA measurement contains an error estimate of $\sqrt2 * 0.5 $ times the plate scale of the detector in use
(one half pixel or digital-element uncertainty for each measurement of an integer quantity).
For the NUV channel, this is 23.5 (mas)/p or $\sqrt2 * 0.5 * 23.5 = 17$ (mas).
For the FUV channel, this is $\approx \sqrt2 * 0.5 * 100 \approx 71$ (mas).}
\end{deluxetable}

% RCSID: "$Id: NimVerT.tex,v 1.1 2018/03/30 20:22:12 penton Exp $"
\begin{deluxetable}{rrrrrrrrrrrrrrrrrrr}
\tabcolsep 0pt
\tabletypesize{\tiny}
\tablecolumns{19}
\tablecaption{COS Imaging \texttt{ACQ/IMAGE} Data}\label{tab:Imagedata}
\tablehead{
\colhead{IPPPSSOOT}&\colhead{EXP}&\colhead{OPT\_ELEM}&\colhead{LAMP}&\colhead{Current}&\colhead{Target}&\colhead{Lamp }&\colhead{WCA\_AD}&\colhead{WCA}&\colhead{SA}&\colhead{SA}&\colhead{WtP}&\colhead{WtP}&\colhead{Lamp}&\colhead{Lamp}&\colhead{WCA}&\colhead{Lamp}&\colhead{Lamp}&\colhead{Target}\\
\colhead{}&\colhead{TYPE}&\colhead{OPT\_ELEM}&\colhead{LAMP}&\colhead{Level}&\colhead{ET(s)}&\colhead{ET (s)}&\colhead{AD}&\colhead{XD}&\colhead{AD}&\colhead{XD}&\colhead{WtP(AD)}&\colhead{XD}&\colhead{counts}&\colhead{cps}&\colhead{bck}&\colhead{CPS}&\colhead{BP}&\colhead{BP}\\
}

\startdata
lcgq01q7q & EXT/SCI & MIRRORB & P2 & MED & 16 & 16 & 717.0 & 214.0 & 763.1 & 588.9 & 46.1 & 374.9 & 4890.0 & 305.6 & 167 & 305.6 & 4.4 & 26.7\\
lcgq01q9q & EXT/SCI & MIRRORA & P2 & MED & 150 & 0 & 479.0 & 370.0 & 550.3 & 739.9 & 71.3 & 369.9 & 1718.0 & \dots & 0 & \dots & \dots & 0.2\\
lcgq01qbq & WAVECAL & MIRRORA & P2 & LOW & 7 & 7 & 503.0 & 372.0 & 596.4 & 869.2 & 93.4 & 497.2 & 2964.0 & 423.4 & 61 & 423.4 & 7.7 & 0.3\\
lcgq01qfq & WAVECAL & MIRRORA & P2 & LOW & 7 & 7 & 503.0 & 372.0 & 652.2 & 793.2 & 149.2 & 421.2 & 2882.0 & 411.7 & 71 & 411.7 & 7.9 & 0.3\\
lcgq01qhq & EXT/SCI & MIRRORB & P2 & MED & 12 & 12 & 718.0 & 212.0 & 762.9 & 589.3 & 44.9 & 377.3 & 3391.0 & 282.6 & 151 & 282.6 & 3.9 & 19.9\\
lcgq02hoq & WAVECAL & MIRRORA & P2 & LOW & 7 & 7 & 529.0 & 372.0 & 891.6 & 635.6 & 362.6 & 263.6 & 2827.0 & 403.9 & 97 & 403.9 & 9.9 & 0.3\\
lcgq02hqq & EXT/SCI & MIRRORB & P2 & LOW & 181 & 0 & 713.0 & 211.0 & 784.4 & 582.7 & 71.4 & 371.7 & 2383.0 & \dots & 0 & \dots & \dots & 0.2\\
lcgq02hsq & WAVECAL & MIRRORB & P2 & MED & 12 & 12 & 738.0 & 212.0 & 898.7 & 439.2 & 160.7 & 227.2 & 3683.0 & 306.9 & 165 & 306.9 & 4.8 & 0.2\\
lcgq02hwq & WAVECAL & MIRRORB & P2 & MED & 12 & 12 & 738.0 & 213.0 & 927.8 & 656.8 & 189.8 & 443.8 & 3575.0 & 297.9 & 145 & 297.9 & 3.9 & 0.2\\
lcgq02hyq & WAVECAL & MIRRORA & P2 & LOW & 10 & 10 & 522.0 & 372.0 & 451.2 & 711.3 & -70.8 & 339.3 & 4173.0 & 417.3 & 147 & 417.3 & 7.7 & 0.2\\
lcgq02icq & WAVECAL & MIRRORA & P1 & LOW & 10 & 10 & 537.0 & 374.0 & 803.3 & 768.3 & 266.3 & 394.3 & 26040.0 & 2604.0 & 120 & 2604.0 & 46.7 & 0.2\\
lcgq02ieq & WAVECAL & MIRRORA & P2 & LOW & 10 & 10 & 538.0 & 374.0 & 559.7 & 667.1 & 21.7 & 293.1 & 4036.0 & 403.6 & 122 & 403.6 & 7.4 & 0.2\\
lcgq02igq & WAVECAL & MIRRORB & P1 & LOW & 30 & 30 & 747.0 & 215.0 & 879.3 & 654.8 & 132.3 & 439.8 & 2659.0 & 88.6 & 364 & 88.6 & 1.3 & 0.1\\
lcgq02iiq & WAVECAL & MIRRORB & P2 & MED & 20 & 20 & 747.0 & 215.0 & 539.0 & 725.6 & -208.0 & 510.6 & 6620.0 & 331.0 & 250 & 331.0 & 4.5 & 0.1\\
lcri01g1q & EXT/SCI & MIRRORB & P2 & MED & 12 & 12 & 722.0 & 210.0 & 767.7 & 584.2 & 45.7 & 374.2 & 3016.0 & 251.3 & 166 & 251.3 & 4.2 & 30.0\\
lcri01g3q & EXT/SCI & MIRRORA & P2 & MED & 150 & 0 & 474.0 & 370.0 & 552.0 & 735.7 & 78.0 & 365.7 & 1964.0 & \dots & 0 & \dots & \dots & 0.2\\
lcri01g5q & WAVECAL & MIRRORA & P2 & LOW & 10 & 10 & 506.0 & 372.0 & 768.5 & 842.0 & 262.5 & 470.0 & 4100.0 & 410.0 & 117 & 410.0 & 9.8 & 0.2\\
lcri01g9q & WAVECAL & MIRRORA & P2 & LOW & 10 & 10 & 506.0 & 371.0 & 278.3 & 582.7 & -227.7 & 211.7 & 3960.0 & 396.0 & 148 & 396.0 & 9.2 & 0.2\\
lcri01gcq & EXT/SCI & MIRRORB & P2 & MED & 14 & 12 & 723.0 & 212.0 & 767.4 & 588.9 & 44.4 & 376.9 & 3381.7 & 281.8 & 148 & 281.8 & 4.0 & 28.3\\
lcri02haq & WAVECAL & MIRRORA & P2 & LOW & 14 & 14 & 526.0 & 372.0 & 644.1 & 719.4 & 118.1 & 347.4 & 5730.0 & 409.3 & 195 & 409.3 & 8.4 & 0.1\\
lcri02hcq & EXT/SCI & MIRRORB & P2 & LOW & 181 & 0 & 715.0 & 211.0 & 782.3 & 578.6 & 67.3 & 367.6 & 2406.0 & \dots & 0 & \dots & \dots & 0.2\\
lcri02heq & WAVECAL & MIRRORB & P2 & MED & 24 & 24 & 737.0 & 213.0 & 853.4 & 647.7 & 116.4 & 434.7 & 7167.0 & 298.6 & 308 & 298.6 & 4.6 & 0.1\\
lcri02hiq & WAVECAL & MIRRORB & P2 & MED & 24 & 24 & 737.0 & 213.0 & 606.7 & 645.2 & -130.3 & 432.2 & 7316.0 & 304.8 & 295 & 304.8 & 4.5 & 0.1\\
lcri02hkq & WAVECAL & MIRRORA & P2 & LOW & 14 & 14 & 519.0 & 372.0 & 551.0 & 580.0 & 32.0 & 208.0 & 5840.0 & 417.1 & 203 & 417.1 & 7.9 & 0.1\\
lcri02hyq & WAVECAL & MIRRORA & P1 & LOW & 14 & 14 & 463.0 & 372.0 & 683.3 & 807.3 & 220.3 & 435.3 & 36245.0 & 2588.9 & 201 & 2588.9 & 45.5 & 0.1\\
lcri02i0q & WAVECAL & MIRRORA & P2 & LOW & 24 & 24 & 463.0 & 372.0 & 781.3 & 778.6 & 318.3 & 406.6 & 9864.0 & 411.0 & 303 & 411.0 & 6.9 & 0.1\\
lcri02i2q & WAVECAL & MIRRORB & P1 & LOW & 30 & 30 & 672.0 & 213.0 & 486.2 & 739.8 & -185.8 & 526.8 & 2864.0 & 95.5 & 415 & 95.5 & 1.3 & 0.1\\
lcri02i4q & WAVECAL & MIRRORB & P2 & MED & 24 & 24 & 671.0 & 212.0 & 884.3 & 415.3 & 213.3 & 203.3 & 8082.0 & 336.8 & 312 & 336.8 & 4.9 & 0.1\\
ld3701gvq & EXT/SCI & MIRRORB & P2 & MED & 16 & 16 & 727.0 & 210.0 & 772.8 & 584.3 & 45.8 & 374.3 & 4147.0 & 259.2 & 184 & 259.2 & 4.3 & 19.0\\
ld3701gxq & EXT/SCI & MIRRORA & P2 & MED & 150 & 0 & 479.0 & 371.0 & 551.2 & 735.8 & 72.2 & 364.8 & 1739.0 & \dots & 0 & \dots & \dots & 0.2\\
ld3701gzq & WAVECAL & MIRRORA & P2 & LOW & 9 & 9 & 505.0 & 372.0 & 413.8 & 701.7 & -91.2 & 329.7 & 3667.0 & 407.4 & 94 & 407.4 & 8.1 & 0.2\\
ld3701h3q & WAVECAL & MIRRORA & P2 & LOW & 10 & 10 & 505.0 & 372.0 & 802.6 & 780.0 & 297.6 & 408.0 & 3999.0 & 399.9 & 107 & 399.9 & 7.6 & 0.2\\
ld3701h5q & EXT/SCI & MIRRORB & P2 & MED & 16 & 16 & 728.0 & 212.0 & 773.4 & 589.0 & 45.4 & 377.0 & 4343.0 & 271.4 & 185 & 271.4 & 4.6 & 19.1\\
ld3702n1q & WAVECAL & MIRRORA & P2 & LOW & 14 & 14 & 515.0 & 371.0 & 886.6 & 659.4 & 371.6 & 288.4 & 5589.0 & 399.2 & 167 & 399.2 & 7.7 & 0.2\\
ld3702n4q & EXT/SCI & MIRRORB & P2 & LOW & 183 & 0 & 723.0 & 213.0 & 774.9 & 577.6 & 51.9 & 364.6 & 2081.0 & \dots & 0 & \dots & \dots & 0.2\\
ld3702n7q & WAVECAL & MIRRORB & P2 & MED & 24 & 24 & 728.0 & 212.0 & 778.9 & 703.3 & 50.9 & 491.3 & 7288.0 & 303.7 & 277 & 303.7 & 4.5 & 0.1\\
ld3702nbq & WAVECAL & MIRRORB & P2 & MED & 24 & 24 & 728.0 & 212.0 & 248.1 & 419.3 & -479.9 & 207.3 & 7140.0 & 297.5 & 274 & 297.5 & 4.5 & 0.1\\
ld3702neq & WAVECAL & MIRRORA & P2 & LOW & 14 & 14 & 507.0 & 372.0 & 911.7 & 878.5 & 404.7 & 506.5 & 5622.0 & 401.6 & 153 & 401.6 & 8.1 & 0.1\\
ld3702o1q & WAVECAL & MIRRORA & P1 & LOW & 14 & 14 & 531.0 & 371.0 & 485.9 & 883.6 & -45.1 & 512.6 & 37530.0 & 2680.7 & 172 & 2680.7 & 45.6 & 0.1\\
ld3702o3q & WAVECAL & MIRRORA & P2 & LOW & 24 & 24 & 531.0 & 371.0 & 665.9 & 888.6 & 134.9 & 517.6 & 9841.0 & 410.0 & 273 & 410.0 & 6.9 & 0.1\\
ld3702o5q & WAVECAL & MIRRORB & P1 & LOW & 30 & 30 & 744.0 & 211.0 & 651.6 & 609.1 & -92.4 & 398.1 & 2375.0 & 79.2 & 319 & 79.2 & 1.5 & 0.1\\
ld3702o7q & WAVECAL & MIRRORB & P2 & MED & 24 & 24 & 743.0 & 211.0 & 940.2 & 700.2 & 197.2 & 489.2 & 6674.0 & 278.1 & 283 & 278.1 & 4.2 & 0.1\\
ldozbadjs & EXT/SCI & MIRRORB & P2 & MED & 16 & 16 & 724.0 & 210.0 & 769.8 & 583.4 & 45.8 & 373.4 & 4005.0 & 250.3 & 138 & 250.3 & 4.4 & 20.2\\
ldozbadlq & EXT/SCI & MIRRORA & P2 & MED & 150 & 0 & 472.0 & 371.0 & 545.1 & 735.6 & 73.1 & 364.6 & 1462.0 & \dots & 0 & \dots & \dots & 0.2\\
ldozbadnq & WAVECAL & MIRRORA & P2 & LOW & 9 & 9 & 499.0 & 372.0 & 889.8 & 583.2 & 390.8 & 211.2 & 3688.0 & 409.8 & 76 & 409.8 & 7.7 & 0.2\\
ldozbadrq & WAVECAL & MIRRORA & P2 & LOW & 10 & 10 & 498.0 & 372.0 & 311.8 & 608.8 & -186.2 & 236.8 & 4009.0 & 400.9 & 97 & 400.9 & 7.0 & 0.2\\
ldozbadtq & EXT/SCI & MIRRORB & P2 & MED & 16 & 16 & 725.0 & 212.0 & 769.8 & 588.9 & 44.8 & 376.9 & 4367.0 & 272.9 & 121 & 272.9 & 3.7 & 21.0\\
ldozbblgq & WAVECAL & MIRRORA & P2 & LOW & 14 & 14 & 507.0 & 372.0 & 748.6 & 911.9 & 241.6 & 539.9 & 5721.0 & 408.6 & 155 & 408.6 & 8.4 & 0.1\\
ldozbbliq & EXT/SCI & MIRRORB & P2 & LOW & 183 & 0 & 713.0 & 213.0 & 776.2 & 578.7 & 63.2 & 365.7 & 2283.0 & \dots & 0 & \dots & \dots & 0.2\\
ldozbblkq & WAVECAL & MIRRORB & P2 & MED & 24 & 24 & 730.0 & 212.0 & 585.6 & 716.8 & -144.4 & 504.8 & 6957.0 & 289.9 & 331 & 289.9 & 4.7 & 0.1\\
ldozbbloq & WAVECAL & MIRRORB & P2 & MED & 24 & 24 & 730.0 & 212.0 & 703.1 & 689.2 & -26.9 & 477.2 & 6983.0 & 291.0 & 305 & 291.0 & 4.0 & 0.1\\
ldozbblqq & WAVECAL & MIRRORA & P2 & LOW & 14 & 14 & 510.0 & 372.0 & 380.6 & 845.9 & -129.4 & 473.9 & 5566.0 & 397.6 & 177 & 397.6 & 7.9 & 0.1\\
ldozbbm4q & WAVECAL & MIRRORA & P1 & LOW & 16 & 16 & 503.0 & 371.0 & 815.0 & 659.6 & 312.0 & 288.6 & 42548.0 & 2659.2 & 189 & 2659.2 & 44.2 & 0.1\\
ldozbbm6q & WAVECAL & MIRRORA & P2 & LOW & 26 & 26 & 503.0 & 371.0 & 772.1 & 616.6 & 269.1 & 245.6 & 10476.0 & 402.9 & 300 & 402.9 & 7.6 & 0.1\\
ldozbbm8q & WAVECAL & MIRRORB & P1 & LOW & 32 & 32 & 715.0 & 211.0 & 252.8 & 463.5 & -462.2 & 252.5 & 2714.0 & 84.8 & 407 & 84.8 & 1.4 & 0.1\\
ldozbbmaq & WAVECAL & MIRRORB & P2 & MED & 26 & 26 & 715.0 & 211.0 & 560.5 & 575.1 & -154.5 & 364.1 & 7768.0 & 298.8 & 340 & 298.8 & 3.7 & 0.1\\
ldozpbf7q & EXT/SCI & MIRRORA & P2 & LOW & 20 & 20 & 511.0 & 370.0 & 555.5 & 741.6 & 44.5 & 371.6 & 7790.0 & 389.5 & 269 & 389.5 & 7.3 & 17.2\\
ldozpbf9q & EXT/SCI & MIRRORB & P2 & MED & 220 & 40 & 734.0 & 210.0 & 779.2 & 582.8 & 45.2 & 372.8 & 12877.2 & 321.9 & 523 & 321.9 & 3.5 & 0.6\\
ldozpbfdq & EXT/SCI & MIRRORB & P2 & MED & 220 & 40 & 734.0 & 211.0 & 780.3 & 584.0 & 46.3 & 373.0 & 13043.9 & 326.1 & 505 & 326.1 & 3.5 & 0.8\\
ldozpbffq & EXT/SCI & MIRRORA & P2 & LOW & 20 & 20 & 514.0 & 370.0 & 559.3 & 743.2 & 45.3 & 373.2 & 7798.0 & 389.9 & 285 & 389.9 & 7.1 & 23.4\\
\enddata
\tablecomments{}
\end{deluxetable}


\begin{deluxetable}{rrrrrrrrrrrrrrrrrrr}
\tablecolumns{19}
\tabcolsep 0pt
\tabletypesize{\tiny}
\tablecaption{COS Spectroscopic \texttt{ACQ/PEAKXD} Data}\label{tab:XDdata}
\tablehead{
\colhead{IPPPSSOOT}&\colhead{EXP}&\colhead{OPT\_ELEM}&\colhead{LAMP}&\colhead{Current}&\colhead{Target}&\colhead{Lamp }&\colhead{WCA\_AD}&\colhead{WCA}&\colhead{SA}&\colhead{SA}&\colhead{WtP}&\colhead{WtP}&\colhead{Lamp}&\colhead{Lamp}&\colhead{WCA}&\colhead{Lamp}&\colhead{Lamp}&\colhead{Target}\\
\colhead{}&\colhead{TYPE}&\colhead{OPT\_ELEM}&\colhead{LAMP}&\colhead{Level}&\colhead{ET(s)}&\colhead{ET (s)}&\colhead{AD}&\colhead{XD}&\colhead{AD}&\colhead{XD}&\colhead{WtP(AD)}&\colhead{XD}&\colhead{counts}&\colhead{cps}&\colhead{bck}&\colhead{CPS}&\colhead{BP}&\colhead{BP}\\
}
\startdata
\hline
lcgq01qlq & EXT/SCI & G230L & P2 & MED & 20 & 7 & 0.0 & 374.0 & 0.0 & 748.0 & 0.0 & 374.0 & 0.0 & 0.0 & 0 & 0.0 & 0.0 & 0.0\\
lcgq01r6q & EXT/SCI & G285M & P2 & MED & 151 & 100 & 0.0 & 355.0 & 0.0 & 728.0 & 0.0 & 373.0 & 0.0 & 0.0 & 0 & 0.0 & 0.0 & 0.0\\
lcgq01r8q & EXT/SCI & G130M & P2 & MED & 20 & 20 & 0.0 & 0.0 & 0.0 & 508.2 & 0.0 & 508.2 & 0.0 & 0.0 & 0 & 0.0 & 0.0 & 0.0\\
lcgq01r8q & EXT/SCI & G130M & P2 & MED & 20 & 20 & 0.0 & 0.0 & 0.0 & 508.2 & 0.0 & 508.2 & 0.0 & 0.0 & 0 & 0.0 & 0.0 & 0.0\\
lcgq01raq & EXT/SCI & G140L & P2 & MED & 7 & 7 & 0.0 & 0.0 & 0.0 & 513.7 & 0.0 & 513.7 & 0.0 & 0.0 & 0 & 0.0 & 0.0 & 0.0\\
lcgq01raq & EXT/SCI & G140L & P2 & MED & 7 & 7 & 0.0 & 0.0 & 0.0 & 513.7 & 0.0 & 513.7 & 0.0 & 0.0 & 0 & 0.0 & 0.0 & 0.0\\
lcgq02i2q & EXT/SCI & G185M & P2 & MED & 40 & 30 & 0.0 & 366.0 & 0.0 & 742.0 & 0.0 & 376.0 & 0.0 & 0.0 & 0 & 0.0 & 0.0 & 0.0\\
lcgq02i4q & EXT/SCI & G225M & P2 & MED & 52 & 30 & 0.0 & 370.0 & 0.0 & 747.0 & 0.0 & 377.0 & 0.0 & 0.0 & 0 & 0.0 & 0.0 & 0.0\\
lcgq02i6q & EXT/SCI & G160M & P2 & MED & 18 & 18 & 0.0 & 0.0 & 0.0 & 503.3 & 0.0 & 503.3 & 0.0 & 0.0 & 0 & 0.0 & 0.0 & 0.0\\
lcgq02i8q & EXT/SCI & G160M & P2 & MED & 22 & 18 & 0.0 & 0.0 & 0.0 & 509.8 & 0.0 & 509.8 & 0.0 & 0.0 & 0 & 0.0 & 0.0 & 0.0\\
lcgq02iaq & EXT/SCI & G160M & P2 & MED & 22 & 18 & 0.0 & 0.0 & 0.0 & 498.4 & 0.0 & 498.4 & 0.0 & 0.0 & 0 & 0.0 & 0.0 & 0.0\\
lcri01ggq & EXT/SCI & G230L & P2 & MED & 20 & 7 & 0.0 & 374.0 & 0.0 & 747.0 & 0.0 & 373.0 & 0.0 & 0.0 & 0 & 0.0 & 0.0 & 0.0\\
lcri01giq & EXT/SCI & G285M & P2 & MED & 151 & 100 & 0.0 & 351.0 & 0.0 & 726.0 & 0.0 & 375.0 & 0.0 & 0.0 & 0 & 0.0 & 0.0 & 0.0\\
lcri01gkq & EXT/SCI & G130M & P2 & MED & 20 & 20 & 0.0 & 532.8 & 0.0 & 449.0 & 0.0 & -83.8 & 0.0 & 0.0 & 0 & 0.0 & 0.0 & 0.0\\
lcri01gkq & EXT/SCI & G130M & P2 & MED & 20 & 20 & 0.0 & 532.8 & 0.0 & 449.0 & 0.0 & -83.8 & 0.0 & 0.0 & 0 & 0.0 & 0.0 & 0.0\\
lcri01h6q & EXT/SCI & G140L & P2 & MED & 7 & 7 & 0.0 & 542.2 & 0.0 & 457.4 & 0.0 & -84.8 & 0.0 & 0.0 & 0 & 0.0 & 0.0 & 0.0\\
lcri01h6q & EXT/SCI & G140L & P2 & MED & 7 & 7 & 0.0 & 542.2 & 0.0 & 457.4 & 0.0 & -84.8 & 0.0 & 0.0 & 0 & 0.0 & 0.0 & 0.0\\
lcri02hoq & EXT/SCI & G225M & P2 & MED & 52 & 30 & 0.0 & 371.0 & 0.0 & 747.0 & 0.0 & 376.0 & 0.0 & 0.0 & 0 & 0.0 & 0.0 & 0.0\\
lcri02hqq & EXT/SCI & G185M & P2 & MED & 40 & 30 & 0.0 & 367.0 & 0.0 & 742.0 & 0.0 & 375.0 & 0.0 & 0.0 & 0 & 0.0 & 0.0 & 0.0\\
lcri02hsq & EXT/SCI & G160M & P2 & MED & 22 & 18 & 0.0 & 521.4 & 0.0 & 442.1 & 0.0 & -79.3 & 0.0 & 0.0 & 0 & 0.0 & 0.0 & 0.0\\
lcri02huq & EXT/SCI & G160M & P2 & MED & 25 & 18 & 0.0 & 520.6 & 0.0 & 449.6 & 0.0 & -71.0 & 0.0 & 0.0 & 0 & 0.0 & 0.0 & 0.0\\
lcri02hwq & EXT/SCI & G160M & P2 & MED & 25 & 18 & 0.0 & 521.0 & 0.0 & 434.2 & 0.0 & -86.8 & 0.0 & 0.0 & 0 & 0.0 & 0.0 & 0.0\\
ld3701h9q & EXT/SCI & G230L & P2 & MED & 21 & 21 & 0.0 & 375.0 & 0.0 & 748.0 & 0.0 & 373.0 & 0.0 & 0.0 & 0 & 0.0 & 0.0 & 0.0\\
ld3701hbq & EXT/SCI & G285M & P2 & MED & 151 & 100 & 0.0 & 352.0 & 0.0 & 727.0 & 0.0 & 375.0 & 0.0 & 0.0 & 0 & 0.0 & 0.0 & 0.0\\
ld3701hdq & EXT/SCI & G130M & P2 & MED & 25 & 25 & 0.0 & 532.7 & 0.0 & 447.4 & 0.0 & -85.3 & 0.0 & 0.0 & 0 & 0.0 & 0.0 & 0.0\\
ld3701hdq & EXT/SCI & G130M & P2 & MED & 25 & 25 & 0.0 & 532.7 & 0.0 & 447.4 & 0.0 & -85.3 & 0.0 & 0.0 & 0 & 0.0 & 0.0 & 0.0\\
ld3701hfq & EXT/SCI & G140L & P2 & MED & 10 & 7 & 0.0 & 541.7 & 0.0 & 455.9 & 0.0 & -85.7 & 0.0 & 0.0 & 0 & 0.0 & 0.0 & 0.0\\
ld3701hfq & EXT/SCI & G140L & P2 & MED & 10 & 7 & 0.0 & 541.7 & 0.0 & 455.9 & 0.0 & -85.7 & 0.0 & 0.0 & 0 & 0.0 & 0.0 & 0.0\\
ld3702nmq & EXT/SCI & G225M & P2 & MED & 53 & 35 & 0.0 & 370.0 & 0.0 & 747.0 & 0.0 & 377.0 & 0.0 & 0.0 & 0 & 0.0 & 0.0 & 0.0\\
ld3702noq & EXT/SCI & G185M & P2 & MED & 40 & 35 & 0.0 & 366.0 & 0.0 & 743.0 & 0.0 & 377.0 & 0.0 & 0.0 & 0 & 0.0 & 0.0 & 0.0\\
ld3702nqq & EXT/SCI & G160M & P2 & MED & 22 & 21 & 0.0 & 523.5 & 0.0 & 441.0 & 0.0 & -82.6 & 0.0 & 0.0 & 0 & 0.0 & 0.0 & 0.0\\
ld3702nsq & EXT/SCI & G160M & P2 & MED & 25 & 24 & 0.0 & 523.7 & 0.0 & 448.4 & 0.0 & -75.3 & 0.0 & 0.0 & 0 & 0.0 & 0.0 & 0.0\\
ld3702nuq & EXT/SCI & G160M & P2 & MED & 25 & 24 & 0.0 & 524.1 & 0.0 & 433.4 & 0.0 & -90.8 & 0.0 & 0.0 & 0 & 0.0 & 0.0 & 0.0\\
ldozbadxq & EXT/SCI & G230L & P2 & MED & 23 & 21 & 0.0 & 374.0 & 0.0 & 748.0 & 0.0 & 374.0 & 0.0 & 0.0 & 0 & 0.0 & 0.0 & 0.0\\
ldozbadzq & EXT/SCI & G285M & P2 & MED & 151 & 100 & 0.0 & 352.0 & 0.0 & 727.0 & 0.0 & 375.0 & 0.0 & 0.0 & 0 & 0.0 & 0.0 & 0.0\\
ldozbae1q & EXT/SCI & G130M & P2 & MED & 25 & 25 & 0.0 & 532.3 & 0.0 & 445.7 & 0.0 & -86.6 & 0.0 & 0.0 & 0 & 0.0 & 0.0 & 0.0\\
ldozbae1q & EXT/SCI & G130M & P2 & MED & 25 & 25 & 0.0 & 532.3 & 0.0 & 445.7 & 0.0 & -86.6 & 0.0 & 0.0 & 0 & 0.0 & 0.0 & 0.0\\
ldozbae3q & EXT/SCI & G140L & P2 & MED & 10 & 7 & 0.0 & 540.7 & 0.0 & 454.3 & 0.0 & -86.4 & 0.0 & 0.0 & 0 & 0.0 & 0.0 & 0.0\\
ldozbae3q & EXT/SCI & G140L & P2 & MED & 10 & 7 & 0.0 & 540.7 & 0.0 & 454.3 & 0.0 & -86.4 & 0.0 & 0.0 & 0 & 0.0 & 0.0 & 0.0\\
ldozbbluq & EXT/SCI & G225M & P2 & MED & 53 & 35 & 0.0 & 370.0 & 0.0 & 747.0 & 0.0 & 377.0 & 0.0 & 0.0 & 0 & 0.0 & 0.0 & 0.0\\
ldozbblwq & EXT/SCI & G185M & P2 & MED & 40 & 35 & 0.0 & 366.0 & 0.0 & 743.0 & 0.0 & 377.0 & 0.0 & 0.0 & 0 & 0.0 & 0.0 & 0.0\\
ldozbblyq & EXT/SCI & G160M & P2 & MED & 22 & 22 & 0.0 & 523.0 & 0.0 & 440.7 & 0.0 & -82.2 & 0.0 & 0.0 & 0 & 0.0 & 0.0 & 0.0\\
ldozbbm0q & EXT/SCI & G160M & P2 & MED & 27 & 25 & 0.0 & 522.9 & 0.0 & 448.2 & 0.0 & -74.7 & 0.0 & 0.0 & 0 & 0.0 & 0.0 & 0.0\\
ldozbbm2q & EXT/SCI & G160M & P2 & MED & 27 & 25 & 0.0 & 523.2 & 0.0 & 433.4 & 0.0 & -89.9 & 0.0 & 0.0 & 0 & 0.0 & 0.0 & 0.0\\
\enddata
\tablecomments{}
\end{deluxetable}

% rcs_id = "$Id: pctaWCA2SANIM.tex,v 1.2 2018/03/30 20:22:12 penton Exp $"
\begin{center}
\begin{table}
\footnotesize
	\centering
	\begin{threeparttable}
	\caption[\textsc{pcta\_[X,Y]ImCalTargetOffset} Values]{\tacq{IMAGE} WCA-to-SA FSW Target Offsets}
		\begin{tabular*}{.895\linewidth}{@{\extracolsep{\fill}}ccccrr}
		\toprule
		Direction & DETector & USER$^{a}$ & \textit{OPT\_ELEM}  &	PSA	&	BOA\\
		(AD or XD) & Coordinate &	Coordinate &	&	\\
		\bottomrule
		\midrule
		\multicolumn{6}{c}{MIRRORA}\\
		\midrule
		AD$^c$	&	Y	&	-X	&	MIRA	&	45.3	&	45.5 \\
		XD$^d$	&	X	&	-Y	&	MIRA	&	372.7	&	368.4 \\
		\midrule
		\multicolumn{6}{c}{MIRRORB {\bf prior} to Oct-20-2014 (2014.283)}\\
		\midrule
		AD	&	Y	&	-X	&	MIRB	&	45.0	&	45.5 \\
		XD	&	X	&	-Y	&	MIRB	&	374.1	&	366.3 \\
		\midrule
		\multicolumn{6}{c}{MIRRORB {\bf after}$^{b}$ to Oct-20-2014 (2014.283)}\\
		\midrule
		AD	&	Y	&	-X	&	MIRB	&	46.0	&	46.5 \\
		XD	&	X	&	-Y	&	MIRB	&	374.0	&	366.2 \\
		\bottomrule
		\end{tabular*}
		\label{tab:pctaWCA2SANIM}
		\footnotesize
		\begin{tablenotes}
			\item[a] COS DETector and USER coordinates are related by equations~\ref{eq:NUVuserX} \& \ref{eq:NUVuserY}. A consequence of
			these definitions, a given \textit{APERTURE}$\times$\textit{OPT\_ELEM} configurations' AD or XD WCA-to-SA offset in one coordinate system is equal to the SA-to-WCA offset in the other.
			For example, the AD PSA$\times$MIRA WCA-to-PSA offset is +45.3 in detector coordinates (Y$_{DET}$), as is the SA-to-WCA AD offset in USER coordinates (X$_{USER}$).
			\item[b] Installed 2014.283 (\pr{79116}, "Update MIRRORB Cal Target Offsets").
			\item[c] XD direction offsets, measured from WCA lamp median to the SA center, stored in the FSW \textsc{pcta\_XImCalTargetOffset} table.
			\item[d] AD direction offsets, measured from WCA lamp median to the SA center, stored in the FSW \textsc{pcta\_YImCalTargetOffset} table.
		\end{tablenotes}
		\normalsize
	\end{threeparttable}
\normalsize
\end{table}
\end{center}


\subsection{WCA Lamp Images (aka, Lamp Family Portraits) \label{subsec:fportrait} }
\normalsize

The four panels of Figures~\ref{fig:FG21}--\ref{fig:FG24} show a `family portrait' of the available COS PtNe Lamp (\plampone{} or \plamptwo{}) + MIRROR (MIRA or MIRB)
combinations possible with \tacq{IMAGE}.
Panel titles give the lamp and mirror configuraton, along with the lamp current setting (in milli-amps, mA) and the exposure times.
The images and subarrays are in `detector' coordinates, as used on-board COS.
The images show the observed counts/pixel/s (cps) as given by the colorbar on the bottom.
The \textcolor{red}{red} dashed boxes show the given cycles' \tacq{IMAGE} WCA subarrays. At the top of the subarrays, text provides the count rate in the brightest pixel (BP) in units of cps.
The \textcolor{blue}{blue} histogram on the bottom edge shows the cross-dispersion (XD) lamp profile in detector `X$_{DET}$' coordinates, while
the \textcolor{green}{green} histogram on the left edge shows the along-dispersion (AD) lamp profile in detector `Y$_{DET}$' coordinates.
The cross-hairs show the median location of the given configurations' lamp events within the TA subarray.
PtNe\#2 (\plamptwo{}) lamp was used for all \tacq{IMAGE}s during C20--24, and was operated at LOW current (6~mA) for MIRA images
and LOW current (3~mA) or MEDium current (10~mA) for the MIRB, depending on the Cycle. Note the separate MIRB images in about a 2:1 ratio, and the asymmetric
(toward -XD) scattered light.

See the figure captions for specific details about each cycles' family portrait.

\begin{figure}[htb]
\noindent\includegraphics[width=0.7\linewidth]{png/C21_13526_FP.png}
\caption[C21 WCA Lamp `Family Portrait']{Cycle~21 (\pid{13526} PtNe Lamp `Family Portrait'
counts per second per pixel (cps) NUV images of the internal PtNe lamps (\plampone{} \& \plamptwo{}) through the
WCA using either MIRRORA (MIRA, left 2 panels) or MIRRORB (MIRB, right 2 panels). The titles
give the exposure \textit{ROOTNAME}, configuration, exposure time and lamp current. Cross hairs show median locations and dashed
lines show the \textsc{LTAIMCAL} TA subarrays.
The insert text gives the Brightest Pixel (BP) in cps and the total counts in the subarray.
AD and XD profiles are given along each axis, and the color bar at the
bottom applies to all four images. Note the separate MIRB images in about a 2:1 ratio, and the asymmetric
(toward -XD) profile and scattered light. All panels are in detector (DET) coordinates.\label{fig:FG21}}
\end{figure}
\begin{center}
	\begin{figure}[htb]
	\noindent\includegraphics*[width=0.7\linewidth]{png/C22_13972_FP.png}
	\caption[C22 WCA Lamp `Family Portrait']{\footnotesize Cycle~22 PtNe Lamp `Family Portrait' (see Fig~\ref{fig:FG21}).
	Note that during both the MIRA and MIRB images, the lamp image is about -50~p from the C21 AD location (Y$_{DET}$).
	This is common, and the TA subarrays must be large enough to account for this offset.\label{fig:FG22}}
	\end{figure}
\end{center}
\begin{center}
\begin{figure}[htb]
\noindent\includegraphics*[width=0.7\linewidth]{png/C23_14440_FP.png}
\caption[C23 WCA Lamp `Family Portrait']{Cycle~23 PtNe Lamp `Family Portrait' (see Fig~\ref{fig:FG21} \& Fig~\ref{fig:FG22}).
	Note that during all images, the lamp image is about +50~p from the C22 AD location (Y$_{DET}$),
	and has returned to its C21 AD position.\label{fig:FG23}}
\end{figure}
\end{center}
\begin{center}
	\begin{figure}[htb]
	\noindent\includegraphics*[width=0.7\linewidth]{png/C24_14857_FP.png}
	\caption[C24 WCA Lamp `Family Portrait']{Cycle~24 PtNe Lamp `Family Portrait' (see Fig~\ref{fig:FG21}--Fig~\ref{fig:FG23}).
	Note that during all images, the lamp image has returned to its C21 position.\label{fig:FG24}}
	\end{figure}
\end{center}

%Begin [OPTIONAL]
\subsection{[OPTIONAL] Reconfiguration of MIRB \tacq{IMAGE}} \label{subsec:newMIRB}
{\bf Note to Reviewers: There are additional Details on the MIRB \tacq{IMAGE} adjust of 2014.
If you feel this document would be a good place to put that information, it could be added here.}\\

\subsection{[OPTIONAL] SIAF Verification} \label{sec:siaf}\label{subsec:siafextra}
{\bf Note to Reviewers: There are additional details on the COS SIAF entries that can be inferred from the FGS-to-SI alignment program than are documented here. They mostly live in spreadsheet, that
should be in the directory in the repo called ``siaf\_extra''. If you feel this document would be a good place to put that information, it could be added here.
Also, nowhere in any ISR are our SIAF entries documented. If requested, they could be added to the ISR either here or in the appendix.}
\section{SIAF Verification} \label{sec:siaf}

\subsection{COS SIAF History}\label{subsec:siafhistory}
The pre-SM4 COS Science Instrument Aperture File (SIAF) is described in detail in Mallo, 2008.
The active COS entries in the 2018 SIAF, and the [V2,V3] aperture positions are given in Table~\ref{tab:activesiaf}.
The changes since SMOV to the COS SIAF, and the continuing Fine Guidance Sensor (FGS) re-alignment efforts are
documented in Table~\ref{tab:siafhistory}.
This table includes all NUV LP1 and FUV LP1--4 entries. Each LP contains entries for the BOA, PSA, BOA when the PSA is being used, and PSA when the BOA is being used.

\input{extrasiaf/siafhistory.tex}

The COS SIAF ``Aperture Names''  start with the ``L"", and are followed by either an ``N'' (NUV), ``F'' (FUV), or
``APT''' indicating that the aperture is only used in APT and not for observations.
``APT'' in the APT entries are immediately followed by an ``N'' or ``F''.
The SA (BOA or PSA) or MAC then follows. MAC represents the APT/SPSS\footnote{SPSS=Science Planning and Scheduling System} MACro aperture used for bright object checking.
Finally, FUV entries end in a number giving the LP\#, while the NUV ``offset'' apertures end with ``OF''.
An illustration of the active COS SIAF entries is given in Figure~\ref{fig:SIAF1}. As described in the individual
FUV LP enabling ISRs, the location of each LP aperture is determined by first selecting the desired XD location on the FUV detector segments. After this selection,
an AD aperture scan, using \texttt{POS\_TARGs} in APT, determines the SIAF entry adjustment corresponding to
the desired LP [V2,V3] aperture center. As shown in Figure~\ref{fig:SIAF1}, this alignment procedure produces
a series of aperture locations that are at an $\approx$ 44.2$\degree$ angle to the -V3 (U3) axis.

\begin{deluxetable}{rcrr}
\tablecaption{Active COS SIAF\tablenotemark{a}~~Entries\label{tab:activesiaf}}
\tabletypesize{\footnotesize}
\tablewidth{0pt}
\tabcolsep 16 pt
\tablecolumns{4}
\tablehead{
\colhead{SIAF} & \colhead{YEAR.DAY} &  \colhead{V2} & \colhead{V3}\\
\colhead{APERNAME} & \colhead{Activated} &  \colhead{(\arcsec)} & \colhead{(\arcsec)}
}
\startdata
\toprule
\multicolumn{4}{c}{NUV LP1}\\
\midrule
LNMAC & 2014.055  & +232.7230 & -237.5150\\
LNBOA & 2014.055  & +232.7230 & -237.5150\\
LNPSA & 2014.055  & +232.7230 & -237.5150\\
LAPTNBOAOF &2014.055  & +223.3488 & -246.8892\\
LAPTNPSAOF &2014.055  & +242.0972 & -228.1408\\
\midrule
\multicolumn{4}{c}{FUV LP1}\\
\midrule
LFMAC      & 2014.055  & +232.7230 & -237.5150\\
LFBOA1     & 2016.151  & +232.7230 & -237.5150\\
LFPSA1     & 2016.151  & +232.7230 & -237.5150\\
LAPTFBOAF1 & 2016.151  & +223.3488 & -246.8892\\
LAPTFPSAF1 & 2016.151  & +242.0972 & -228.1408\\
\midrule
\multicolumn{4}{c}{FUV LP2}\\
\midrule
LFBOA2      & 2016.151  & +235.1580 & -235.0100\\
LFPSA2      & 2016.151  & +235.1580 & -235.0100\\
LAPTFBOAF2  & 2016.151  & +225.7838 & -244.3842\\
LAPTFPSAF2  & 2016.151  & +244.5322 & -225.6358\\
\midrule
\multicolumn{4}{c}{FUV LP3}\\
\midrule
LFBOA3      & 2016.151  & +230.9137 & -239.2749\\
LFPSA3      & 2016.151  & +230.9137 & -239.2749\\
LAPTFBOAF3  & 2016.151  & +221.5395 & -248.6491\\
LAPTFPSAF3  & 2016.151  & +240.2879 & -229.9007\\
\midrule
\multicolumn{4}{c}{FUV LP4}\\
\midrule
LFBOA4      & 2017.031  & +229.1328 & -241.0575\\
LFPSA4      & 2017.031  & +229.1328 & -241.0575\\
LAPTFBOAF4  & 2017.031  & +219.7586 & -250.4317\\
LAPTFPSAF4  & 2017.031  & +238.5070 & -231.6833\\
\bottomrule
\enddata
\tablenotetext{a}{SIAF = Science Instrument Aperture File.}
\tablecomments{COS SIAF ``Aperture Names'' (APERNAME) start with the ``L"", and are
followed by either an ``N'' (NUV), ``F'' (FUV), or
``APT''' indicating that the aperture is only used in APT and not for observations.
``APT'' in the APT entries are immediately followed by an ``N'' or ``F''.
The SA (BOA or PSA) or MAC then follows.
MAC represents the APT/SPSS MACro aperture used for bright object checking.
Finally, FUV entries end in a number giving the LP\#, while the NUV ''offset'' aperturens end with ``OF''.}
\end{deluxetable}


The orientation of the COS AD and XD directions in relation to the HST mechanical [V2,V3] coordinates
and U-frame are shown in Figure~\ref{fig:ADXDV23}. The HST U-frame of [U2,U3]=[-V2,-V3] and is used by APT
and SPSS.  Converting the orientation shown in Figure~\ref{fig:SIAF1} to
the nomenclature of the SIAF, the COS FUV apertures are aligned as follows:
$\beta_{x}=135.8\degree$, $\beta_y=45.8\degree$, and parity=+1. As shown in Figure~\ref{fig:SIAF1}, for COS:
\begin{eqnarray}
\setlength\itemsep{0.1em}
AD = [+V2,-V3]\label{eq:ADV23}\\
XD = [+V2,+V3]\label{eq:XDV23}\\
V2 = [+AD,+XD]\label{eq:V2ADXD}\\
V3 = [-AD,+XD]\label{eq:V3ADXD}
\end{eqnarray}
\normalsize
The NUV coordinate system orientation was measured during SMOV (Hartig et al., 2010 and Goudfrooij et al., 2010).
This NUV angle was determined to be 0.52 $\pm 0.01\degree$ from +Y \texttt{POS\_TARG} in the +X \texttt{POS\_TARG} direction
($\beta_{x}=135.5\degree$, $\beta_y=45.5\degree$). The COS SIAF currently uses $\beta_{x}=135\degree$, $\beta_y=45\degree$ for both NUV and FUV.
All conversions between [AD,XD] and [V2,V3] in this ISR use the current SIAF values for  $\beta_{x}$ and $\beta_{y}$.
The conversion between [V2,V3] and [AD,XD] in terms of $\beta_{x}$ and $\beta_{y}$ are :\footnote{See http://www.stsci.edu/hst/observatory/apertures/siaf.html}
\begin{eqnarray}
\setlength\itemsep{0.1em}
	V2 = S_{AD} \cdot sin(\beta_x) \cdot XD	+ S_{XD} \cdot sin(\beta_y) \cdot XD \label{eq:V2beta}\label{eq:V2BETA}\\
	V3 = S_{XD} \cdot cos(\beta_x) \cdot XD	+ S_{XD} \cdot cos(\beta_y) \cdot XD \label{eq:V3beta}\label{eq:V3BETA}
\end{eqnarray}
Where, $S_{AD}$ is the AD plate scale (0.02352\arcsec/p), $S_{XD}$ is the XD plate scale (0.02362\arcsec/p ), and the aperture center was taken as the reference point.

\begin{figure}[htb]
\noindent\includegraphics*[width=0.485\linewidth]{png/LP4_SIAF_positions.png}
\noindent\includegraphics*[width=0.485\linewidth]{png/LP4_SIAF_positions_sname.png}
\caption[Illustration of COS SIAF Entries]{
Illustration of the COS SIAF Entries.  All FUV LP1 entries are shown in \textcolor{RED}{RED},
LP2 entries are shown in \textcolor{GREEN}{GREEN}, LP3 entries are shown in \textcolor{BLUE}{BLUE},
and the LP4 entries are shown in \textcolor{MAGENTA}{MAGENTA}.
The NUV SIAF entries are coincident with the \textcolor{RED}{LP1} FUV entries.
The left panel shows the actual SIAF entry names, while the right panel gives a more readable translation.\label{fig:SIAF1}}
\end{figure}

\begin{figure}[htb]
\begin{center}
\noindent\includegraphics*[width=0.7\linewidth]{png/ADXD_V23.png}
\noindent\includegraphics*[width=0.795\linewidth]{pdf/COS_COORDS.pdf}
\end{center}
\caption[COS Aperture Orientation]{The upper panel shows the COS aperture orientation versus the [V2,V3] telescope coordinate system (Lallo 2008).
The U ([U2,U3]=[-V2,-V3]) frame is used during the proposal process in APT and SPSS. The lower panel (Osbourne, 2004) shows all the pre-launch COS coordinate
systems. From left to right, the ``P''hysical coordinate system is aligned with
the COS enclosure, and shares a common origin with the [X,Y,Z] ``DET''ector coordinates.
The X, Y, and Z axes of the STOPT (Space Telescope OPTical) coordinate system are shown slightly below and aft of these.
The HST [V1,V2,V3] coordinate system is shown on the right.\label{fig:ADXDV23}}
\end{figure}

\subsection{C17--C23 COS to SIAF Alignment \label{subsec:siafalign}}

The FGS-to-SI programs provide the opportunity to estimate the co-alignment
of the COS SIAF entry to the actual center of the COS SAs. The FGS-to-SI programs
concludes with two cos \tacq{IMAGE}s and a target that is approximately centered in the COS
aperture. By comparing the [V2,V3] position after the first of these \tacq{IMAGE}s
(the Configuration\#1 or Config\#1 PSA$\times$MIRA \tacq{IMAGE}), a direct comparison is possible.
Table~\ref{tab:fgs2siInit} gives these results for both the Spring and the Fall C17--C23 FGS-to-SI alignment programs.

The columns of Table~\ref{tab:fgs2siInit} are:
\footnotesize
\begin{enumerate}
\item \textit{PROPOSID} gives the HST program id (PID).
\item YEAR.DAY gives the Year and day of the year of the observation.
\item \textit{DATE-OBS} gives the observation date as reported in the
fits header in DY-Mon-YEAR format.
\item gives HST [V2,V3] coordinates (in \arcsec) of
the initial HST pointing before the (Config\#1) PSA$\times$MIRA \tacq{IMAGE}.
\item gives HST [V2,V3] coordinates (in \arcsec) of
the intermediate HST pointing after the (Config\#1) PSA$\times$MIRA \tacq{IMAGE}
and before the (Config\#2) PSA$\times$MIRB \tacq{IMAGE}.
\item gives HST [V2,V3] ``Miss-Distances'' (in \arcsec) of
the initial HST pointing before the (Config\#1) PSA$\times$MIRA \tacq{IMAGE}
and are the Initial Pointing coordinates subtracted from the \textit{LNPSA}
SIAF entry active at the time of the observation.
\item gives HST [V2,V3] ``Miss-Distances'' (in \arcsec) of
the intermediate HST pointing after the PSA$\times$MIRA \tacq{IMAGE}.
\item ``SIAF Dates'' gives the dates the [V2,V3] SIAF entry in the following ``SIAF Entry'' column was active.
\item ``SIAF Entry'' gives the [V2,V3] entries that were active at the time
of the observations of this row.
\end{enumerate}
\normalsize

\begin{deluxetable}{rrrrrrrrrr}
\tablecaption{FGS-to-SI Program Initial Pointing Determinations\label{tab:fgs2siInit}}
\tablecolumns{10}
\tabletypesize{\footnotesize}
\tablehead{
\colhead{PID} & \colhead{YEAR.DAY} & \colhead{DATE-OBS} & \multicolumn{2}{c}{Initial Pointing} & \multicolumn{2}{c}{Miss-Distance} & \colhead{SIAF} & \multicolumn{2}{c}{Active SIAF Entry}\\
\colhead{ } & \colhead{} & \colhead{} & \colhead{V2 (\arcsec)} & \colhead{V3 (\arcsec)} & \colhead{V2 (\arcsec)} & \colhead{V3 (\arcsec)} & \colhead{V2 (\arcsec)} & \colhead{Dates\tablenotemark{a}} & \colhead{V3 (\arcsec)}\\
\colhead{(1)}&\colhead{(2)} & \colhead{(3)}&\colhead{(4)} & \colhead{(5)}&\colhead{(6)} & \colhead{(7)}&\colhead{(8)} & \colhead{(9)}&\colhead{(10)}
}
\startdata
\hline
\pid{11878} & 2009.338 & 04-Dec-2009 & 232.581 & -237.544 & -0.191 & -0.033 & 3-Aug-2009 & 232.772 & -237.511\\
\pid{11878} & 2010.074 & 15-Mar-2010 & 232.488 & -237.462 & -0.284 & 0.049 & \dots & 232.772 & -237.511\\
\pid{11878} & 2010.110 & 20-Apr-2010 & 232.481 & -237.457 & -0.291 & 0.054 & \dots & 232.772 & -237.511\\
\pid{11878} & 2010.309 & 05-Nov-2010 & 232.604 & -237.561 & -0.168 & -0.050 & \dots & 232.772 & -237.511\\
\pid{12399} & 2011.070 & 11-Mar-2011 & 232.645 & -237.438 & -0.127 & 0.073 & 20-Jun-2011 & 232.772 & -237.511\\
\hline
\pid{12399} & 2011.255 & 12-Sep-2011 & 232.737 & -237.507 & 0.091 & -0.062 & 21-Jun-2011 & 232.646 & -237.445\\
\pid{12781} & 2012.087 & 27-Mar-2012 & 232.622 & -237.515 & -0.024 & -0.070 & \dots & 232.646 & -237.445\\
\pid{12781} & 2012.268 & 24-Sep-2012 & 232.713 & -237.578 & 0.067 & -0.133 & \dots &232.646 & -237.445\\
\pid{13171} & 2013.061 & 02-Mar-2013 & 232.647 & -237.590 & 0.001 & -0.145 & \dots & 232.646 & -237.445\\
\pid{13171} & 2013.244 & 01-Sep-2013 & 232.723 & -237.515 & {\bf 0.077}\tablenotemark{b} & {\bf -0.070}\tablenotemark{b} & 23-Feb-2014 & 232.646 & -237.445\\
\hline
\pid{13616} & 2014.055 & 06-Apr-2014 & 232.535 & -237.497 & -0.188 & 0.018 & 24-Feb-2014 & 232.723 & -237.515\\
\pid{13616} & 2014.300 & 27-Oct-2014 & 232.841 & -237.465 & {\bf 0.118} & {\bf 0.050} & \dots & 232.723 & -237.515\\
\pid{14035} & 2015.104 & 14-May-2015 & 232.617 & -237.464 & -0.106 & 0.051 & \dots & 232.723 & -237.515\\
\pid{14035} & 2015.275 & 02-Oct-2015 & 232.788 & -237.462 & {\bf 0.065} & {\bf0.053} & \dots & 232.723 & -237.515\\
\pid{14452} & 2016.092 & 01-Apr-2016 & 232.742 & -237.485 & 0.019 & 0.030 & \dots & 232.723 & -237.515\\
\hline
\enddata
\tablenotetext{a}{Dates in this column show the dates that the [V2,V3] SIAF entries in the this and the following rows were active.}
\tablenotetext{b}{These exposures, and the offsets measured here, were used to adjust the COS SIAF entries on 2014.055 (STScI PR\#76982).}
\tablecomments{Items in {\bf bold} are used in the analysis of this ISR.}
\end{deluxetable}



\subsection{[OPTIONAL] Importance of S/N to \tacq{IMAGE}} \label{sec:snai}
{\bf Note to Reviewers: A great deal of effort was
exerted in 2017 to analysis the S/N requirement of
\tacq{IMAGE}s. This allowed the relaxing of the S/N requirements that allowed many of the Mdwarf exposures to proceed.
If you feel this document would be a good place to put that information, it could be added here, OR
a new ISR could be initiated to document these efforts. Please let me know your thoughts on this.}
\begin{figure}[htb]
\noindent\includegraphics*[width=0.795\linewidth]{png/C24_14857_Error_vs_lampSN.png}
\caption[OPTIONAL:Example of Lamp S/N centering concerns]{Example of C24 centering changes with S/N. \label{fig:FG24e}.
{\bf Note to reviewers, this is a sample of the data available for the S/N vs centering accuracy discussion.} }
\end{figure}
%End OPTIONAL
\clearpage
% $Id: spVER.tex,v 1.5 2018/03/30 20:22:12 penton Exp $
\section{Spectroscopic TA Verification}\label{sec:spVER}
\normalsize
After the series of \texttt{ACQ/IMAGE}s that start each visit, the target should be accurately centered.
We take advantage of this to monitor certain aspects of COS spectroscopic TAs.

COS spectroscopic TAs consist of up to three stages \texttt{ACQ/SEARCH}, \texttt{PEAKD}, and \texttt{PEAKXD}.
The COS spectroscopic \texttt{ACQ/SEARCH} and \texttt{PEAKD} algorithms do not use any FSW patchable constants, and do not flash the
internal calibration lamps. The only monitoring required for these TA phases is to ensure that the mechanisms were in their proper
positions and that the TA subarrays defined in the HST ground commanding are proper for the mechanism positions used during the acquisitions.
As discussed in \S~\ref{sec:subarray}, the majority of the details will be addressed for each FUV LP in its enabling ISR, or have already been verified
for the NUV and FUV LP1 positions in Penton \& Keyes (2011).

COS NUV (LP1) and FUV LP2--4 spectroscopic TA in the XD direction uses \texttt{ACQ/PEAKXD} and requires the use of the XD WCA-to-PSA offsets with the nominal \numposone~ algorithm.
These offsets are contained for both the NUV and FUV channels in the FSW patchable constant table \textsc{pcta\_CalTargetOffset}, and are provided for reference for all COS LPs in Table~\ref{tab:wcatopsa}.
This ISR only attempts to verify that these offsets were appropriate for all data obtained during the annual monitoring programs.

Each FUV central wavelength setting (\cenwave) uses a unique OSM1 rotation, whereas all NUV TAs use the same OSM1 rotation independent of \cenwave.
However, each NUV \cenwave uses a different OSM2 rotation during TA. Each FUV \cenwave has it's own set of TA subarrays (up to four per segment), while the NUV TA subarrays are not \cenwave
specific, but are grating specific.

The verification process is for \texttt{ACQ/PEAKXD} is simple, take a normal spectrum with a target signal-to-noise ratio of least 50 for the entire spectrum (2500 target counts),
and directly measure the WCA-to-PSA offset and compare it the FSW value. For NUV exposures, this is almost always \texttt{STRIPE=B}, and for the FUV, only events from FUVA are used at LP2--4.
TA subarrays are used to mask out any detector hot-spots or Geocoronal light that could interfere with the centering process. These spectra are also compared to the TA subarrays to
verify that they satisfactory.

% $Id: pctaWCA2SA.tex,v 1.5 2018/03/30 20:22:12 penton Exp $
\begin{deluxetable}{lrrr}
%\tablewidth{0pt}
\tabcolsep 10 pt
%\tabletypesize{\footnotesize}
\tablecolumns{4}
\tablecaption{\tacq{PEAKXD} WCA-to-PSA offsets \label{tab:wcatopsa}}
\tablehead{
\colhead{\textit{OPT\_ELEM}}&\colhead{LP1}&\colhead{LP2}&\colhead{LP3}\\
}

\startdata
\hline
\multicolumn{4}{c}{FUV\tablenotemark{f}}\\
\hline
G130M	&	 -898	&	-943	&	-892 \\
G140L	&	 -884	&	-950	&	-857 \\
G160M	&	 -898	&	-933	&	-901 \\
\hline
\multicolumn{4}{c}{NUV\tablenotemark{n}}\\
\hline
G185M	&	3742	&	\dots	&	\dots \\
G225M	&	3746	&	\dots	&	\dots \\
G230L	&	3734	&	\dots	&	\dots \\
G285M	&	3749	&	\dots	&	\dots \\
\hline
\enddata
\tablenotetext{f}{Divide the FUV numbers by -10 to get the number of XD rows between the PSA and WCA spectra for a target centered in the aperture.}
\tablenotetext{n}{Divide the NUV numbers by 10 to get the NUV WCA-to-PSA offset. }
\tablecomments{The FSW patchable constant \textsc{pcta\_CalTargetOffsetScale} determines the FSW scaling (currently set to 10).
FUV scalings are "negative" due to parity of HST slews relative to the COS coordinate system. {\bf Note to reviewers: Do you think I should keep the numbers in their FSW
values (not scaled), or should I go ahead and scale them ?}}
\end{deluxetable}


\subsection{NUV Spectroscopic TA verification}\label{subsec:NspVER}
The \plamptwo{} WCA lamp and target XD locations for all NUV spectroscopic exposures are given in Table~\ref{tab:XDdataNUV}.
As shown in the two rightmost ``$|\Delta|$'' columns, all measured WCA-to-PSA offsets were within 3~p in XD of their FSW values. This equates to a $<$ 0.07\arcsec{} XD offset due to TA
for all NUV monitoring exposures over C20--24.
A visual inspection of the spectra showed all NUV spectra to continue to be well centered in the \tacq{PEAKXD}, \tacq{PEAKD}, and \tacq{SEARCH} NUV spectroscopic subarrays.
%{\bf Note to reviewers: Table~\ref{tab:XDdataNUV} doesn't actually show the subarray check. This was just a visual check to
%make sure that the NUV spectrum was well contained in the subarray. If you think that a table comparing the XD line centers to
%the subarray edges is worthwhile, it can be easily incorporated.}\\

\begin{deluxetable}{rrrcrrrrrr}
\tablecolumns{10}
\tabletypesize{\footnotesize}
\tabcolsep 6 pt
\tablecaption{NUV Spectroscopic \texttt{ACQ/PEAKXD} Monitoring}\label{tab:XDdataNUV}
\tablehead{
\colhead{\textit{ROOTNAME}}&\colhead{\textit{DATE-OBS}}&\colhead{\textit{OPT\_ELEM}}&\colhead{LP}&
\colhead{WCA\tablenotemark{a}}&\colhead{PSA\tablenotemark{b}}&\colhead{WtP\tablenotemark{c}}&\colhead{eWtP\tablenotemark{d}}&\colhead{$|\Delta|$\tablenotemark{e}}&\colhead{$|\Delta|$\tablenotemark{f}}\\
\colhead{}&\colhead{}&\colhead{}&\colhead{}&\colhead{(p)}&\colhead{(p)}&\colhead{(p)}&\colhead{(p)}&\colhead{(p)}&\colhead{(\arcsec{}))}\\
\colhead{(1)}&\colhead{(2)} & \colhead{(3)}&\colhead{(4)} &
\colhead{(5)}&\colhead{(6)} & \colhead{(7)}&\colhead{(8)} &
\colhead{(9)}&\colhead{(10)}
}
\startdata
\toprule
\multicolumn{10}{c}{C21 (\pid{13526})}\\
\midrule
lcgq02i2q& 2014-11-17&  G185M& 1& 366.0 & 742.0 & 376.0 & 374.20&  1.80 &  0.04\\
lcgq02i4q& 2014-11-17&  G225M& 1& 370.0 & 747.0 & 377.0 & 374.60&  2.40 &  0.06\\
lcgq01r6q& 2014-11-19&  G285M& 1& 355.0 & 728.0 & 373.0 & 374.90&  1.90 &  0.04\\
lcgq01qlq& 2014-11-19&  G230L& 1& 374.0 & 748.0 & 374.0 & 373.40&  0.60 &  0.01\\
\midrule
\multicolumn{10}{c}{C22 (\pid{13972})}\\
\midrule
lcri02hqq& 2015-10-06&  G185M& 1& 367.0 & 742.0 & 375.0 & 374.20&  0.80 &  0.02\\
lcri02hoq& 2015-10-06&  G225M& 1& 371.0 & 747.0 & 376.0 & 374.60&  1.40 &  0.03\\
lcri01giq& 2015-10-06&  G285M& 1& 351.0 & 726.0 & 375.0 & 374.90&  0.10 &  $<$0.01\\
lcri01ggq& 2015-10-06&  G230L& 1& 374.0 & 747.0 & 373.0 & 373.40&  0.40 &  0.01\\
\midrule
\multicolumn{10}{c}{C23 (\pid{14440})}\\
\midrule
ld3702noq& 2016-10-19&  G185M& 1& 366.0 & 743.0 & 377.0 & 374.20&  2.80 &  0.07\\
ld3702nmq& 2016-10-19&  G225M& 1& 370.0 & 747.0 & 377.0 & 374.60&  2.40 &  0.06\\
ld3701hbq& 2016-10-18&  G285M& 1& 352.0 & 727.0 & 375.0 & 374.90&  0.10 &  $<$0.01\\
ld3701h9q& 2016-10-18&  G230L& 1& 375.0 & 748.0 & 373.0 & 373.40&  0.40 &  0.01\\
\midrule
\multicolumn{10}{c}{C24 (\pid{14857})}\\
\midrule
ldozbblwq& 2017-09-06&  G185M& 1& 366.0 & 743.0 & 377.0 & 374.20&  2.80 &  0.07\\
ldozbbluq& 2017-09-06&  G225M& 1& 370.0 & 747.0 & 377.0 & 374.60&  2.40 &  0.06\\
ldozbadzq& 2017-09-04&  G285M& 1& 352.0 & 727.0 & 375.0 & 374.90&  0.10 &  $<$0.01\\
ldozbadxq& 2017-09-04&  G230L& 1& 374.0 & 748.0 & 374.0 & 373.40&  0.60 &  0.01\\
\bottomrule
\enddata
\tablenotetext{a}{XD centroid of the WCA spectrum. For NUV spectra, this is the median calibration lamp location.}
\tablenotetext{b}{XD centroid of the target spectrum taken through the PSA, using the same centroid method as the WCA.}
\tablenotetext{c}{WtP is absolute value of XD location difference of measured WCA and PSA spectra ($WtP = |PSA-WCA|$)}
\tablenotetext{d}{eWtP = Expected WCA-to-PSA offset from FSW table \textsc{XimCalTargetOffset} (see Table~\ref{tab:wcatopsa}).}
\tablenotetext{e}{Offset of WtP from a perfectly centered target measured in XD rows.}
\tablenotetext{f}{Offset of WtP in arcseconds (\arcsec{}). Note that the platescales are different for each grating, as shown in Table~\ref{tab:FSWplatescales}.}
\tablecomments{All spectra taken at \texttt{FP-POS=3}. All verifications ($|\Delta| = |WtP-eWtp|$) easily exceeded both the $\pm$0.3\arcsec{} requirement and the $\pm$0.1\arcsec{} goal.}
\end{deluxetable}
\begin{deluxetable}{rrrcrrrrrr}
\tabletypesize{\footnotesize}
\tablecolumns{10}
\tabcolsep 6 pt
\tablecaption{FUV Spectroscopic \texttt{ACQ/PEAKXD} Monitoring}\label{tab:XDdataFUV}
\tablehead{
\colhead{\textit{ROOTNAME}}&\colhead{\textit{DATE-OBS}}&\colhead{\textit{OPT\_ELEM}}&\colhead{LP}&
\colhead{WCA\tablenotemark{a}}&\colhead{PSA\tablenotemark{b}}&\colhead{WtP\tablenotemark{c}}&\colhead{eWtP\tablenotemark{d}}&\colhead{$|\Delta|$\tablenotemark{e}}&\colhead{$|\Delta|$\tablenotemark{f}}\\
\colhead{}&\colhead{}&\colhead{}&\colhead{}&\colhead{(p)}&\colhead{(p)}&\colhead{(p)}&\colhead{(p)}&\colhead{(p)}&\colhead{(\arcsec))}\\
\colhead{(1)}&\colhead{(2)} &
\colhead{(3)}&\colhead{(4)} &
\colhead{(5)}&\colhead{(6)} &
\colhead{(7)}&\colhead{(8)} &
\colhead{(9)}&\colhead{(10)}
}
\startdata
\toprule
\multicolumn{10}{c}{C21 (\pid{13526})}\\
\midrule
lcgq01r8q& 2014-11-19&  G130M& 2& 602.15& 508.31& -93.84& -92.80&  1.04& 0.12 \\
lcgq01raq& 2014-11-19&  G140L& 2& 608.76& 513.48& -95.28& -93.50&  1.78& 0.20 \\
lcgq02i6q& 2014-11-17&  G160M& 2& 596.07& 503.35& -92.71& -91.80&  0.91& 0.11 \\
\midrule
\multicolumn{10}{c}{C22 (\pid{13972})}\\
\midrule
lcri01gkq& 2015-10-06&  G130M& 3& 537.32& 448.98& -88.34& -89.20&  0.86& 0.08 \\
lcri01h6q& 2015-10-06&  G140L& 3& 544.55& 457.36& -87.20& -85.70&  1.50& 0.15 \\
lcri02hsq& 2015-10-06&  G160M& 3& 531.78& 442.13& -89.65& -90.10&  0.45& 0.04 \\
\midrule
\multicolumn{10}{c}{C23 (\pid{14440})}\\
\midrule
ld3701hdq& 2016-10-18&  G130M& 3& 536.33& 447.41& -88.92& -89.20&  0.28&  0.03 \\
ld3701hfq& 2016-10-18&  G140L& 3& 543.43& 455.95& -87.48& -85.70&  1.78&  0.17 \\
ld3702nqq& 2016-10-19&  G160M& 3& 531.09& 440.96& -90.13& -90.10&  0.03&  0.00 \\
\midrule
\multicolumn{10}{c}{C24 (\pid{14857})}\\
\midrule
ldozbae1q& 2017-09-04&  G130M& 3& 535.26& 445.66& -89.61& -89.20&  0.41&  0.04 \\
ldozbae3q& 2017-09-04&  G140L& 3& 541.76& 454.25& -87.51& -85.70&  1.81&  0.18 \\
ldozbblyq& 2017-09-06&  G160M& 3& 530.84& 440.75& -90.09& -90.10&  0.01&  0.00 \\
\bottomrule
\enddata
\tablenotetext{a}{XD centroid of the WCA spectrum.For FUV spectra, this is mean lamp photon location.}
\tablenotetext{b}{XD centroid of the target spectrum taken through the PSA, using the same centroid method as the WCA.}
\tablenotetext{c}{WtP is absolute value of XD location difference of measured WCA and PSA spectra ($WtP = |PSA-WCA|$)}
\tablenotetext{d}{eWtP = Expected WCA-to-PSA offset from FSW table \textsc{XimCalTargetOffset} (see Table~\ref{tab:wcatopsa}).}
\tablenotetext{e}{Offset of WtP from a perfectly centered target measured in XD rows.}
\tablenotetext{f}{Offset of WtP in \arcsec{}. Note that the platescales are different for each FUV grating and LP, as shown in Table~\ref{tab:FSWplatescales}.}
\tablecomments{All spectra taken at \texttt{FP-POS=3}. All FUV verifications ($|\Delta|$ = $|WtP-eWtp|$) exceeded both the $\pm$0.3\arcsec{} requirement,
but spectra taken near LP2 lifetime end, and all G140L spectra, exceeded the $\pm$0.1\arcsec{} goal.}
\end{deluxetable}

% Here is the +/- 0.7" table if we want to put it in
% Note the POSTARG extra column at the end
%13526& lcgq02i8q& 2014-11-17&  G160M& 2& 596.45& 509.80& -86.66& -91.80&  5.14&  0.61 &	+0.7\\
%13526& lcgq02iaq& 2014-11-17&  G160M& 2& 596.35& 498.41& -97.94& -91.80&  6.14&  0.73 &	-0.7\\
%13972& lcri02huq& 2015-10-06&  G160M& 3& 531.48& 449.64& -81.84& -90.10&  8.26&  0.75 &	+0.7\\
%13972& lcri02hwq& 2015-10-06&  G160M& 3& 531.73& 434.22& -97.51& -90.10&  7.41&  0.67 &	-0.7\\
%14440& ld3702nsq& 2016-10-19&  G160M& 3& 531.00& 448.43& -82.57& -90.10&  7.53&  0.68 &	+0.7\\
%14440& ld3702nuq& 2016-10-19&  G160M& 3& 531.24& 433.36& -97.88& -90.10&  7.78&  0.70 &	-0.7\\
%14857& ldozbbm0q& 2017-09-06&  G160M& 3& 530.70& 448.18& -82.52& -90.10&  7.58&  0.68 &	+0.7\\
%14857& ldozbbm2q& 2017-09-06&  G160M& 3& 530.75& 433.39& -97.36& -90.10&  7.26&  0.66 &	-0.7\\

\input{fspVER.tex}
\clearpage
\section{Results}\label{sec:results}
The main results of the HST Cycle~21--24 COS TA monitoring program are as follows:
\begin{description}
\item{\bf SIAF:}{
	All COS NUV \texttt{ACQ/IMAGE}s~use identical SIAF entries ({\it LFPSA} or {\it LFBOA}).
	Previously, the exposures in the Cycle~23 FGS-to-SI Alignment program (14452) gave a good estimate of the accuracy of the existing NUV LP1 {\it LFPSA}/{\it LFBOA} SIAF entries
	as P14452 performed a PSA/MIRRORA \texttt{ACQ/IMAGE} on a target whose position was already determined by cross-calibration of the other HST Science Instruments (SI).
	For Cycle~23, data from P14452 indicated that the NUV SIAF entry was accurate to at least [AD,XD] = [0.02,0.08]$\arcsec$.\footnote{As determined from the initial pointing before the first COS \texttt{ACQ/IMAGE}~of the program.}
	No SIAF adjustments were identified as being needed for NUV (LP1) or FUV (LP3) from this program.\footnote{Long term SIAF monitoring is used to track any mechanical drift in the location of the COS aperture mechanism or any changes to the FGS-to-SI alignment that will need adjusting.
	The last such adjustment was in Cycle~22 (February 2, 2014), while COS FUV observations were at LP2. At this time, all COS entries (NUV and FUV) were adjusted in [V2,V3] by [0.077, -0.070]". }

}
\item{\bf TA Subarrays:} Visual inspection of NUV images, and a review of the photon lists of the NUV and FUV spectra, indicate that all TA subarrays are appropriately defined for Cycle~24 and no adjustments were necessary.
\item{\bf NUV Imaging TAs:}
	The COS \texttt{ACQ/IMAGE}~ tests in P14452 indicate that the centering achieved with a PSA/MIRRORB \texttt{ACQ/IMAGE}~is co-aligned with a PSA/MIRRORA \texttt{ACQ/IMAGE}~to within [AD,XD] $\approx [0.010,0.020]\arcsec$, with a measurement error of approximately $0.014\arcsec$.
	\texttt{ACQ/IMAGE}~ tests in P14857 reveal that BOA/MIRRORA is co-aligned with PSA/MIRRORB to within [AD,XD] $\approx [0.015,0.100]\arcsec$,
	\footnote{The larger XD alignment error is due to a frequent 1 aperture XD (XAPER) step mechanism position error (1 step ~ $0.048\arcsec$).}
	and that BOA/MIRRORB is co-aligned with BOA/MIRRORA to within [AD,XD] $\approx [0.007,0.062]\arcsec$.

	As shown in Figure~\ref{fig:FP}, P14587 obtained a `family portrait' of Cycle~24 wavelength calibration aperture (WCA) lamp images. These images of PtNe lamp light seen through the WCA
	are used during the LTAIMCAL portion of the LTAIMAGE (ACQ/IMAGE) TA FSW routine to locate the position of the aperture mechanism before centering the target.
	While COS TAs have used the PtNe\#2 lamp for all TAs since installation, images of both lamps (PtNe\#1 and PtNe\#2) are taken annually with both MIRRORs
	(MIRRORA and MIRRORB) to monitor the observed count rates. No changes were observed in the PtNe lamp count rates between Cycles~23 and 24.
	\clearpage
\item{\bf NUV Spectroscopic TAs:}
	The G285M and G230L WCA-to-PSA offsets were measured after a PSA/MIRRORB \texttt{ACQ/IMAGE}, and were within a XD offset of $0.020\arcsec$ of the FSW value for each grating.
	\footnote{Spectroscopic NUV WCA-to-PSA offsets are determined using a median photon lamp and/or target XD position in the appropriate subarray. The difference between the positions is compared to the FSW value, accounting for any measured offset in the preceding \texttt{ACQ/IMAGE}.}
	The G185M and G225M offsets were measured after a BOA/MIRRORA \texttt{ACQ/IMAGE}, and were measured to be within a XD offset of $0.070\arcsec$ and $0.060\arcsec$, respectively, of the FSW value.
	Spectroscopic TAs for all NUV gratings met both the $0.3\arcsec$ requirement and the $0.1\arcsec$ goal.
\item{\bf FUV Spectroscopic TAs:}
	The G130M and G140L WCA-to-PSA offsets were measured after the same PSA/MIRRORB \texttt{ACQ/IMAGE}~as the G285M and G230L observations.
	The measured offsets were determined to be offset from the FSW values by $\approx -0.030\arcsec$ and $-0.170\arcsec$, respectively, with a measurement error estimated at $0.070\arcsec$.
	The G160M offset was measured after the BOA/MIRRORA \texttt{ACQ/IMAGE}~used for the G185M and G225M observations. The G160M offset was determined to have a WCA-to-PSA XD offset of $-0.020 \pm 0.070\arcsec$ of the FSW WCA-to-PSA value.
footnote{Spectroscopic FUV WCA-to-PSA offsets are determined using a mean photon lamp and/or target XD position in the appropriate subarray. The difference between the positions is compared to the FSW value, accounting for any measured offset in the preceding \texttt{ACQ/IMAGE}.}
	Spectroscopic TAs for all FUV gratings met the $0.3\arcsec$ requirement and the G130M and G160M gratings achieved the $0.1\arcsec$ goal.

\end{description}

\vspace{-0.3cm}
\section{Conclusions.\label{theend} }
\vspace{-0.3cm}
Through constant monitoring, and periodic FSW, ground commanding, and operations updates,
HST+COS TA has performed remarkablely well during Cycles 21--24. The STScI Team thanks the
GSFC and STScI personal for there outstanding cooperation and contributions in these efforts

NUV detector background has been the biggest source of concern
for NUV TAs, while FUV gain-sag induced Y-walk and inherent detector geometric distortions
were the biggest concerns of FUV TAs at LP1--3. At FUV LP4, Y-walk will not be as big a concern as
the \numposone \tacq{PEAKXD} is not affected by either Y-walk or geometric distortions.

{\bf Notes to reviewers: This section will be continue to be completed as the review process continues.}

\clearpage
%%%%%%%%
%Acknowledgements
%%%%%%%%
\vspace{-0.3cm}
\section{Acknowledgements}
{\bf Notes to reviewers: This section will be completed as the review process continues.}
To be acknowledeged:
GFSC: Mike Kelly, Ben Teasdel, Olivia Lupie, Scott Swain,
STScI: John Bacinski, George Chapman, Merle Reinhart, James White, Sean Lockwood, Brian York, David Sahnow, Karla Peterson, Josh Goldberg.

\vspace{-0.3cm}
%%%%%%%%
%Change History
%%%%%%%%
\vspace{0.3cm}
%Put instrument, year, and ISR number
\section{Change History for COS ISR 2018-XX}\label{sec:History}
\vspace{0.3cm}
%Put publication date
Version 0.01: 30-March-2018 Original Draft Document for Review
Version 0.02: 03-April-2018 NUV Image Verification Section Debuts with  Minor tweeks to tables and text (no external comments)
{\bf Note to reviewers: I will be documenting updates here, until Version 1.0 is released, then the notes
will be moved to comments.}
%%%%%%%%
%References
%%%%%%%%
\vspace{0.3cm}
\section{References}\label{sec:References}
\vspace{0.3cm}

\small
Dixon, W.~V., et al. 2010, Cosmic Origins Instrument Handbook (IHB), Version 2.0 (Baltimore, STScI)\\
Keyes, T., \& Penton, S. 2010, COS ISR 2010-14 (v1) (HST+COS Target Acquisition Guidelines, Recommendations, and Interpretation)\\
Fix, M.~B., 2018, COS ISR 2018-03 (v1) (COS NUV Dark Monitor Summary)\\
Fox, A., at al. 2017, Cosmic Origins Spectosgraph Instrument Handbook (IHB), Version 9.0 (Baltimore, STScI)\\
Holland, S. T., et al. 2014, Cosmic Origins Spectrograph Instrument Handbook (IHB), Version 6.0 (Baltimore: STScI)\\
Penton, S,, 2001, COS-11-0024A, ``TAACOS: Phase I NUV Report''\\
Penton, S., 2001, COS-11-0014B, ``Recommended TA FSW and Operations Changes Based upon the TAACOS Phase I Reports for the FUV and NUV Channels"\\
Penton, S., 2002, COS-11-0016A, ``TAACOS: Phase I FUV Report''\\
Penton, S., \& Keyes, T., 2011, COS TIR 2010-03 (On-Orbit Target Acquisitions with HST+COS)\\
Penton, S., 2016, COS ISR 2016-09 (Cycle~22 HST+COS Target Acquisition Monitoring Summary (\pid{13972})\\
Penton, S., 2017, COS ISR 2017-18 (Cycle~23 HST+COS Target Acquisition Monitoring Summary (\pid{14440})\\
Penton, S., COS ISR 2018-{\bf XX} (HST+COS LP2 Target Acquisition Enabling (LENA3, \pid{12797})\\
Penton, S., COS ISR 2018-{\bf XX} (HST+COS LP3 Target Acquisition Enabling (FENA4, \pid{13636})\\
Penton, S. \& White, J. 2018, COS ISR 2018-{\bf XX} (HST+COS LP4 Target Acquisition Enabling (\pid{14907}\\
Roman-Duval, J., 2015, COS ISR 2015-02 (Summary of the COS Cycle 20 Calibration Program: \pid{13124})\\
Rose, S., et al., 2017, HST Cycle~25 Phase II Proposal Instructions (V25.0)\\
Sana, H., et. al., 2015, COS ISR 2015-06 (Summary of the COS Cycle 21 Calibration Program: \pid{13526}))\\
Smith, H., et al., 2004, ``Hubble Space Telescope Cosmic Origins Spectrograph Contract End Item (CEI) Specification'' (NASA STE-63, HST \#TM-025984) (2004)\\
Sonnetrucker, P., et. al., 2016, COS ISR 2016-03 (Summary of the COS Cycle 22 Calibration Program : \pid{13972}) \\
Taylor, J., 2017, COS ISR 2017-13 (v1) (Cycle 23 COS/NUV Spectroscopic Sensitivity Monitor)\\
\normalsize
\newpage
\clearpage
%%%%%%%%
%Appendix
%%%%%%%%
%\vspace{-0.3cm}
\section{[OPTIONAL]Appendix A}\label{sec:Appendix}
{\bf Note To Reviewer: If You Think That A Complete Listing Of All TA FSW Parameters And Tables Is Appropriate, I Am Willing To Include These Here.
Here is a sample table, that is current referenced.}
% $Id$
%pcmech_ApMXDispPosition
%% $Id$
%pcmech_ApMXDispPosition
%% $Id$
%pcmech_ApMXDispPosition
%\input{pcmechApMXDispPosition}

\begin{deluxetable}{ccccc}
\tablecolumns{5}
\tablewidth{5 in}
\tablecaption{Cross-Dispersion (XD) Aperture Positions (\textit{APERXPOS})\label{tab:ApMXDispPosition}}
\tablehead{
\colhead{\textit{LIFE\_ADJ}} &    \multicolumn{2}{c}{NUV} & \multicolumn{2}{c}{FUV} \\
\colhead{(LP)} & \colhead{PSA\tablenotemark{a}$/$WCA\tablenotemark{b}} & \colhead{BOA\tablenotemark{c}$/$FCA\tablenotemark{d}} & \colhead{PSA$/$WCA} & \colhead{BOA$/$FCA} \\
\colhead{(1)}&\colhead{(2)} & \colhead{(3)}&\colhead{(4)} & \colhead{(5)}
}
\startdata
\toprule
LP1 &  126	&	-153 	& 126	&	153\\
LP2 &  53	&	-226 	& \dots	&	\dots\\
LP3 &  181	&	 -98	& \dots	&	\dots\\
LP4 &  234	&	 -45 	& \dots	&	\dots\\
\bottomrule
\enddata
\footnotesize
\tablenotetext{a}{PSA=Primary Science Aperture}
\tablenotetext{b}{WCA=Wavelength Calibration Aperture}
\tablenotetext{c}{BOA=Bright Object Aperture}
\tablenotetext{d}{FCA=Flat-field Calibration Aperture}
\tablecomments{COS XD aperture positions (\textit{APERXPOS}) are stored in the \textsc{pcmech\_ApMXDispPosition} FSW table. Although LP1-8 are defined in that table for both NUV and FUV, only the NUV LP1 and FUV LP1--4 entries listed here have been used for science observations.
Values used for FCA calibration observations are different from those listed here, and are commanded via APT special commanding (e.g., during the semi-annual FUV Gain Map programs, {\bf REFERENCE}).
Along-Dispersion (AD) values (\textit{APERYPOS}) are stored in the \textsc{pcmech\_ApMDispPosition} FSW table. All COS apertures and LPs use \textit{APERYPOS=22}. }
\normalsize
\end{deluxetable}


\begin{deluxetable}{ccccc}
\tablecolumns{5}
\tablewidth{5 in}
\tablecaption{Cross-Dispersion (XD) Aperture Positions (\textit{APERXPOS})\label{tab:ApMXDispPosition}}
\tablehead{
\colhead{\textit{LIFE\_ADJ}} &    \multicolumn{2}{c}{NUV} & \multicolumn{2}{c}{FUV} \\
\colhead{(LP)} & \colhead{PSA\tablenotemark{a}$/$WCA\tablenotemark{b}} & \colhead{BOA\tablenotemark{c}$/$FCA\tablenotemark{d}} & \colhead{PSA$/$WCA} & \colhead{BOA$/$FCA} \\
\colhead{(1)}&\colhead{(2)} & \colhead{(3)}&\colhead{(4)} & \colhead{(5)}
}
\startdata
\toprule
LP1 &  126	&	-153 	& 126	&	153\\
LP2 &  53	&	-226 	& \dots	&	\dots\\
LP3 &  181	&	 -98	& \dots	&	\dots\\
LP4 &  234	&	 -45 	& \dots	&	\dots\\
\bottomrule
\enddata
\footnotesize
\tablenotetext{a}{PSA=Primary Science Aperture}
\tablenotetext{b}{WCA=Wavelength Calibration Aperture}
\tablenotetext{c}{BOA=Bright Object Aperture}
\tablenotetext{d}{FCA=Flat-field Calibration Aperture}
\tablecomments{COS XD aperture positions (\textit{APERXPOS}) are stored in the \textsc{pcmech\_ApMXDispPosition} FSW table. Although LP1-8 are defined in that table for both NUV and FUV, only the NUV LP1 and FUV LP1--4 entries listed here have been used for science observations.
Values used for FCA calibration observations are different from those listed here, and are commanded via APT special commanding (e.g., during the semi-annual FUV Gain Map programs, {\bf REFERENCE}).
Along-Dispersion (AD) values (\textit{APERYPOS}) are stored in the \textsc{pcmech\_ApMDispPosition} FSW table. All COS apertures and LPs use \textit{APERYPOS=22}. }
\normalsize
\end{deluxetable}


\begin{deluxetable}{ccccc}
\tablecolumns{5}
\tablewidth{5 in}
\tablecaption{Cross-Dispersion (XD) Aperture Positions (\textit{APERXPOS})\label{tab:ApMXDispPosition}}
\tablehead{
\colhead{\textit{LIFE\_ADJ}} &    \multicolumn{2}{c}{NUV} & \multicolumn{2}{c}{FUV} \\
\colhead{(LP)} & \colhead{PSA\tablenotemark{a}$/$WCA\tablenotemark{b}} & \colhead{BOA\tablenotemark{c}$/$FCA\tablenotemark{d}} & \colhead{PSA$/$WCA} & \colhead{BOA$/$FCA} \\
\colhead{(1)}&\colhead{(2)} & \colhead{(3)}&\colhead{(4)} & \colhead{(5)}
}
\startdata
\toprule
LP1 &  126	&	-153 	& 126	&	153\\
LP2 &  53	&	-226 	& \dots	&	\dots\\
LP3 &  181	&	 -98	& \dots	&	\dots\\
LP4 &  234	&	 -45 	& \dots	&	\dots\\
\bottomrule
\enddata
\footnotesize
\tablenotetext{a}{PSA=Primary Science Aperture}
\tablenotetext{b}{WCA=Wavelength Calibration Aperture}
\tablenotetext{c}{BOA=Bright Object Aperture}
\tablenotetext{d}{FCA=Flat-field Calibration Aperture}
\tablecomments{COS XD aperture positions (\textit{APERXPOS}) are stored in the \textsc{pcmech\_ApMXDispPosition} FSW table. Although LP1-8 are defined in that table for both NUV and FUV, only the NUV LP1 and FUV LP1--4 entries listed here have been used for science observations.
Values used for FCA calibration observations are different from those listed here, and are commanded via APT special commanding (e.g., during the semi-annual FUV Gain Map programs, {\bf REFERENCE}).
Along-Dispersion (AD) values (\textit{APERYPOS}) are stored in the \textsc{pcmech\_ApMDispPosition} FSW table. All COS apertures and LPs use \textit{APERYPOS=22}. }
\normalsize
\end{deluxetable}

%% RCSID="$Id: FSWappendix.tex,v 1.1 2018/03/29 19:16:24 penton Exp $"
\section{Appendix A : FSW TA Tables}\label{sec:Appendix}
% $Id$
%pcmech_ApMXDispPosition
%% $Id$
%pcmech_ApMXDispPosition
%\input{pcmechApMXDispPosition}

\begin{deluxetable}{ccccc}
\tablecolumns{5}
\tablewidth{5 in}
\tablecaption{Cross-Dispersion (XD) Aperture Positions (\textit{APERXPOS})\label{tab:ApMXDispPosition}}
\tablehead{
\colhead{\textit{LIFE\_ADJ}} &    \multicolumn{2}{c}{NUV} & \multicolumn{2}{c}{FUV} \\
\colhead{(LP)} & \colhead{PSA\tablenotemark{a}$/$WCA\tablenotemark{b}} & \colhead{BOA\tablenotemark{c}$/$FCA\tablenotemark{d}} & \colhead{PSA$/$WCA} & \colhead{BOA$/$FCA} \\
\colhead{(1)}&\colhead{(2)} & \colhead{(3)}&\colhead{(4)} & \colhead{(5)}
}
\startdata
\toprule
LP1 &  126	&	-153 	& 126	&	153\\
LP2 &  53	&	-226 	& \dots	&	\dots\\
LP3 &  181	&	 -98	& \dots	&	\dots\\
LP4 &  234	&	 -45 	& \dots	&	\dots\\
\bottomrule
\enddata
\footnotesize
\tablenotetext{a}{PSA=Primary Science Aperture}
\tablenotetext{b}{WCA=Wavelength Calibration Aperture}
\tablenotetext{c}{BOA=Bright Object Aperture}
\tablenotetext{d}{FCA=Flat-field Calibration Aperture}
\tablecomments{COS XD aperture positions (\textit{APERXPOS}) are stored in the \textsc{pcmech\_ApMXDispPosition} FSW table. Although LP1-8 are defined in that table for both NUV and FUV, only the NUV LP1 and FUV LP1--4 entries listed here have been used for science observations.
Values used for FCA calibration observations are different from those listed here, and are commanded via APT special commanding (e.g., during the semi-annual FUV Gain Map programs, {\bf REFERENCE}).
Along-Dispersion (AD) values (\textit{APERYPOS}) are stored in the \textsc{pcmech\_ApMDispPosition} FSW table. All COS apertures and LPs use \textit{APERYPOS=22}. }
\normalsize
\end{deluxetable}


\begin{deluxetable}{ccccc}
\tablecolumns{5}
\tablewidth{5 in}
\tablecaption{Cross-Dispersion (XD) Aperture Positions (\textit{APERXPOS})\label{tab:ApMXDispPosition}}
\tablehead{
\colhead{\textit{LIFE\_ADJ}} &    \multicolumn{2}{c}{NUV} & \multicolumn{2}{c}{FUV} \\
\colhead{(LP)} & \colhead{PSA\tablenotemark{a}$/$WCA\tablenotemark{b}} & \colhead{BOA\tablenotemark{c}$/$FCA\tablenotemark{d}} & \colhead{PSA$/$WCA} & \colhead{BOA$/$FCA} \\
\colhead{(1)}&\colhead{(2)} & \colhead{(3)}&\colhead{(4)} & \colhead{(5)}
}
\startdata
\toprule
LP1 &  126	&	-153 	& 126	&	153\\
LP2 &  53	&	-226 	& \dots	&	\dots\\
LP3 &  181	&	 -98	& \dots	&	\dots\\
LP4 &  234	&	 -45 	& \dots	&	\dots\\
\bottomrule
\enddata
\footnotesize
\tablenotetext{a}{PSA=Primary Science Aperture}
\tablenotetext{b}{WCA=Wavelength Calibration Aperture}
\tablenotetext{c}{BOA=Bright Object Aperture}
\tablenotetext{d}{FCA=Flat-field Calibration Aperture}
\tablecomments{COS XD aperture positions (\textit{APERXPOS}) are stored in the \textsc{pcmech\_ApMXDispPosition} FSW table. Although LP1-8 are defined in that table for both NUV and FUV, only the NUV LP1 and FUV LP1--4 entries listed here have been used for science observations.
Values used for FCA calibration observations are different from those listed here, and are commanded via APT special commanding (e.g., during the semi-annual FUV Gain Map programs, {\bf REFERENCE}).
Along-Dispersion (AD) values (\textit{APERYPOS}) are stored in the \textsc{pcmech\_ApMDispPosition} FSW table. All COS apertures and LPs use \textit{APERYPOS=22}. }
\normalsize
\end{deluxetable}


%\vspace{-0.3cm}
\end{document}
