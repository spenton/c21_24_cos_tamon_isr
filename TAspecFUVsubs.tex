\subsection{COS FUV Spectroscopic TA Subarrays}\label{subsec:FUVsupSUBS}
The FUV spectroscopic TA subarrays for the WCA are the same for \tacq{SEARCH},  \tacq{PEAKD}, and \tacq{PEAKXD}
and are given in Table~\ref{tab:TAsubWCAfuv} for both FUVA and FUVB.
Only one subarray is used for the WCA for each FUV segment, these are labeled `A1' and `B1'.
As the data are taken in ``detector'' coordinates, all FUV TA subarrays values are valid only for the normal operating temperature range of COS. FUVB is not used in G140L TAs.

The FUV spectroscopic subarrays used for all external targets at LP1--4 for FUVA are given in Table~\ref{tab:FUVsubA} and for FUVB in Table~\ref{tab:FUVsubB}.
There are two subarrays used for each FUV segment, these are labeled `A1', `A2', `B1', and `B2'.
The COS FSW uses the same subarrays for the PSA and BOA as the offset between the FUV spectra is small ($\Delta$~XD$\sim$3p).
As with the other HST spectrographs, FUV TAs are susceptible to contamination from geocoronal light , particularly Ly$\alpha$ 1216\AA, {\rm O}\textsc{I} 1302\AA, and {\rm Si}{\sc II}1304\AA\ (Penton \& Keyes, 2010).
The FUV TA subarrays outlined in tables~\ref{tab:FUVsubA} and \ref{tab:FUVsubB} have been tailored to remove regions
of the target spectrum that may contain Geocoronal light.
The Geocoronal light fills the aperture and has a very different XD profile which could cause problems with FUV TAs.

In 2014--5, several ``hot-spots'' appeared during solar maximum. On April 20, 2015 (2015.110) with STScI PR\#80571, the FUV LP3 subarrays were adjusted to avoid these hot-spots.
Details are given in \S~\ref{subsec:hotspots}, and the adjusted FUVB subarrays are also given in Table~\ref{tab:FUVsubB}.

\begin{deluxetable}{r|rrrr|rrrr}
\tablewidth{0pt}
\tabcolsep 10 pt
\tabletypesize{\scriptsize}
\tablecolumns{9}
\tablecaption{FUV WCA Subarrays for LP1--4\label{tab:TAsubWCAfuv}.}
\tablehead{
\colhead{} & \multicolumn{4}{c}{A1 Subarray} & \multicolumn{4}{c}{B1 Subarray} \\
\colhead{\textit{OPT\_ELEM}} & \colhead{XC} & \colhead{YC} & \colhead{XS} & \colhead{YS} & \colhead{XC} & \colhead{YC} & \colhead{XS} & \colhead{YS}\\
\colhead{(1)}&\colhead{(2)} & \colhead{(3)}&\colhead{(4)} &
\colhead{(5)}&\colhead{(6)} & \colhead{(7)}&\colhead{(8)} & \colhead{(9)}
}
\startdata
	\hline
	\multicolumn{9}{c}{LP1}\\
	\hline
	G130M & 1201 & 541\tablenotemark{a} & 13799                   & 44 & 1501  & 585                 & 13799 & 44\\
	G160M & 1201 & 535\tablenotemark{a} & 13799                   & 44 & 1501  & 579\tablenotemark{a}& 13799 & 44\\
	G140L & 1201 & 547\tablenotemark{a} & 13799                   & 44 &\dots&\dots&\dots&\dots\\
	G140L & 4701 & 547\tablenotemark{b} & 10299\tablenotemark{b}  & 44 &\dots&\dots&\dots&\dots\\
	\hline
	\multicolumn{9}{c}{LP2\tablenotemark{c}}\\
	\hline
	G130M & 1201 & 581 & 13799  & 44 & 1501 & 630 & 13799  & 44 \\
	G160M & 1201 & 568 & 13799  & 44 & 1501 & 617 & 13799  & 44 \\
	G140L & 4701 & 587 & 10299  & 44 &\dots&\dots&\dots&\dots\\
	\hline
	\multicolumn{9}{c}{LP3\tablenotemark{d}}\\
	\hline
	G130M & 1201  & 515 & 13799 & 44 & 1501  & 567 & 13799 & 44\\
	G160M & 1201  & 504 & 13799 & 44 & 1501  & 559 & 13799 & 44\\
	G140L & 4701  & 521 & 10299 & 44 &\dots&\dots&\dots&\dots\\
	\hline
	\multicolumn{9}{c}{LP4\tablenotemark{e}}\\
	\hline
	G130M & 1201 & 483 & 13799 & 52 & 1501 &  539 & 13799 & 52 \\
	G160M & 1201 & 475 & 13799 & 52 & 1501 &  534 & 13799 & 52 \\
	G140L & 4701 & 491 & 10299 & 52 &\dots&\dots&\dots&\dots\\
\enddata
\tablenotetext{a}{These values were updated on 2009.201 (July 20, 2009) with STScI PR\#63095, some very early COS calibration and ERO datasets used slightly different TA subarrays.}
\tablenotetext{b}{Additional G140L updates were made on Dec. 4, 2012 (2012.339) with STScI PR\#72193 to futher optimize the G140L subarrays.}
\tablenotetext{c}{Updated for LP2 operations on July 18, 2012 (2012.200) with STScI PR\#70548.}
\tablenotetext{d}{Updated for LP3 operations on Aug. 26, 2014 (2014.238) with STScI PR\#78747.}
\tablenotetext{e}{Updated for LP4 operations on Feb. 20, 2017 (2017.051) with STScI PR\#86945.}
\end{deluxetable}

\begin{deluxetable}{lc|rrrr|rrrr}
\tablewidth{0pt}
\tabletypesize{\scriptsize}
\tabcolsep 4 pt
\tablecolumns{10}
\tablecaption{FUVA PSA/BOA Subarrays for LP1--4\label{tab:FUVsubA}}
\tablehead{
\colhead{\textit{OPT\_ELEM}} & \colhead{\cenwave} & \multicolumn{4}{c}{A1 Subarray} & \multicolumn{4}{c}{A2 Subarray}\\
\colhead{} & \colhead{(\AA)} &
\colhead{XC} & \colhead{YC} & \colhead{XS} & \colhead{YS} & \colhead{XC} & \colhead{YC} &\colhead{XS} & \colhead{YS}\\
\colhead{(1)}&\colhead{(2)} & \colhead{(3)}&\colhead{(4)} &
\colhead{(5)}&\colhead{(6)} & \colhead{(7)}&\colhead{(8)} &
\colhead{(9)}&\colhead{(10)}
}
\startdata
\hline
\multicolumn{10}{c}{LP1}\\
\hline
G130M & 1291 & 1201& 6555\tablenotemark{b}   & 437\tablenotemark{a}& 76  & 4078 & 8896\tablenotemark{b} & 437\tablenotemark{a} & 76\\
G130M & 1300 & 1201& 7559\tablenotemark{b}   & 437\tablenotemark{a}& 76  & 4078 & 9900\tablenotemark{b} & 437\tablenotemark{a} & 76\\
G130M & 1309 & 1201& 8562\tablenotemark{b}   & 437\tablenotemark{a}& 76  & 4097\tablenotemark{b}  & 10903\tablenotemark{b} & 437\tablenotemark{a} & 76 \\
G130M & 1318 & 1201& 9465\tablenotemark{b}   & 437\tablenotemark{a}& 76  & 3194\tablenotemark{b}  & 11806\tablenotemark{b} & 437\tablenotemark{a} & 76 \\
G130M & 1327 & 1201& 10489\tablenotemark{b}  & 437\tablenotemark{a}& 76  & 2170\tablenotemark{b}  & 12830\tablenotemark{b} & 437\tablenotemark{a} & 76 \\
G160M & ALL  & 1201& 13799 & 432\tablenotemark{a,b} & 76 &\dots&\dots&\dots&\dots\\
G140L & 1105 & 1201& 10458\tablenotemark{c} & 445\tablenotemark{a,b}& 76  & 457  & 14543 & 445\tablenotemark{a,b} & 76\\
G140L & 1230\tablenotemark{g} & 1201& 12216\tablenotemark{c} & 445\tablenotemark{a,b}& 76  &\dots&\dots&\dots&\dots\\
G140L & 1280 & 1201& 12216\tablenotemark{c} & 445\tablenotemark{a,b}& 76  &\dots&\dots&\dots&\dots\\
\hline
G140L & 1105 & 4701\tablenotemark{c}& 6958\tablenotemark{c} & 445\tablenotemark{a,b} & 76  & 457 & 14543 & 445\tablenotemark{a,b} & 76\\
G140L & 1230\tablenotemark{g} & 4701\tablenotemark{c}&8716\tablenotemark{c}& 445\tablenotemark{a,b}   & 76  &\dots&\dots&\dots&\dots\\
G140L & 1280 & 6201\tablenotemark{c}&7400\tablenotemark{c} & 445\tablenotemark{a,b}  & 76  &\dots&\dots&\dots&\dots\\
\hline
\multicolumn{10}{c}{LP2\tablenotemark{d}}\\
\hline
G130M & 1291 & 1201 & 472 & 6555 & 76 & 8896 & 472 & 4078 & 76\\
G130M & 1300 & 1201 & 472 & 7559 & 76 & 9900 & 472 & 4078 & 76\\
G130M & 1309 & 1201 & 472 & 8562 & 76 & 10903 & 472 & 4097 & 76\\
G130M & 1318 & 1201 & 472 & 9465 & 76 & 11806 & 472 & 3194 & 76\\
G130M & 1327 & 1201 & 472 & 10489 & 76 & 12830 & 472 & 2170 & 76\\
G160M & ALL & 1201 & 466 & 13799 & 76 &\dots&\dots&\dots& \dots\\
G140L & 1105 & 4701 & 479 & 6958 & 76 & 14543 & 479 & 457 & 76\\
G140L & 1280 & 6201 & 479 & 7400 & 76 &\dots& 34 &\dots&\dots\\
G140L & 1105 & 4701 & 479 & 6958 & 76 & 14543 & 479 & 457 & 76\\
G140L & 1280 & 6201 & 479 & 7400 & 76 &\dots&\dots&\dots&\dots\\
\hline
\multicolumn{10}{c}{LP3\tablenotemark{e}}\\
\hline
G130M & 1291 & 1201 & 409 & 6555 & 76 & 8896 & 409 & 4078 & 76\\
G130M & 1300 & 1201 & 409 & 7559 & 76 & 9900 & 409 & 4078 & 76\\
G130M & 1309 & 1201 & 409 & 8562 & 76 & 10903 & 409 & 4097 & 76\\
G130M & 1318 & 1201 & 409 & 9465 & 76 & 11806 & 409 & 3194 & 76\\
G130M & 1327 & 1201 & 409 & 10489 & 76 & 12830 & 409 & 2170 & 76\\
G160M & ALL & 1201 & 403 & 13799 & 76 &\dots&\dots&\dots& \dots\\
G140L & 1105 & 4701 & 418 & 6958 & 76 & 14543 & 418 & 457 & 76\\
G140L & 1280 & 6201 & 418 & 7400 & 76 &\dots&\dots&\dots& \dots\\
\hline
\multicolumn{10}{c}{LP4\tablenotemark{f}}\\
\hline
G130M & 1291 & 1201 & 362 & 6555 & 112 & 8896 & 362 & 4078 & 112\\
G130M & 1300 & 1201 & 362 & 7559 & 112 & 9900 & 362 & 4078 & 112\\
G130M & 1309 & 1201 & 362 & 8562 & 112 & 10903 & 362 & 4097 & 112\\
G130M & 1318 & 1201 & 362 & 9465 & 112 & 11806 & 362 & 3194 & 112\\
G130M & 1327 & 1201 & 362 & 10489 & 112 & 12830 & 362 & 2170 & 112\\
G160M & ALL  & 1201 & 356 & 13799 & 112 &\dots&\dots&\dots&\dots\\
G140L & 1105 & 4701 & 372 & 6958 & 112 & 14543 & 372 & 457 & 112\\
G140L & 1280 & 6201 & 372 & 7400 & 112 &\dots&\dots&\dots& \dots
\enddata
\tablenotetext{a}{Updated on 2009.201 (July 20, 2009) with STScI PR\#63095, some very early COS calibration and ERO datasets used slightly different TA subarrays.}
\tablenotetext{b}{Updated early in LP1 on Aug 27, 2009 (2009.2239) with STScI PR\#63378.}
\tablenotetext{c}{Additional G140L updates were made on Dec. 4, 2012 (2012.339) with STScI PR\#72193 to futher optimize the G140L subarrays.}
\tablenotetext{d}{Updated for LP2 July 18, 2012 (2012.200) with STScI PR\#70548.}
\tablenotetext{e}{Updated for LP3 operations on Aug. 26, 2014 (2014.238) with STScI PR\#78747.}
\tablenotetext{f}{Updated for LP4 operations on Feb. 20, 2017 (2017.051) with STScI PR\#86945.}
\tablenotetext{g}{Starting with C18, the C1230 \textit{CENWAVE} was replaced with C1280 due to first-order light issues.
For further details see the C18 COS Instrument Handbook (Dixon et al., 2010) (STScI PR\#64041 and \#64659).}
\end{deluxetable}

\begin{deluxetable}{lc|rrrr|rrrr}
\tablewidth{0pt}
\tabcolsep 9pt
\tablecolumns{10}
\tabletypesize{\scriptsize}
\tablecaption{FUVB PSA/BOA Subarrays for LP1--4\label{tab:FUVsubB}.}
\tablehead{
\colhead{\textit{OPT\_ELEM}} & \colhead{\cenwave} & \multicolumn{4}{c}{B1 Subarray} & \multicolumn{4}{c}{B2 Subarray} \\
\colhead{ } & \colhead{(\AA)} &
\colhead{XC} & \colhead{YC} & \colhead{XS} & \colhead{YS} & \colhead{XC} & \colhead{YC} & \colhead{XS} & \colhead{YS}\\
\colhead{(1)}&\colhead{(2)} & \colhead{(3)}&\colhead{(4)} &
\colhead{(5)}&\colhead{(6)} & \colhead{(7)}&\colhead{(8)} & \colhead{(9)}&\colhead{(10)}
}
\startdata
\multicolumn{10}{c}{LP1\tablenotemark{c}}\\
\hline
G130M & 1291 & 5036\tablenotemark{b} & 76 & 1501 & 483 & 7477\tablenotemark{b} & 76 & 7773\tablenotemark{b} & 483\tablenotemark{a,b} \\
G130M & 1300 & 6039\tablenotemark{b} & 76 & 1501 & 483 & 6474\tablenotemark{b} & 76 & 8776\tablenotemark{b} & 483\tablenotemark{a,b} \\
G130M & 1309 & 7023\tablenotemark{b} & 76 & 1501 & 483 & 5490\tablenotemark{a} & 76 & 9760\tablenotemark{a} & 483\tablenotemark{a,b} \\
G130M & 1318 & 7977\tablenotemark{b} & 76 & 1501 & 483 & 4536\tablenotemark{b} & 76 & 10714\tablenotemark{b} & 483\tablenotemark{a,b} \\
G130M & 1327 & 7629\tablenotemark{b} & 76 & 2792\tablenotemark{b} & 483 & 3593\tablenotemark{b} & 76 & 11657\tablenotemark{b} & 483\tablenotemark{a,b} \\
G160M & ALL  & 13749 & 76 & 1501 & 477\tablenotemark{a,b} &\dots&\dots&\dots&\dots\\
G140L & 1105 &\dots&\dots&\dots&\dots&\dots&\dots&\dots&\dots\\
G140L & 1230\tablenotemark{g} &\dots&\dots&\dots&\dots&\dots&\dots&\dots&\dots\\
G140L & 1280 &\dots&\dots&\dots&\dots&\dots&\dots&\dots&\dots\\
\hline
\multicolumn{10}{c}{LP2\tablenotemark{c}}\\
\hline
G130M & 1291 & 1501 & 522 & 5036 & 76 & 7773 & 522 & 7477 & 76\\
G130M & 1300 & 1501 & 522 & 6039 & 76 & 8776 & 522 & 6474 & 76\\
G130M & 1309 & 1501 & 522 & 7023 & 76 & 9760 & 522 & 5490 & 76\\
G130M & 1318 & 1501 & 522 & 7977 & 76 & 10714 & 522 & 4536 & 76\\
G130M & 1327 & 2792 & 522 & 7629 & 76 & 11657 & 522 & 3593 & 76\\
G160M & ALL & 1501 & 515 & 13749 & 76        &\dots&\dots&\dots&\dots\\
G140L & 1105 &\dots&\dots&\dots&\dots&\dots&\dots&\dots&\dots\\
G140L & 1280 &\dots&\dots&\dots&\dots&\dots&\dots&\dots&\dots\\
\hline
\multicolumn{10}{c}{LP3\tablenotemark{d}~~(Pre FUVB ``Hot-Spot'')}\\
\hline
G130M & 1291 & 1501 & 460 & 5036 & 76 & 7773 & 460 & 7477 & 76\\
G130M & 1300 & 1501 & 460 & 6039 & 76 & 8776 & 460 & 6474 & 76\\
G130M & 1309 & 1501 & 460 & 7023 & 76 & 9760 & 460 & 5490 & 76\\
G130M & 1318 & 1501 & 460 & 7977 & 76 & 10714 & 460 & 4536 & 76\\
G130M & 1327 & 2792 & 460 & 7629 & 76 & 11657 & 460 & 3593 & 76\\
G160M & ALL & 1501 & 453 & 13749 & 76        &\dots&\dots&\dots&\dots\\
G140L & 1105 &\dots&\dots&\dots&\dots&\dots&\dots&\dots&\dots\\
G140L & 1280 &\dots&\dots&\dots&\dots&\dots&\dots&\dots&\dots\\
\hline
\multicolumn{10}{c}{LP3\tablenotemark{e}~~(Post FUVB ``Hot-Spot'')\tablenotemark{d}}\\
\hline
G130M & 1291 & 1501 & 460 & 5036 & 76 & 7773 & 460 & 7060\tablenotemark{e} & 76\\
G130M & 1300 & 1501 & 460 & 6039 & 76 & 8776 & 460 & 6057\tablenotemark{e} & 76\\
G130M & 1309 & 1501 & 460 & 7023 & 76 & 9760 & 460 & 5073\tablenotemark{e} & 76\\
G130M & 1318 & 1501 & 460 & 7977 & 76 & 10714 & 460 & 4119\tablenotemark{e} & 76\\
G130M & 1327 & 2792 & 460 & 7629 & 76 & 11657 & 460 & 3176\tablenotemark{e} & 76\\
G160M & ALL & 1501 & 453 & 13332 & 76        &\dots&\dots&\dots&\dots\\
G140L & 1105 &\dots&\dots&\dots&\dots&\dots&\dots&\dots&\dots\\
G140L & 1280 &\dots&\dots&\dots&\dots&\dots&\dots&\dots&\dots\\
\hline
\multicolumn{10}{c}{LP4\tablenotemark{f}}\\
\hline
G130M & 1291 & 1501 & 419 &  5036 & 112 &  7773 & 419 & 7060 & 112 \\
G130M & 1300 & 1501 & 419 &  6039 & 112 &  8776 & 419 & 6057 & 112 \\
G130M & 1309 & 1501 & 419 &  7023 & 112 &  9760 & 419 & 5073 & 112 \\
G130M & 1318 & 1501 & 419 &  7977 & 112 & 10714 & 419 & 4119 & 112 \\
G130M & 1327 & 2792 & 419 &  7629 & 112 & 11657 & 419 & 3176 & 112 \\
G160M &  ALL & 1501 & 416 & 13332 & 112      &\dots&\dots&\dots&\dots\\
G140L & 1105 &\dots&\dots&\dots&\dots&\dots&\dots&\dots&\dots\\
G140L & 1280 &\dots&\dots&\dots&\dots&\dots&\dots&\dots&\dots\\
\enddata
\tablenotetext{a}{Updated during SMOV (2009.201) with STScI PR\#63095.}
\tablenotetext{b}{Updated for LP2 operations on July 18, 2012 (2012.200) with STScI PR\#70548.}
\tablenotetext{c}{Due to gain sag induced 'Y-walk', the use of the FUVB segment for \tacq{PEAKXD} (\texttt{NUM\_POS=1})
TAs was deprecated on April 8, 2011 (2011.098) with STScI PR\#67985. FUVB is still used for \tacq{SEARCH} and \tacq{PEAKD} TA exposures.}
\tablenotetext{d}{Updated for LP3 operations on Aug. 26, 2014 (2014.238) with STScI PR\#78747.}
\tablenotetext{e}{Updated for post ``Hot-Spot'' LP3 TA operations on April 20, 2015 (2015.110) with STScI PR\#80571.}
\tablenotetext{f}{Updated for LP4 operations on Feb. 20, 2017 (2017.051) with STScI PR\#86945. At LP4, both FUVA and FUVB are supported for all \tacq{PEAKXD} (\textit{NUM\_POS$>1$}) TA exposures. These subarrays also avoid the FUVB ``Hot-Spot''.}
\tablenotetext{g}{Starting with C18, the C1230 \textit{CENWAVE} was replaced with C1280 due to first-order light issues.
For further details see the C18 COS Instrument Handbook (Dixon et al., 2010) (STScI PR\#64041 and \#64659).}
\end{deluxetable}

\subsection{Trimming of COS FUV TA subarrays due to FUVB ``Hot-Spot''.}\label{subsec:hotspots}
A ``hot-spot'' appeared on the COS FUVB segment coincident with increased solar activity in 2014--15.
This spot produced enough counts that it could cause mis-centering during all phases of the FUV LP3 (\& LP4) spectroscopic TAs.
This mis-centerings could be in significant in either the AD or XD. All affected
LP3 FUVB TA subarrays were adjusted on April 20, 2015 (2015.110)\footnote{See STScI PR\#80571 for futher details.}.

In FUVB detector coordinates, the approximate location of the hot-spot is at [X,Y]=[14895,482]. As this is near the detector edge, we are able to avoid this hotspot by stopping the last subarray of the FUVB subarrays at X=14833.
For the COS FUV gratings and the FUVB TA subarrays, the impacts were:\\
\begin{description}
	\item[G140L:] Not affected as no FUVB TA subarrays are used for G140L
	\item[G160M:] One FUVB subarray is used for each \textit{CENWAVE} with XC1=1501, XS1=13749. These were all changed to XS=13332 (no change in Y).
	\item[G130M:] Two \textit{CENWAVE}-specific FUVB subarrays are used to avoid Geocoronal Ly$\alpha$. The X-size (XS) of the second subarray (XS2) will be trimmed to avoid the hotspot (XC1, XS1, XC2 and all the Y definitions do not change).
\end{description}

As of March 2018, no additional hot-spots have appeared on either FUVA or FUVB that required adjustment of the TA subarrays.\footnote{{\bf NOTE TO REVIEWER: IF YOU THINK THAT ADDITIONAL DETAILS ARE WARRANTED HERE, I CAN ADD MORE DETAILS.}}
Due to the possibility of future hot-spots, the number of allow FUV TA subarrays per segment was increased from two to four on  Sept 21, 2015 (2015.264) with STScI PR\#81263.
