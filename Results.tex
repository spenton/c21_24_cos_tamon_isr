% $Id: Results.tex,v 1.9 2018/04/18 04:10:05 penton Exp $
\section{Results and Conclusion}\label{sec:results}
The main results of the HST Cycles 20--24 COS TA monitoring programs are as follows:
\begin{description}
\item{\bf SIAF:}{
	All COS NUV \tacq{IMAGE}s use identically-valued SIAF entries ({\it LFPSA} \& {\it LFBOA}), although the changed twice over C18--C24.
	The exposures in the FGS-to-SI Alignment programs gave great estimates of the accuracy of the existing NUV LP1 {\it LFPSA}/{\it LFBOA} SIAF entries
	as they performed a PSA$\times$MIRA \tacq{IMAGE}.
	For C20--24, this results of this ISR indicate that the NUV SIAF entry was accurate to :
	\begin{itemize}
	\item 2009.215--2011.171 $\Delta$[AD,XD] $\sim$ [-0.169,-0.141]\arcsec{} ($\Delta$[V2,V3]=[-0.044,-0.215]\arcsec{}).
	\item 2011.172--2014.054 $\Delta$[AD,XD] $\sim$ [ 0.091,-0.042]\arcsec{} ($\Delta$[V2,V3]=[-0.071, 0.041]\arcsec{}),
	\item 2014.055--2016.276 $\Delta$[AD,XD] $\sim$ [-0.031,-0.027]\arcsec{} ($\Delta$[V2,V3]=[ 0.038, 0.065]\arcsec{}),
	\end{itemize}
	SIAF entry offsets affect the ``blind pointing'' of COS, but the COS TA modes are designed to center any target within the aperture to the center, so small offsets do not affect the final COS post-TA pointing.
	Long term SIAF monitoring is used to track any mechanical drift in the location of the COS aperture mechanism or any changes to the FGS-to-SI alignment that will need adjusting.
	The last such adjustment was in C22 (February 24, 2014; 2014.055, \pr{76982}), while COS FUV observations were at LP2. At this time, all COS entries (NUV and FUV) were adjusted in [V2,V3] by [0.077, -0.070]".
}
\item{\bf Spectroscopic TA Subarrays:} {
	Visual inspection of NUV and FUV images, and a comparison of the NUV and FUV spectra XD centroids, indicate that all spectroscopic TA subarrays were appropriately defined for C20--C24.
	However, NUV PtNe lamp (WCA) monitoring should be continued, as OSM1 and OSM2 secular drift continues to move the WCA lamp images in the AD direction, and XD aperture offsets are common, especially when switching to and from NUV and FUV.
	Combined with the increased detector background of the NUV channel, some of the approved NUV central wavelength settings for COS TA
	have loss viability, for further details see \S~2.6 of the C25 COS IHB (Fox et al., 2017).
	Hot-spot monitoring must be continued for both FUVA and FUVB as COS TAs are particularly susceptible to contamination from variable localized detector background.
}
\item{\bf NUV Imaging TAs and Subarrays:}{
	C20--C24 \tacq{IMAGE} tests indicate that the following average co-alignment between \tacq{IMAGE} configurations to PSA$\times$MIRA \tacq{IMAGE}~to within [AD,XD] $\approx [0.010,0.020]$\arcsec{}, with a measurement error of approximately $0.014$\arcsec{}.
	\begin{itemize}
	\item PSA$\times$MIRB was aligned with PSA$\times$MIRA to [AD, XD] $\sim$ [{\bf 0.002},{\bf 0.015}] $\pm$ {\bf 0.012}\arcsec.
	\item BOA$\times$MIRA was aligned with PSA$\times$MIRA to [AD, XD] $\sim$ [{\bf-0.021},{\bf 0.082}] $\pm$ {\bf 0.014}\arcsec.
	\item BOA$\times$MIRB was aligned with PSA$\times$MIRA to [AD, XD] $\sim$ [{\bf-0.016},{\bf 0.047}] $\pm$ {\bf 0.016}\arcsec.
	\end{itemize}
	Larger XD alignment errors due to a frequent $\pm 1$~p XD (\textit{APERYPOS}) step mechanism position errors (1 step ~ $0.053$\arcsec{}).
	As shown in the PtNe lamp `family portraits' of Figures~\ref{fig:FG21}--\ref{fig:FG24}, and used during the \textsc{LTAIMCAL} portion of \tacq{IMAGE},
	the COS PtNe lamps are still performing well although their positions on the detector must continue to be monitored as there is
	considerable AD ($\pm$ 50~p) and XD ($\pm$ 5~p) non-repeatability.
	No changes of concern were observed in the PtNe lamp count rates between C20--C24.
}
\item{\bf HST+COS TA Slew Accuracy:}{
	As determined from C17--C24 observations FGS-to-SI programs, slews commanded by \tacq{IMAGE}s move very close to their expected positions.
	The average measured offset was $\Delta$[AD,XD]=[-0.002,-0.006]\arcsec{}, with an RMS difference of [AD,XD]=[0.003,0.010]\arcsec{}..
}
\item{\bf NUV Spectroscopic TAs:}{
	Spectroscopic TAs for all NUV gratings in all Cycles met both the 0.3\arcsec{} requirement and the 0.1\arcsec{} goal.
}
\item{\bf FUV Spectroscopic TAs:}{
	C20--C24 spectroscopic TAs for all FUV gratings met the $0.3$\arcsec{} requirement and the G130M and G160M gratings achieved the 0.1\arcsec{} goal for C20--C24.
	However, spectra taken near the end of the LP2 lifetime, and all G140L spectra, did not achieve the $\pm 0.1$\arcsec{} goal.
}
\end{description}

Through constant monitoring, and periodic FSW, ground commanding, and operations updates,
HST+COS TA has performed remarkably well during Cycles 17--24. The STScI Team thanks the
GSFC and STScI personal for their outstanding cooperation and contributions in these efforts

NUV detector background has been the biggest source of concern
for NUV TAs, while FUV gain-sag induced Y-walk, hot-spots, and inherent detector geometric distortions
were the biggest concerns of FUV TAs at LP1--3. At FUV LP4, Y-walk will not be as big a concern as
the \numposone{} \tacq{PEAKXD} is not affected by either Y-walk or geometric distortions.

With continued monitoring, and occasional corrective actions, COS TAs should continue their excellent performance
in future HST Cycles.
