\subsection{SIAF Verification} \label{sec:siaf}\label{subsec:siafextra}
The pre-SM4 COS Science Instrument Aperture File is described in detail in Mallo 2008. The changes
since SMOV to the COS SIAF, and the continuing Fine Guidance Sensor (FGS) re-alignment efforts are
documented it Table~\ref{tab:siafhistory}. The COS entries in the 2018 SIAF, and the [V2,V3] positions are given in Table~\ref{tab:activesiaf}.

The COS SIAF ``Aperture Names''  start with the ``L"", and are followed by either an ``N'' (NUV), ``F'' (FUV), or
``APT''' indicating that the aperture is only used in APT and not for observations.
``APT'' in the APT entries are immediately followed by an ``N'' or ``F''.
The SA (BOA or PSA) or MAC then follows. MAC represents the APT/SPSS MACro aperture used for bright object checking.
Finally, FUV entries end in a number giving the LP\#, while the NUV ''offset'' aperturens end with ``OF''.
An illustration of the active COS SIAF entries is given in Figure~\ref{fig:SIAF1}. As described in the individual
FUV LP enabling ISRs (Penton, 201X (LP2), Penton 201Y(LP3), and Penton 201Z (LP4)), the location of each LP aperture
is determined by first selecting the desired XD location on the FUV detector segments. After this selection,
an AD aperture scan, using \textit{POS\_TARGs} in APT, determines the SIAF entry adjustment corresponding to
the desired LP [V2,V3] aperture center. As shown in Figure~\ref{fig;SIAF1}, this alignment procedure produces
a series of aperture locations that are at an $\approx$ 44.2$\degree$ angle to the -V3 (U3) axis.

The orientation of the COS AD and XD directions in relation to the HST mechanical [V2,V3] coordinates
and U-frame are shown in Figure~\ref{fig:ADXDV23}. The HST U-frame of [U2,U3]=[-V2,-V3] and is used by APT
and SPSS\footnote{SPSS=Science Planning and Scheduling System}.  Converting the orientation shown in Figure~\ref{fig:SIAF1} to
the nomenclature of the SIAF, the COS FUV apertures are aligned as follows:
$\beta_{x}=135.8\degree$, $\beta_y=45.8\degree$, and parity=+1. As shown in Figure~\ref{fig:SIAF1}, the COS AD direction is [+V2,-V3],
while XD is [+V2,+V3].

\begin{figure}[htb]
\noindent\includegraphics*[width=0.485\linewidth]{png/LP4_SIAF_positions.png}
\noindent\includegraphics*[width=0.485\linewidth]{png/LP4_SIAF_positions_sname.png}
\caption[Illustration of COS SIAF Entries]{
Illustration of the COS SIAF Entries.  All FUV LP1 entries are shown in \textcolor{RED}{RED},
LP2 entries are shown in \textcolor{GREEN}{GREEN}, LP3 entries are shown in \textcolor{BLUE}{BLUE},
and the LP4 entries are shown in \textcolor{MAGENTA}{MAGENTA}.
The NUV SIAF entries are coincident with the \textcolor{RED}{LP1} FUV entries.
The left panel shows the actual SIAF entry names, while the right panel gives a more readable translation.\label{fig:SIAF1}}
\end{figure}

\begin{figure}[htb]
\noindent\includegraphics*[width=0.795\linewidth]{png/ADXD_V23.png}
\caption[Illustration of COS Aperture Orientation]{Illustration of COS aperture orientation versus the [V2,V3] telescope coordinate system (Lallo 2008).
The U ([U2,U3]=[-V2,-V3]) frame is used during the proposal process in APT and SPSS.\label{fig:ADXDV23}}
\end{figure}

\begin{deluxetable}{rcrr}
\tablecaption{FGS-to-SI Program Initial Pointing Determinations\label{table:activesiaf}}
\tablecolumns{4}
\tablehead{
\colhead{APERNAME} & \colhead{YEAR.DAY Activated} &  \colhead{V2 (\arcsec)} & \colhead{V3 (\arcsec)}
}
\startdata
\hline
\multicolumn{4}{c}{NUV LP1}\\
\hline
LNMAC & 2014.055  & +232.7230 & -237.5150\\
LNBOA & 2014.055  & +232.7230 & -237.5150\\
LNPSA & 2014.055  & +232.7230 & -237.5150\\
LAPTNBOAOF &2014.055  & +223.3488 & -246.8892\\
LAPTNPSAOF &2014.055  & +242.0972 & -228.1408\\
\hline
\multicolumn{4}{c}{FUV LP1}\\
\hline
LFBOA1     & 2016.151  & +232.7230 & -237.5150\\
LFPSA1     & 2016.151  & +232.7230 & -237.5150\\
LAPTFBOAF1 & 2016.151  & +223.3488 & -246.8892\\
LAPTFPSAF1 & 2016.151  & +242.0972 & -228.1408\\
\hline
\multicolumn{4}{c}{FUV LP2}\\
\hline
LFBOA2     & 2016.151  & +235.1580 & -235.0100\\
LFPSA2      & 2016.151  & +235.1580 & -235.0100\\
LAPTFBOAF2  & 2016.151  & +225.7838 & -244.3842\\
LAPTFPSAF2  & 2016.151  & +244.5322 & -225.6358\\
\hline
\multicolumn{4}{c}{FUV LP3}\\
\hline
LFBOA3      & 2016.151  & +230.9137 & -239.2749\\
LFPSA3      & 2016.151  & +230.9137 & -239.2749\\
LAPTFBOAF3  & 2016.151  & +221.5395 & -248.6491\\
LAPTFPSAF3  & 2016.151  & +240.2879 & -229.9007\\
\hline
\multicolumn{4}{c}{FUV LP4}\\
\hline
LFBOA4      & 2017.031  & +229.1328 & -241.0575\\
LFPSA4      & 2017.031  & +229.1328 & -241.0575\\
LAPTFBOAF4  & 2017.031  & +219.7586 & -250.4317\\
LAPTFPSAF4  & 2017.031  & +238.5070 & -231.6833\\
\hline
\enddata
\tablecomments{Explain APERNAME and reference figures}
\end{deluxetable}

\begin{deluxetable}{rrrrrrrrrr}
\tablecaption{FGS-to-SI Program Initial Pointing Determinations\label{table:fgs2siInit}}
\tablecolumns{10}
\tabletypesize{\footnotesize}
\tablehead{
\colhead{HST} & \colhead{YEAR.DAY} & \colhead{DATE-OBS} & \multicolumn{2}{c}{Initial Pointing} & \multicolumn{2}{c}{Miss-Distance} & {SIAF} & \multicolumn{2}{c}{Active SIAF Entry}\\
\colhead{PID} & \colhead{} & \colhead{} & \colhead{V2 (\arcsec)} & \colhead{V3 (\arcsec)} & \colhead{V2 (\arcsec)} & \colhead{V3 (\arcsec)} & \colhead{V2 (\arcsec)} & \colhead{Dates\tablenotemark{a}} & \colhead{V3 (\arcsec)}\\
\colhead{(1)}&\colhead{(2)} & \colhead{(3)}&\colhead{(4)} & \colhead{(5)}&\colhead{(6)} & \colhead{(7)}&\colhead{(8)} & \colhead{(9)}&\colhead{(10)}
}
\startdata
\hline
\pid{11878} & 2009.338 & 04-Dec-2009 & 232.581 & -237.544 & -0.191 & -0.033 & 3-Aug-2009 & 232.772 & -237.511\\
\pid{11878} & 2010.074 & 15-Mar-2010 & 232.488 & -237.462 & -0.284 & 0.049 & \dots & 232.772 & -237.511\\
\pid{11878} & 2010.110 & 20-Apr-2010 & 232.481 & -237.457 & -0.291 & 0.054 & \dots & 232.772 & -237.511\\
\pid{11878} & 2010.309 & 05-Nov-2010 & 232.604 & -237.561 & -0.168 & -0.050 & \dots & 232.772 & -237.511\\
\pid{12399} & 2011.070 & 11-Mar-2011 & 232.645 & -237.438 & -0.127 & 0.073 & 20-Jun-2011 & 232.772 & -237.511\\
\hline
\pid{12399} & 2011.255 & 12-Sep-2011 & 232.737 & -237.507 & 0.091 & -0.062 & 21-Jun-2011 & 232.646 & -237.445\\
\pid{12781} & 2012.087 & 27-Mar-2012 & 232.622 & -237.515 & -0.024 & -0.070 & \dots & 232.646 & -237.445\\
\pid{12781} & 2012.268 & 24-Sep-2012 & 232.713 & -237.578 & 0.067 & -0.133 & \dots &232.646 & -237.445\\
\pid{13171} & 2013.061 & 02-Mar-2013 & 232.647 & -237.590 & 0.001 & -0.145 & 23-Feb-2014 & 232.646 & -237.445\\
\hline
\pid{13171} & 2013.244 & 01-Sep-2013 & 232.723 & -237.515 & 0.077 & -0.070 & 24-Feb-2014 & 232.723 & -237.515\\
\pid{13616} & 2014.055 & 06-Apr-2014 & 232.535 & -237.497 & -0.188 & 0.018 & \dots & 232.723 & -237.515\\
\pid{13616} & 2014.300 & 27-Oct-2014 & 232.841 & -237.465 & 0.118 & 0.050 & \dots & 232.723 & -237.515\\
\pid{14035} & 2015.104 & 14-May-2015 & 232.617 & -237.464 & -0.106 & 0.051 & \dots & 232.723 & -237.515\\
\pid{14035} & 2015.275 & 02-Oct-2015 & 232.788 & -237.462 & 0.065 & 0.053 & \dots & 232.723 & -237.515\\
\pid{14452} & 2016.092 & 01-Apr-2016 & 232.742 & -237.485 & 0.019 & 0.030 & \dots & 232.723 & -237.515\\
\hline
\enddata
\tablenotetext{a}{Dates in this column show the dates that the [V2,V3] SIAF entries in the this and the following rows were active.}
\tablecomments{Comments to be added here.}
\end{deluxetable}

%RCSID="$Id: siafhistory.tex,v 1.1 2018/04/17 18:39:53 penton Exp $"
%Taken from Space Telescope Science Operations Database (SCIOPSDB)
%http://prd.stsci.edu/prd/sciopsdb/uvm/UVMelem.cgi?ELEM=sdb/cgg5_cos.dat&REV=Latest
%
%Space Telescope Science Operations Database (SCIOPSDB)
%
%UVM Lib sdb / Element
%
%cgg5_cos.dat
%
%   Option:
%Format (Optional)	Version1:Version2
%Example (Optional)	1.6:1.12
% Option  set version (optional)
%
%Revision: "Latest" = 1.16     text_only
%
%LFMAC     2014.055:00:00:00  1    232.7230   -237.5150    0.022600    0.094300    135.0000     45.0000   8192.0000    512.0000
%LFBOA     2014.055:00:00:00  1    232.7230   -237.5150    0.022600    0.094300    135.0000     45.0000   8192.0000    512.0000
%LFPSA     2014.055:00:00:00  1    232.7230   -237.5150    0.022600    0.094300    135.0000     45.0000   8192.0000    512.0000
%LNMAC     2014.055:00:00:00  1    232.7230   -237.5150    0.025800    0.025800    135.0000     45.0000    512.0000    512.0000
%LNBOA     2014.055:00:00:00  1    232.7230   -237.5150    0.025800    0.025800    135.0000     45.0000    512.0000    512.0000
%LNPSA     2014.055:00:00:00  1    232.7230   -237.5150    0.025800    0.025800    135.0000     45.0000    512.0000    512.0000
%LAPTFBOAOF2014.055:00:00:00  1    223.3488   -246.8892    0.022600    0.094300    135.0000     45.0000   8192.0000    512.0000
%LAPTFPSAOF2014.055:00:00:00  1    242.0972   -228.1408    0.022600    0.094300    135.0000     45.0000   8192.0000    512.0000
%LAPTNBOAOF2014.055:00:00:00  1    223.3488   -246.8892    0.025800    0.025800    135.0000     45.0000    512.0000    512.0000
%LAPTNPSAOF2014.055:00:00:00  1    242.0972   -228.1408    0.025800    0.025800    135.0000     45.0000    512.0000    512.0000
%LFBOAA    2015.040:00:00:00  1    235.1580   -235.0100    0.022600    0.094300    135.0000     45.0000   8192.0000    512.0000
%LFPSAA    2015.040:00:00:00  1    235.1580   -235.0100    0.022600    0.094300    135.0000     45.0000   8192.0000    512.0000
%LNBOAA    2014.055:00:00:00  1    232.7230   -237.5150    0.025800    0.025800    135.0000     45.0000    512.0000    512.0000
%LNPSAA    2014.055:00:00:00  1    232.7230   -237.5150    0.025800    0.025800    135.0000     45.0000    512.0000    512.0000
%LAPTFBOAFA2015.040:00:00:00  1    225.7838   -244.3842    0.022600    0.094300    135.0000     45.0000   8192.0000    512.0000
%LAPTFPSAFA2015.040:00:00:00  1    244.5322   -225.6358    0.022600    0.094300    135.0000     45.0000   8192.0000    512.0000
%LAPTNBOAFA2014.055:00:00:00  1    223.3488   -246.8892    0.025800    0.025800    135.0000     45.0000    512.0000    512.0000
%LAPTNPSAFA2014.055:00:00:00  1    242.0972   -228.1408    0.025800    0.025800    135.0000     45.0000    512.0000    512.0000
%LFBOAB    2015.040:00:00:00  1    230.9137   -239.2749    0.022600    0.094300    135.0000     45.0000   8192.0000    512.0000
%LFPSAB    2015.040:00:00:00  1    230.9137   -239.2749    0.022600    0.094300    135.0000     45.0000   8192.0000    512.0000
%LNBOAB    2014.055:00:00:00  1    232.7230   -237.5150    0.025800    0.025800    135.0000     45.0000    512.0000    512.0000
%LNPSAB    2014.055:00:00:00  1    232.7230   -237.5150    0.025800    0.025800    135.0000     45.0000    512.0000    512.0000
%LAPTFBOAFB2015.040:00:00:00  1    221.5395   -248.6491    0.022600    0.094300    135.0000     45.0000   8192.0000    512.0000
%LAPTFPSAFB2015.040:00:00:00  1    240.2879   -229.9007    0.022600    0.094300    135.0000     45.0000   8192.0000    512.0000
%LAPTNBOAFB2014.055:00:00:00  1    223.3488   -246.8892    0.025800    0.025800    135.0000     45.0000    512.0000    512.0000
%LAPTNPSAFB2014.055:00:00:00  1    242.0972   -228.1408    0.025800    0.025800    135.0000     45.0000    512.0000    512.0000
%LFBOA1    2016.151:00:00:00  1    232.7230   -237.5150    0.022600    0.094300    135.0000     45.0000   8192.0000    512.0000
%LFPSA1    2016.151:00:00:00  1    232.7230   -237.5150    0.022600    0.094300    135.0000     45.0000   8192.0000    512.0000
%LAPTFBOAF12016.151:00:00:00  1    223.3488   -246.8892    0.022600    0.094300    135.0000     45.0000   8192.0000    512.0000
%LAPTFPSAF12016.151:00:00:00  1    242.0972   -228.1408    0.022600    0.094300    135.0000     45.0000   8192.0000    512.0000
%LFBOA2    2016.151:00:00:00  1    235.1580   -235.0100    0.022600    0.094300    135.0000     45.0000   8192.0000    512.0000
%LFPSA2    2016.151:00:00:00  1    235.1580   -235.0100    0.022600    0.094300    135.0000     45.0000   8192.0000    512.0000
%LAPTFBOAF22016.151:00:00:00  1    225.7838   -244.3842    0.022600    0.094300    135.0000     45.0000   8192.0000    512.0000
%LAPTFPSAF22016.151:00:00:00  1    244.5322   -225.6358    0.022600    0.094300    135.0000     45.0000   8192.0000    512.0000
%LFBOA3    2016.151:00:00:00  1    230.9137   -239.2749    0.022600    0.094300    135.0000     45.0000   8192.0000    512.0000
%LFPSA3    2016.151:00:00:00  1    230.9137   -239.2749    0.022600    0.094300    135.0000     45.0000   8192.0000    512.0000
%LAPTFBOAF32016.151:00:00:00  1    221.5395   -248.6491    0.022600    0.094300    135.0000     45.0000   8192.0000    512.0000
%LAPTFPSAF32016.151:00:00:00  1    240.2879   -229.9007    0.022600    0.094300    135.0000     45.0000   8192.0000    512.0000
%LFBOA4    2017.058:00:00:00  1    229.1328   -241.0575    0.022600    0.094300    135.0000     45.0000   8192.0000    512.0000
%LFPSA4    2017.058:00:00:00  1    229.1328   -241.0575    0.022600    0.094300    135.0000     45.0000   8192.0000    512.0000
%LAPTFBOAF42017.058:00:00:00  1    219.7586   -250.4317    0.022600    0.094300    135.0000     45.0000   8192.0000    512.0000
%LAPTFPSAF42017.058:00:00:00  1    238.5070   -231.6833    0.022600    0.094300    135.0000     45.0000   8192.0000    512.0000
%LFBOA5    2016.151:00:00:00  1    230.9500   -239.2500    0.022600    0.094300    135.0000     45.0000   8192.0000    512.0000
%LFPSA5    2016.151:00:00:00  1    230.9500   -239.2500    0.022600    0.094300    135.0000     45.0000   8192.0000    512.0000
%LAPTFBOAF52016.151:00:00:00  1    221.5500   -248.6500    0.022600    0.094300    135.0000     45.0000   8192.0000    512.0000
%LAPTFPSAF52016.151:00:00:00  1    240.2500   -229.9500    0.022600    0.094300    135.0000     45.0000   8192.0000    512.0000
%LFBOA6    2016.151:00:00:00  1    230.9600   -239.2600    0.022600    0.094300    135.0000     45.0000   8192.0000    512.0000
%LFPSA6    2016.151:00:00:00  1    230.9600   -239.2600    0.022600    0.094300    135.0000     45.0000   8192.0000    512.0000
%LAPTFBOAF62016.151:00:00:00  1    221.5600   -248.6600    0.022600    0.094300    135.0000     45.0000   8192.0000    512.0000
%LAPTFPSAF62016.151:00:00:00  1    240.2600   -229.9600    0.022600    0.094300    135.0000     45.0000   8192.0000    512.0000
%LFBOA7    2016.151:00:00:00  1    230.9700   -239.2700    0.022600    0.094300    135.0000     45.0000   8192.0000    512.0000
%LFPSA7    2016.151:00:00:00  1    230.9700   -239.2700    0.022600    0.094300    135.0000     45.0000   8192.0000    512.0000
%LAPTFBOAF72016.151:00:00:00  1    221.5700   -248.6700    0.022600    0.094300    135.0000     45.0000   8192.0000    512.0000
%LAPTFPSAF72016.151:00:00:00  1    240.2700   -229.9700    0.022600    0.094300    135.0000     45.0000   8192.0000    512.0000
%LFBOA8    2016.151:00:00:00  1    230.9800   -239.2800    0.022600    0.094300    135.0000     45.0000   8192.0000    512.0000
%LFPSA8    2016.151:00:00:00  1    230.9800   -239.2800    0.022600    0.094300    135.0000     45.0000   8192.0000    512.0000
%LAPTFBOAF82016.151:00:00:00  1    221.5800   -248.6800    0.022600    0.094300    135.0000     45.0000   8192.0000    512.0000
%LAPTFPSAF82016.151:00:00:00  1    240.2800   -229.9800    0.022600    0.094300    135.0000     45.0000   8192.0000    512.0000
%! Feb 2017Shift four apertures
%LFBOA4    2016.346:06:00:00  1    229.1073   -241.0320    0.022600    0.094300    135.0000     45.0000   8192.0000    512.0000
%LFPSA4    2016.346:06:00:00  1    229.1073   -241.0320    0.022600    0.094300    135.0000     45.0000   8192.0000    512.0000
%LAPTFBOAF42016.346:06:00:00  1    219.7331   -250.4062    0.022600    0.094300    135.0000     45.0000   8192.0000    512.0000
%LAPTFPSAF42016.346:06:00:00  1    238.4815   -231.6578    0.022600    0.094300    135.0000     45.0000   8192.0000    512.0000
%! 11/21/2016 Install Lifetime 4 positions
%LFBOA4    2016.151:00:00:00  1    229.1389   -241.0013    0.022600    0.094300    135.0000     45.0000   8192.0000    512.0000
%LFPSA4    2016.151:00:00:00  1    229.1389   -241.0013    0.022600    0.094300    135.0000     45.0000   8192.0000    512.0000
%LAPTFBOAF42016.151:00:00:00  1    219.7647   -250.3755    0.022600    0.094300    135.0000     45.0000   8192.0000    512.0000
%LAPTFPSAF42016.151:00:00:00  1    238.5131   -231.6271    0.022600    0.094300    135.0000     45.0000   8192.0000    512.0000
%!4/12/2016New apertures for extra lifetime positions
%! 12/17 20 Switch four apertures between sets A and B
%!
%LFBOAA    2014.265:00:00:00  1    230.9137   -239.2749    0.022600    0.094300    135.0000     45.0000   8192.0000    512.0000
%LFPSAA    2014.265:00:00:00  1    230.9137   -239.2749    0.022600    0.094300    135.0000     45.0000   8192.0000    512.0000
%LAPTFBOAFA2014.265:00:00:00  1    221.5395   -248.6491    0.022600    0.094300    135.0000     45.0000   8192.0000    512.0000
%LAPTFPSAFA2014.265:00:00:00  1    240.2879   -229.9007    0.022600    0.094300    135.0000     45.0000   8192.0000    512.0000
%LFBOAB    2014.055:00:00:00  1    235.1580   -235.0100    0.022600    0.094300    135.0000     45.0000   8192.0000    512.0000
%LFPSAB    2014.055:00:00:00  1    235.1580   -235.0100    0.022600    0.094300    135.0000     45.0000   8192.0000    512.0000
%LAPTFBOAFB2014.055:00:00:00  1    225.7838   -244.3842    0.022600    0.094300    135.0000     45.0000   8192.0000    512.0000
%LAPTFPSAFB2014.055:00:00:00  1    244.5322   -225.6358    0.022600    0.094300    135.0000     45.0000   8192.0000    512.0000
%!
%! Revised positions for FUV alternate apertures
%LFBOAA    2014.188:00:00:00  1    230.9384   -239.2996    0.022600    0.094300    135.0000     45.0000   8192.0000    512.0000
%LFPSAA    2014.188:00:00:00  1    230.9384   -239.2996    0.022600    0.094300    135.0000     45.0000   8192.0000    512.0000
%LAPTFBOAFA2014.188:00:00:00  1    221.5642   -248.6738    0.022600    0.094300    135.0000     45.0000   8192.0000    512.0000
%LAPTFPSAFA2014.188:00:00:00  1    240.3126   -229.9254    0.022600    0.094300    135.0000     45.0000   8192.0000    512.0000
%!
%! Alternate FUV apertures shifted to prepare for next Lifetime position
%!
%LFBOAA    2014.055:00:00:00  1    235.1580   -235.0100    0.022600    0.094300    135.0000     45.0000   8192.0000    549.0000
%LFPSAA    2014.055:00:00:00  1    235.1580   -235.0100    0.022600    0.094300    135.0000     45.0000   8192.0000    549.0000
%LAPTFBOAFA2014.055:00:00:00  1    225.7838   -244.3842    0.022600    0.094300    135.0000     45.0000   8192.0000    512.0000
%LAPTFPSAFA2014.055:00:00:00  1    244.5322   -225.6358    0.022600    0.094300    135.0000     45.0000   8192.0000    512.0000
%!
%! All apertures shifted in V2,V3 by (0.077, -0.070) arcsec
%!
%LFMAC     2011.172:21:00:00  1    232.6460   -237.4450    0.022600    0.094300    135.0000     45.0000   8192.0000    512.0000
%LFBOA     2011.172:21:00:00  1    232.6460   -237.4450    0.022600    0.094300    135.0000     45.0000   8192.0000    512.0000
%LFPSA     2011.172:21:00:00  1    232.6460   -237.4450    0.022600    0.094300    135.0000     45.0000   8192.0000    512.0000
%LNMAC     2011.172:21:00:00  1    232.6460   -237.4450    0.025800    0.025800    135.0000     45.0000    512.0000    512.0000
%LNBOA     2011.172:21:00:00  1    232.6460   -237.4450    0.025800    0.025800    135.0000     45.0000    512.0000    512.0000
%LNPSA     2011.172:21:00:00  1    232.6460   -237.4450    0.025800    0.025800    135.0000     45.0000    512.0000    512.0000
%LAPTFBOAOF2011.172:21:00:00  1    223.2718   -246.8192    0.022600    0.094300    135.0000     45.0000   8192.0000    512.0000
%LAPTFPSAOF2011.172:21:00:00  1    242.0202   -228.0708    0.022600    0.094300    135.0000     45.0000   8192.0000    512.0000
%LAPTNBOAOF2011.172:21:00:00  1    223.2718   -246.8192    0.025800    0.025800    135.0000     45.0000    512.0000    512.0000
%LAPTNPSAOF2011.172:21:00:00  1    242.0202   -228.0708    0.025800    0.025800    135.0000     45.0000    512.0000    512.0000
%LFBOAA    2012.135:00:00:00  1    235.0810   -234.9400    0.022600    0.094300    135.0000     45.0000   8192.0000    549.0000
%LFPSAA    2012.135:00:00:00  1    235.0810   -234.9400    0.022600    0.094300    135.0000     45.0000   8192.0000    549.0000
%LNBOAA    2012.086:00:00:00  1    232.6460   -237.4450    0.025800    0.025800    135.0000     45.0000    512.0000    512.0000
%LNPSAA    2012.086:00:00:00  1    232.6460   -237.4450    0.025800    0.025800    135.0000     45.0000    512.0000    512.0000
%LAPTFBOAFA2012.135:00:00:00  1    225.7068   -244.3142    0.022600    0.094300    135.0000     45.0000   8192.0000    512.0000
%LAPTFPSAFA2012.135:00:00:00  1    244.4552   -225.5658    0.022600    0.094300    135.0000     45.0000   8192.0000    512.0000
%LAPTNBOAFA2012.086:00:00:00  1    223.2718   -246.8192    0.025800    0.025800    135.0000     45.0000    512.0000    512.0000
%LAPTNPSAFA2012.086:00:00:00  1    242.0202   -228.0708    0.025800    0.025800    135.0000     45.0000    512.0000    512.0000
%LFBOAB    2012.205:00:00:00  1    235.0810   -234.9400    0.022600    0.094300    135.0000     45.0000   8192.0000    512.0000
%LFPSAB    2012.205:00:00:00  1    235.0810   -234.9400    0.022600    0.094300    135.0000     45.0000   8192.0000    512.0000
%LNBOAB    2012.086:00:00:00  1    232.6460   -237.4450    0.025800    0.025800    135.0000     45.0000    512.0000    512.0000
%LNPSAB    2012.086:00:00:00  1    232.6460   -237.4450    0.025800    0.025800    135.0000     45.0000    512.0000    512.0000
%LAPTFBOAFB2012.205:00:00:00  1    225.7068   -244.3142    0.022600    0.094300    135.0000     45.0000   8192.0000    512.0000
%LAPTFPSAFB2012.205:00:00:00  1    244.4552   -225.5658    0.022600    0.094300    135.0000     45.0000   8192.0000    512.0000
%LAPTNBOAFB2012.086:00:00:00  1    223.2718   -246.8192    0.025800    0.025800    135.0000     45.0000    512.0000    512.0000
%LAPTNPSAFB2012.086:00:00:00  1    242.0202   -228.0708    0.025800    0.025800    135.0000     45.0000    512.0000    512.0000
%!
%! Change FUV B apertures to match A values
%!
%LFBOAB    2012.086:00:00:00  1    232.6460   -237.4450    0.022600    0.094300    135.0000     45.0000   8192.0000    512.0000
%LFPSAB    2012.086:00:00:00  1    232.6460   -237.4450    0.022600    0.094300    135.0000     45.0000   8192.0000    512.0000
%LAPTFBOAFB2012.086:00:00:00  1    223.2718   -246.8192    0.022600    0.094300    135.0000     45.0000   8192.0000    512.0000
%LAPTFPSAFB2012.086:00:00:00  1    242.0202   -228.0708    0.022600    0.094300    135.0000     45.0000   8192.0000    512.0000
%!
%! Further correction to  the four FUV apertures in alternate set
%!
%LFBOAA    2012.100:00:00:00  1    235.1510   -235.0100    0.022600    0.094300    135.0000     45.0000   8192.0000    549.0000
%LFPSAA    2012.100:00:00:00  1    235.1510   -235.0100    0.022600    0.094300    135.0000     45.0000   8192.0000    549.0000
%LAPTFBOAFA2012.100:00:00:00  1    225.7768   -244.3842    0.022600    0.094300    135.0000     45.0000   8192.0000    512.0000
%LAPTFPSAFA2012.100:00:00:00  1    244.5252   -225.6358    0.022600    0.094300    135.0000     45.0000   8192.0000    512.0000
%!
%! Small correction to the four FUV apertures in the alternate set
%!
%LFBOAA    2012.086:00:00:00  1    235.1160   -234.9750    0.022600    0.094300    135.0000     45.0000   8192.0000    549.0000
%LFPSAA    2012.086:00:00:00  1    235.1160   -234.9750    0.022600    0.094300    135.0000     45.0000   8192.0000    549.0000
%LAPTFBOAFA2012.086:00:00:00  1    225.7418   -244.3492    0.022600    0.094300    135.0000     45.0000   8192.0000    512.0000
%LAPTFPSAFA2012.086:00:00:00  1    244.4902   -225.6008    0.022600    0.094300    135.0000     45.0000   8192.0000    512.0000
%!
%! Added 8 new apertures to support new COS Lifetime positions
%! Update to match FGS realignment solution June 2011
%!
%LFMAC     2009.215:00:00:00  1    232.7720   -237.5110    0.022600    0.094300    135.0000     45.0000   8192.0000    512.0000
%LFBOA     2009.215:00:00:00  1    232.7720   -237.5110    0.022600    0.094300    135.0000     45.0000   8192.0000    512.0000
%LFPSA     2009.215:00:00:00  1    232.7720   -237.5110    0.022600    0.094300    135.0000     45.0000   8192.0000    512.0000
%LNMAC     2009.215:00:00:00  1    232.7720   -237.5110    0.025800    0.025800    135.0000     45.0000    512.0000    512.0000
%LNBOA     2009.215:00:00:00  1    232.7720   -237.5110    0.025800    0.025800    135.0000     45.0000    512.0000    512.0000
%LNPSA     2009.215:00:00:00  1    232.7720   -237.5110    0.025800    0.025800    135.0000     45.0000    512.0000    512.0000
%LAPTFPSAOF2009.215:00:00:00  1    242.1462   -228.1368    0.022600    0.094300    135.0000     45.0000   8192.0000    512.0000
%LAPTFBOAOF2009.215:00:00:00  1    223.3978   -246.8852    0.022600    0.094300    135.0000     45.0000   8192.0000    512.0000
%LAPTNPSAOF2009.215:00:00:00  1    242.1462   -228.1368    0.025800    0.025800    135.0000     45.0000    512.0000    512.0000
%LAPTNBOAOF2009.215:00:00:00  1    223.3978   -246.8852    0.025800    0.025800    135.0000     45.0000    512.0000    512.0000
%!
%! Update following alignment exercise. July 2009
%!
%LFMAC     2001.001:00:00:00  1    231.6600   -236.2114    0.022600    0.094300    135.0000     45.0000   8192.0000    512.0000
%LFBOA     2001.001:00:00:00  1    231.6600   -236.2114    0.022600    0.094300    135.0000     45.0000   8192.0000    512.0000
%LFPSA     2001.001:00:00:00  1    231.6600   -236.2114    0.022600    0.094300    135.0000     45.0000   8192.0000    512.0000
%LNMAC     2001.001:00:00:00  1    231.6600   -236.2114    0.025800    0.025800    135.0000     45.0000    512.0000    512.0000
%LNBOA     2001.001:00:00:00  1    231.6600   -236.2114    0.025800    0.025800    135.0000     45.0000    512.0000    512.0000
%LNPSA     2001.001:00:00:00  1    231.6600   -236.2114    0.025800    0.025800    135.0000     45.0000    512.0000    512.0000
%LAPTFPSAOF2001.001:00:00:00  1    241.0342   -226.8372    0.022600    0.094300    135.0000     45.0000   8192.0000    512.0000
%LAPTFBOAOF2001.001:00:00:00  1    222.2858   -245.5856    0.022600    0.094300    135.0000     45.0000   8192.0000    512.0000
%LAPTNPSAOF2001.001:00:00:00  1    241.0342   -226.8372    0.025800    0.025800    135.0000     45.0000    512.0000    512.0000
%LAPTNBOAOF2001.001:00:00:00  1    222.2858   -245.5856    0.025800    0.025800    135.0000     45.0000    512.0000    512.0000
\begingroup
%\renewcommand{\arraystretch}{0.9}
%\renewcommand{\baselinestretch}{0.9}
\begin{deluxetable}{llll}
\tablecolumns{4}
\tablecaption{History of COS SIAF Changes and FGS Alignment Activities\label{tab:siafhistory}}
\tabletypesize{\footnotesize}
\tablehead{
\colhead{YEAR.DAY}	&	\colhead{STScI \pr{}\tablenotemark{a}}	&	\colhead{Entries Changed\tablenotemark{b}}	& \colhead{Comment}
}
\startdata
2001.001	&		N/A	&	Original 10\tablenotemark{c} & Estimates for all COS apertures from Ground Testing \\
\midrule
2009.054	&	61897	&	None				& FGS1R Alignment Update \\
\midrule
2009.215	&	63138	&	Original 10			& On-orbit positions based upon SMOV alignment \\
\midrule
2010.055	&	64538	&	Original 10			&  Update following December 2009 re-alignment\\
\midrule
2010.263	&	65904	&	None				& FGS2R2 Distortion/Scale Update\\
\midrule
2011.172	&	68498	&	Original 10					 & Update original 10 to match June 2011 FGS realignment \\
			&			&	$+$ 16 New entries\tablenotemark{d}		 & LP2 prep. 2 x copies of 8 original non-MACro apertures.\\
			&			&								 & 8 (4 FUV, 4 NUV) have an '{\bf A}' (ALTERNATE) added to the end.\\
			&			&								 & 8 (4 FUV, 4 NUV) have a '{\bf B}' (BEST) added to end.\\
\midrule
2011.206	&	68299	&	None						 & FGS2R2 Allowed to be Dominant in GS Pairs\\
\midrule
2012.086	&	70792	&	LF*{\bf A}		 & Initial correction to 4 LP2 FUV LP*{\bf A}lternate entries \\
\midrule
2012.100	&	70903	&	LF*{\bf A}			 & LP2 correction to 4 FUV LP*{\bf A}lternate entries.\\
\midrule
2012.135	&	71160	&	LF*{\bf A}	 & \pr{70903} adjustment incorrectly applied, corrected. \\
\midrule
2012.205	&	71568	&	LF*{\bf A}	and LF*{\bf B}	 & Swap FUV {\bf A} and {\bf B} entries. After, LF*{\bf A}=LP1 \& LP*{\bf B}=LP2 \\
\midrule
2013.205	&	75035	&		None					&	FGS re-alignment update loaded to HST\\
\midrule
2014.055	&	76982	&			All Entries			 & SIAF update to match FGS re-alignment (all apertures)\\
			&			&								 &	$\Delta$[V2,V3] = [0.077, -0.070]" \\
\midrule
2014.188	&	78255	&			LF*{\bf A}			 & LP3 initial estimates installed in LF*{\bf A} \\
\midrule
2014.245	&	78801	&		LF*{\bf A}	 & LP3 Final Refinement\\
\midrule
2014.265	&	78775	&	LF*{\bf A}	and LF*{\bf B}		 & LP3 Move {\bf B}est/{\bf A}lternate Swap: LP1=Original, LP2=LF*{\bf A} \& \\
			&			&									 & LP3=LF*{\bf B}. Activated 2014.351 (SIAF entries use 2014.265)\\
\midrule
2015.327	&			&		None						 & COS stopped using FGS2R2 as DOMinant FGS \\
\midrule
2016.095	&	83878	&		None						&	New FGS2R2 Calibration installed on HST \\
			&			&									&	FGS2R2 DOM GS still OFF for COS, but on for STIS \\
\midrule
2016.123	&			&		None						&	Based upon STIS data, FGS2R2 for DOM GS re-enabled for COS\\
\midrule
2016.151	&	84188	&	Add 32 LP1-8 entries & Install new LP\# nomeclature\\
			&			&	LF*{\bf 1,2,3}	updated		 & Original, {\bf A}lternate, and {\bf B}est copied to LP1, LP2 \& LP3 \\
			&			&	LF*{\bf 4,5,6,7,8}		 & LP4-8 entries set to LP3 values \\
\midrule
2016.346	&	86315	&	LF*{\bf 4}					 & Initial LP4 Estimates \\
\midrule
2017.058	&	86877	&	LF*{\bf 4}					 & LP4 Position Update ($\Delta$[V2,V3]=[0.0255, -0.0255])\\
\bottomrule
\enddata
\vspace{-0.5cm}
\tablenotetext{a}{\footnotesize Problem Report \#s refer to the SCIOPSDB delivery PRs.}
\tablenotetext{b}{\footnotesize Trailing characters not part of the original nomenclature are shown in {\bf bold}.}
\tablenotetext{c}{\footnotesize The original 5 NUV SIAF entries were LNMAC, LNBOA, LNPSA, LAPTNPSAOF \& LAPTNBOAOF.  The original 5 FUV entries were LFMAC, LFBOA, LFPSA, LAPTFPSAOF \& LAPTFBOAOF.}
\tablenotetext{d}{\footnotesize The 8 new {\bf A}lternate entries were LNBOA{\bf A}, LNPSA{\bf A}, LFBOA{\bf A}, LFPSA{\bf A}, LAPTFPSAF{\bf A}, LAPTFBOAF{\bf A}, LAPTNPSAF{\bf A} \& LAPTNBOAF{\bf A}.
The 8 new {\bf B}est entries are identical to the {\bf A}ternates, but end with {\bf B} instead of {\bf A}. The penultimate letter, ``O'', in the offset apertures has been removed in create room for the trailing {\bf A} or {\bf B}.}
\end{deluxetable}
\endgroup
%13203    07/22 - Bower
%
%2 calendars - break W dayshift @ 2013.205:15:38:00
%2nd calendar uses updated PRD and FGS calib files
%FGS Alignment Update at end of 1st calendar and start of 2nd calendar

