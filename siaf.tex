\section{SIAF Verification} \label{sec:siaf}

\subsection{COS SIAF History}\label{subsec:siafhistory}
The pre-SM4 COS Science Instrument Aperture File (SIAF) is described in detail in Mallo, 2008.
The active COS entries in the 2018 SIAF, and the [V2,V3] aperture positions are given in Table~\ref{tab:activesiaf}.
The changes since SMOV to the COS SIAF, and the continuing Fine Guidance Sensor (FGS) re-alignment efforts are
documented in Table~\ref{tab:siafhistory}.
This table includes all NUV LP1 and FUV LP1--4 entries. Each LP contains entries for the BOA, PSA, BOA when the PSA is being used, and PSA when the BOA is being used.

\input{extrasiaf/siafhistory.tex}

The COS SIAF ``Aperture Names''  start with the ``L"", and are followed by either an ``N'' (NUV), ``F'' (FUV), or
``APT''' indicating that the aperture is only used in APT and not for observations.
``APT'' in the APT entries are immediately followed by an ``N'' or ``F''.
The SA (BOA or PSA) or MAC then follows. MAC represents the APT/SPSS\footnote{SPSS=Science Planning and Scheduling System} MACro aperture used for bright object checking.
Finally, FUV entries end in a number giving the LP\#, while the NUV ``offset'' apertures end with ``OF''.
An illustration of the active COS SIAF entries is given in Figure~\ref{fig:SIAF1}. As described in the individual
FUV LP enabling ISRs, the location of each LP aperture is determined by first selecting the desired XD location on the FUV detector segments. After this selection,
an AD aperture scan, using \texttt{POS\_TARGs} in APT, determines the SIAF entry adjustment corresponding to
the desired LP [V2,V3] aperture center. As shown in Figure~\ref{fig:SIAF1}, this alignment procedure produces
a series of aperture locations that are at an $\approx$ 44.2$\degree$ angle to the -V3 (U3) axis.

\begin{deluxetable}{rcrr}
\tablecaption{Active COS SIAF\tablenotemark{a}~~Entries\label{tab:activesiaf}}
\tabletypesize{\footnotesize}
\tablewidth{0pt}
\tabcolsep 16 pt
\tablecolumns{4}
\tablehead{
\colhead{SIAF} & \colhead{YEAR.DAY} &  \colhead{V2} & \colhead{V3}\\
\colhead{APERNAME} & \colhead{Activated} &  \colhead{(\arcsec)} & \colhead{(\arcsec)}
}
\startdata
\toprule
\multicolumn{4}{c}{NUV LP1}\\
\midrule
LNMAC & 2014.055  & +232.7230 & -237.5150\\
LNBOA & 2014.055  & +232.7230 & -237.5150\\
LNPSA & 2014.055  & +232.7230 & -237.5150\\
LAPTNBOAOF &2014.055  & +223.3488 & -246.8892\\
LAPTNPSAOF &2014.055  & +242.0972 & -228.1408\\
\midrule
\multicolumn{4}{c}{FUV LP1}\\
\midrule
LFMAC      & 2014.055  & +232.7230 & -237.5150\\
LFBOA1     & 2016.151  & +232.7230 & -237.5150\\
LFPSA1     & 2016.151  & +232.7230 & -237.5150\\
LAPTFBOAF1 & 2016.151  & +223.3488 & -246.8892\\
LAPTFPSAF1 & 2016.151  & +242.0972 & -228.1408\\
\midrule
\multicolumn{4}{c}{FUV LP2}\\
\midrule
LFBOA2      & 2016.151  & +235.1580 & -235.0100\\
LFPSA2      & 2016.151  & +235.1580 & -235.0100\\
LAPTFBOAF2  & 2016.151  & +225.7838 & -244.3842\\
LAPTFPSAF2  & 2016.151  & +244.5322 & -225.6358\\
\midrule
\multicolumn{4}{c}{FUV LP3}\\
\midrule
LFBOA3      & 2016.151  & +230.9137 & -239.2749\\
LFPSA3      & 2016.151  & +230.9137 & -239.2749\\
LAPTFBOAF3  & 2016.151  & +221.5395 & -248.6491\\
LAPTFPSAF3  & 2016.151  & +240.2879 & -229.9007\\
\midrule
\multicolumn{4}{c}{FUV LP4}\\
\midrule
LFBOA4      & 2017.031  & +229.1328 & -241.0575\\
LFPSA4      & 2017.031  & +229.1328 & -241.0575\\
LAPTFBOAF4  & 2017.031  & +219.7586 & -250.4317\\
LAPTFPSAF4  & 2017.031  & +238.5070 & -231.6833\\
\bottomrule
\enddata
\tablenotetext{a}{SIAF = Science Instrument Aperture File.}
\tablecomments{COS SIAF ``Aperture Names'' (APERNAME) start with the ``L"", and are
followed by either an ``N'' (NUV), ``F'' (FUV), or
``APT''' indicating that the aperture is only used in APT and not for observations.
``APT'' in the APT entries are immediately followed by an ``N'' or ``F''.
The SA (BOA or PSA) or MAC then follows.
MAC represents the APT/SPSS MACro aperture used for bright object checking.
Finally, FUV entries end in a number giving the LP\#, while the NUV ''offset'' aperturens end with ``OF''.}
\end{deluxetable}


The orientation of the COS AD and XD directions in relation to the HST mechanical [V2,V3] coordinates
and U-frame are shown in Figure~\ref{fig:ADXDV23}. The HST U-frame of [U2,U3]=[-V2,-V3] and is used by APT
and SPSS.  Converting the orientation shown in Figure~\ref{fig:SIAF1} to
the nomenclature of the SIAF, the COS FUV apertures are aligned as follows:
$\beta_{x}=135.8\degree$, $\beta_y=45.8\degree$, and parity=+1. As shown in Figure~\ref{fig:SIAF1}, for COS:
\begin{eqnarray}
\setlength\itemsep{0.1em}
AD = [+V2,-V3]\label{eq:ADV23}\\
XD = [+V2,+V3]\label{eq:XDV23}\\
V2 = [+AD,+XD]\label{eq:V2ADXD}\\
V3 = [-AD,+XD]\label{eq:V3ADXD}
\end{eqnarray}
\normalsize
The NUV coordinate system orientation was measured during SMOV (Hartig et al., 2010 and Goudfrooij et al., 2010).
This NUV angle was determined to be 0.52 $\pm 0.01\degree$ from +Y \texttt{POS\_TARG} in the +X \texttt{POS\_TARG} direction
($\beta_{x}=135.5\degree$, $\beta_y=45.5\degree$). The COS SIAF currently uses $\beta_{x}=135\degree$, $\beta_y=45\degree$ for both NUV and FUV.
All conversions between [AD,XD] and [V2,V3] in this ISR use the current SIAF values for  $\beta_{x}$ and $\beta_{y}$.
The conversion between [V2,V3] and [AD,XD] in terms of $\beta_{x}$ and $\beta_{y}$ are :\footnote{See http://www.stsci.edu/hst/observatory/apertures/siaf.html}
\begin{eqnarray}
\setlength\itemsep{0.1em}
	V2 = S_{AD} \cdot sin(\beta_x) \cdot XD	+ S_{XD} \cdot sin(\beta_y) \cdot XD \label{eq:V2beta}\label{eq:V2BETA}\\
	V3 = S_{XD} \cdot cos(\beta_x) \cdot XD	+ S_{XD} \cdot cos(\beta_y) \cdot XD \label{eq:V3beta}\label{eq:V3BETA}
\end{eqnarray}
Where, $S_{AD}$ is the AD plate scale (0.02352\arcsec/p), $S_{XD}$ is the XD plate scale (0.02362\arcsec/p ), and the aperture center was taken as the reference point.

\begin{figure}[htb]
\noindent\includegraphics*[width=0.485\linewidth]{png/LP4_SIAF_positions.png}
\noindent\includegraphics*[width=0.485\linewidth]{png/LP4_SIAF_positions_sname.png}
\caption[Illustration of COS SIAF Entries]{
Illustration of the COS SIAF Entries.  All FUV LP1 entries are shown in \textcolor{RED}{RED},
LP2 entries are shown in \textcolor{GREEN}{GREEN}, LP3 entries are shown in \textcolor{BLUE}{BLUE},
and the LP4 entries are shown in \textcolor{MAGENTA}{MAGENTA}.
The NUV SIAF entries are coincident with the \textcolor{RED}{LP1} FUV entries.
The left panel shows the actual SIAF entry names, while the right panel gives a more readable translation.\label{fig:SIAF1}}
\end{figure}

\begin{figure}[htb]
\begin{center}
\noindent\includegraphics*[width=0.7\linewidth]{png/ADXD_V23.png}
\noindent\includegraphics*[width=0.795\linewidth]{pdf/COS_COORDS.pdf}
\end{center}
\caption[COS Aperture Orientation]{The upper panel shows the COS aperture orientation versus the [V2,V3] telescope coordinate system (Lallo 2008).
The U ([U2,U3]=[-V2,-V3]) frame is used during the proposal process in APT and SPSS. The lower panel (Osbourne, 2004) shows all the pre-launch COS coordinate
systems. From left to right, the ``P''hysical coordinate system is aligned with
the COS enclosure, and shares a common origin with the [X,Y,Z] ``DET''ector coordinates.
The X, Y, and Z axes of the STOPT (Space Telescope OPTical) coordinate system are shown slightly below and aft of these.
The HST [V1,V2,V3] coordinate system is shown on the right.\label{fig:ADXDV23}}
\end{figure}

\subsection{C17--C23 COS to SIAF Alignment \label{subsec:siafalign}}

The FGS-to-SI programs provide the opportunity to estimate the co-alignment
of the COS SIAF entry to the actual center of the COS SAs. The FGS-to-SI programs
concludes with two cos \tacq{IMAGE}s and a target that is approximately centered in the COS
aperture. By comparing the [V2,V3] position after the first of these \tacq{IMAGE}s
(the Configuration\#1 or Config\#1 PSA$\times$MIRA \tacq{IMAGE}), a direct comparison is possible.
Table~\ref{tab:fgs2siInit} gives these results for both the Spring and the Fall C17--C23 FGS-to-SI alignment programs.

The columns of Table~\ref{tab:fgs2siInit} are:
\footnotesize
\begin{enumerate}
\item \textit{PROPOSID} gives the HST program id (PID).
\item YEAR.DAY gives the Year and day of the year of the observation.
\item \textit{DATE-OBS} gives the observation date as reported in the
fits header in DY-Mon-YEAR format.
\item gives HST [V2,V3] coordinates (in \arcsec) of
the initial HST pointing before the (Config\#1) PSA$\times$MIRA \tacq{IMAGE}.
\item gives HST [V2,V3] coordinates (in \arcsec) of
the intermediate HST pointing after the (Config\#1) PSA$\times$MIRA \tacq{IMAGE}
and before the (Config\#2) PSA$\times$MIRB \tacq{IMAGE}.
\item gives HST [V2,V3] ``Miss-Distances'' (in \arcsec) of
the initial HST pointing before the (Config\#1) PSA$\times$MIRA \tacq{IMAGE}
and are the Initial Pointing coordinates subtracted from the \textit{LNPSA}
SIAF entry active at the time of the observation.
\item gives HST [V2,V3] ``Miss-Distances'' (in \arcsec) of
the intermediate HST pointing after the PSA$\times$MIRA \tacq{IMAGE}.
\item ``SIAF Dates'' gives the dates the [V2,V3] SIAF entry in the following ``SIAF Entry'' column was active.
\item ``SIAF Entry'' gives the [V2,V3] entries that were active at the time
of the observations of this row.
\end{enumerate}
\normalsize

\begin{deluxetable}{rrrrrrrrrr}
\tablecaption{FGS-to-SI Program Initial Pointing Determinations\label{tab:fgs2siInit}}
\tablecolumns{10}
\tabletypesize{\footnotesize}
\tablehead{
\colhead{PID} & \colhead{YEAR.DAY} & \colhead{DATE-OBS} & \multicolumn{2}{c}{Initial Pointing} & \multicolumn{2}{c}{Miss-Distance} & \colhead{SIAF} & \multicolumn{2}{c}{Active SIAF Entry}\\
\colhead{ } & \colhead{} & \colhead{} & \colhead{V2 (\arcsec)} & \colhead{V3 (\arcsec)} & \colhead{V2 (\arcsec)} & \colhead{V3 (\arcsec)} & \colhead{V2 (\arcsec)} & \colhead{Dates\tablenotemark{a}} & \colhead{V3 (\arcsec)}\\
\colhead{(1)}&\colhead{(2)} & \colhead{(3)}&\colhead{(4)} & \colhead{(5)}&\colhead{(6)} & \colhead{(7)}&\colhead{(8)} & \colhead{(9)}&\colhead{(10)}
}
\startdata
\hline
\pid{11878} & 2009.338 & 04-Dec-2009 & 232.581 & -237.544 & -0.191 & -0.033 & 3-Aug-2009 & 232.772 & -237.511\\
\pid{11878} & 2010.074 & 15-Mar-2010 & 232.488 & -237.462 & -0.284 & 0.049 & \dots & 232.772 & -237.511\\
\pid{11878} & 2010.110 & 20-Apr-2010 & 232.481 & -237.457 & -0.291 & 0.054 & \dots & 232.772 & -237.511\\
\pid{11878} & 2010.309 & 05-Nov-2010 & 232.604 & -237.561 & -0.168 & -0.050 & \dots & 232.772 & -237.511\\
\pid{12399} & 2011.070 & 11-Mar-2011 & 232.645 & -237.438 & -0.127 & 0.073 & 20-Jun-2011 & 232.772 & -237.511\\
\hline
\pid{12399} & 2011.255 & 12-Sep-2011 & 232.737 & -237.507 & 0.091 & -0.062 & 21-Jun-2011 & 232.646 & -237.445\\
\pid{12781} & 2012.087 & 27-Mar-2012 & 232.622 & -237.515 & -0.024 & -0.070 & \dots & 232.646 & -237.445\\
\pid{12781} & 2012.268 & 24-Sep-2012 & 232.713 & -237.578 & 0.067 & -0.133 & \dots &232.646 & -237.445\\
\pid{13171} & 2013.061 & 02-Mar-2013 & 232.647 & -237.590 & 0.001 & -0.145 & \dots & 232.646 & -237.445\\
\pid{13171} & 2013.244 & 01-Sep-2013 & 232.723 & -237.515 & {\bf 0.077}\tablenotemark{b} & {\bf -0.070}\tablenotemark{b} & 23-Feb-2014 & 232.646 & -237.445\\
\hline
\pid{13616} & 2014.055 & 06-Apr-2014 & 232.535 & -237.497 & -0.188 & 0.018 & 24-Feb-2014 & 232.723 & -237.515\\
\pid{13616} & 2014.300 & 27-Oct-2014 & 232.841 & -237.465 & {\bf 0.118} & {\bf 0.050} & \dots & 232.723 & -237.515\\
\pid{14035} & 2015.104 & 14-May-2015 & 232.617 & -237.464 & -0.106 & 0.051 & \dots & 232.723 & -237.515\\
\pid{14035} & 2015.275 & 02-Oct-2015 & 232.788 & -237.462 & {\bf 0.065} & {\bf0.053} & \dots & 232.723 & -237.515\\
\pid{14452} & 2016.092 & 01-Apr-2016 & 232.742 & -237.485 & 0.019 & 0.030 & \dots & 232.723 & -237.515\\
\hline
\enddata
\tablenotetext{a}{Dates in this column show the dates that the [V2,V3] SIAF entries in the this and the following rows were active.}
\tablenotetext{b}{These exposures, and the offsets measured here, were used to adjust the COS SIAF entries on 2014.055 (STScI PR\#76982).}
\tablecomments{Items in {\bf bold} are used in the analysis of this ISR.}
\end{deluxetable}


