\subsection{NUV Spectroscopic TA verification}\label{subsec:NspVER}
The \plamptwo{} WCA lamp and target XD locations for all NUV spectroscopic exposures are given in Table~\ref{tab:XDdataNUV}.
As shown in the two rightmost ``$|\Delta|$'' columns, all measured WCA-to-PSA offsets were within 3~p in XD of their FSW values. This equates to a $<$ 0.07\arcsec{} XD offset due to TA
for all NUV monitoring exposures over C20--24.
A visual inspection of the spectra showed all NUV spectra to continue to be well centered in the \tacq{PEAKXD}, \tacq{PEAKD}, and \tacq{SEARCH} NUV spectroscopic subarrays.
%{\bf Note to reviewers: Table~\ref{tab:XDdataNUV} doesn't actually show the subarray check. This was just a visual check to
%make sure that the NUV spectrum was well contained in the subarray. If you think that a table comparing the XD line centers to
%the subarray edges is worthwhile, it can be easily incorporated.}\\

\begin{deluxetable}{rrrcrrrrrr}
\tablecolumns{10}
\tabletypesize{\footnotesize}
\tabcolsep 6 pt
\tablecaption{NUV Spectroscopic \texttt{ACQ/PEAKXD} Monitoring}\label{tab:XDdataNUV}
\tablehead{
\colhead{\textit{ROOTNAME}}&\colhead{\textit{DATE-OBS}}&\colhead{\textit{OPT\_ELEM}}&\colhead{LP}&
\colhead{WCA\tablenotemark{a}}&\colhead{PSA\tablenotemark{b}}&\colhead{WtP\tablenotemark{c}}&\colhead{eWtP\tablenotemark{d}}&\colhead{$|\Delta|$\tablenotemark{e}}&\colhead{$|\Delta|$\tablenotemark{f}}\\
\colhead{}&\colhead{}&\colhead{}&\colhead{}&\colhead{(p)}&\colhead{(p)}&\colhead{(p)}&\colhead{(p)}&\colhead{(p)}&\colhead{(\arcsec{}))}\\
\colhead{(1)}&\colhead{(2)} & \colhead{(3)}&\colhead{(4)} &
\colhead{(5)}&\colhead{(6)} & \colhead{(7)}&\colhead{(8)} &
\colhead{(9)}&\colhead{(10)}
}
\startdata
\toprule
\multicolumn{10}{c}{C21 (\pid{13526})}\\
\midrule
lcgq02i2q& 2014-11-17&  G185M& 1& 366.0 & 742.0 & 376.0 & 374.20&  1.80 &  0.04\\
lcgq02i4q& 2014-11-17&  G225M& 1& 370.0 & 747.0 & 377.0 & 374.60&  2.40 &  0.06\\
lcgq01r6q& 2014-11-19&  G285M& 1& 355.0 & 728.0 & 373.0 & 374.90&  1.90 &  0.04\\
lcgq01qlq& 2014-11-19&  G230L& 1& 374.0 & 748.0 & 374.0 & 373.40&  0.60 &  0.01\\
\midrule
\multicolumn{10}{c}{C22 (\pid{13972})}\\
\midrule
lcri02hqq& 2015-10-06&  G185M& 1& 367.0 & 742.0 & 375.0 & 374.20&  0.80 &  0.02\\
lcri02hoq& 2015-10-06&  G225M& 1& 371.0 & 747.0 & 376.0 & 374.60&  1.40 &  0.03\\
lcri01giq& 2015-10-06&  G285M& 1& 351.0 & 726.0 & 375.0 & 374.90&  0.10 &  $<$0.01\\
lcri01ggq& 2015-10-06&  G230L& 1& 374.0 & 747.0 & 373.0 & 373.40&  0.40 &  0.01\\
\midrule
\multicolumn{10}{c}{C23 (\pid{14440})}\\
\midrule
ld3702noq& 2016-10-19&  G185M& 1& 366.0 & 743.0 & 377.0 & 374.20&  2.80 &  0.07\\
ld3702nmq& 2016-10-19&  G225M& 1& 370.0 & 747.0 & 377.0 & 374.60&  2.40 &  0.06\\
ld3701hbq& 2016-10-18&  G285M& 1& 352.0 & 727.0 & 375.0 & 374.90&  0.10 &  $<$0.01\\
ld3701h9q& 2016-10-18&  G230L& 1& 375.0 & 748.0 & 373.0 & 373.40&  0.40 &  0.01\\
\midrule
\multicolumn{10}{c}{C24 (\pid{14857})}\\
\midrule
ldozbblwq& 2017-09-06&  G185M& 1& 366.0 & 743.0 & 377.0 & 374.20&  2.80 &  0.07\\
ldozbbluq& 2017-09-06&  G225M& 1& 370.0 & 747.0 & 377.0 & 374.60&  2.40 &  0.06\\
ldozbadzq& 2017-09-04&  G285M& 1& 352.0 & 727.0 & 375.0 & 374.90&  0.10 &  $<$0.01\\
ldozbadxq& 2017-09-04&  G230L& 1& 374.0 & 748.0 & 374.0 & 373.40&  0.60 &  0.01\\
\bottomrule
\enddata
\tablenotetext{a}{XD centroid of the WCA spectrum. For NUV spectra, this is the median calibration lamp location.}
\tablenotetext{b}{XD centroid of the target spectrum taken through the PSA, using the same centroid method as the WCA.}
\tablenotetext{c}{WtP is absolute value of XD location difference of measured WCA and PSA spectra ($WtP = |PSA-WCA|$)}
\tablenotetext{d}{eWtP = Expected WCA-to-PSA offset from FSW table \textsc{XimCalTargetOffset} (see Table~\ref{tab:wcatopsa}).}
\tablenotetext{e}{Offset of WtP from a perfectly centered target measured in XD rows.}
\tablenotetext{f}{Offset of WtP in arcseconds (\arcsec{}). Note that the platescales are different for each grating, as shown in Table~\ref{tab:FSWplatescales}.}
\tablecomments{All spectra taken at \texttt{FP-POS=3}. All verifications ($|\Delta| = |WtP-eWtp|$) easily exceeded both the $\pm$0.3\arcsec{} requirement and the $\pm$0.1\arcsec{} goal.}
\end{deluxetable}
\begin{deluxetable}{rrrcrrrrrr}
\tabletypesize{\footnotesize}
\tablecolumns{10}
\tabcolsep 6 pt
\tablecaption{FUV Spectroscopic \texttt{ACQ/PEAKXD} Monitoring}\label{tab:XDdataFUV}
\tablehead{
\colhead{\textit{ROOTNAME}}&\colhead{\textit{DATE-OBS}}&\colhead{\textit{OPT\_ELEM}}&\colhead{LP}&
\colhead{WCA\tablenotemark{a}}&\colhead{PSA\tablenotemark{b}}&\colhead{WtP\tablenotemark{c}}&\colhead{eWtP\tablenotemark{d}}&\colhead{$|\Delta|$\tablenotemark{e}}&\colhead{$|\Delta|$\tablenotemark{f}}\\
\colhead{}&\colhead{}&\colhead{}&\colhead{}&\colhead{(p)}&\colhead{(p)}&\colhead{(p)}&\colhead{(p)}&\colhead{(p)}&\colhead{(\arcsec))}\\
\colhead{(1)}&\colhead{(2)} &
\colhead{(3)}&\colhead{(4)} &
\colhead{(5)}&\colhead{(6)} &
\colhead{(7)}&\colhead{(8)} &
\colhead{(9)}&\colhead{(10)}
}
\startdata
\toprule
\multicolumn{10}{c}{C21 (\pid{13526})}\\
\midrule
lcgq01r8q& 2014-11-19&  G130M& 2& 602.15& 508.31& -93.84& -92.80&  1.04& 0.12 \\
lcgq01raq& 2014-11-19&  G140L& 2& 608.76& 513.48& -95.28& -93.50&  1.78& 0.20 \\
lcgq02i6q& 2014-11-17&  G160M& 2& 596.07& 503.35& -92.71& -91.80&  0.91& 0.11 \\
\midrule
\multicolumn{10}{c}{C22 (\pid{13972})}\\
\midrule
lcri01gkq& 2015-10-06&  G130M& 3& 537.32& 448.98& -88.34& -89.20&  0.86& 0.08 \\
lcri01h6q& 2015-10-06&  G140L& 3& 544.55& 457.36& -87.20& -85.70&  1.50& 0.15 \\
lcri02hsq& 2015-10-06&  G160M& 3& 531.78& 442.13& -89.65& -90.10&  0.45& 0.04 \\
\midrule
\multicolumn{10}{c}{C23 (\pid{14440})}\\
\midrule
ld3701hdq& 2016-10-18&  G130M& 3& 536.33& 447.41& -88.92& -89.20&  0.28&  0.03 \\
ld3701hfq& 2016-10-18&  G140L& 3& 543.43& 455.95& -87.48& -85.70&  1.78&  0.17 \\
ld3702nqq& 2016-10-19&  G160M& 3& 531.09& 440.96& -90.13& -90.10&  0.03&  0.00 \\
\midrule
\multicolumn{10}{c}{C24 (\pid{14857})}\\
\midrule
ldozbae1q& 2017-09-04&  G130M& 3& 535.26& 445.66& -89.61& -89.20&  0.41&  0.04 \\
ldozbae3q& 2017-09-04&  G140L& 3& 541.76& 454.25& -87.51& -85.70&  1.81&  0.18 \\
ldozbblyq& 2017-09-06&  G160M& 3& 530.84& 440.75& -90.09& -90.10&  0.01&  0.00 \\
\bottomrule
\enddata
\tablenotetext{a}{XD centroid of the WCA spectrum.For FUV spectra, this is mean lamp photon location.}
\tablenotetext{b}{XD centroid of the target spectrum taken through the PSA, using the same centroid method as the WCA.}
\tablenotetext{c}{WtP is absolute value of XD location difference of measured WCA and PSA spectra ($WtP = |PSA-WCA|$)}
\tablenotetext{d}{eWtP = Expected WCA-to-PSA offset from FSW table \textsc{XimCalTargetOffset} (see Table~\ref{tab:wcatopsa}).}
\tablenotetext{e}{Offset of WtP from a perfectly centered target measured in XD rows.}
\tablenotetext{f}{Offset of WtP in \arcsec{}. Note that the platescales are different for each FUV grating and LP, as shown in Table~\ref{tab:FSWplatescales}.}
\tablecomments{All spectra taken at \texttt{FP-POS=3}. All FUV verifications ($|\Delta|$ = $|WtP-eWtp|$) exceeded both the $\pm$0.3\arcsec{} requirement,
but spectra taken near LP2 lifetime end, and all G140L spectra, exceeded the $\pm$0.1\arcsec{} goal.}
\end{deluxetable}

% Here is the +/- 0.7" table if we want to put it in
% Note the POSTARG extra column at the end
%13526& lcgq02i8q& 2014-11-17&  G160M& 2& 596.45& 509.80& -86.66& -91.80&  5.14&  0.61 &	+0.7\\
%13526& lcgq02iaq& 2014-11-17&  G160M& 2& 596.35& 498.41& -97.94& -91.80&  6.14&  0.73 &	-0.7\\
%13972& lcri02huq& 2015-10-06&  G160M& 3& 531.48& 449.64& -81.84& -90.10&  8.26&  0.75 &	+0.7\\
%13972& lcri02hwq& 2015-10-06&  G160M& 3& 531.73& 434.22& -97.51& -90.10&  7.41&  0.67 &	-0.7\\
%14440& ld3702nsq& 2016-10-19&  G160M& 3& 531.00& 448.43& -82.57& -90.10&  7.53&  0.68 &	+0.7\\
%14440& ld3702nuq& 2016-10-19&  G160M& 3& 531.24& 433.36& -97.88& -90.10&  7.78&  0.70 &	-0.7\\
%14857& ldozbbm0q& 2017-09-06&  G160M& 3& 530.70& 448.18& -82.52& -90.10&  7.58&  0.68 &	+0.7\\
%14857& ldozbbm2q& 2017-09-06&  G160M& 3& 530.75& 433.39& -97.36& -90.10&  7.26&  0.66 &	-0.7\\
