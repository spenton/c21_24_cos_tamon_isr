\section{Spectroscopic TA Verification}\label{sec:spVER}

After the series of \texttt{ACQ/IMAGE}s that start each visit, the target should be accurately centered. We take advantage of this to monitor COS spectroscopic TAs.

COS spectroscopic TAs consist of up to three stages \texttt{ACQ/SEARCH}, \texttt{ACQ/PEAKD}, and \texttt{ACQ/PEAKXD}.
The COS spectroscopic \texttt{ACQ/SEARCH} and \texttt{ACQ/PEAKD} algorithms do not use any FSW patchable constants, and do not flash the
internal calibration lamps. The only monitoring required for these TA phases is to ensure that the mechanisms were in their proper
positions and that the TA subarrays defined in the HST ground commanding are proper for the mechanism positions used during the acquisitions.

Each FUV central wavelength setting (CENWAVE) uses a unique OSM1 rotation, whereas all NUV TAs use the same OSM1 rotation independent of CENWAVE.
However, each NUV CENWAVE uses a different OSM2 rotation during TA. Each FUV CENWAVE has it's own set of TA subarrays (up to four per segment), while the NUV TA subarrays are not CENWAVE
specific, but are grating specific.

COS TA in the XD direction (\texttt{ACQ/PEAKXD}) requires the use of XD WCA-to-PSA offsets with the nominal {{\bf NUM$\_$POS}\rm}~=1 algorithm.
These values are contained for both the NUV and FUV channels in the FSW patchable constant table \textsc{pcta\_CalTargetOffset}, and are provided for reference in Table~\ref{tab:wcatopsa}.
All NUV and FUV (LP1, LP2 and LP3) \texttt{ACQ/PEAKXDs} monitored in this ISR used the original {{\bf NUM$\_$POS}\rm}~=1 algorithm).

The verification process is for \texttt{ACQ/PEAKXD} is simple, take a normal spectrum with a target signal-to-noise ratio of least 50 for the entire spectrum (2500 target counts) used by \texttt{ACQ/PEAKXD}.
For NUV exposures, this is the `B' stripe only, and for the FUV, only events on FUVA are used.
TA subarrays are used to mask out any detector hot-spots or Geocoronal light that could interfere with the centering process.

% $Id: pctaWCA2SA.tex,v 1.5 2018/03/30 20:22:12 penton Exp $
\begin{table}
\centering
	\begin{threeparttable}[tbc]
	\caption{\tacq{PEAKXD} WCA-to-PSA offsets}
	\begin{tabular*}{.75\linewidth}{@{\extracolsep{\fill}}lrrr}
		\toprule
		\textit{OPT\_ELEM} &	LP1	&	LP2	&	LP3	\\
		\midrule
		\multicolumn{4}{c}{FUV\tnote{1}}\\
		\midrule
		G130M	&	 -898	&	-943	&	-892 \\
		G140L	&	 -884	&	-950	&	-857 \\
		G160M	&	 -898	&	-933	&	-901 \\
		\midrule
		\multicolumn{4}{c}{NUV\tnote{2}}\\
		\midrule
		G130M	&	 -898	&	-943	&	-892 \\
		G140L	&	 -884	&	-950	&	-857 \\
		G160M	&	 -898	&	-933	&	-901 \\
		\bottomrule
	\end{tabular*}
	\footnotesize
		\begin{tablenotes}
			\item[1] {Divide the FUV numbers by -10 to get the number of XD rows between the PSA and WCA spectra for a target centered in the aperture.}
			\item[2] {Divide the NUV numbers by 10 to get the NUV WCA-to-PSA offset. }
		\end{tablenotes}
	The FSW patchable constant \textsc{pcta\_CalTargetOffsetScale} determines the FSW scaling (currently set to 10).
	FUV scalings are "negative" due to the parity of HST slews relative to the COS coordinate system.
	\label{tab:wcatopsa}
	\normalsize
	\end{threeparttable}
\end{table}

%\begin{deluxetable}{lrrr}
%\tablewidth{0pt}
%\tabcolsep 12 pt
%%\tabletypesize{\footnotesize}
%\tablecolumns{4}
%\tablecaption{\tacq{PEAKXD} WCA-to-PSA offsets \label{tab:wcatopsa}}
%\tablehead{
%\colhead{\textit{OPT\_ELEM}}&\colhead{LP1}&\colhead{LP2}&\colhead{LP3}\\
%}
%\startdata
%\toprule
%\multicolumn{4}{c}{FUV\tablenotemark{f}}\\
%\midrule
%G130M	&	 -898	&	-943	&	-892 \\
%G140L	&	 -884	&	-950	&	-857 \\
%G160M	&	 -898	&	-933	&	-901 \\
%\midrule
%\multicolumn{4}{c}{NUV\tablenotemark{n}}\\
%\midrule
%G185M	&	3742	&	\dots	&	\dots \\
%G225M	&	3746	&	\dots	&	\dots \\
%G230L	&	3734	&	\dots	&	\dots \\
%G285M	&	3749	&	\dots	&	\dots \\
%\bottomrule
%\enddata
%\tablenotetext{f}{Divide the FUV numbers by -10 to get the number of XD rows between the PSA and WCA spectra for a target centered in the aperture.}
%\tablenotetext{n}{Divide the NUV numbers by 10 to get the NUV WCA-to-PSA offset. }
%\tablecomments{The FSW patchable constant \textsc{pcta\_CalTargetOffsetScale} determines the FSW scaling (currently set to 10).
%FUV scalings are "negative" due to the parity of HST slews relative to the COS coordinate system.
%%{\bf Note to reviewers: Do you think I should keep the numbers in their FSW values (not scaled), or should I go ahead and scale them ?}}
%\end{deluxetable}

% $Id: tamon_output.tex,v 1.7 2018/04/16 21:16:03 penton Exp $

\begin{deluxetable}{rrrrrrrrrrrrrrrrrrr}
\tabcolsep 2pt
\tabletypesize{\tiny}
\tablecolumns{19}
\tablecaption{COS TA Monitor \texttt{ACQ/IMAGE} Data}\label{tab:Imagedata}
\tablehead{
\colhead{\textit{ROOTNAME}}&\colhead{\textit{EXPTYPE}}&\colhead{\textit{OPT\_ELEM}}&\colhead{LAMP}&\colhead{Current}&\colhead{Target ET}&\colhead{Lamp ET}&\colhead{WCA}&\colhead{WCA}&\colhead{SA}&\colhead{SA}&\colhead{WtP}&\colhead{WtP}&\colhead{Lamp}&\colhead{Lamp}&\colhead{WCA}&\colhead{Lamp}&\colhead{Lamp}&\colhead{Target}\\
\colhead{}&\colhead{}&\colhead{ }&\colhead{}&\colhead{}&\colhead{(s)}&\colhead{(s)}&\colhead{AD}&\colhead{XD}&\colhead{AD}&\colhead{XD}&\colhead{AD}&\colhead{XD}&\colhead{counts}&\colhead{cps}&\colhead{bck}&\colhead{CPS}&\colhead{BP}&\colhead{BP}\\
\colhead{(1)}&\colhead{(2)} & \colhead{(3)}&\colhead{(4)} &
\colhead{(5)}&\colhead{(6)} & \colhead{(7)}&\colhead{(8)} &
\colhead{(9)}&\colhead{(10)} & \colhead{(11)} &\colhead{(12)} &
\colhead{(13)}&\colhead{(14)} & \colhead{(15)}&\colhead{(16)} &
\colhead{(17)}&\colhead{(18)} & \colhead{(19)}
}
\startdata
\toprule
lcgq01q7q & EXT/SCI & MIRB & P2 & Med &  16 &  16 & 717 & 214 & 763.1 & 588.9 & 46.1 & 374.9 & 4890.0 & 305.6 & 167 & 305.6 & 4.4 & 26.7\\
lcgq01q9q & EXT/SCI & MIRA & P2 & Med & 150 & 150 & 479 & 370 & 550.3 & 739.9 & 71.3 & 369.9 & 1718.0 &\dots&\dots&\dots&\dots& 0.2\\
lcgq01qbq & WAVECAL & MIRA & P2 & Low &   7 &\dots& 503 & 372 & 596.4 & 869.2 & 93.4 & 497.2 & 2964.0 & 423.4 & 61 & 423.4 & 7.7 & 0.3\\
lcgq01qfq & WAVECAL & MIRA & P2 & Low &   7 &\dots& 503 & 372 & 652.2 & 793.2 & 149.2 & 421.2 & 2882.0 & 411.7 & 71 & 411.7 & 7.9 & 0.3\\
lcgq01qhq & EXT/SCI & MIRB & P2 & Med &  12 &  12 & 718 & 212 & 762.9 & 589.3 & 44.9 & 377.3 & 3391.0 & 282.6 & 151 & 282.6 & 3.9 & 19.9\\
lcgq02hoq & WAVECAL & MIRA & P2 & Low &   7 &\dots& 529 & 372 & 891.6 & 635.6 & 362.6 & 263.6 & 2827.0 & 403.9 & 97 & 403.9 & 9.9 & 0.3\\
lcgq02hqq & EXT/SCI & MIRB & P2 & Low & 181 &\dots& 713 & 211 & 784.4 & 582.7 & 71.4 & 371.7 & 2383.0 &\dots&\dots&\dots&\dots& 0.2\\
lcgq02hsq & WAVECAL & MIRB & P2 & Med &  12 &\dots& 738 & 212 & 898.7 & 439.2 & 160.7 & 227.2 & 3683.0 & 306.9 & 165 & 306.9 & 4.8 & 0.2\\
lcgq02hwq & WAVECAL & MIRB & P2 & Med &  12 &\dots& 738 & 213 & 927.8 & 656.8 & 189.8 & 443.8 & 3575.0 & 297.9 & 145 & 297.9 & 3.9 & 0.2\\
lcgq02hyq & WAVECAL & MIRA & P2 & Low &  10 &\dots& 522 & 372 & 451.2 & 711.3 & -70.8 & 339.3 & 4173.0 & 417.3 & 147 & 417.3 & 7.7 & 0.2\\
lcgq02icq & WAVECAL & MIRA & P1 & Low &  10 &\dots& 537 & 374 & 803.3 & 768.3 & 266.3 & 394.3 & 26040.0 & 2604.0 & 120 & 2604.0 & 46.7 & 0.2\\
lcgq02ieq & WAVECAL & MIRA & P2 & Low &  10 &\dots& 538 & 374 & 559.7 & 667.1 & 21.7 & 293.1 & 4036.0 & 403.6 & 122 & 403.6 & 7.4 & 0.2\\
lcgq02igq & WAVECAL & MIRB & P1 & Low &  30 &\dots& 747 & 215 & 879.3 & 654.8 & 132.3 & 439.8 & 2659.0 & 88.6 & 364 & 88.6 & 1.3 & 0.1\\
lcgq02iiq & WAVECAL & MIRB & P2 & Med &  20 &\dots& 747 & 215 & 539.0 & 725.6 & -208.0 & 510.6 & 6620.0 & 331.0 & 250 & 331.0 & 4.5 & 0.1\\
lcri01g1q & EXT/SCI & MIRB & P2 & Med &  12 &  12 & 722 & 210 & 767.7 & 584.2 & 45.7 & 374.2 & 3016.0 & 251.3 & 166 & 251.3 & 4.2 & 30.0\\
lcri01g3q & EXT/SCI & MIRA & P2 & Med & 150 &\dots& 474 & 370 & 552.0 & 735.7 & 78.0 & 365.7 & 1964.0 &\dots&\dots&\dots&\dots& 0.2\\
lcri01g5q & WAVECAL & MIRA & P2 & Low &  10 &\dots& 506 & 372 & 768.5 & 842.0 & 262.5 & 470.0 & 4100.0 & 410.0 & 117 & 410.0 & 9.8 & 0.2\\
lcri01g9q & WAVECAL & MIRA & P2 & Low &  10 &\dots& 506 & 371 & 278.3 & 582.7 & -227.7 & 211.7 & 3960.0 & 396.0 & 148 & 396.0 & 9.2 & 0.2\\
lcri01gcq & EXT/SCI & MIRB & P2 & Med &  14 &  12 & 723 & 212 & 767.4 & 588.9 & 44.4 & 376.9 & 3381.7 & 281.8 & 148 & 281.8 & 4.0 & 28.3\\
lcri02haq & WAVECAL & MIRA & P2 & Low &  14 &\dots& 526 & 372 & 644.1 & 719.4 & 118.1 & 347.4 & 5730.0 & 409.3 & 195 & 409.3 & 8.4 & 0.1\\
lcri02hcq & EXT/SCI & MIRB & P2 & Low & 181 & 181 & 715 & 211 & 782.3 & 578.6 & 67.3 & 367.6 & 2406.0 &\dots&\dots&\dots&\dots& 0.2\\
lcri02heq & WAVECAL & MIRB & P2 & Med &  24 &\dots& 737 & 213 & 853.4 & 647.7 & 116.4 & 434.7 & 7167.0 & 298.6 & 308 & 298.6 & 4.6 & 0.1\\
lcri02hiq & WAVECAL & MIRB & P2 & Med &  24 &\dots& 737 & 213 & 606.7 & 645.2 & -130.3 & 432.2 & 7316.0 & 304.8 & 295 & 304.8 & 4.5 & 0.1\\
lcri02hkq & WAVECAL & MIRA & P2 & Low &  14 &\dots& 519 & 372 & 551.0 & 580.0 & 32.0 & 208.0 & 5840.0 & 417.1 & 203 & 417.1 & 7.9 & 0.1\\
lcri02hyq & WAVECAL & MIRA & P1 & Low &  14 &\dots& 463 & 372 & 683.3 & 807.3 & 220.3 & 435.3 & 36245.0 & 2588.9 & 201 & 2588.9 & 45.5 & 0.1\\
lcri02i0q & WAVECAL & MIRA & P2 & Low &  24 &\dots& 463 & 372 & 781.3 & 778.6 & 318.3 & 406.6 & 9864.0 & 411.0 & 303 & 411.0 & 6.9 & 0.1\\
lcri02i2q & WAVECAL & MIRB & P1 & Low &  30 &\dots& 672 & 213 & 486.2 & 739.8 & -185.8 & 526.8 & 2864.0 & 95.5 & 415 & 95.5 & 1.3 & 0.1\\
lcri02i4q & WAVECAL & MIRB & P2 & Med &  24 &\dots& 671 & 212 & 884.3 & 415.3 & 213.3 & 203.3 & 8082.0 & 336.8 & 312 & 336.8 & 4.9 & 0.1\\
ld3701gvq & EXT/SCI & MIRB & P2 & Med &  16 &  16 & 727 & 210 & 772.8 & 584.3 & 45.8 & 374.3 & 4147.0 & 259.2 & 184 & 259.2 & 4.3 & 19.0\\
ld3701gxq & EXT/SCI & MIRA & P2 & Med & 150 & 150 & 479 & 371 & 551.2 & 735.8 & 72.2 & 364.8 & 1739.0 &\dots&\dots&\dots&\dots& 0.2\\
ld3701gzq & WAVECAL & MIRA & P2 & Low &   9 &\dots& 505 & 372 & 413.8 & 701.7 & -91.2 & 329.7 & 3667.0 & 407.4 & 94 & 407.4 & 8.1 & 0.2\\
ld3701h3q & WAVECAL & MIRA & P2 & Low &  10 &\dots& 505 & 372 & 802.6 & 780.0 & 297.6 & 408.0 & 3999.0 & 399.9 & 107 & 399.9 & 7.6 & 0.2\\
ld3701h5q & EXT/SCI & MIRB & P2 & Med &  16 & 1 6 & 728 & 212 & 773.4 & 589.0 & 45.4 & 377.0 & 4343.0 & 271.4 & 185 & 271.4 & 4.6 & 19.1\\
ld3702n1q & WAVECAL & MIRA & P2 & Low &  14 &\dots& 515 & 371 & 886.6 & 659.4 & 371.6 & 288.4 & 5589.0 & 399.2 & 167 & 399.2 & 7.7 & 0.2\\
ld3702n4q & EXT/SCI & MIRB & P2 & Low & 183 & 183 & 723 & 213 & 774.9 & 577.6 & 51.9 & 364.6 & 2081.0 &\dots&\dots&\dots&\dots& 0.2\\
ld3702n7q & WAVECAL & MIRB & P2 & Med &  24 &\dots& 728 & 212 & 778.9 & 703.3 & 50.9 & 491.3 & 7288.0 & 303.7 & 277 & 303.7 & 4.5 & 0.1\\
ld3702nbq & WAVECAL & MIRB & P2 & Med &  24 &\dots& 728 & 212 & 248.1 & 419.3 & -479.9 & 207.3 & 7140.0 & 297.5 & 274 & 297.5 & 4.5 & 0.1\\
ld3702neq & WAVECAL & MIRA & P2 & Low &  14 &\dots& 507 & 372 & 911.7 & 878.5 & 404.7 & 506.5 & 5622.0 & 401.6 & 153 & 401.6 & 8.1 & 0.1\\
ld3702o1q & WAVECAL & MIRA & P1 & Low &  14 &\dots& 531 & 371 & 485.9 & 883.6 & -45.1 & 512.6 & 37530.0 & 2680.7 & 172 & 2680.7 & 45.6 & 0.1\\
ld3702o3q & WAVECAL & MIRA & P2 & Low &  24 &\dots& 531 & 371 & 665.9 & 888.6 & 134.9 & 517.6 & 9841.0 & 410.0 & 273 & 410.0 & 6.9 & 0.1\\
ld3702o5q & WAVECAL & MIRB & P1 & Low &  30 &\dots& 744 & 211 & 651.6 & 609.1 & -92.4 & 398.1 & 2375.0 & 79.2 & 319 & 79.2 & 1.5 & 0.1\\
ld3702o7q & WAVECAL & MIRB & P2 & Med &  24 &\dots& 743 & 211 & 940.2 & 700.2 & 197.2 & 489.2 & 6674.0 & 278.1 & 283 & 278.1 & 4.2 & 0.1\\
ldozbadjs & EXT/SCI & MIRB & P2 & Med &  16 & 16  & 724 & 210 & 769.8 & 583.4 & 45.8 & 373.4 & 4005.0 & 250.3 & 138 & 250.3 & 4.4 & 20.2\\
ldozbadlq & EXT/SCI & MIRA & P2 & Med &  150& 150 & 472 & 371 & 545.1 & 735.6 & 73.1 & 364.6 & 1462.0 &\dots&\dots&\dots&\dots& 0.2\\
ldozbadnq & WAVECAL & MIRA & P2 & Low &   9 &\dots& 499 & 372 & 889.8 & 583.2 & 390.8 & 211.2 & 3688.0 & 409.8 & 76 & 409.8 & 7.7 & 0.2\\
ldozbadrq & WAVECAL & MIRA & P2 & Low &  10 &\dots& 498 & 372 & 311.8 & 608.8 & -186.2 & 236.8 & 4009.0 & 400.9 & 97 & 400.9 & 7.0 & 0.2\\
ldozbadtq & EXT/SCI & MIRB & P2 & Med &  16 &  16 & 725 & 212 & 769.8 & 588.9 & 44.8 & 376.9 & 4367.0 & 272.9 & 121 & 272.9 & 3.7 & 21.0\\
ldozbblgq & WAVECAL & MIRA & P2 & Low &  14 &\dots& 507 & 372 & 748.6 & 911.9 & 241.6 & 539.9 & 5721.0 & 408.6 & 155 & 408.6 & 8.4 & 0.1\\
ldozbbliq & EXT/SCI & MIRB & P2 & Low &  183&\dots& 713 & 213 & 776.2 & 578.7 & 63.2 & 365.7 & 2283.0 &\dots&\dots&\dots&\dots& 0.2\\
ldozbblkq & WAVECAL & MIRB & P2 & Med &  24 &\dots& 730 & 212 & 585.6 & 716.8 & -144.4 & 504.8 & 6957.0 & 289.9 & 331 & 289.9 & 4.7 & 0.1\\
ldozbbloq & WAVECAL & MIRB & P2 & Med &  24 &\dots& 730 & 212 & 703.1 & 689.2 & -26.9 & 477.2 & 6983.0 & 291.0 & 305 & 291.0 & 4.0 & 0.1\\
ldozbblqq & WAVECAL & MIRA & P2 & Low &  14 &\dots& 510 & 372 & 380.6 & 845.9 & -129.4 & 473.9 & 5566.0 & 397.6 & 177 & 397.6 & 7.9 & 0.1\\
ldozbbm4q & WAVECAL & MIRA & P1 & Low &  16 &\dots& 503 & 371 & 815.0 & 659.6 & 312.0 & 288.6 & 42548.0 & 2659.2 & 189 & 2659.2 & 44.2 & 0.1\\
ldozbbm6q & WAVECAL & MIRA & P2 & Low &  26 &\dots& 503 & 371 & 772.1 & 616.6 & 269.1 & 245.6 & 10476.0 & 402.9 & 300 & 402.9 & 7.6 & 0.1\\
ldozbbm8q & WAVECAL & MIRB & P1 & Low &  32 &\dots& 715 & 211 & 252.8 & 463.5 & -462.2 & 252.5 & 2714.0 & 84.8 & 407 & 84.8 & 1.4 & 0.1\\
ldozbbmaq & WAVECAL & MIRB & P2 & Med &  26 &\dots& 715 & 211 & 560.5 & 575.1 & -154.5 & 364.1 & 7768.0 & 298.8 & 340 & 298.8 & 3.7 & 0.1\\
ldozpbf7q & EXT/SCI & MIRA & P2 & Low &  20 &  20 & 511 & 370 & 555.5 & 741.6 & 44.5 & 371.6 & 7790.0 & 389.5 & 269 & 389.5 & 7.3 & 17.2\\
ldozpbf9q & EXT/SCI & MIRB & P2 & Med & 220 &  40 & 734 & 210 & 779.2 & 582.8 & 45.2 & 372.8 & 12877.2 & 321.9 & 523 & 321.9 & 3.5 & 0.6\\
ldozpbfdq & EXT/SCI & MIRB & P2 & Med & 220 &  40 & 734 & 211 & 780.3 & 584.0 & 46.3 & 373.0 & 13043.9 & 326.1 & 505 & 326.1 & 3.5 & 0.8\\
ldozpbffq & EXT/SCI & MIRA & P2 & Low &  20 &  20 & 514 & 370 & 559.3 & 743.2 & 45.3 & 373.2 & 7798.0 & 389.9 & 285 & 389.9 & 7.1 & 23.4\\
\enddata
\tablecomments{{\bf Note to reviewer: Some of the numbers in this table are odd, I am researching.}}
\end{deluxetable}

\subsection{NUV Spectroscopic TA verification}\label{subsec:NspVER}
\subsection{FUV Spectroscopic TA verification}\label{subsec:FspVER}
