% $Id: NimVer.tex,v 1.5 2018/03/30 20:22:12 penton Exp $

\subsection{Verifying the \tacq{IMAGE} WCA-to-SA Offsets.}\label{subsec:WCA2SAVER}

The verification of the \tacq{IMAGE} WCA-to-SA (PSA or BOA) offsets is a multi-stage bootstrap
process similar to the one used to measure the initial offsets in the SMOV enabling program (\pid{11471}, COS FUV Target Acquisition Algorithm Verification).

Each visit of each cycles monitoring program directly compares two \tacq{IMAGE} combinations.
We can bootstrap these back to PSA$\times$MIRA to test the co-alignment of all four combinations.
We call this the `baseline` bootstrapping, the results of which are shown in Table~\ref{tab:basebootstrap}
and discussed in \S~\ref{subsec:basebootstrap}

These measurements have certain limitations, so as the $\pm 0.5p$ measurement uncertainity in both directions
when measuring the WCA centroid. The WCA lamp exposures of each cycles program assist in removing
these limitations from the bootstrapping. The PSA \tacq{IMAGE} visits level the shutter open when
taking these WCA lamp images so that a co-eval target+lamp TT image is acquired. This allows a direct
calculation of the WCA-to-PSA offset using any desired centroiding algoritm. The BOA \tacq{IMAGE} visits
take sequential lamp and BOA target images to measure the WCA-to-BOA offsets, but these are not co-eval,
and the aperture has been moved between the exposures, which often causes a $\pm$ one step offset.
Fortunately, this offset can be tracked with the telemetry keywords and accounted for.


\subsection{Baseline Bootstrap of \tacq{IMAGE modes}}\label{subsec:basebootstrap}

The baseline bootstrapping data is given in Table~\ref{tab:tamonbasicnimverT}.

% tamon_basic_nimver
%
% RCSID="$Id$"
%
\begin{deluxetable}{llrccccccrrrrr}
\tablecolumns{14}
\tabcolsep 3pt
\tablecaption{C17--C20 PSA \tacq{IMAGE} Co-alignment Measurements\label{tab:basebootstrap}}
\tabletypesize{\scriptsize}
\tablehead{
\colhead{\textit{PROP}}&\colhead{\textit{ROOT}}&\colhead{Configuration}  &
\multicolumn{2}{c}{WCA-Msrd\tablenotemark{a}}  &
\multicolumn{2}{c}{PSA-Msrd}  &
\multicolumn{2}{c}{PSA-Centered}  &
\multicolumn{2}{c}{SA-to-WCA}     &
\multicolumn{2}{c}{TA Centering}  &
\colhead{\textit{DATE-}}\\

\colhead{\textit{OSID}}& \colhead{\textit{NAME}} & \colhead{\textit{APERTURE}}&
\colhead{\textit{MXCR}}&\colhead{\textit{MYCR}}&
\colhead{\textit{ACQ}}&\colhead{\textit{ACQ}}&
\colhead{\textit{ACQ}}&\colhead{\textit{ACQ}}&
\colhead{} &\colhead{}&
\colhead{\textit{ACQ}}&\colhead{\textit{ACQ}}&
\colhead{\textit{OBS}}\\

\colhead{}&\colhead{}&\colhead{$\times$}&
\colhead{\textit{LAMP}}&\colhead{\textit{LAMP}}&
\colhead{\textit{CENTX}}&\colhead{\textit{CENTY}}&
\colhead{\textit{PREFX}}&\colhead{\textit{PREFY}}&
\colhead{} & \colhead{} &
\colhead{\textit{SLEWX}}&\colhead{\textit{SLEWY}}&
\colhead{}\\

\colhead{(PID)}&\colhead{}&\colhead{\textit{OPT\_ELEM}}&
\colhead{AD} &\colhead{XD\tablenotemark{b}}&
\colhead{AD} &\colhead{XD\tablenotemark{b}}&
\colhead{AD} &\colhead{XD\tablenotemark{b}}&
\colhead{AD} &\colhead{XD\tablenotemark{b}}&
\colhead{AD} &\colhead{XD\tablenotemark{b}}&
\colhead{}\\

\colhead{(1)} & \colhead{(2)} & \colhead{(3)}&
\colhead{(4)} & \colhead{(5)} & \colhead{(6)} & \colhead{(7)} &
\colhead{(8)} & \colhead{(9)} & \colhead{(10)} & \colhead{(11)} &
\colhead{(12)} & \colhead{(13)} & \colhead{(14)}
}
\startdata
\toprule
\multicolumn{14}{c}{C17} \\
\midrule
\pid{11878} &lbcla3s3q&PSA$\times$MIRA&555&653&509.3&282.7&509.7&280.3& 45.3&372.7&  -0.010&   0.056&2010-11-05 \\
\pid{11878} &lbcla3s7q&PSA$\times$MIRB&342&813&296.4&439.8&297.0&438.9& 45.0&374.1&  -0.015&   0.021&2010-11-05 \\
\midrule
\multicolumn{14}{c}{C18} \\
\midrule
\pid{12399} &lbm7a2ahq&PSA$\times$MIRA&529&653&475.2&279.5&483.7&280.3& 45.3&372.7&  -0.200&  -0.019&2011-09-12 \\
\pid{12399} &lbm7a2ajq&PSA$\times$MIRB&317&813&271.2&439.3&272.0&438.9& 45.0&374.1&  -0.018&   0.008&2011-09-12 \\
\midrule
\multicolumn{14}{c}{C19} \\
\midrule
\pid{12781} &lbx1a2ffq&PSA$\times$MIRA&503&650&450.0&280.6&457.7&277.3& 45.3&372.7&  -0.183&   0.078&2012-09-24 \\
\pid{12781} &lbx1a2fhq&PSA$\times$MIRB&296&811&249.8&436.0&251.0&436.9& 45.0&374.1&  -0.028&  -0.021&2012-09-24 \\
\midrule
\multicolumn{14}{c}{C20} \\
\midrule
\pid{13171} &lc6ka2imq&PSA$\times$MIRA&508&650&459.7&284.0&462.7&277.3& 45.3&372.7&  -0.070&   0.158&2013-09-01 \\
\pid{13171} &lc6ka2ioq&PSA$\times$MIRB&304&811&258.2&436.5&259.0&436.9& 45.0&374.1&  -0.019&  -0.009&2013-09-01 \\
\midrule
\bottomrule
\enddata
\tablenotetext{a}{Non-repeatability of the OSM and aperture mechanisms, along with environmental factors, result in
lamp center offsets of up to 3~p in AD and to 52~p in XD in these exposures. }
%\tablenotetext{b}{BOA \tacq{IMAGE}s move the aperture in the XD direction to obtain the WCA lamp image. Occasionally, the aperture mechanism misses the desired location by $\pm 1 $ \textit{APERXPOS} step of $\sim0.05\arcsec$.}
%\tablenotetext{c}{These PSA$\times$MIRA \tacq{IMAGE}s were part of the FGS-to-SI programs and do {\bf NOT} have a proceeding TA. The TA centering adjustments presented here are to be compared to the FGS-to-SI post processing results presented in Table~\ref{tab:fgs2siInit}.}
\tablecomments{If the table caption is in the \texttt{Courier} font, this value was taken directly from the indicated \textsc{\_rawacq.fits} header keyword. In DETector coordinates, +AD is -Y$_{DET}$, +XD is -X$_{DET}$, in USER coordinates, +AD is +Y$_{USER}$, +XD is +X$_{USER}$.}
\end{deluxetable}

\begin{deluxetable}{llrccccccrrrrr}
\tablecolumns{14}
\tabcolsep 3pt
\tablecaption{C21--24 \tacq{IMAGE} Bootstrapping Results\label{tab:basebootstrap}}
\tabletypesize{\scriptsize}
\tablehead{
\colhead{\textit{PROP}}&\colhead{\textit{ROOT}}&\colhead{Configuration}  &
\multicolumn{2}{c}{WCA-Msrd\tablenotemark{a}}  &
\multicolumn{2}{c}{SA-Msrd\tablenotemark{b}}  &
\multicolumn{2}{c}{SA-Centered}  &
\multicolumn{2}{c}{SA-to-WCA}     &
\multicolumn{2}{c}{TA Centering}  &
\colhead{\textit{DATE-}}\\

\colhead{\textit{OSID}}& \colhead{\textit{NAME}} & \colhead{\textit{APERTURE}}&
\colhead{\textit{MXCR}}&\colhead{\textit{MYCR}}&
\colhead{\textit{ACQ}}&\colhead{\textit{ACQ}}&
\colhead{\textit{ACQ}}&\colhead{\textit{ACQ}}&
\colhead{} &\colhead{}&
\colhead{\textit{ACQ}}&\colhead{\textit{ACQ}}&
\colhead{\textit{OBS}}\\

\colhead{}&\colhead{}&\colhead{$\times$}&
\colhead{\textit{LAMP}}&\colhead{\textit{LAMP}}&
\colhead{\textit{CENTX}}&\colhead{\textit{CENTY}}&
\colhead{\textit{PREFX}}&\colhead{\textit{PREFY}}&
\colhead{} & \colhead{} &
\colhead{\textit{SLEWX}}&\colhead{\textit{SLEWY}}&
\colhead{}\\

\colhead{(PID)}&\colhead{}&\colhead{\textit{OPT\_ELEM}}&
\colhead{AD} &\colhead{XD}&
\colhead{AD} &\colhead{XD}&
\colhead{AD} &\colhead{XD}&
\colhead{AD} &\colhead{XD}&
\colhead{AD} &\colhead{XD}&
\colhead{}\\

\colhead{(1)} & \colhead{(2)} & \colhead{(3)}&
\colhead{(4)} & \colhead{(5)} & \colhead{(6)} & \colhead{(7)} &
\colhead{(8)} & \colhead{(9)} & \colhead{(10)} & \colhead{(11)} &
\colhead{(12)} & \colhead{(13)} & \colhead{(14)}
}
\startdata
\toprule
\multicolumn{14}{c}{C21} \\
\midrule
\pid{13616} &lci4a2e3q&PSA$\times$MIRA&517&650&471.7&282.9&471.7&277.3& 45.3&372.7&   0.001&   0.133&2014-10-27 \\
\pid{13616} &lci4a2e5q&PSA$\times$MIRB&305&809&259.7&436.1&259.0&435.0& 46.0&374.0&   0.016&   0.027&2014-10-27 \\
\midrule
\pid{13526} &lcgq03dbq&PSA$\times$MIRA&566&648&527.7&268.2&520.7&275.3& 45.3&372.7&   0.166&  -0.168&2014-10-06 \\
\pid{13526} &lcgq03dlq&PSA$\times$MIRB&357&808&310.8&434.2&312.0&433.9& 45.0&374.1&  -0.029&   0.006&2014-10-06 \\
\pid{13526} &lcgq03dtq&PSA$\times$MIRA&574&649&530.1&275.2&528.7&276.3& 45.3&372.7&   0.032&  -0.026&2014-10-06 \\
\pid{13526} &lcgq01q5q&PSA$\times$MIRB&306&809&222.4&447.7&260.0&435.0& 46.0&374.0&  -0.888&   0.298&2014-11-19 \\
\pid{13526} &lcgq01qdq&BOA$\times$MIRA&520&651&472.9&283.2&474.5&282.6& 45.5&368.4&  -0.038&   0.013&2014-11-19 \\
\pid{13526} &lcgq01qjq&PSA$\times$MIRB&305&811&260.2&433.7&259.0&437.0& 46.0&374.0&   0.027&  -0.078&2014-11-19 \\
\pid{13526} &lcgq02hmq&BOA$\times$MIRA&495&649&452.0&293.6&449.5&280.6& 45.5&368.4&   0.058&   0.305&2014-11-17 \\
\pid{13526} &lcgq02huq&BOA$\times$MIRB&285&811&237.9&440.3&238.5&444.8& 46.5&366.2&  -0.015&  -0.105&2014-11-17 \\
\pid{13526} &lcgq02i0q&BOA$\times$MIRA&501&651&455.7&285.3&455.5&282.6& 45.5&368.4&   0.006&   0.063&2014-11-17 \\
\midrule
\multicolumn{14}{c}{C22} \\
\midrule
\pid{14035} &lcsla2bhq&PSA$\times$MIRA&505&653&458.8&284.8&459.7&280.3& 45.3&372.7&  -0.020&   0.105&2015-10-02 \\
\pid{14035} &lcsla2bjq&PSA$\times$MIRB&293&813&247.9&439.3&247.0&439.0& 46.0&374.0&   0.022&   0.007&2015-10-02 \\
\pid{13972} &lcri01fzq&PSA$\times$MIRB&302&813&264.0&427.4&256.0&439.0& 46.0&374.0&   0.189&  -0.273&2015-10-06 \\
\pid{13972} &lcri01g7q&BOA$\times$MIRA&517&651&471.0&287.2&471.5&282.6& 45.5&368.4&  -0.011&   0.109&2015-10-06 \\
\pid{13972} &lcri01geq&PSA$\times$MIRB&300&812&255.5&434.1&254.0&438.0& 46.0&374.0&   0.036&  -0.092&2015-10-06 \\
\pid{13972} &lcri02h8q&BOA$\times$MIRA&499&654&462.0&272.2&453.5&285.6& 45.5&368.4&   0.201&  -0.315&2015-10-06 \\
\pid{13972} &lcri02hgq&BOA$\times$MIRB&286&810&240.6&444.2&239.5&443.8& 46.5&366.2&   0.026&   0.010&2015-10-06 \\
\pid{13972} &lcri02hmq&BOA$\times$MIRA&504&651&457.7&284.2&458.5&282.6& 45.5&368.4&  -0.018&   0.038&2015-10-06 \\
\midrule
\multicolumn{14}{c}{C23} \\
\midrule
\pid{14452} &ld3la2ojq&PSA$\times$MIRA&527&654&481.0&284.5&481.7&281.3& 45.3&372.7&  -0.017&   0.075&2016-10-02 \\
\pid{14452} &ld3la2onq&PSA$\times$MIRB&306&813&259.9&439.9&260.0&439.0& 46.0&374.0&  -0.002&   0.021&2016-10-02 \\
\pid{14440} &ld3701gtq&PSA$\times$MIRB&297&813&246.6&442.7&251.0&439.0& 46.0&374.0&  -0.104&   0.088&2016-10-18 \\
\pid{14440} &ld3701h1q&BOA$\times$MIRA&518&651&471.7&287.1&472.5&282.6& 45.5&368.4&  -0.018&   0.105&2016-10-18 \\
\pid{14440} &ld3701h7q&PSA$\times$MIRB&295&811&249.7&434.0&249.0&437.0& 46.0&374.0&   0.016&  -0.071&2016-10-18 \\
\pid{14440} &ld3702mzq&BOA$\times$MIRA&509&654&480.0&299.3&463.5&285.6& 45.5&368.4&   0.390&   0.322&2016-10-19 \\
\pid{14440} &ld3702n9q&BOA$\times$MIRB&295&811&248.5&446.1&248.5&444.8& 46.5&366.2&   0.001&   0.030&2016-10-19 \\
\pid{14440} &ld3702nhq&BOA$\times$MIRA&516&651&470.0&285.0&470.5&282.6& 45.5&368.4&  -0.012&   0.057&2016-10-19 \\
\midrule
\multicolumn{14}{c}{C24} \\
\midrule
\pid{14857} &ldozpbf5q&PSA$\times$MIRA&515&654&467.3&266.2&469.7&281.3& 45.3&372.7&  -0.058&  -0.355&2017-09-10 \\
\pid{14857} &ldozpbfbq&PSA$\times$MIRB&289&813&243.8&440.0&243.0&439.0& 46.0&374.0&   0.020&   0.023&2017-09-10 \\
\pid{14857} &ldozpbfhq&PSA$\times$MIRA&510&653&463.9&279.8&464.7&280.3& 45.3&372.7&  -0.020&  -0.011&2017-09-10 \\
\pid{14857} &ldozbadhq&PSA$\times$MIRB&299&813&237.5&397.9&253.0&439.0& 46.0&374.0&  -0.366&  -0.967&2017-09-04 \\
\pid{14857} &ldozbadpq&BOA$\times$MIRA&524&651&477.9&287.4&478.5&282.6& 45.5&368.4&  -0.013&   0.113&2017-09-04 \\
\pid{14857} &ldozbadvq&PSA$\times$MIRB&298&811&253.2&434.5&252.0&437.0& 46.0&374.0&   0.029&  -0.059&2017-09-04 \\
\pid{14857} &ldozbbleq&BOA$\times$MIRA&518&653&484.8&289.5&472.5&284.6& 45.5&368.4&   0.290&   0.114&2017-09-06 \\
\pid{14857} &ldozbblmq&BOA$\times$MIRB&293&811&246.7&444.3&246.5&444.8& 46.5&366.2&   0.006&  -0.011&2017-09-06 \\
\pid{14857} &ldozbblsq&BOA$\times$MIRA&514&651&467.9&285.1&468.5&282.6& 45.5&368.4&  -0.015&   0.060&2017-09-06 \\
\bottomrule
\enddata
\tablenotetext{a}{Non-repeatability of the OSM and aperture mechanisms, along with environmental factors, result in
lamp center offsets of up to 6~p in AD and $> 50$~p in XD in these exposures.}
\tablenotetext{b}{BOA \tacq{IMAGE}s move the aperture in the XD direction to obtain the WCA lamp image. Occasionally, the aperture mechanism misses the desired location by $\pm 1$ \textit{APERYPOS} step of $\sim0.05\arcsec$.}
%\tablenotetext{c}{These PSA$\times$MIRA \tacq{IMAGE}s were part of the FGS-to-SI programs and do not have a proceeding \tacq{IMAGE}. The TA centerings presented here are to be compared to the FGS-to-SI post processing results presented in Table~\ref{tab:fgs2siInit}.}
\tablecomments{If the table caption is in the \textit{ITALICS}, this value was taken directly from the indicated \textsc{\_rawacq.fits} header keyword. Columns 4-11 are in units of NUV pixels (p). Columns 12 \& 13 are in arcseconds (\arcsec). In DETector coordinates, +AD is -Y$_{DET}$, +XD is -X$_{DET}$, in USER coordinates, +AD is +Y$_{USER}$, +XD is +X$_{USER}$.}
\end{deluxetable}

\begin{deluxetable}{llrccccccrrrrr}
\tablecolumns{14}
\tabcolsep 4pt
\tablecaption{\tacq{IMAGE} Bootstrapping Measurements Sorted by Configuration\label{tab:bootstrapAligned}}
\tabletypesize{\scriptsize}
\tablehead{
\colhead{\textit{PROP}}&\colhead{\textit{ROOT}}&
\colhead{HST}  &
\multicolumn{2}{c}{WCA-Msrd\tablenotemark{a}}  &
\multicolumn{2}{c}{SA-Msrd\tablenotemark{b}}  &
\multicolumn{2}{c}{SA-Center}  &
\multicolumn{2}{c}{SA-to-WCA}     &
\multicolumn{2}{c}{TA Centering}&\colhead{\textit{DATE}}\\

\colhead{\textit{OSID}}& \colhead{\textit{NAME}} & \colhead{Cycle}&
\colhead{\textit{LAMP}}&\colhead{\textit{LAMP}}&
\colhead{\textit{ACQ}}&\colhead{\textit{ACQ}}&
\colhead{\textit{ACQ}}&\colhead{\textit{ACQ}}&
\colhead{}& \colhead{} &\colhead{\textit{ACQ}}&\colhead{\textit{ACQ}}& \colhead{\textit{OBS}} \\

\colhead{}& \colhead{} & \colhead{}&
\colhead{\textit{MXCR}}&\colhead{\textit{MYCR}}&
\colhead{\textit{MSRDX}}&\colhead{\textit{MSRDY}}&
\colhead{\textit{PREFX}}&\colhead{\textit{PREFY}}&
\colhead{}& \colhead{} &\colhead{\textit{SLEWX}}&\colhead{\textit{SLEWY}}& \\
\colhead{(PID)}  &     \colhead{}     & \colhead{} &
\colhead{AD} &\colhead{XD}&
\colhead{AD} &\colhead{XD}&
\colhead{AD} &\colhead{XD}&
\colhead{AD} &\colhead{XD}& \colhead{} \\
\colhead{(1)}  &\colhead{(2)} & \colhead{(3)}&\colhead{(4)} &
\colhead{(5)}  &\colhead{(6)} & \colhead{(7)}&\colhead{(8)} &
\colhead{(9)}  &\colhead{(10)} & \colhead{(11)} &\colhead{(12)} &
\colhead{(13)} &\colhead{(14)}
}
\startdata
\toprule
\multicolumn{14}{c}{PSA$\times$MIRRORA}\\
\midrule
\pid{11878} &lbcla3s3q&C17&555&653&509.3&282.7&509.7&280.3& 45.3&372.7&  -0.010&   0.056&2010-11-05 \\
\pid{12399} &lbm7a2ahq&C18&529&653&475.2&279.5&483.7&280.3& 45.3&372.7&  -0.200&  -0.019&2011-09-12 \\
\pid{12781} &lbx1a2ffq&C19&503&650&450.0&280.6&457.7&277.3& 45.3&372.7&  -0.183&   0.078&2012-09-24 \\
\pid{13171} &lc6ka2imq&C20&508&650&459.7&284.0&462.7&277.3& 45.3&372.7&  -0.070&   0.158&2013-09-01 \\
\pid{13616} &lci4a2e3q&C21&517&650&471.7&282.9&471.7&277.3& 45.3&372.7&   0.001&   0.133&2014-10-27 \\
\pid{14035} &lcsla2bhq&C22&505&653&458.8&284.8&459.7&280.3& 45.3&372.7&  -0.020&   0.105&2015-10-02 \\
\pid{14452} &ld3la2ojq&C23&527&654&481.0&284.5&481.7&281.3& 45.3&372.7&  -0.017&   0.075&2016-10-02 \\
\pid{14857} &ldozpbf5q&C24&515&654&467.3&266.2&469.7&281.3& 45.3&372.7&  -0.058&  -0.355&2017-09-10 \\
\pid{14857} &ldozpbfhq&C24&510&653&463.9&279.8&464.7&280.3& 45.3&372.7&  -0.020&  -0.011&2017-09-10 \\
\midrule
\multicolumn{14}{c}{PSA$\times$MIRRORB\tablenotemark{c}}\\
\midrule
\pid{11878} &lbcla3s7q&C17&342&813&296.4&439.8&297.0&438.9& 45.0&374.1&  -0.015&   0.021&2010-11-05 \\
\pid{12399} &lbm7a2ajq&C18&317&813&271.2&439.3&272.0&438.9& 45.0&374.1&  -0.018&   0.008&2011-09-12 \\
\pid{12781} &lbx1a2fhq&C19&296&811&249.8&436.0&251.0&436.9& 45.0&374.1&  -0.028&  -0.021&2012-09-24 \\
\pid{13171} &lc6ka2ioq&C20&304&811&258.2&436.5&259.0&436.9& 45.0&374.1&  -0.019&  -0.009&2013-09-01 \\
\hline
\pid{13616} &lci4a2e5q&C21&305&809&259.7&436.1&259.0&435.0& 46.0&374.0&   0.016&   0.027&2014-10-27 \\
\pid{14035} &lcsla2bjq&C22&293&813&247.9&439.3&247.0&439.0& 46.0&374.0&   0.022&   0.007&2015-10-02 \\
\pid{13972} &lcri01fzq&C22&302&813&264.0&427.4&256.0&439.0& 46.0&374.0&   0.189&  -0.273&2015-10-06 \\
\pid{13972} &lcri01geq&C22&300&812&255.5&434.1&254.0&438.0& 46.0&374.0&   0.036&  -0.092&2015-10-06 \\
\pid{14452} &ld3la2onq&C23&306&813&259.9&439.9&260.0&439.0& 46.0&374.0&  -0.002&   0.021&2016-10-02 \\
\pid{14440} &ld3701gtq&C23&297&813&246.6&442.7&251.0&439.0& 46.0&374.0&  -0.104&   0.088&2016-10-18 \\
\pid{14440} &ld3701h7q&C23&295&811&249.7&434.0&249.0&437.0& 46.0&374.0&   0.016&  -0.071&2016-10-18 \\
\pid{14857} &ldozbadhq&C24&299&813&237.5&397.9&253.0&439.0& 46.0&374.0&  -0.366&  -0.967&2017-09-04 \\
\pid{14857} &ldozbadvq&C24&298&811&253.2&434.5&252.0&437.0& 46.0&374.0&   0.029&  -0.059&2017-09-04 \\
\pid{14857} &ldozpbfbq&C24&289&813&243.8&440.0&243.0&439.0& 46.0&374.0&   0.020&   0.023&2017-09-10 \\
\midrule
\multicolumn{14}{c}{BOA$\times$MIRRORA}\\
\midrule
\pid{13526} &lcgq01qdq&C21&520&651&472.9&283.2&474.5&282.6& 45.5&368.4&  -0.038&   0.013&2014-11-19 \\
\pid{13972} &lcri01g7q&C22&517&651&471.0&287.2&471.5&282.6& 45.5&368.4&  -0.011&   0.109&2015-10-06 \\
\pid{13972} &lcri02h8q&C22&499&654&462.0&272.2&453.5&285.6& 45.5&368.4&   0.201&  -0.315&2015-10-06 \\
\pid{13972} &lcri02hmq&C22&504&651&457.7&284.2&458.5&282.6& 45.5&368.4&  -0.018&   0.038&2015-10-06 \\
\pid{14440} &ld3701h1q&C23&518&651&471.7&287.1&472.5&282.6& 45.5&368.4&  -0.018&   0.105&2016-10-18 \\
\pid{14440} &ld3702mzq&C23&509&654&480.0&299.3&463.5&285.6& 45.5&368.4&   0.390&   0.322&2016-10-19 \\
\pid{14440} &ld3702nhq&C23&516&651&470.0&285.0&470.5&282.6& 45.5&368.4&  -0.012&   0.057&2016-10-19 \\
\pid{14857} &ldozbadpq&C24&524&651&477.9&287.4&478.5&282.6& 45.5&368.4&  -0.013&   0.113&2017-09-04 \\
\pid{14857} &ldozbbleq&C24&518&653&484.8&289.5&472.5&284.6& 45.5&368.4&   0.290&   0.114&2017-09-06 \\
\pid{14857} &ldozbblsq&C24&514&651&467.9&285.1&468.5&282.6& 45.5&368.4&  -0.015&   0.060&2017-09-06 \\
\midrule
\multicolumn{14}{c}{BOA$\times$MIRRORB\tablenotemark{c}}\\
\midrule
\pid{13526} &lcgq02huq&C21&285&811&237.9&440.3&238.5&444.8& 46.5&366.2&  -0.015&  -0.105&2014-11-17 \\
\pid{13972} &lcri02hgq&C22&286&810&240.6&444.2&239.5&443.8& 46.5&366.2&   0.026&   0.010&2015-10-06 \\
\pid{14440} &ld3702n9q&C23&295&811&248.5&446.1&248.5&444.8& 46.5&366.2&   0.001&   0.030&2016-10-19 \\
\pid{14857} &ldozbblmq&C24&293&811&246.7&444.3&246.5&444.8& 46.5&366.2&   0.006&  -0.011&2017-09-06 \\
\bottomrule
\enddata
\tablenotetext{a}{Environmental factors, and non-repeatability of the OSM and aperture mechanisms, results in
lamp offsets of up to 6~p in AD and $ > 50$~p in XD in these exposures. }
\tablenotetext{b}{BOA \tacq{IMAGE}s move the aperture in the XD direction to obtain the WCA lamp image. Occasionally, the aperture mechanism misses the desired location by $\pm 1 $ \textit{APERXPOS} step of $\sim0.05\arcsec$.}
%\tablenotetext{c}{These PSA$\times$MIRA \tacq{IMAGE}s were part of the FGS-to-SI programs and do {\bf NOT} have a proceeding TA. The TA centering adjustments presented here are to be compared to the FGS-to-SI post processing results presented in Table~\ref{tab:fgs2siInit}.}
\tablenotetext{c}{On November 6, 2014, the MIRB \texttt{ACQ/IMAGE} lamp exposure was changed in duration and current. FSW
tables were also updated at this time (\pr{67139}).}
\tablecomments{If the table caption is in the \texttt{Courier} font, this value was taken directly from the indicated \textsc{\_rawacq.fits} header keyword. In DETector coordinates, +AD is -Y$_{DET}$, +XD is -X$_{DET}$, in USER coordinates, +AD is +Y$_{USER}$, +XD is +X$_{USER}$.}
%           FROM   TO
%   PSA_B   3741   3740
%   BOA_B   3663   3662
%           FROM   TO
%   PSA_B   450    460
%   BOA_B   455    465
\end{deluxetable}


%\begin{deluxetable}{rrrrrrr}
%\tablecaption{Basic \tacq{IMAGE} Bootstrapping Results\label{tab:tamonbasicnimverB}}
%\tablecolumns{7}
%\tablhead{
%\colhead{ROOTNAME} & \colhead{\texttt{APERTURE}}& \colhead{\texttt{OPT\_ELEM}} & \colhead{}  & \colhead{} & \colhead{}  & \colhead{}\\
%\colhead{} & \colhead{}\colhead{} & \colhead{AD (Y)}  & \colhead{XD (X)} & \colhead{AD (Y)}  & \colhead{XD (X)}
%}
%\startdata
%\hline
%\multicolumn{7}{c}{C21 (\pid{})}\\
%\hline
%l	&	PSA &	MIRB	&	&	&	&	& \\
%l	&	BOA &	MIRA	&	&	&	&	& \\
%l	&	BOA &	MIRB	&	&	&	&	& \\
%\hline
%\multicolumn{7}{c}{C22 (\pid{})}\\
%\hline
%l	&	PSA &	MIRB	&	&	&	&	& \\
%l	&	BOA &	MIRA	&	&	&	&	& \\
%l	&	BOA &	MIRB	&	&	&	&	& \\
%\hline
%\multicolumn{7}{c}{C23 (\pid{})}\\
%\hline
%l	&	PSA &	MIRB	&	&	&	&	& \\
%l	&	BOA &	MIRA	&	&	&	&	& \\
%l	&	BOA &	MIRB	&	&	&	&	& \\
%\hline
%\multicolumn{7}{c}{C24 (\pid{})}\\
%\hline
%l	&	PSA &	MIRB	&	&	&	&	& \\
%l	&	BOA &	MIRA	&	&	&	&	& \\
%l	&	BOA &	MIRB	&	&	&	&	& \\
%\hline
%\enddata
%\tablecomments{+AD is -Y detector, +XD is -X detector.}
%\end{deluxetable}


% $Id: tamon_output.tex,v 1.6 2018/03/30 20:22:12 penton Exp $

\begin{deluxetable}{rrrrrrrrrrrrrrrrrrr}
\tabcolsep 2pt
\tabletypesize{\tiny}
\tablecolumns{19}
\tablecaption{COS TA Monitor \texttt{ACQ/IMAGE} Data}\label{tab:Imagedata}
\tablehead{
\colhead{\textit{ROOTNAME}}&\colhead{\textit{EXPTYPE}}&\colhead{\textit{OPT\_ELEM}}&\colhead{LAMP}&\colhead{Current}&\colhead{Target ET}&\colhead{Lamp ET}&\colhead{WCA}&\colhead{WCA}&\colhead{SA}&\colhead{SA}&\colhead{WtP}&\colhead{WtP}&\colhead{Lamp}&\colhead{Lamp}&\colhead{WCA}&\colhead{Lamp}&\colhead{Lamp}&\colhead{Target}\\
\colhead{}&\colhead{}&\colhead{ }&\colhead{}&\colhead{}&\colhead{(s)}&\colhead{(s)}&\colhead{AD}&\colhead{XD}&\colhead{AD}&\colhead{XD}&\colhead{AD}&\colhead{XD}&\colhead{counts}&\colhead{cps}&\colhead{bck}&\colhead{CPS}&\colhead{BP}&\colhead{BP}\\
\colhead{(1)}&\colhead{(2)} &
\colhead{(3)}&\colhead{(4)} &
\colhead{(5)}&\colhead{(6)} &
\colhead{(7)}&\colhead{(8)} &
\colhead{(9)}&\colhead{(10)} &
\colhead{(11)} &\colhead{(12)} &
\colhead{(13)}&\colhead{(14)} &
\colhead{(15)}&\colhead{(16)} &
\colhead{(17)}&\colhead{(18)} &
\colhead{(19)}
}

\startdata
lcgq01q7q & EXT/SCI & MIRB & P2 & MED & 16 & 16 & 717.0 & 214.0 & 763.1 & 588.9 & 46.1 & 374.9 & 4890.0 & 305.6 & 167 & 305.6 & 4.4 & 26.7\\
lcgq01q9q & EXT/SCI & MIRA & P2 & MED & 150 & 150 & 479.0 & 370.0 & 550.3 & 739.9 & 71.3 & 369.9 & 1718.0 & \dots & \dots & \dots & \dots & 0.2\\

lcgq02hoq & WAVECAL & MIRA & P2 & LOW & 7 & \dots & 529.0 & 372.0 & 891.6 & 635.6 & 362.6 & 263.6 & 2827.0 & 403.9 & 97 & 403.9 & 9.9 & 0.3\\
lcgq02hqq & EXT/SCI & MIRB & P2 & LOW & 181 & \dots & 713.0 & 211.0 & 784.4 & 582.7 & 71.4 & 371.7 & 2383.0 & \dots & \dots & \dots & \dots & 0.2\\

lcri01g1q & EXT/SCI & MIRB & P2 & MED & 12 & 12 & 722.0 & 210.0 & 767.7 & 584.2 & 45.7 & 374.2 & 3016.0 & 251.3 & 166 & 251.3 & 4.2 & 30.0\\
lcri01g3q & EXT/SCI & MIRA & P2 & MED & 150 & \dots & 474.0 & 370.0 & 552.0 & 735.7 & 78.0 & 365.7 & 1964.0 & \dots & \dots & \dots & \dots & 0.2\\

lcri02hcq & EXT/SCI & MIRB & P2 & LOW & 181 & 181 & 715.0 & 211.0 & 782.3 & 578.6 & 67.3 & 367.6 & 2406.0 & \dots & \dots & \dots & \dots & 0.2\\

ld3701gvq & EXT/SCI & MIRB & P2 & MED & 16 & 16 & 727.0 & 210.0 & 772.8 & 584.3 & 45.8 & 374.3 & 4147.0 & 259.2 & 184 & 259.2 & 4.3 & 19.0\\
ld3701gxq & EXT/SCI & MIRA & P2 & MED & 150 & 150 & 479.0 & 371.0 & 551.2 & 735.8 & 72.2 & 364.8 & 1739.0 & \dots & \dots & \dots & \dots & 0.2\\

ld3702n1q & WAVECAL & MIRA & P2 & LOW & 14 & \dots & 515.0 & 371.0 & 886.6 & 659.4 & 371.6 & 288.4 & 5589.0 & 399.2 & 167 & 399.2 & 7.7 & 0.2\\
ld3702n4q & EXT/SCI & MIRB & P2 & LOW & 183 & 183 & 723.0 & 213.0 & 774.9 & 577.6 & 51.9 & 364.6 & 2081.0 & \dots & \dots & \dots & \dots & 0.2\\

ldozbadjs & EXT/SCI & MIRB & P2 & MED & 16 & 16  & 724.0 & 210.0 & 769.8 & 583.4 & 45.8 & 373.4 & 4005.0 & 250.3 & 138 & 250.3 & 4.4 & 20.2\\
ldozbadlq & EXT/SCI & MIRA & P2 & MED & 150 & 150 & 472.0 & 371.0 & 545.1 & 735.6 & 73.1 & 364.6 & 1462.0 & \dots & \dots & \dots & \dots & 0.2\\

ldozbblgq & WAVECAL & MIRA & P2 & LOW & 14 & \dots & 507.0 & 372.0 & 748.6 & 911.9 & 241.6 & 539.9 & 5721.0 & 408.6 & 155 & 408.6 & 8.4 & 0.1\\
ldozbbliq & EXT/SCI & MIRB & P2 & LOW & 183 & \dots & 713.0 & 213.0 & 776.2 & 578.7 & 63.2 & 365.7 & 2283.0 & \dots & \dots & \dots & \dots & 0.2\\
\enddata
\tablecomments{{\bf Note to reviewer: Some of the numbers in this table are odd, I am researching.}}
\end{deluxetable}




The basic steps in the verification process are:
\begin{enumerate}
\item{Step1: Perform a PSA$\times$MIRA \tacq{IMAGE} {\bf with} a separate WCA lamp image, preferably in TT mode.
	\begin{enumerate}
		\item{If the PSA$\times$MIRA \tacq{IMAGE} was taken as part of an FGS-to-SI alignment program, then use this information to estimate the
		accuracy of the NUV SIAF entry by comparing the slew from the \tacq{IMAGE} to the known offset inferred from evaluation of the FGS-to-SI program data (from Colin Cox).}
		\item{Measure the [AD,XD] median of the lamp image (as done in \texttt{LTAIMCAL}), and the center of the target (in the same image) using both the \texttt{LTAIMAGE}
		9$\times$9 checkbox + flux-weighted centroid algorithm,
		and a 2D Gaussian fitting profile.}
	\end{enumerate}
	}
\item {Step 2}
\item {Step 3}
\item {Step 4}
\item {Step 5}
\end{enumerate}

These results can be combined to show the measured offsets of PSA+MIRB, BOA+MIRA, and BOA+MIRB when compared to the initial PSA+MIRA \tacq{IMAGE} of Visit `A2' of \pid{14035}. These results are shown in Table~\ref{tab:ai}.
Combined offsets from PSA+MIRA are provided in both NUV pixels (p) and in arcseconds (\arcsec).
\clearpage
The results of \pid{13972} and \pid{14035} show that, for \tacq{IMAGE}s :
\footnotesize
\begin{itemize}
\item PSA+MIRA is aligned with PSA+MIRB to [AD, XD] $\le$ [0.022, 0.007]\arcsec\ (14035, Visit `A2')
\item PSA+MIRB is aligned with BOA+MIRA to [AD, XD] $\le$ [0.023, 0.100]\arcsec\ (13972, Visit `01')
\item BOA+MIRA is aligned with BOA+MIRB to [AD, XD] $\le$ [0.022, 0.024]\arcsec\ (13972, Visit `02')
\end{itemize}

Discuss PR\#81834 : COS ACQ/IMAGE WCA2SCI[X,Y] not calculated properly

\begin{deluxetable}{rrrrrr}
\tabcolsep 8 pt
%\tabletypesize{\footnotesize}
\tablecolumns{6}
%\tablewidth{0 pt}
\tablecaption{\tacq{IMAGE} WCA-to-SA offsets from PSA+MIRA\label{tab:ai}}
\tablehead{\colhead{Aperture}&\colhead{MIRROR}&\colhead{AD Offset} & \colhead{XD Offset} & \colhead{AD Offset}& \colhead{XD Offset}\\
\colhead{}&\colhead{}&\colhead{(\arcsec)} & \colhead{(\arcsec)} & \colhead{(p)} & \colhead{(p)}\\
}
\startdata
\hline
\multicolumn{6}{c}{C21}\\
\hline
\hline
\multicolumn{6}{c}{C22}\\
\hline
\hline
\multicolumn{6}{c}{C23}\\
\hline
\hline
\multicolumn{6}{c}{C24}\\
\hline
PSA & B & 0.021 &-0.049 & 0.298 & 0.893\\
BOA & A & 0.010 & 0.060 & 0.425 & 2.550\\
BOA & B & 0.036 & 0.070 & 1.530 & 2.975 \\
\hline
\enddata
\end{deluxetable}

\begin{deluxetable}{lclcccr}
%\tablewidth{0pt}
\tabcolsep 6pt
\tablecolumns{7}
%\tabletypesize{\footnotesize}
\tablecaption{COS TA \tacq{IMAGE} Monitoring Results Summary\label{tab:airesults}}
\tablehead{
\colhead{\tacq{}} & \colhead{COS} & \colhead{Optical} & \colhead{Direction} & \colhead{Measured Offset\tablenotemark{a}} & \colhead{Requirement} & \colhead{Goal}\\
\colhead{Mode} & \colhead{Channel} & \colhead{Configuration} & \colhead{AD or XD} & \colhead{(mas)} & \colhead{(mas)} & \colhead{(mas)}
}

\startdata
\hline
\multicolumn{7}{c}{C21}\\
\hline

\hline
\multicolumn{7}{c}{C22}\\
\hline

\hline
\multicolumn{7}{c}{C23}\\
\hline

\hline
\multicolumn{7}{c}{C24}\\
\hline
IMAGE	&	NUV	&	PSA+MIRA	&	AD	&	20$\pm$14	&	41--105	&	40\\
IMAGE	&	NUV	&	PSA+MIRB	&	AD	&	10$\pm$14	&	41--105	&	40\\
IMAGE	&	NUV	&	BOA+MIRA	&	AD	&	20$\pm$14	&	41--105	&	40\\
IMAGE	&	NUV	&	BOA+MIRB	&	AD	&	15$\pm$14	&	41--105	&	40\\
\hline
IMAGE	&	NUV	&	PSA+MIRA	&	XD	&	75$\pm$14	&	300	&	100\\
IMAGE	&	NUV	&	PSA+MIRB	&	XD	&	20$\pm$14	&	300	&	100\\
IMAGE	&	NUV	&	BOA+MIRA	&	XD	&	95$\pm$14	&	300	&	100\\
IMAGE	&	NUV	&	BOA+MIRB	&	XD	&	12$\pm$14	&	300	&	100\\
\hline
PEAKXD	&	NUV	&	G185M	&	XD	&	 70$\pm$17	&	300	&	100\\
PEAKXD	&	NUV	&	G225M	&	XD	&	 60$\pm$17	&	300	&	100\\
PEAKXD	&	NUV	&	G285M	&	XD	&	 20$\pm$17	&	300	&	100\\
PEAKXD	&	NUV	&	G230L	&	XD	&	 20$\pm$17	&	300	&	100\\
PEAKXD	&	FUVA	&	G130M	&	XD	&	-30$\pm$71	&	300	&	100\\
PEAKXD	&	FUVA	&	G160M	&	XD	&	-20$\pm$71	&	300	&	100\\
PEAKXD	&	FUVA	&	G140L	&	XD	&	-170$\pm$71	&	300	&	100\\
\hline
\enddata
\tablenotetext{a}{The quoted error bars are associated with a 0.5 uncertainty when measuring the integer WCA coordinate,
and 1/3 of an NUV pixel when using the \tacq{IMAGE}~checkbox centering algorithm. Added in quadrature, the approximate
\tacq{IMAGE}~measurement error is $\approx 0.6$ NUV pixels, or 14 (mas).
Each \tacq{PEAKXD}~ WCA-to-SA measurement contains an error estimate of $\sqrt2 * 0.5 $ times the plate scale of the detector in use
(one half pixel or digital-element uncertainty for each measurement of an integer quantity).
For the NUV channel, this is 23.5 (mas)/p or $\sqrt2 * 0.5 * 23.5 = 17$ (mas).
For the FUV channel, this is $\approx \sqrt2 * 0.5 * 100 \approx 71$ (mas).}
\end{deluxetable}
