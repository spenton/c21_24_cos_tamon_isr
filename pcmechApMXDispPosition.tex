% $Id$
%pcmech_ApMXDispPosition
%% $Id$
%pcmech_ApMXDispPosition
%% $Id$
%pcmech_ApMXDispPosition
%% $Id$
%pcmech_ApMXDispPosition
%\input{pcmechApMXDispPosition}

\begin{deluxetable}{ccccc}
\tablecolumns{5}
\tablewidth{5 in}
\tablecaption{Cross-Dispersion (XD) Aperture Positions (\textit{APERXPOS})\label{tab:ApMXDispPosition}}
\tablehead{
\colhead{\textit{LIFE\_ADJ}} &    \multicolumn{2}{c}{NUV} & \multicolumn{2}{c}{FUV} \\
\colhead{(LP)} & \colhead{PSA\tablenotemark{a}$/$WCA\tablenotemark{b}} & \colhead{BOA\tablenotemark{c}$/$FCA\tablenotemark{d}} & \colhead{PSA$/$WCA} & \colhead{BOA$/$FCA} \\
\colhead{(1)}&\colhead{(2)} & \colhead{(3)}&\colhead{(4)} & \colhead{(5)}
}
\startdata
\toprule
LP1 &  126	&	-153 	& 126	&	153\\
LP2 &  53	&	-226 	& \dots	&	\dots\\
LP3 &  181	&	 -98	& \dots	&	\dots\\
LP4 &  234	&	 -45 	& \dots	&	\dots\\
\bottomrule
\enddata
\footnotesize
\tablenotetext{a}{PSA=Primary Science Aperture}
\tablenotetext{b}{WCA=Wavelength Calibration Aperture}
\tablenotetext{c}{BOA=Bright Object Aperture}
\tablenotetext{d}{FCA=Flat-field Calibration Aperture}
\tablecomments{COS XD aperture positions (\textit{APERXPOS}) are stored in the \textsc{pcmech\_ApMXDispPosition} FSW table. Although LP1-8 are defined in that table for both NUV and FUV, only the NUV LP1 and FUV LP1--4 entries listed here have been used for science observations.
Values used for FCA calibration observations are different from those listed here, and are commanded via APT special commanding (e.g., during the semi-annual FUV Gain Map programs, {\bf REFERENCE}).
Along-Dispersion (AD) values (\textit{APERYPOS}) are stored in the \textsc{pcmech\_ApMDispPosition} FSW table. All COS apertures and LPs use \textit{APERYPOS=22}. }
\normalsize
\end{deluxetable}


\begin{deluxetable}{ccccc}
\tablecolumns{5}
\tablewidth{5 in}
\tablecaption{Cross-Dispersion (XD) Aperture Positions (\textit{APERXPOS})\label{tab:ApMXDispPosition}}
\tablehead{
\colhead{\textit{LIFE\_ADJ}} &    \multicolumn{2}{c}{NUV} & \multicolumn{2}{c}{FUV} \\
\colhead{(LP)} & \colhead{PSA\tablenotemark{a}$/$WCA\tablenotemark{b}} & \colhead{BOA\tablenotemark{c}$/$FCA\tablenotemark{d}} & \colhead{PSA$/$WCA} & \colhead{BOA$/$FCA} \\
\colhead{(1)}&\colhead{(2)} & \colhead{(3)}&\colhead{(4)} & \colhead{(5)}
}
\startdata
\toprule
LP1 &  126	&	-153 	& 126	&	153\\
LP2 &  53	&	-226 	& \dots	&	\dots\\
LP3 &  181	&	 -98	& \dots	&	\dots\\
LP4 &  234	&	 -45 	& \dots	&	\dots\\
\bottomrule
\enddata
\footnotesize
\tablenotetext{a}{PSA=Primary Science Aperture}
\tablenotetext{b}{WCA=Wavelength Calibration Aperture}
\tablenotetext{c}{BOA=Bright Object Aperture}
\tablenotetext{d}{FCA=Flat-field Calibration Aperture}
\tablecomments{COS XD aperture positions (\textit{APERXPOS}) are stored in the \textsc{pcmech\_ApMXDispPosition} FSW table. Although LP1-8 are defined in that table for both NUV and FUV, only the NUV LP1 and FUV LP1--4 entries listed here have been used for science observations.
Values used for FCA calibration observations are different from those listed here, and are commanded via APT special commanding (e.g., during the semi-annual FUV Gain Map programs, {\bf REFERENCE}).
Along-Dispersion (AD) values (\textit{APERYPOS}) are stored in the \textsc{pcmech\_ApMDispPosition} FSW table. All COS apertures and LPs use \textit{APERYPOS=22}. }
\normalsize
\end{deluxetable}


\begin{deluxetable}{ccccc}
\tablecolumns{5}
\tablewidth{5 in}
\tablecaption{Cross-Dispersion (XD) Aperture Positions (\textit{APERXPOS})\label{tab:ApMXDispPosition}}
\tablehead{
\colhead{\textit{LIFE\_ADJ}} &    \multicolumn{2}{c}{NUV} & \multicolumn{2}{c}{FUV} \\
\colhead{(LP)} & \colhead{PSA\tablenotemark{a}$/$WCA\tablenotemark{b}} & \colhead{BOA\tablenotemark{c}$/$FCA\tablenotemark{d}} & \colhead{PSA$/$WCA} & \colhead{BOA$/$FCA} \\
\colhead{(1)}&\colhead{(2)} & \colhead{(3)}&\colhead{(4)} & \colhead{(5)}
}
\startdata
\toprule
LP1 &  126	&	-153 	& 126	&	153\\
LP2 &  53	&	-226 	& \dots	&	\dots\\
LP3 &  181	&	 -98	& \dots	&	\dots\\
LP4 &  234	&	 -45 	& \dots	&	\dots\\
\bottomrule
\enddata
\footnotesize
\tablenotetext{a}{PSA=Primary Science Aperture}
\tablenotetext{b}{WCA=Wavelength Calibration Aperture}
\tablenotetext{c}{BOA=Bright Object Aperture}
\tablenotetext{d}{FCA=Flat-field Calibration Aperture}
\tablecomments{COS XD aperture positions (\textit{APERXPOS}) are stored in the \textsc{pcmech\_ApMXDispPosition} FSW table. Although LP1-8 are defined in that table for both NUV and FUV, only the NUV LP1 and FUV LP1--4 entries listed here have been used for science observations.
Values used for FCA calibration observations are different from those listed here, and are commanded via APT special commanding (e.g., during the semi-annual FUV Gain Map programs, {\bf REFERENCE}).
Along-Dispersion (AD) values (\textit{APERYPOS}) are stored in the \textsc{pcmech\_ApMDispPosition} FSW table. All COS apertures and LPs use \textit{APERYPOS=22}. }
\normalsize
\end{deluxetable}


\begin{deluxetable}{ccccc}
\tablecolumns{5}
\tablewidth{5 in}
\tablecaption{Cross-Dispersion (XD) Aperture Positions (\textit{APERXPOS})\label{tab:ApMXDispPosition}}
\tablehead{
\colhead{\textit{LIFE\_ADJ}} &    \multicolumn{2}{c}{NUV} & \multicolumn{2}{c}{FUV} \\
\colhead{(LP)} & \colhead{PSA\tablenotemark{a}$/$WCA\tablenotemark{b}} & \colhead{BOA\tablenotemark{c}$/$FCA\tablenotemark{d}} & \colhead{PSA$/$WCA} & \colhead{BOA$/$FCA} \\
\colhead{(1)}&\colhead{(2)} & \colhead{(3)}&\colhead{(4)} & \colhead{(5)}
}
\startdata
\toprule
LP1 &  126	&	-153 	& 126	&	153\\
LP2 &  53	&	-226 	& \dots	&	\dots\\
LP3 &  181	&	 -98	& \dots	&	\dots\\
LP4 &  234	&	 -45 	& \dots	&	\dots\\
\bottomrule
\enddata
\footnotesize
\tablenotetext{a}{PSA=Primary Science Aperture}
\tablenotetext{b}{WCA=Wavelength Calibration Aperture}
\tablenotetext{c}{BOA=Bright Object Aperture}
\tablenotetext{d}{FCA=Flat-field Calibration Aperture}
\tablecomments{COS XD aperture positions (\textit{APERXPOS}) are stored in the \textsc{pcmech\_ApMXDispPosition} FSW table. Although LP1-8 are defined in that table for both NUV and FUV, only the NUV LP1 and FUV LP1--4 entries listed here have been used for science observations.
Values used for FCA calibration observations are different from those listed here, and are commanded via APT special commanding (e.g., during the semi-annual FUV Gain Map programs, {\bf REFERENCE}).
Along-Dispersion (AD) values (\textit{APERYPOS}) are stored in the \textsc{pcmech\_ApMDispPosition} FSW table. All COS apertures and LPs use \textit{APERYPOS=22}. }
\normalsize
\end{deluxetable}
