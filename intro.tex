% $Id: intro.tex,v 1.8 2018/04/18 04:10:05 penton Exp $
\section{Introduction}\label{sec:Introduction}
Preliminary results of the Hubble Space Telescopes' (HST) Cosmic Origins Spectrograph (COS)
annual target acquisition (TA) monitoring programs reviewed here were previously reported in the following COS ISRs:
\begingroup
\small
\begin{itemize}
\setlength\itemsep{0.1em}
\item{COS ISR 2015-02 (Summary of the COS Cycle 20 Calibration Program)}
\item{COS ISR 2015-06 (Summary of the COS Cycle 21 Calibration Program)}
\item{COS ISR 2016-03 (Summary of the COS Cycle 22 Calibration Program)}
\item{COS ISR 2016-09 (Cycle 22 COS Target Acquisition Monitor Summary)}
\item{COS ISR 2017-18 (Cycle 23 COS Target Acquisition Monitor Summary)}
\item{COS ISR 2018-09 (Cycle 24 COS Target Acquisition Monitor Summary)}
\end{itemize}
\normalsize
\endgroup
\noindent The information in this ISR supersedes any previous preliminary results or conclusions.\\

This ISR provides details on the following HST+COS calibration {\bf P}rograms:
\small
\begin{itemize}
\setlength\itemsep{0.1em}
\item{\pid{13124} (COS Imaging TA and Spectroscopic WCA-PSA/BOA offset verifications, Cycle 20)}
\item{\pid{13526} (COS Imaging TA and Spectroscopic WCA-PSA/BOA offset verifications, Cycle 21)}
\item{\pid{13972} (COS Imaging TA and Spectroscopic WCA-PSA/BOA offset verifications, Cycle 22)}
\item{\pid{14440} (COS Imaging TA and Spectroscopic WCA-PSA/BOA offset verifications, Cycle 23)}
\item{\pid{14857} (COS Imaging TA and Spectroscopic WCA-PSA/BOA offset verifications, Cycle 24)}
\end{itemize}
\normalsize
\clearpage
\subsection{Introductory Notes and Conventions}\label{subsec:conventions}
%\vspace{-0.3cm}
There are a few COS conventions to be established before discussing the TA monitoring in detail.
\begin{enumerate}
	\item{COS TAs are performed in raw ``detector'' coordinates, not the ``user'' coordinate system of calibrated
		COS files. Therefore, they are not corrected in any way for detector effects such as thermal distortion, geometric distortion,
		PHA\footnote{Pulse Height Amplitude} filtering, walk correction, etc. To avoid confusion over the different coordinate systems, we will use along-dispersion (AD) and cross-dispersion (XD) whenever possible.
		\dotuline{All references to the coordinates ``X$_{DET}$'' and ``Y$_{DET}$'' are in the detector coordinate system unless otherwise specified.}
		In NUV detector coordinates, +X$_{DET}$ is -XD and +Y$_{DET}$ is -AD. In FUV detector coordinates, +X is -AD and +Y is +XD.
		The transformations between user and detector coordinates are:
		\begin{equation} NUV: X_{USER} = 1023 - Y_{DET} \label{eq:NUVuserX}\end{equation}
		\begin{equation} NUV: Y_{USER} = 1023 - X_{DET} \label{eq:NUVuserY}\end{equation}
		\begin{equation} FUV: X_{USER} = 16383 - X_{DET} \label{eq:FUVuserX}\end{equation}
		\begin{equation} FUV: Y_{USER} = Y_{DET} \label{eq:FUVuserY}\end{equation}
		See the COS Data Handbook (DHB, Rafelski et al. 2015) for further information about COS coordinate systems.
		}
	\item{When referencing NUV pixels, we will abbreviate pixel as p. For the FUV, we use rows (for Y) or columns (for X) to reference the FUV digital elements, as the FUV detector does not have physical pixels.}
	\item{When discussing the various subarrays used during COS TA, boxes will be specified by giving the lowest
		valued corner (C) and full size (S) for both X and Y. A box is fully specified by giving its XC, XS, YC \& YS. In this ISR, these will always be given in detector coordinates
		as these are how they are implemented in the HST ground commanding.}
	\item{To clarify the names and locations of TA parameters, the following convention will be used:
		\begin{itemize}
			\item{COS TA modes and their Astronomers Proposal Tool (APT) optional parameters will be in \texttt{Courier} (e.g., \tacq{IMAGE} and \numpos).}
			\item{Keywords in FITS headers will be in \textit{ITALICIZED ALL CAPITALS} (e.g., \textit{ACQSLEWY}).}
			\item{Flight SoftWare (FSW) parameters and routines will be in \textsc{small capitals}.
			All TA FSW patchable constants begin with ``\textsc{pcta\_}'' (e.g., \textsc{pcta\_CalTargetOffset}).
			In this ISR, this prefix is considered implied after the initial introduction of a \textsc{pcta\_} paramater, and will not always be included.
			FSW patchable constants relating to mechanism positions begin with \textsc{pcmech\_} and will always be included in references.}
			\item{Archived COS files are in FITS (.fits) format. FITS filenames, or portions of a filename, will be in {\sf sans-serif} (e.g., {\sf ld9mg2nrq\_rawtag.fits} or {\sf \_spt.fits}).
			COS filenames are in the form {\sf IPPPSSOOT\_{\it extension}.fits}.
			The HST naming convention breaks down for COS as I=Instrument=``L'', PPP=Program ID, SS=Visit ID, OO=Exposure ID,
			and T=``Q'' for nominally recorded observations. See the COS DHB for a full breakdown of the HST IPPPSSOOT naming conventions.
			COS TA files have the {\it extension} of {\sf rawacq}, and additional
			information useful for TA analysis is contained in the {\sf IPPPSSOOT\_{\it spt}.fits} files known as the support file,
			and in the {\sf IPPPSSOOT\_{\it jit/f}.fits} files known as the jitter files.}
		\end{itemize}
	}
%	\item{There are three centering options during \tacq{SEARCH} and \tacq{PEAKD}. In the Astronomers Proposal Tool (APT), these are
%		referred to as \texttt{CENTER}=\texttt{FLUX-WT}, \texttt{FLUX-WT-FLR}, and \texttt{BRIGHTEST}.
%		These parameters have slightly different names in the IHB, the FITS keywords, and the FSW.
%		In this ISR, we will refer to the centering options as \texttt{CENTER}=\texttt{Flux-Weighted (FW)}, \texttt{Flux-Weighted-Floor (FWF)}, and \texttt{Return-To-Brightest (RTB)}.
%	}
	\item{COS contains numerous FUV and NUV central wavelength settings, which are defined in the FSW by the OSM1 or OSM2 rotation positions
	(see the COS Instrument Handbook (IHB, Fischer et al. 2018) for complete details).
	In this ISR, the term \cenwave{}, which is also the FITS keyword name, will be used to mean any of the pre-defined OSM1 + OSM2 rotation settings that uniquely define a central wavelength setting.}
	\item{COS \cenwaves{} are named for the (predicted) lowest wavelength that lands on the FUVA detector segment for \textit{FP-POS}=3. For convenience, when referring to
	a specific \cenwave{} we will either call out the grating and \cenwave{} in use as GRATING/\cenwave{}  (e.g. G130M/1222), or just use a leading ``C'' to identify a particular \cenwave{} (e.g., C1222) in the same manner as ``G" is used for GRATING (e.g., G130M).
	Note that the FITS header keyword equivalent of GRATING is \textit{OPT\_ELEM}.}
	\item{The COS FUV detector has two independent segments, Segment-A and Segment-B. In this ISR, they will be referred to as FUVA \& FUVB.}
	\item{Following the conventions used in APT and the Phase~II Proposal Instructions (Rose et al., 2017), NUV \tacq{PEAKXD} exposures will specify which \texttt{STRIPE}\footnote{\texttt{STRIPE} is the optional parameter name in APT, therefore the \texttt{Courier} font is used.} is used during TA. In this ISR, we will always use
	the APT default (\texttt{STRIPE=DEF}) for a given \cenwave{}. This default is \texttt{STRIPE=MEDIUM} (or STRIPE=B) for all \cenwaves{}, except G230L/3360 where it is \texttt{STRIPE=SHORT} (STRIPE=A).}
	\item{Unless specified, all spectroscopic exposures were taken at \textit{FP-POS}=3.}
	\item{COS has two Science Apertures (SA), the Primary Science Aperture (PSA) and the Bright Object Aperture (BOA). When referring to either aperture without distinction, the abbreviation SA will be used.}
	\item{When referring to an HST program number, we will use either ``HST PID" or a leading ``P" in a similar fashion an ``C=\cenwave{}'' and ``G=GRATING''.}
	\item{\tacq{IMAGE} can use either of two ``MIRROR'' modes, MIRRORA or MIRRORB. In this ISR, they will sometimes be referred to as MIRA \& MIRB to save space, mainly in tables and captions.
	In the FSW, MIRRORA is referred to as the TA1 mirror, and the MIRRORB configuration is referred to as TA1BRT (TA1 bright).}
	\item{When referring to a particular day, we will use YEAR.DAY. For example, day 60 of 2010 will be referred to as \psiafdate.
	%We will also occasionally use decimal years. In these cases, there will only be a single digit in the fractional part (e.g., 2009.9).
	}
	\item{HST observations are grouped in approximately annual ``cycles''. `C\#\#' will be used as shorthand for ``HST Cycle \#\#'' (e.g., Cycle~19~=~C19).}
	\item{Unit abbreviations:
		\begin{itemize}
		\item{Milli-arcseconds (0.001\arcsec) will be abbreviated as mas.}
		\item{Milli-amperes (0.001~A) will be abbreviated as mA.}
		\item{Counts per second will be abbreviated as cps.}
		\end{itemize}
	}
	\item{COS has two internal PtNe wavelength calibrations lamps that send light through the Wavelength Calibration Aperture (WCA) and onto the detectors.The two PtNe lamps are referred to in this ISR
	as \plampone{} and \plamptwo{}. Each lamp has three current settings, LOW, MEDIUM (MED) or HIGH. On-orbit, both lamps have been operated at LOW and MED, but neither has been operated at HIGH.
	The \plampone{} lamp is used for spectroscopic lamp flashes during science exposures (``TAGFLASH''es), while the \plamptwo{} lamp is used for all TA exposures.
	Both lamps have MED current settings of 10~mA, but the \plampone{} lamp has LOW/HIGH current settings of 6/18~mA. The \plamptwo{} lamp
	has LOW/HIGH current settings of 3/14~mA. COS lamp output generally scales as current$^{2}$ ($P=I^2 R$).}
	\item{STScI uses a problem reporting (PR) system to track HST changes.
		Where applicable, these STScI PR identifying numbers will be included as \pr{}.}
	\item{Goddard Spaceflight Center (GSFC) use a Software Change Request (SCR) system to document HST FSW changes.
	COS requests are identified by their SCRC\#. }
	\item{Each COS \tacq{} mode calls one or more FSW routines which interact with HST+COS to perform the TA.
		The FSW routine names begin with \fsw{}.
		The FSW routines called by each \tacq{} mode are given in Table~\ref{tab:fsw}.
		Details of the \tacq{} modes are given in the IHB.
		}
	\item{We use the FSW NUV imaging plate scales values\footnote{The C26 COS IHB (Fischer et al., 2018) does not differentiate between AD and XD NUV imaging plate scales, and lists 0.0235 "/p for both AD and XD.} of 0.02352\arcsec/p (AD) and 0.02362\arcsec/p (XD) as these
	are in agreement with the SMOV results of Goudfrooij et al., 2010. We also assume
	the NUV detector orientation as described in Goudfrooij et al., 2010 (a 0.52$\degree$ rotation between the
	NUV detector coordinates and the APT \texttt{POS\_TARG} system). }
\end{enumerate}

\begin{deluxetable}{lll}
	\tablecolumns{3}
	\tablecaption{\tacq{} to \fsw{} Conversion Table\label{tab:fsw}}
	\tablehead{
	\colhead{\tacq{} Mode}	&	\colhead{FSW}	& \colhead{Comments}\\
%	\colhead{\tacq{} Mode} & \colhead{FSW routine (\textsc{LTA})} & \colhead{Comments} \\
	}
	\startdata
	\toprule
	\tacq{IMAGE} & \fsw{IMCAL}  & Calibrate Image Aperture Location for \fsw{IMAGE} \\
				 & \fsw{IMAGE}  & NUV Image Acquisition \\
	\midrule
	\tacq{SEARCH} & \fsw{SEARCH} & Spiral Target Search\\
	\midrule
	\tacq{PEAKD}  & \fsw{PKD} & Peakup in the Dispersion Direction (AD)\\
	\midrule
	\tacq{PEAKXD} (\numposone) & \fsw{CAL}  & Calibrate Aperture Location for \fsw{PKXD} \\
							   & \fsw{PKXD} & Peakup in the Cross-Dispersion Direction  (XD) \\
	\tacq{PEAKXD} (\numposgtone)    & \fsw{PKD}  & Peakup in the Dispersion Direction, but \\
								 &  			 & modified in LV54 for XD \\
	\bottomrule
	\enddata
	\vspace{-0.5cm}
	\tablecomments{LV54 is the COS FSW Version 4.16 update. GSFC \textsc{SCRC\#352} adapted \textsc{LTAPKD} for XD use in LV54 and
	was installed on HST on 2014.133.}
\end{deluxetable}

% $Id: History.tex,v 1.4 2018/03/30 20:22:12 penton Exp $
\subsection{COS TA Monitoring Program History}\label{subsec:History}
After the installation of COS into HST in 2009 (STS-125), and the
servicing mission orbital verification (SMOV) phase,
a series of calibration programs in NUV imaging mode carefully determined the two-dimensional offset from the COS WCA to the center of the PSA when observed with MIRA.
These X and Y offsets were loaded in the FSW TA parameters \textsc{XImCalTargetOffset} and \textsc{YImCalTargetOffset}.
A target was then centered using a PSA$\times$MIRA \texttt{ACQ/IMAGE}, and a target image was taken along with a MIRB image
of the WCA image. These images were used to determine the AD (Y) and XD (X) offsets of the image target and WCA centroids.
These values were uploaded in the FSW paramaters. This bootstrapping procedure was repeated with the BOA$\times$MIRA
and BOA$\times$MIRB \texttt{ACQ/IMAGE} modes until all four \texttt{ACQ/IMAGE} modes were co-aligned.

The FGS-to-SI programs perform a PSA$\times$MIRA \texttt{ACQ/IMAGE} on a target that should be $\approx$ centered in the aperture.
After some post-processing analysis of the spacecraft telemetry, the PSA$\times$MIRA \texttt{ACQ/IMAGE} can be used to estimate the accuracy of the NUV PSA aperture position in the SIAF\footnote{Science Instrument Aperture File (Mallo, 2008)}.

%\footnote{On November 6, 2014, the MIRB \texttt{ACQ/IMAGE} wavelength calibration lamp (P2) exposure was changed from a 30 second exposure
%at LOW current (3~mA) to a 12 second exposure at MED current (10~mA). At this point the \textsc{pcta\_XImCalTargetOffset} and \textsc{pcta\_YImCalTargetOffset}
%FSW parameters were also updated to reflect a small change in the WCA-to-SA imaging MIRB offsets.}

\subsection{COS centroid measurements}
	The COS FSW uses either a mean or a median to calculate spectral XD locations and imaging wavelength lamp center centers.
On the NUV channel, medians are always used, while for FUV, a mean is always used. This
behavior is controlled by the following FSW patchable constants\footnote{``Current Value'' indicates the LV61 value. These values have worked
well and there is no reason to consider changing these values at this time.} :\\

\footnotesize
\begin{enumerate}
 \setlength{\itemsep}{1pt}
  \setlength{\parskip}{0pt}
  \setlength{\parsep}{0pt}

\item{\textsc{\bf pcta\_UseMedian4CAL4FUV}}
	\begin{description}
	\item[\underline{\rm Description}:]Flag to indicate whether to use ``median'' or ``mean'' for the calculation of the cross-dispersion coordinate of the wavelength calibration lamp spectrum in the phase \textsc{LTACAL} for the FUV detector.
	\item[\underline{\rm Format}:]    Boolean
	\item[\underline{\rm Units}:]     None
	\item[\underline{\rm Limits/Ranges}:]  TRUE = use median;  FALSE = use mean
	\item[\underline{\rm Scaling}:]   None
	\item[\underline{\rm Current Value }:]   FALSE (use mean)
\end{description}

\item{\textsc{\bf pcta\_UseMedian4CAL4NUV}}
	\begin{description}
	\item[\underline{\rm Description}:]Flag to indicate whether to use 'median' or 'mean' for the calculation of the cross-dispersion coordinate of the cal lamp spectrum in the phase \textsc{LTACAL} for the NUV detector.
	\item[\underline{\rm Format}:]    Boolean
	\item[\underline{\rm Units}:]     None
	\item[\underline{\rm Limits/Ranges}:]  TRUE = use median;  FALSE = use mean
	\item[\underline{\rm Scaling}:]   None
	\item[\underline{\rm Current Value }:]   TRUE (use median)
\end{description}

\item{\textsc{\bf pcta\_UseMedian4PKXD4FUV}}
	\begin{description}
	\item[\underline{\rm Description}:]Flag to indicate whether to use 'median' or 'mean' for the calculation of the cross-dispersion coordinate of the target spectrum in the phase \textsc{LTAPKXD} for the FUV detector.
	\item[\underline{\rm Format}:]    Boolean
	\item[\underline{\rm Units}:]     None
	\item[\underline{\rm Limits/Ranges}:]  TRUE = use median;  FALSE = use mean
	\item[\underline{\rm Scaling}:]   None
	\item[\underline{\rm Current Value }:]   FALSE (use mean)
\end{description}

\item{\textsc{\bf pcta\_UseMedian4PKXD4NUV}}
	\begin{description}
	\item[\underline{\rm Description}:]Flag to indicate whether to use 'median' or 'mean' for the calculation of the cross-dispersion coordinate of the target spectrum in the phase \textsc{LTAPKXD} for the NUV detector.
	\item[\underline{\rm Format}:]    Boolean
	\item[\underline{\rm Units}:]     None
	\item[\underline{\rm Limits/Ranges}:]  TRUE = use median;  FALSE = use mean
	\item[\underline{\rm Scaling}:]   None
	\item[\underline{\rm Current Value }:]   TRUE (use median)
\end{description}
\end{enumerate}
\normalsize

The COS aperture mechanism is only repeatable in the XD direction to $\pm 1$ motor step (0.053\arcsec). In addition, the WCA location
phase of the \tacq{IMAGE} (\textsc{LTAIMCAL}), which uses the median integer pixel location as the lamp location, cannot measure the WCA position to better than $\frac{1}{2}$ pixel in either AD or XD.
On the NUV detector, an imaging pixel is $\sim$ 0.02352\arcsec (AD) and $\sim$ 0.02362\arcsec (XD), so there is an intrinsic radial uncertainty of $\sim$0.017\arcsec after each \textsc{LTAIMCAL}.


During TA, all \tacq{} procedures operate in ACCUM mode (no individual photon events, no pulse-height information, and no calibrations available) and operate using integer values only.
For \tacq{IMAGE}, the WCA lamp image location is determined using a median in each coordinate. Therefore, a $\pm$ 0.5p uncertainty is present during each \textsc{LTAIMCAL} measurement when determining the center of the SA position for the
\textsc{LTAIMAGE} portion of the \tacq{IMAGE}. The target location phase of \tacq{IMAGE} (\textsc{LTAIMAGE}) uses a flux-weighted centroid over a 9$\times$9 checkbox, which is described in detail in the COS IHB (C25, Fox et al., 2017: Section 8.4, ``ACQ/IMAGE Acquisition Mode'') and in \S~4.2
of COS TIR 2010-03 (Penton \& Keyes, 2010). A point source in a PSA$\times$MIRA image produces an approximately Gaussian image with a FWHM of ~2.5p.
The 9$\times$9 checkbox considers the majority of the target ($>70\%$\footnote{The PSA$\times$MIRB, BOA$\times$MIRA, and BOA$\times$MIRB 9$\times$9 checkbox fractions are $\approx$ 51\%, 38\%, and 28\%, respectively.})
light while minimizing background contamination,\footnote{As of April, 2018, the average NUV detector background was $\approx$ 8.2E-4 counts/s/p.} and should find the target center to within $\pm \frac{1}{3}$~p.
Combined \textsc{LTAIMCAL} and \textsc{LTAIMAGE} TA stages have a combined uncertainty of $\sqrt( \frac{1}{2}^2 + \frac{1}{3}^2 )~p = 0.6$~p.
\tacq{IMAGE} relies upon the WCA-to-SA offset, which was measured in a similar way and has the same uncertainty. Therefore, the total uncertainty
of a PSA$\times$MIRA \tacq{IMAGE} is $sqrt(2) 0.6$~p = 0.85~p in each direction($\sim$0.020\arcsec). As the
\tacq{IMAGE} configurations were bootstrapped from the PSA$\times$MIRA configuration, there uncertianties
are given in Table~\ref{tab:unc}.

\begin{table}[htb]
	\caption{\tacq{IMAGE} Measurement Uncertainties\label{tab:unc}}
	\begin{tabular}{lcrc}
	\toprule
	Configuration &	  WCA-to-SA     	& \tacq{IMAGE}  & Total \\
				  &    offset           & Measurement   & \tacq{IMAGE} \\
				  &  Uncertainty      &  Uncertainty   & Uncertainty \\
	\midrule
	PSA$\times$MIRA &	$\sqrt(\frac{1}{2}^2 + \frac{1}{3}^2 )~p = 0.6$~p	& 0.6~p	& $\sqrt(2) * 0.6$p = 0.85p ($\sim$ 0.020\arcsec) \\
	PSA$\times$MIRB &	$\sqrt(2) * 0.6$~p =	0.85p		& 0.6~p & $\sqrt(3) * 0.6$p = 1.04 p ($\sim$ 0.024\arcsec) \\
	BOA$\times$MIRA &	$\sqrt(3) * 0.6$~p =	1.04p		& 0.6~p & $\sqrt(4) * 0.6$p = 1.20 p ($\sim$ 0.028\arcsec) \\
	BOA$\times$MIRB &	$\sqrt(4) * 0.6$~p =	1.20p		& 0.6~p & $\sqrt(5) * 0.6$p = 1.34 p ($\sim$ 0.032\arcsec) \\
	\bottomrule
	\end{tabular}
\end{table}

For NUV \tacq{PEAKXD}, the same $\pm$ 0.5p uncertainty is present in both the spectral and target locations portions of the \textsc{LTAPKXD}. Combined in quadrature, this implies that
an \textsc{LTAPKXD} has an inherent XD centering accuracy of no less than $\sqrt(2)$ 0.5 p = 0.7p = 0.017\arcsec. For FUV \textsc{LTAPKXD}, a mean is used to measure both the WCA lamp spectrum XD location and the target XD location.
For FUV LP1--3, uncorrected geometric and thermal distortions can cause targets with different spectral energy distributions (SEDs) to center differently. This effect has been measured (Penton \& Keyes, 2010) to be as large at $\pm 2$ DE (rows) or
$\sim 0.2$\arcsec.\footnote{At FUV LP4 this effect is even more pronounced and prohibits \textsc{LTAPKXD} (\numposone~\tacq{PEAKXD}) from achieving the centering requirement of $\pm 0.3$\arcsec. For this reason, the \tacq{PEAKD} FSW routine \textsc{LTAPKD} was enabled
for XD usage in FSW version LV58 (installed 2014.132 and initially tested on-orbit on 2014.300 in \pid{13636}).}

\subsection{ISR Organization}\label{subsec:org}
In \S~\ref{sec:TAoperations} we will discuss the concepts involved
in the TA monitoring strategy along with a basic review of COS TA operations and
centering requirements (\S~\ref{subsec:requirements}).

In \S~\ref{sec:programs} we discuss the programs discussed in this ISR.
\S~\ref{sec:structure} discusses the details of the individual
COS TA monitoring programs and, the annual HST FGS-to-SI alignment programs and their
connection to the COS TA monitoring programs are discussed in \S~\ref{subsec:fgs2si}.
In \S~\ref{subsec:elists} we list and discuss the individual program exposures.

In \S~\ref{sec:subarray} we discuss the numerous detektor subarrays used in COS TA, and their verification by the programs in this ISR.
The NUV imaging subarrays are discussed in \S~\ref{subsec:NUVimSUBS},
and the NUV and FUV spectroscopic subarrays are discussed in  \S~\ref{subsec:NUVspSUBS},
and  \S~\ref{subsec:FUVspSUBS}, respectively.

In \S~\ref{sec:NimVER} we discuss the verification co-alignment of the four COS \tacq{IMAGE} configurations.

In \S~\ref{sec:siaf} we discuss the relationship and history of the COS SIAF (Science Instrument Aperture File),
and discuss the verification of the C17--C24 NUV entries (\S~\ref{subsec:siafalign}).

In \S~\ref{sec:slewaccuracy} we use information from the FGS-to-SI programs to estimate the accuracy of COS TA slews.

In \S~\ref{sec:spVER} we discuss the verification of the FSW parameters, lamp operations, and subarrays associated with COS spectroscopic TAs.
% Not sure if we are actually going to be able to do the SIAF VER in this document
%As part of this process, we will verify the active COS NUV SIAF (Science Instrument Aperture File) entries.
%In the FUV sub-section we will discuss and the verification of COS FUV SIAF entries.

Results and Conclusions are presented in \S~\ref{sec:results}.
