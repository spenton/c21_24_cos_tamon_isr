% $Id: pctaPS.tex,v 1.1 2018/04/17 18:38:43 penton Exp $
\begin{table}
\footnotesize
\centering
	\begin{threeparttable}[tbc]
	\caption{\fsw{PKXD} Plate Scales}
	\begin{tabular*}{.8\linewidth}{@{\extracolsep{\fill}}lrrr}
		\toprule
		\textit{OPT\_ELEM} &	LP1	&	LP2	&	LP3	\\
		\midrule
		\multicolumn{4}{c}{FUV\tnote{1}}\\
		\midrule
		G130M &  -11086 &  -11400 &   -9550 \\
		G140L &  -10549 &  -11839 &   -9800 \\
		G160M &   -9881 &  -11178 &   -9040 \\
		\midrule
		\multicolumn{4}{c}{NUV\tnote{2}}\\
		\midrule
			ALL	&	2384	&	\dots	&	\dots \\
		\bottomrule
	\end{tabular*}
	\scriptsize
		\begin{tablenotes}
			\item[] {The FSW patchable constant \textsc{pcta\_MilliArcsecsPerPixelScale} determines the FSW scaling (currently set to 100). FUV plate scales are ``negative'' due to the parity of HST slews relative to the COS coordinate system.\\}
			\item[1] {Divide the FUV numbers by 100 to get the FUV plate scale in mas/row.}
			\item[2] {Divide the NUV numbers by 100 to get the NUV plate scale in mas/p. }
		\end{tablenotes}
	\label{tab:platescales}
	\normalsize
	\end{threeparttable}
\normalsize
\end{table}
%\begin{deluxetable}{lrrr}
%\tablewidth{0pt}
%\tabcolsep 12 pt
%%\tabletypesize{\footnotesize}
%\tablecolumns{4}
%\tablecaption{\tacq{PEAKXD} WCA-to-PSA offsets \label{tab:wcatopsa}}
%\tablehead{
%\colhead{\textit{OPT\_ELEM}}&\colhead{LP1}&\colhead{LP2}&\colhead{LP3}\\
%}
%\startdata
%\toprule
%\multicolumn{4}{c}{FUV\tablenotemark{f}}\\
%\midrule
%G130M	&	 -898	&	-943	&	-892 \\
%G140L	&	 -884	&	-950	&	-857 \\
%G160M	&	 -898	&	-933	&	-901 \\
%\midrule
%\multicolumn{4}{c}{NUV\tablenotemark{n}}\\
%\midrule
%G185M	&	3742	&	\dots	&	\dots \\
%G225M	&	3746	&	\dots	&	\dots \\
%G230L	&	3734	&	\dots	&	\dots \\
%G285M	&	3749	&	\dots	&	\dots \\
%\bottomrule
%\enddata
%\tablenotetext{f}{Divide the FUV numbers by -10 to get the number of XD rows between the PSA and WCA spectra for a target centered in the aperture.}
%\tablenotetext{n}{Divide the NUV numbers by 10 to get the NUV WCA-to-PSA offset. }
%\tablecomments{The FSW patchable constant \textsc{pcta\_CalTargetOffsetScale} determines the FSW scaling (currently set to 10).
%FUV scalings are "negative" due to the parity of HST slews relative to the COS coordinate system.
%%{\bf Note to reviewers: Do you think I should keep the numbers in their FSW values (not scaled), or should I go ahead and scale them ?}}
%\end{deluxetable}
