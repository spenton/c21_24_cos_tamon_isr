% $Id: TAsubarrays.tex,v 1.4 2018/03/30 22:05:03 penton Exp $
\section{Verifying the TA (\tacq{}) Subarrays}\label{sec:subarray}

COS TA subarrays are loaded during the HST ground commanding uniquely for each TA exposure,
and are \tacq{} mode, NUV stripe (\textit{STRIPE}), \cenwave, and FUV segment \textit{SEGMENT} and LP dependent.
Additionally, these subarrays change with time in response to detector (e.g., increasing background or ``hot-spots)
or mechanism (e.g., secular OSM drift) monitoring.
There are two stages to the TA verification,
1) ensuring that the intended subarrays were commanded as intended, and
2) that those subarrays were valid for the entire duration of usage.

Ideally, one would compare that commanded subarrays for all exposures to those reported in the \textsf{\_rawacq.fits}.
However, due to issues with the COS TA subarrays\footnote{This issues should be addressed for C26 with the corrections outlined in \pr{64849}, \pr{64874}, \& \pr{66840}},
in this ISR, the subarrays were inferred from the telemetry reported in the \textsf{\_spt.fits} files.

Tables~\ref{tab:TAnuvIMAGEsmov}, \ref{tab:TAnuvIMAGEpost}, \ref{tab:TAnuvSEARCH}, \ref{tab:TAnuvIMAGEmirB} give the TA subarrays for all imaging modes
as it has evolved since SMOV.
Table~\ref{tab:NUVspSUBSsad2} gives the TA subarrays for all NUV spectroscopic \tacq{SEARCH} and \tacq{PEAKKD} that have executed use since SMOV, and
Table~\ref{tab:NUVspSUBSxd} gives the TA subarrays for all NUV \tacq{PEAKXD}s since SMOV.
Table~\ref{tab:TAsubWCAfuv} gives the TA subarrays for the WCA portion of all FUV \tacq{PEAKXD}s separated by LP and \cenwave.
Table~\ref{tab:TAsubWCAfuv} gives the TA subarrays for the WCA portion of all FUV \tacq{PEAKXD}s separated by LP and \cenwave.

Table~\ref{tab:FUVsubA} gives the TA subarrays for the PSA/BOA portion of all FUVA \tacq{}s separated by LP and \cenwave,
and Table~\ref{tab:FUVsubB} gives the TA subarrays for all LPs and \cenwaves for FUVB.
Note that TA has not been enabled for all FUV \cenwaves, so only the TA subarrays that are in use are listed.
The FUV table includes subarrays for all four COS LPs even though only the LP2 and LP3 subarrays were monitored in this ISR.

All values indicate that the intended subarrays are being used for all TA and science exposures. All FUV spectra were visually
inspected to verify that the TA subarrays were successfully excluding all known detector hot-spots and the
bright Geocoronal emission lines that can negatively affect TAs.  No action is required based upon this
analysis of the TA and science subarrays used in C21.

COS TA subarrays are defined in detector coordinates, and are specified by giving the [X,Y] corners ([XC,YC]) and sizes ([XS,YS]).
Table~\ref{tab:NUVsubs} below gives the NUV spectroscopic TA subarrays used for \tacq{SEARCH} and \tacq{PEAKD}, which have not changed since SMOV.
Table~\ref{tab:NUVsubsXD} below gives the NUV spectroscopic TA subarrays used for \tacq{PEAKXD}, which include subarrays to measure the
calibration lamp XD location (WCA) as well as the target spectral location of \textit{STRIPE=B (MEDIUM)}).
These have not changed there updated in 2010 as \pr{XXX}.

In this section, we describe the various subarrays used in COS TA.
These subarrays are defined by giving the detector coordinate of the lowest valued corner (C) and the full size (S) for both X and Y.
A subarray is fully specified by giving its XC, XS, YC, and YS. Unless noted, coordinates are in detector coordinates as this is the system in which COS TAs are performed.

TA subarrays are necessary to remove unwanted detector background or spectral or detector features not associated with the target, such as detector
``hot-spots'' or Geocoronal emission (see Penton \& Keyes, 2011). All COS \tacq{} modes use subarrays, but they different for each mode, detector or detector segment,
and \cenwave.  The explanation for the sizes and locations of the TA subarrays are beyond the scope of this ISR, but can be found in the TIR COS-2010-03 (Penton \& Keyes, 2011),
the pre-launch estimates (driven by ray-trace predictions; COS-11-0024A (Penton, 2001), COS-11-0014B (Penton, 2002), \& COS-11-0016A (Penton, 2001) and for
FUV LP2--4 in their respective enabling ISRs (Penton 2018 (LP2), Penton 2018 (LP3) and Penton \& White 2018 (LP4).)
The programs discussed in this ISR do not contain any FUV or NUV spectroscopic \tacq{} exposures, therefore, the bulk of the discussion for the TA subarrays for
spectroscopic TAs are contained in the respective enabling ISRs. The spectroscopic exposures discussed in this ISR will, however, be used to verify
the appropriateness of the XD locations of the subarrays in \S~\ref{subsec:NspVER} (NUV) and \S~\ref{subsec:FspVER}.

\subsection{NUV Imaging TA subarrays}\label{subsec:NUVimSUBS}
The original (2009) NUV imaging \tacq{IMAGE} and \tacq{SEARCH} TA subarrays are given in Table~\ref{tab:TAnuvIMAGEsmov}.
This table includes entries for the WCA and PSA and both MIRA and MIRB.
The COS FSW uses the same subarrays for the PSA and BOA as the offset on the detector between the aperture locationss is small ($\Delta$~[AD,XD]$\sim$[11.0,0.4]p).

Due to rising NUV detector background, and supported by an analysis of OSM repeatability, reductions to the PSA/BOA \tacq{SEARCH}  and
WCA \tacq{IMAGE} subarrays sizes were implemented on 2011.017 (STScI \pr{67139})\footnote{On-orbit analysis of the OSM positions showed that the the WCA \textsc{LTAIMCAL} MIRA and MIRB lamp image detector
locatians were fairly repeatable (usually with $\pm$ 50 AD p and $\pm$ 10 XDp). As discussed in Penton \& Keyes (2011), the WCA TA \tacq{IMAGE} subarrays were reduced by $\pm$ 50p in XD and $\pm$ 180p in AD,
and \tacq{SEARCH} subarrays were reduced by 125 AD p and 346 XD p.}.
During \tacq{IMAGE}, the region of the detector used to determine the source location is small, and is given by the square of the TA parameter \textsc{pcta\_CheckboxSize}, which is currently set to 9p (81 total pixels).
There no adjustment was warranted for the PSA/BOA \tacq{IMAGE} subarrays. However, during \tacq{SEARCH}, the counts in the full subarray are used (currently $345 \times 816$=19,376p).
NUV \tacq{SEARCH} TAs are therefore much more vulnerable (by a factor of 3500) to contamination from background events and SAA passages (Penton \& Keyes, 2011).
The updated \tacq{SEARCH} values are given in Table~\ref{tab:TAnuvSEARCH}, and the updated \tacq{IMAGE} subarrays are given in Table~\ref{tab:TAnuvIMAGEpost}.
\begin{center}
\begin{deluxetable}{llrrrr}
\tabcolsep 12 pt
%\tabletypesize{\footnotesize}
\tablecolumns{6}
\tablewidth{4.5 in}
\tablecaption{2009--2011.016 \tacq{IMAGE} and \tacq{SEARCH} Subarrays\label{tab:TAnuvIMAGEsmov}.}
\tablehead{\colhead{Aperture}&\colhead{MIRROR}&\colhead{XC}&\colhead{YC}&\colhead{XS}&\colhead{YS}}
\startdata
WCA & MIRA & 268 & 95 & 200 & 660\\
WCA & MIRB & 103 & 361 & 200 & 660\\
PSA/BOA & MIRA & 572 & 108 & 345 & 816\\
PSA/BOA & MIRB & 410 & 200 & 345 & 816
\enddata
\footnotesize
\tablecomments{Due to increased detector background, these were updated on 2011.017 (\pr{67139}) as described in Penton \& Keyes (2011). }
\normalsize
\end{deluxetable}
\end{center}

\begin{center}
\begin{deluxetable}{llrrrr}
\tabcolsep  12 pt
%\tabletypesize{\footnotesize}
\tablecolumns{6}
\tablewidth{4.5 in}
\tablecaption{2011.017--2014.299 \tacq{IMAGE} Subarrays\label{tab:TAnuvIMAGEpost}.}
\tablehead{\colhead{Aperture}&\colhead{MIRROR}&\colhead{XC}&\colhead{YC}&\colhead{XS}&\colhead{YS}}
\startdata
WCA & MIRA & {\bf 345} & {\bf 324} & {\bf 50} & {\bf 300}\\
WCA & MIRB & {\bf 184} & {\bf 539} & {\bf 50} & {\bf 300}\\
PSA/BOA & MIRA & 572 & 108 & 345 & 816\\
PSA/BOA & MIRB & 410 & 200 & 345 & 816
\enddata
\footnotesize
\tablecomments{{\bf Bold} values in this table were updated on 2011.017 (\pr{67139}) due to increased detector background, as described in Penton \& Keyes (2011). }
\normalsize
\end{deluxetable}
\end{center}
\begin{center}
\begin{deluxetable}{llrrrr}
\tabcolsep 12 pt
%\tabletypesize{\footnotesize}
\tablecolumns{6}
\tablewidth{4.5 in}
\tablecaption{2011.017--Present COS \tacq{SEARCH} TA Subarrays\label{tab:TAnuvSEARCH}.}
\tablehead{\colhead{\textit{APERTURE}}&\colhead{MIRROR}&\colhead{XC}&\colhead{YC}&\colhead{XS}&\colhead{YS}}
\startdata
PSA/BOA & MIRA & 630 & 284 & 220 & 470\\
PSA/BOA & MIRB & 469 & 499 & 220 & 470
\enddata
\footnotesize
\tablecomments{Updated on 2011.017 (\pr{67139}), as described in Penton \& Keyes (2011).}
\normalsize
\end{deluxetable}
\end{center}


\begin{center}
\begin{deluxetable}{llrrrr}
\tabcolsep 12 pt
%\tabletypesize{\footnotesize}
\tablecolumns{6}
\tablewidth{5 in}
\tablecaption{2014.300--Present \tacq{IMAGE} Subarrays\label{tab:TAnuvIMAGEmirB}.}
\tablehead{\colhead{\textit{APERTURE}}&\colhead{MIRROR}&\colhead{XC}&\colhead{YC}&\colhead{XS}&\colhead{YS}}
\startdata
WCA & MIRA & 345 & 324 & 50 & 300\\
WCA & MIRB & {\bf 187} & {\bf 566} & 50 & 300\\
PSA/BOA & MIRA & 572 & 108 & 345 & 816\\
PSA/BOA & MIRB & 410 & 200 & 345 & 816
\enddata
\footnotesize
\tablecomments{Due to errors in determining the WCA AD position due to increased NUV detector background, the {\bf bold} subarrays values were updated on Oct. 6, 2014 (2014.279) with \pr{78749}. Simultaneously, the MIRB \tacq{} lamp exposure time and current settings (for the P2 lamp)
were changed from 17s @ LOW current to 12s @ MED current (\pr{78749}). The FSW WCA-to-SA offsets (\textsc{[X,Y]imCalTargetOffset}) were adjusted accordingly (\pr{79116}) on October 16, 2014 (2014.289),
prior to use by HST users on Nov. 10, 2014 (2014.314). The default exposure time and current for the P1 lamp image ``TAGFLASH''s were changed later with \pr{84463}.}
\normalsize
\end{deluxetable}
\end{center}

\begin{deluxetable}{rrr|rrrr|rrrr|r}
\tablecolumns{12}
\tablewidth{0pt}
\tabcolsep 5 pt
\tabletypesize{\scriptsize}
\tablecaption{FITS Reported TA \tacq{IMAGE} Subarrays\label{tab:IMsubs}}
\tablehead{
\colhead{\textit{PROPOSID}} & \colhead{} & \colhead{\textit{APERTURE}} & \multicolumn{4}{c}{WCA} & \multicolumn{4}{|c|}{SA (PSA or BOA)} & \colhead{\textit{DATE-OBS}}\\
\colhead{(PID)} & \colhead{IPPPSSOOT} & \colhead{$\times$MIRROR} & \multicolumn{1}{|r}{XC} & \colhead{YC} & \colhead{XS} & \colhead{YS} & \colhead{XC} & \colhead{YC} & \colhead{XS} & \colhead{YS} & \colhead{}
}
\startdata
\toprule
\multicolumn{12}{c}{MIRA \tacq{IMAGE} Subarrays}\\
\midrule
13171& lc6ka1i1q & PSA$\times$MIRA & 345 & 324 & 50 & 300 & 572 & 108 & 345 & 816 & 2013-03-02 \\
13171& lc6ka2imq & PSA$\times$MIRA & 345 & 324 & 50 & 300 & 572 & 108 & 345 & 816 & 2013-09-01 \\
13616& lci4a1dcq & PSA$\times$MIRA & 345 & 324 & 50 & 300 & 572 & 108 & 345 & 816 & 2014-04-03 \\
13616& lci4a2e3q & PSA$\times$MIRA & 345 & 324 & 50 & 300 & 572 & 108 & 345 & 816 & 2014-10-27 \\
13526& lcgq01qdq & BOA$\times$MIRA & 345 & 324 & 50 & 300 & 572 & 108 & 345 & 816 & 2014-11-19 \\
13526& lcgq02hmq & BOA$\times$MIRA & 345 & 324 & 50 & 300 & 572 & 108 & 345 & 816 & 2014-11-17 \\
13526& lcgq02i0q & BOA$\times$MIRA & 345 & 324 & 50 & 300 & 572 & 108 & 345 & 816 & 2014-11-17 \\
13526& lcgq03dbq & PSA$\times$MIRA & 345 & 324 & 50 & 300 & 572 & 108 & 345 & 816 & 2014-10-06 \\
13526& lcgq03dtq & PSA$\times$MIRA & 345 & 324 & 50 & 300 & 572 & 108 & 345 & 816 & 2014-10-06 \\
14035& lcsla1i4q & PSA$\times$MIRA & 345 & 324 & 50 & 300 & 572 & 108 & 345 & 816 & 2015-04-14 \\
14035& lcsla2bhq & PSA$\times$MIRA & 345 & 324 & 50 & 300 & 572 & 108 & 345 & 816 & 2015-10-02 \\
14452& lcsla1i4q & PSA$\times$MIRA & 345 & 324 & 50 & 300 & 572 & 108 & 345 & 816 & 2016-04-01 \\
14452& lcsla2bhq & PSA$\times$MIRA & 345 & 324 & 50 & 300 & 572 & 108 & 345 & 816 & 2016-10-02 \\
13972& lcri01g7q & BOA$\times$MIRA & 345 & 324 & 50 & 300 & 572 & 108 & 345 & 816 & 2015-10-06 \\
13972& lcri02h8q & BOA$\times$MIRA & 345 & 324 & 50 & 300 & 572 & 108 & 345 & 816 & 2015-10-06 \\
13972& lcri02hmq & BOA$\times$MIRA & 345 & 324 & 50 & 300 & 572 & 108 & 345 & 816 & 2015-10-06 \\
14440& ld3701h1q & BOA$\times$MIRA & 345 & 324 & 50 & 300 & 572 & 108 & 345 & 816 & 2016-10-18 \\
14440& ld3702mzq & BOA$\times$MIRA & 345 & 324 & 50 & 300 & 572 & 108 & 345 & 816 & 2016-10-19 \\
14440& ld3702nhq & BOA$\times$MIRA & 345 & 324 & 50 & 300 & 572 & 108 & 345 & 816 & 2016-10-19 \\
14857& ldozbadpq & BOA$\times$MIRA & 345 & 324 & 50 & 300 & 572 & 108 & 345 & 816 & 2017-09-04 \\
14857& ldozbbleq & BOA$\times$MIRA & 345 & 324 & 50 & 300 & 572 & 108 & 345 & 816 & 2017-09-06 \\
14857& ldozbblsq & BOA$\times$MIRA & 345 & 324 & 50 & 300 & 572 & 108 & 345 & 816 & 2017-09-06 \\
14857& ldozpbf5q & PSA$\times$MIRA & 345 & 324 & 50 & 300 & 572 & 108 & 345 & 816 & 2017-09-10 \\
14857& ldozpbfhq & PSA$\times$MIRA & 345 & 324 & 50 & 300 & 572 & 108 & 345 & 816 & 2017-09-10 \\
\midrule
\multicolumn{12}{c}{MIRB \tacq{IMAGE} Subarrays}\\
\midrule
13171& lc6ka1i3q & PSA$\times$MIRB & 184 & 539 & 50 & 300 & 411 & 200 & 345 & 816 & 2013-03-02 \\
13171& lc6ka1i3q & PSA$\times$MIRB & 184 & 539 & 50 & 300 & 411 & 200 & 345 & 816 & 2013-03-02 \\
13171& lc6ka2ioq & PSA$\times$MIRB & 184 & 539 & 50 & 300 & 411 & 200 & 345 & 816 & 2013-09-01 \\
13171& lc6ka2ioq & PSA$\times$MIRB & 184 & 539 & 50 & 300 & 411 & 200 & 345 & 816 & 2013-09-01 \\
13616& lci4a1deq & PSA$\times$MIRB & 184 & 539 & 50 & 300 & 411 & 200 & 345 & 816 & 2014-04-03 \\
\midrule
\multicolumn{12}{c}{MIRB \tacq{IMAGE} Subarray Size Change (\pr{78749})}\\
\midrule
13616& lci4a2e5q & PSA$\times$MIRB & 187 & 566 & 50 & 300 & 411 & 200 & 345 & 816 & 2014-10-27 \\
13526& lcgq01q5q & PSA$\times$MIRB & 187 & 566 & 50 & 300 & 411 & 200 & 345 & 816 & 2014-11-19 \\
13526& lcgq01qjq & PSA$\times$MIRB & 187 & 566 & 50 & 300 & 411 & 200 & 345 & 816 & 2014-11-19 \\
13526& lcgq02huq & BOA$\times$MIRB & 187 & 566 & 50 & 300 & 411 & 200 & 345 & 816 & 2014-11-17 \\
13526& lcgq03dlq & PSA$\times$MIRB & 187 & 566 & 50 & 300 & 411 & 200 & 345 & 816 & 2014-10-06 \\
13972& lcri01fzq & PSA$\times$MIRB & 187 & 566 & 50 & 300 & 411 & 200 & 345 & 816 & 2015-10-06 \\
13972& lcri01geq & PSA$\times$MIRB & 187 & 566 & 50 & 300 & 411 & 200 & 345 & 816 & 2015-10-06 \\
13972& lcri02hgq & BOA$\times$MIRB & 187 & 566 & 50 & 300 & 411 & 200 & 345 & 816 & 2015-10-06 \\
14035& lcsla1i6q & PSA$\times$MIRB & 187 & 566 & 50 & 300 & 411 & 200 & 345 & 816 & 2015-04-14 \\
14035& lcsla2bjq & PSA$\times$MIRB & 187 & 566 & 50 & 300 & 411 & 200 & 345 & 816 & 2015-10-02 \\
14452& ld3la1csq & PSA$\times$MIRB & 187 & 566 & 50 & 300 & 411 & 200 & 345 & 816 & 2016-04-01 \\
14452& ld3la2onq & PSA$\times$MIRB & 187 & 566 & 50 & 300 & 411 & 200 & 345 & 816 & 2016-10-02 \\
14440& ld3701gtq & PSA$\times$MIRB & 187 & 566 & 50 & 300 & 411 & 200 & 345 & 816 & 2016-10-18 \\
14440& ld3701h7q & PSA$\times$MIRB & 187 & 566 & 50 & 300 & 411 & 200 & 345 & 816 & 2016-10-18 \\
14440& ld3702n9q & BOA$\times$MIRB & 187 & 566 & 50 & 300 & 411 & 200 & 345 & 816 & 2016-10-19 \\
14857& ldozbadhq & PSA$\times$MIRB & 187 & 566 & 50 & 300 & 411 & 200 & 345 & 816 & 2017-09-04 \\
14857& ldozbadvq & PSA$\times$MIRB & 187 & 566 & 50 & 300 & 411 & 200 & 345 & 816 & 2017-09-04 \\
14857& ldozbblmq & BOA$\times$MIRB & 187 & 566 & 50 & 300 & 411 & 200 & 345 & 816 & 2017-09-06 \\
14857& ldozpbfbq & PSA$\times$MIRB & 187 & 566 & 50 & 300 & 411 & 200 & 345 & 816 & 2017-09-10 \\
\bottomrule
\enddata
\tablecomments{As correctly reported in the FITS files, the  MIRB subarrays values were updated from [XC,YC]=[184,539] to [187,566] on Oct. 6, 2014 (2014.279) with \pr{78749}. Simultaneously, the MIRB \tacq{} lamp exposure time and current settings (for the P2 lamp)
TAsubarrays.tex:were changed from 17s @ LOW current to 12s @ MED current (\pr{78749}). The FSW WCA-to-SA offsets (\textsc{[X,Y]imCalTargetOffset}) were adjusted accordingly (\pr{79116}) on October 16, 2014 (2014.289)}
\end{deluxetable}

