\section{Verifying the TA (\tacq{ }) Subarrays}\label{sec:subarray}

COS TA subarrays are loaded during the HST ground commanding uniquely for each TA exposure,
and are NUV stripe, FUV segment, \texttt{ACQ} mode, and CENWAVE dependent.
There are two stages to the TA verification, 1) ensuring that the intended subarrays were commanded and
2) that those subarrays are valid for the entire Cycle of usage.

Ideally, one would compare that commanded subarrays for all exposures to those
reported in the \textsc{\_rawacq.fits}. However, due to issues with the
COS TA subarrays\footnote{This issues should be addressed for Cycle~26 with the corrections outlined in PR\#XXXXX},
the subarrays were inferred from the telemetry reported in the \textsc{\_spt.fits} files.

Table~\ref{tab:NUVimsubs} gives the TA subarrays for all imaging modes,
Table~\ref{tab:NUVspecsubs} gives the TA subarrays for all NUV spectroscopic modes,
and Table~\ref{tab:FUVspecsubs} gives the TA all FUV CENWAVEs. Note that TA has not
been enabled for all CENWAVES, so only the TA subarrays that are in use are listed.
The FUV table includes subarrays for all four COS LPs even though only the LP2 and LP3
subarrys were used in this ISR.

All values indicate that the intended subarrays are being used for all TA and science exposures. All FUV spectra were visually
inspected to verify that the TA subarrays were successfully excluding all known detector hot-spots and the
bright Geocoronal emission lines that can negatively affect TAs.  No action is required based upon this
analysis of the TA and science sub-arrays used in HST Cycle 21.

COS TA subarrays are defined in detector coordinates, and are specified by giving the [X,Y] corners ([XC,YC]) and sizes ([XS,YS]).
Table~\ref{tab:NUVsubs} below gives the NUV spectroscopic TA subarrays used for \tacq{SEARCH} and \tacq{PEAKD}, which have not changed since SMOV.
Table~\ref{tab:NUVsubsXD} below gives the NUV spectroscopic TA subarrays used for \tacq{PEAKXD}, which include subarrays to measure the
calibration lamp XD location (Cal) as well as the target spectral location of the "B" (WAVELENGTH=MEDIUM) stripe.
These have not changed there updated in 2010 as STScI PR\#{}.

In this section, we describe the various subarrays used in COS TA.
These subarrays are defined by giving the detector coordinate of the lowest valued corner (C) and the full size (S) for both X and Y.
A subarray is fully specified by giving its XC, XS, YC, and YS. Unless noted, coordinates are in detector coordinates as this is the system in which COS TAs are performed.

\subsection{NUV Imaging TA subarrays}\label{subsec:NUVimSUBS}
The NUV imaging TA subarrays are given in Table~\ref{tab:TAnuvIMAGEsmov}.
These subarrays are used for both the \tacq{IMAGE} and \tacq{SEARCH} TA phases.
This table includes entries for the WCA and PSA and both
MIRRORA and MIRRORB. The COS FSW uses the same subarrays for the PSA
and BOA as the offset between the apertures is small ($\Delta$~[AD,XD]$\sim$[11.0,0.4]p).
As discussed in \S~\ref{sec:TAback}, the rising NUV detector background necessitates a reduction
in the TA subarray size for WCA+MIRRORB (\texttt{LTAIMCAL}).
The OSM positions, and hence the WCA \texttt{LTAIMCAL} MIRRORA and MIRROB image placements (see Figure~\ref{LTAIMCALpos}), are fairly repeatable and
it is recommended that both WCA TA subarrays be reduced by 50p in XD and 100p in AD as outlined in Table~\ref{tab:TAnuvIMAGEupdate}.

During \tacq{IMAGE}, the region of the detector used to determine the source location is small,
and is given by the square of the TA parameter \textsc{pcta\_CheckboxSize}, which is currently set to 9p (81 total pixels).
However, during \tacq{SEARCH}, the counts in the full subarray are used (currently $345 \times 816$=19,376p).
NUV \tacq{SEARCH} TAs are therefore much more vulnerable (by a factor of 3500)
to contamination from background events and SAA passages as described in \S~\ref{sec:TAback}.

\begin{deluxetable}{|l|l|r|r|r|r|}
\tabcolsep 14pt
\tabletypesize{\footnotesize}
\tablecolumns{6}
\tablewidth{0 pt}
\tablecaption{2009-20XX Imaging NUV \tacq{IMAGE} and \tacq{SEARCH} TA Subarrays\label{tab:TAnuvIMAGEsmov}.}
\tablehead{\colhead{Aperture}&\colhead{MIRROR}&\colhead{XC}&\colhead{YC}&\colhead{XS}&\colhead{YS}}
\startdata
WCA & MIRRORA & CHECK THESE  & 324 & 50 & 300\\
WCA & MIRRORB & 184 & 539 & 50 & 300\\
PSA/BOA & MIRRORA & 630 & 284 & 220 & 470\\
PSA/BOA & MIRRORB & 469 & 499 & 220 & 470
\enddata
\tablecomments{Updated on SMOV-SMS201 with PR\#63095 and PR\#67139.}
\end{deluxetable}

\begin{deluxetable}{|l|l|r|r|r|r|}
\tabcolsep 14pt
\tabletypesize{\footnotesize}
\tablecolumns{6}
\tablewidth{0 pt}
\tablecaption{Cycle~24 Imaging NUV \tacq{IMAGE} and \tacq{SEARCH} TA Subarrays\label{tab:TAimage09}.}
\tablehead{\colhead{Aperture}&\colhead{MIRROR}&\colhead{XC}&\colhead{YC}&\colhead{XS}&\colhead{YS}}
\startdata
WCA & MIRRORA & 345 & 324 & 50 & 300\\
WCA & MIRRORB & 184 & 539 & 50 & 300\\
PSA/BOA & MIRRORA & 630 & 284 & 220 & 470\\
PSA/BOA & MIRRORB & 469 & 499 & 220 & 470
\enddata
\tablecomments{Updated on SMOV-SMS201 with PR\#63095 and PR\#67139.}
\end{deluxetable}

Need to insert new MIRRORB TABLE here.

\begin{deluxetable}{|l|l|r|r|r|r|}
\tabcolsep 14pt
\tabletypesize{\footnotesize}
\tablecolumns{6}
\tablewidth{150 pt}
\tablecaption{2017 Imaging NUV \tacq{IMAGE} and \tacq{SEARCH} TA Subarrays\label{tab:TAnuvIMAGEpost}.}
\tablehead{\colhead{Aperture}&\colhead{MIRROR}&\colhead{XC}&\colhead{YC}&\colhead{XS}&\colhead{YS}}
\startdata
%WCA &MIRRORA&268&95&200&660\\
%WCA &MIRRORB&108&200&200&660\\
WCA &MIRRORA&268&95&200&660\\
WCA &MIRRORB&103&361&200&660\\
PSA/BOA &MIRRORA&572&108&345&816\\
PSA/BOA &MIRRORB&411&200&345&816\\
\enddata
\tablecomments{Updated on SMOV-SMS201 with PR\#63095.}
\end{deluxetable}

\begin{deluxetable}{|l|l|r|r|r|r|}
\tabcolsep 14pt
\tabletypesize{\footnotesize}
\tablecolumns{6}
\tablewidth{200 pt}
\tablecaption{Imaging NUV \tacq{SEARCH} TA Subarrays\label{tab:TAnuvIMAGEupdate}.}
\tablehead{\colhead{Aperture}&\colhead{MIRROR}&\colhead{XC}&\colhead{YC}&\colhead{XS}&\colhead{YS}}
\startdata
%WCA &MIRRORA&\textbf{293}&\textbf{195}&\textbf{150}&\textbf{560}\\
%WCA &MIRRORB&\textbf{133}&\textbf{300}&\textbf{150}&\textbf{560}\\
%PSA &MIRRORA&570&60&345&816\\
%PSA &MIRRORB&410&200&345&816\\
%\hline
%WCA &MIRRORA&\textbf{320} & \textbf{274} & \textbf{100} & \textbf{400}\\
%WCA &MIRRORB&\textbf{159} & \textbf{486} & \textbf{100} & \textbf{400}\\
%PSA/BOA &MIRRORA&\textbf{605} & \textbf{234} & \textbf{270} & \textbf{570}\\
%PSA/BOA &MIRRORB&\textbf{444} & \textbf{446} & \textbf{270} & \textbf{570}\\
WCA &MIRRORA&\textbf{345} & \textbf{324} & \textbf{50} & \textbf{300}\\
WCA &MIRRORB&\textbf{184} & \textbf{539} & \textbf{50} & \textbf{300}\\
PSA/BOA &MIRRORA&\textbf{630} & \textbf{284} & \textbf{220} & \textbf{470}\\
PSA/BOA &MIRRORB&\textbf{469} & \textbf{499} & \textbf{220} & \textbf{470}\\
\enddata
%Updated April 14, 2010
\tablecomments{To be updated in PR\#67139 (2011.017). New values are in \textbf{bold}.
These subarrays are the dashed ones displayed in Figure~\ref{LTAIMCALpos}.
Before this PR, the NUV \tacq{IMAGE} subarrays were identical to the NUV \tacq{SEARCH} subarrays.}
\end{deluxetable}

\section{COS Spectroscopic TA subarrays}\label{sec:taSUBS}
\subsection{COS NUV Spectroscopic TA Subarrays}\label{subsec:NUVspSUBS}
%\vspace{-0.3cm}
The NUV spectroscopic TA subarrays for the \tacq{SEARCH} and \tacq{PEAKD} phases are identical, and are given in Table~\ref{tab:NUVspSUBSsad}.
These subarrays are not grating-specific and are large enough to capture the flux from all three stripes (two for G230L, stripe C is not used).
The COS FSW uses the same subarrays for the PSA and BOA as the offset between the NUV spectra is small ($\Delta$~XD$\sim$5p).

The NUV spectroscopic TA subarrays for the \tacq{PEAKXD} are given in Table~\ref{tab:NUVspSUBSxd}.
These subarrays are large enough to only capture the flux from a single stripe.
Stripe-specific subarrays are defined for both the WCA and PSA.
The COS FSW uses the same subarrays for the PSA
and BOA as the offset between the apertures is small.
If used with an extended source, these subarrays are vulnerable to
cross-contamination of stripe light. In this table, only the values
of XC are listed, for all NUV \tacq{PEAKXD} YC=0, YS=1024, and XS=81.

\begin{deluxetable}{|r|r|r|r|r|}
\tabcolsep 10pt
\tabletypesize{\footnotesize}
\tablecolumns{5}
\tablewidth{0 pt}
\tablecaption{NUV \tacq{SEARCH} and \tacq{PEAKD} Spectroscopic TA subarrays \label{tab:NUVspSUBSsad}}
\tablehead{\colhead{GRating}&\colhead{XC}&\colhead{YC}&\colhead{XS}&\colhead{YS}\\
}
\startdata
\hline
G185M&509&0&420&1024\\
G225M&512&0&420&1024\\
G285M&499&0&420&1024\\
G230L&659&0&275&1024\\
\hline
\enddata
\tablecomments{These NUV TA subarrays were installed in HST commanding in PR\#{}.}
\end{deluxetable}

NEED TO DETERMINE WHICH OF THESE NUV PEAKXD tables is CORRECT !
\begin{deluxetable}{|l|r|r|r|r|r|r|}
\tablewidth{0pt}
\tabcolsep 8pt
\tabletypesize{\footnotesize}
\tablecolumns{7}
\tablecaption{XC Values for NUV \tacq{PEAKXD} TA Subarrays\label{tab:TAnuvPEAKXDxc}.}
\tablehead{\colhead{Grating}&\colhead{WCA-A}& \colhead{WCA-B} & \colhead{WCA-C} & \colhead{PSA-A} &\colhead{PSA-B}& \colhead{PSA-C}
}
\startdata
G185M&418&327&192&794&740&565\\
G225M&440&327&186&804&743&560\\
G285M&417&333&180&782&728&545\\
G230L&443&344&194&807&747&564\\
\enddata
\tablecomments{Updated after SMOV with PR \#63095. For all NUV \tacq{PEAKXD} TA subarrays: YC=0, YS=1024, and XS=81.}
\end{deluxetable}

\begin{deluxetable}{|r|r|r|r|r|r|r|}
\tabcolsep 10pt
\tabletypesize{\footnotesize}
\tablecolumns{7}
\tablewidth{0 pt}
\tablecaption{NUV \tacq{PEAKXD} WCA and PSA/BOA TA Subarrays \label{tab:NUVspSUBSxd}}
\tablehead{\colhead{OPT\_ELEM}&\colhead{CAL-A}&\colhead{CAL-B}&\colhead{CAL-C} &\colhead{SCI-A}&\colhead{SCI-B}&\colhead{SCI-C}
}
\startdata
\hline
G185M	&	418	&	327	&	192	&	794	&	\bf{700}	&	565	\\
G225M	&	430	&	327	&	186	&	804	&	703	&	560	\\
G285M	&	407	&	313	&	180	&	782	&	688	&	555	\\
G230L	&	433	&	334	&	194	&	807	&	707	&	564	\\
\hline
\enddata
\tablecomments{Updated after SMOV with PR \#63095. For all NUV \tacq{PEAKXD} TA subarrays: YC=0, YS=1024, and XS=81.}
\end{deluxetable}

\subsection{COS FUV Spectroscopic TA Subarrays}\label{subsec:FUVsupSUBS}
The FUV spectroscopic TA subarrays for the WCA are the same for \tacq{SEARCH},  \tacq{PEAKD}, and \tacq{PEAKXD}
and are given in Table~\ref{tab:TAsubWCAfuv} for both FUVA and FUVB.
Only one subarray is used for each FUV segment, these are labeled `A1' and `B1'.
As the data are taken in ``detector'' coordinates, all FUV TA subarrays values are valid only for the normal operating temperature range of COS. FUVB is not used in G140L TAs.

The FUV spectroscopic subarrays used for all exposures at LP1, LP2, and LP3 for FUVA are given in Table~\ref{tab:FUVsubA} and for FUVB in Table~\ref{tab:FUVsubB}.
There are two subarrays used for each FUV segment, these are labeled `A1', `A2', `B1', and `B1'.
The COS FSW uses the same subarrays for the PSA and BOA as the offset between the FUV spectra is small ($\Delta$~XD$\sim$3p).
As with the other HST spectrographs, FUV TA is susceptible to contamination from geocoronal light as outlined in
Table~\ref{tab:GEO}, particularly Ly$\alpha$ 1216\AA, {\rm O}\textsc{I} 1302\AA, and {\rm Si}{\sc II}1304\AA.
The FUV TA subarrays outlined in tables~\ref{tab:FUVsubA} and \ref{tab:FUVsubB} have been tailored to remove regions
of the target spectrum that may contain Geocoronal light.
The Geocoronal light fills the aperture and has a very different XD profile which could cause problems with FUV TAs.

%The FUV spectroscopic TA subarrays for LP2 are given for FUVA in Table~\ref{tab:FUVsubA2} and for FUVB in Table~\ref{tab:FUVsubB2}.
%The initial FUV spectroscopic TA subarrays for LP3 are given for FUVA in Table~\ref{tab:FUVsubA3} and for FUVB in Table~\ref{tab:FUVsubB3}.
In 201X, several ``hot-spots'' appeared during solar maximum.
On XX,YY/201X (PR\#XXXXX) the FUV LP3 subarrays were adjusted to avoid these hotspots.
Details are given in \S~\ref{sec:hotspots}, and the adjusted FUVB subarrays are given in Table~\ref{tab:FUVsubB3hs}.

\begin{deluxetable}{|r|rrrr|rrrr|}
\tablewidth{0pt}
\tabcolsep 8pt
\tabletypesize{\footnotesize}
\tablecolumns{9}
\tablecaption{FUV WCA Subarrays\label{tab:TAsubWCAfuv}.}
\tablehead{
\colhead{} &
\colhead{XS} & \colhead{YS} & \colhead{XC} & \colhead{YC} &
\colhead{XS} & \colhead{YS} & \colhead{XC} & \colhead{YC}\\

\colhead{Grating} &
\colhead{A1} & \colhead{A1} & \colhead{A1} & \colhead{A1} &
\colhead{B1} & \colhead{B1} & \colhead{B1} & \colhead{B1}
}
\startdata
\multicolumn{9}{|c|}{LP1}\\ \hline
G130M & 13799 & 44 & 1201 & 541\tablenotemark{a} & 13799 & 44 & 1501 & 585 \\
G160M & 13799 & 44 & 1201 & 535\tablenotemark{a} & 13799 & 44 & 1501 & 579\tablenotemark{a} \\ \hline
G140L & 13799 & 44 & 1201 & 547\tablenotemark{a} & NA  & NA & NA & NA    \\ \hline
\multicolumn{9}{|c|}{LP2}\\ \hline
\multicolumn{9}{|c|}{LP3}\\ \hline
\enddata
\tablenotetext{a}{Updated for SMOV--SMS201 with PR\#63095.}
\end{deluxetable}

\begin{deluxetable}{|l|r||rrrr|rrrr||rrrr|rrrr||}
\tablewidth{0pt}
\tabletypesize{\footnotesize}
\tabcolsep 10pt
\tablecolumns{10}
\tablecaption{PSA/BOA FUVA TA Subarrays\label{tab:FUVsubA}}
\tablehead{
\colhead{Grating} & \colhead{Cenwave} &
\colhead{XS} & \colhead{YS} & \colhead{XC} & \colhead{YC} &
\colhead{XS} & \colhead{YS} & \colhead{XC} & \colhead{YC} \\

\colhead{} & \colhead{(\AA)} &
\colhead{A1} & \colhead{A1} & \colhead{A1} & \colhead{A1} &
\colhead{A2} & \colhead{A2} & \colhead{A2} & \colhead{A2}
}
\startdata
\multicolumn{10}{c}{LP1}\\ \hline
G130M & 1291 & 6555\tablenotemark{b}  & 76 & 1201 & 437\tablenotemark{a} & 4078 & 76 & 8896\tablenotemark{b} & 437\tablenotemark{a} \\
G130M & 1300 & 7559\tablenotemark{b}  & 76 & 1201 & 437\tablenotemark{a} & 4078 & 76 & 9900\tablenotemark{b} & 437\tablenotemark{a} \\
G130M & 1309 & 8562\tablenotemark{b}  & 76 & 1201 & 437\tablenotemark{a} & 4097\tablenotemark{b} & 76 & 10903\tablenotemark{b} & 437\tablenotemark{a}\\
G130M & 1318 & 9465\tablenotemark{b}  & 76 & 1201 & 437\tablenotemark{a} & 3194\tablenotemark{b} & 76 & 11806\tablenotemark{b} & 437\tablenotemark{a} \\
G130M & 1327 & 10489\tablenotemark{b} & 76 & 1201 & 437\tablenotemark{a} & 2170\tablenotemark{b} & 76 & 12830\tablenotemark{b} & 437\tablenotemark{a} \\ \hline
G160M & ALL     & 13799 & 76 & 1201 & 432\tablenotemark{a,b} & \dots & \dots & \dots & \dots \\ \hline
G140L & 1105 & 10458 & 76 & 1201 & 445\tablenotemark{a,b} & 457 & 76 & 14543 & 445\tablenotemark{a,b} \\
G140L & 1230 & 12216 & 76 & 1201 & 445\tablenotemark{a,b} & \dots & \dots & \dots & \dots \\ \hline
\multicolumn{10}{c}{LP2}\\ \hline
G130M & 1291 & 6555\tablenotemark{b}  & 76 & 1201 & 437\tablenotemark{a} & 4078 & 76 & 8896\tablenotemark{b} & 437\tablenotemark{a} \\
G130M & 1300 & 7559\tablenotemark{b}  & 76 & 1201 & 437\tablenotemark{a} & 4078 & 76 & 9900\tablenotemark{b} & 437\tablenotemark{a} \\
G130M & 1309 & 8562\tablenotemark{b}  & 76 & 1201 & 437\tablenotemark{a} & 4097\tablenotemark{b} & 76 & 10903\tablenotemark{b} & 437\tablenotemark{a}\\
G130M & 1318 & 9465\tablenotemark{b}  & 76 & 1201 & 437\tablenotemark{a} & 3194\tablenotemark{b} & 76 & 11806\tablenotemark{b} & 437\tablenotemark{a} \\
G130M & 1327 & 10489\tablenotemark{b} & 76 & 1201 & 437\tablenotemark{a} & 2170\tablenotemark{b} & 76 & 12830\tablenotemark{b} & 437\tablenotemark{a} \\ \hline
G160M & ALL     & 13799 & 76 & 1201 & 432\tablenotemark{a,b} & \dots & \dots & \dots & \dots \\ \hline
G140L & 1105 & 10458 & 76 & 1201 & 445\tablenotemark{a,b} & 457 & 76 & 14543 & 445\tablenotemark{a,b} \\
G140L & 1230 & 12216 & 76 & 1201 & 445\tablenotemark{a,b} & \dots & \dots & \dots & \dots \\ \hline
\multicolumn{10}{|c|}{LP3}\\
\enddata
\tablenotetext{a}{Updated during SMOV (2009.201) with PR\#63095.}
\tablenotetext{b}{Updated for LP2 operations (201X.215) with PR\#63378.}
\tablenotetext{c}{Updated for LP3 operations on 20xx with PR\#63095.}
\end{deluxetable}

\begin{deluxetable}{|l|r||rrrr|rrrr||}
\tablewidth{0pt}
\tabcolsep 10pt
\tablecolumns{10}
\tabletypesize{\footnotesize}
\tablecaption{PSA/BOA FUVB TA Subarrays\label{tab:FUVsubB}.}
\tablehead{
\colhead{Grating} & \colhead{Cenwave} &
\colhead{XS} & \colhead{YS} & \colhead{XC} & \colhead{YC} &
\colhead{XS} & \colhead{YS} & \colhead{XC} & \colhead{YC}\\

\colhead{} & \colhead{(\AA)} &
\colhead{B1} & \colhead{B1} & \colhead{B1} & \colhead{B1} &
\colhead{B2} & \colhead{B2} & \colhead{B2} & \colhead{B2}
}
\startdata
\multicolumn{9}{|c|}{LP1}\\
\hline
G130M & 1291 & 5036\tablenotemark{b} & 76 & 1501 & 483 & 7477\tablenotemark{b} & 76 & 7773\tablenotemark{b} & 483\tablenotemark{a,b} \\
G130M & 1300 & 6039\tablenotemark{b} & 76 & 1501 & 483 & 6474\tablenotemark{b} & 76 & 8776\tablenotemark{b} & 483\tablenotemark{a,b} \\
G130M & 1309 & 7023\tablenotemark{b} & 76 & 1501 & 483 & 5490\tablenotemark{a} & 76 & 9760\tablenotemark{a} & 483\tablenotemark{a,b} \\
G130M & 1318 & 7977\tablenotemark{b} & 76 & 1501 & 483 & 4536\tablenotemark{b} & 76 & 10714\tablenotemark{b} & 483\tablenotemark{a,b} \\
G130M & 1327 & 7629\tablenotemark{b} & 76 & 2792\tablenotemark{b} & 483 & 3593\tablenotemark{b} & 76 & 11657\tablenotemark{b} & 483\tablenotemark{a,b} \\ \hline
G160M & ALL  & 13749 & 76 & 1501 & 477\tablenotemark{a,b} & \dots & \dots & \dots & \dots \\ \hline
G140L & 1105 & \dots & \dots & \dots & \dots & \dots & \dots & \dots & \dots \\
G140L & 1230 & \dots & \dots & \dots & \dots & \dots & \dots & \dots & \dots \\ \hline
\hline
\multicolumn{9}{|c|}{LP2}\\
\hline
G130M & 1291 & 5036\tablenotemark{b} & 76 & 1501 & 483 & 7477\tablenotemark{b} & 76 & 7773\tablenotemark{b} & 483\tablenotemark{a,b} \\
G130M & 1300 & 6039\tablenotemark{b} & 76 & 1501 & 483 & 6474\tablenotemark{b} & 76 & 8776\tablenotemark{b} & 483\tablenotemark{a,b} \\
G130M & 1309 & 7023\tablenotemark{b} & 76 & 1501 & 483 & 5490\tablenotemark{a} & 76 & 9760\tablenotemark{a} & 483\tablenotemark{a,b} \\
G130M & 1318 & 7977\tablenotemark{b} & 76 & 1501 & 483 & 4536\tablenotemark{b} & 76 & 10714\tablenotemark{b} & 483\tablenotemark{a,b} \\
G130M & 1327 & 7629\tablenotemark{b} & 76 & 2792\tablenotemark{b} & 483 & 3593\tablenotemark{b} & 76 & 11657\tablenotemark{b} & 483\tablenotemark{a,b} \\ \hline
G160M & ALL  & 13749 & 76 & 1501 & 477\tablenotemark{a,b} & \dots & \dots & \dots & \dots \\ \hline
G140L & 1105 & \dots & \dots & \dots & \dots & \dots & \dots & \dots & \dots \\
G140L & 1230 & \dots & \dots & \dots & \dots & \dots & \dots & \dots & \dots \\ \hline
\hline
\multicolumn{9}{|c|}{LP3 (Pre Hot-Spot)}\\
\hline
G130M & 1291 & 5036 & 76 & 1501 & 460 & 5036 & 76 & 7773 & 460 \\
G130M & 1300 & 6039 & 76 & 1501 & 460 & 6039 & 76 & 8776 & 460 \\
G130M & 1309 & 7023 & 76 & 1501 & 460 & 7023 & 76 & 9760 & 460 \\
G130M & 1318 & 7977 & 76 & 1501 & 460 & 7977 & 76 & 10714 & 460 \\
G130M & 1327 & 7629 & 76 & 2792 & 460 & 7629 & 76 & 11657 & 460 \\ \hline
G160M & ALL  & 13749 & 76 & 1501 & 453 & \dots & \dots & \dots & \dots \\ \hline
G140L & 1105 & \dots & \dots & \dots & \dots & \dots & \dots & \dots & \dots \\
G140L & 1230 & \dots & \dots & \dots & \dots & \dots & \dots & \dots & \dots \\ \hline
\hline
\multicolumn{9}{|c|}{LP3 (Post Hot-Spot) -- CHECK THESE}\\
\hline
G130M & 1291 & 5036\tablenotemark{d} & 76 & 1501 & 483 & 7477\tablenotemark{d} & 76 & 7773\tablenotemark{d} & 483 \\
G130M & 1300 & 6039\tablenotemark{d} & 76 & 1501 & 483 & 6474\tablenotemark{d} & 76 & 8776\tablenotemark{d} & 483 \\
G130M & 1309 & 7023\tablenotemark{d} & 76 & 1501 & 483 & 5490\tablenotemark{d} & 76 & 9760\tablenotemark{d} & 483 \\
G130M & 1318 & 7977\tablenotemark{d} & 76 & 1501 & 483 & 4536\tablenotemark{d} & 76 & 10714\tablenotemark{d} & 483 \\
G130M & 1327 & 7629\tablenotemark{d} & 76 & 2792\tablenotemark{b} & 483 & 3593\tablenotemark{d} & 76 & 11657\tablenotemark{d} & 483\tablenotemark{a,b} \\ \hline
G160M & ALL  & 13332 & 76 & 1501 & 477\tablenotemark{a,b} & \dots & \dots & \dots & \dots \\ \hline
G140L & 1105 & \dots & \dots & \dots & \dots & \dots & \dots & \dots & \dots \\
G140L & 1230 & \dots & \dots & \dots & \dots & \dots & \dots & \dots & \dots \\ \hline
\enddata
\tablenotetext{a}{Updated during SMOV (2009.201) with PR\#63095.}
\tablenotetext{b}{Updated for LP2 operations (201X.215) with PR\#63378.}
\tablenotetext{c}{Updated for LP3 operations on 20xx.YYY with PR\#63095.}
\tablenotetext{d}{Updated for continuing LP3 operations on 20xx.YYY with PR\#80571.}
\end{deluxetable}

\subsection{Trimming of FUV-B TA subarrays due to FUVB ``Hot-Spot''.}\label{subsec:hotspot}
A ``Hot Spot'' appeared on the COS FUV-B segment coincident with the solar maximum of 20XX.
This spot produced enough counts that it could cause mis-centering during all phases of the FUV LP3 spectroscopic TAs.
This mis-centerings could be in significant in either the AD or XD. On 201X\footnote{STScI PR\#{}.}, all affected
LP3 FUV subarrays were changed.

In raw FUV-B detector coordinates, the approximate location of the `Hot Spot' is at [X,Y]=[14895,482].
As this is near the detector edge, we are able to avoid this hotspot by stopping the last subarray of the FUVB subarrays at X=14833.
For the COS FUV gratings and the FUVB TA subarrays, the impacts were
\begin{enumerate}
	\item[G140L:]{NOT affected as no FUVB TA subarrays are used for G140L }
	\item[G160M:]{One subarray is used, all have XC1=1501, XS1=13749. These will all change to XS1=13332. (no change in Y)}
	\item[G130M:]{Two subarrays are used to avoid Geocoronal Ly$\alpha$. The X-size of the second subarray (XS2) will be trimmed to avoid the hotspot (XC1, XS1, XC2 and all the Y definitions do not change).}
\end{enumerate}
