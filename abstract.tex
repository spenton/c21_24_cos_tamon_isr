\abstract{
This ISR documents the  Cosmic Origins Spectrograph (COS) Target Acquisition (TA) monitoring programs for HST Cycles 21-24. During this period, FUV exposures were executed at Lifetime Positions LP2 and LP3, and all NUV exposures were obtained at the nominal (LP1) position.
These programs were designed to monitor numerous aspects of both imaging and spectroscopic COS TAs, including checking the TA subarrays, monitoring the required flashes of the internal PtNe lamps, and evaluating the accuracy of numerous COS flight software (FSW) patchable constants required for TA.
This project verified that all three COS TA modes (FUV spectroscopic, NUV spectroscopic, and NUV imaging) were, on large, behaving nominally in Cycle 21-24, and determined that no SIAF or FSW parameter updates were required during this time, with the exception of changes to MIRRORB \texttt{ACQ/IMAGE} MIRRORB in 2014.
These changes included a changing of the lamp current from LOW to MEDIUM, an adjustment of the \texttt{LTACAL} exposure time, and a modification of both the MIRRORB WCA and PSA/BOA  \texttt{ACQ/IMAGE} TA subarrays.\footnote{On November 6, 2014, the MIRRORB \texttt{ACQ/IMAGE} wavelength calibration lamp exposure was changed from a 30 second exposure
at LOW current (3mA) to a 12 second exposure at MEDIUM current. At this point the \textsc{pcta\_XImCalTargetOffset} and \textsc{pcta\_YImCalTargetOffset}
FSW parameters were also updated to reflect a small change in the WCA-to-SA imaging MIRRORB offsets. The P14440 program was the first to monitor the updated offsets.}
}
