% $Id: abstract.tex,v 1.5 2018/04/18 04:10:05 penton Exp $
\abstract{
This ISR documents the  Cosmic Origins Spectrograph (COS) Target Acquisition (TA) annual monitoring programs for HST Cycles 20--24.
During this period, NUV exposures were obtained at the nominal (LP1) position, and FUV exposures were executed at Lifetime Positions LP2 and LP3.
These programs were designed to monitor numerous aspects of both imaging and spectroscopic TAs, including checking the TA subarrays, the NUV SIAF entries, the telescope slew distances, and evaluating the accuracy of numerous COS flight software (FSW) patchable constants required for TA.
This project verified that all three COS TA modes (FUV spectroscopic, NUV spectroscopic, and NUV imaging) were, on large, behaving nominally in Cycle 20--24, and determined that no SIAF or FSW parameter updates were required during this time, with the exception of changes to MIRRORB \texttt{ACQ/IMAGE} in 2014.
These changes included a changing of the lamp current from LOW to MEDIUM, an adjustment of the \textsc{LTACAL} exposure time, and a modification of both the MIRRORB WCA and PSA/BOA  \texttt{ACQ/IMAGE} TA subarrays.}
