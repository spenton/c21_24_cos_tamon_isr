\begin{deluxetable}{rcrr}
\tablecaption{Active COS SIAF\tablenotemark{a}~~Entries\label{tab:activesiaf}}
\tabletypesize{\footnotesize}
\tablewidth{0pt}
\tabcolsep 16 pt
\tablecolumns{4}
\tablehead{
\colhead{SIAF} & \colhead{YEAR.DAY} &  \colhead{V2} & \colhead{V3}\\
\colhead{APERNAME} & \colhead{Activated} &  \colhead{(\arcsec)} & \colhead{(\arcsec)}
}
\startdata
\toprule
\multicolumn{4}{c}{NUV LP1}\\
\midrule
LNMAC & 2014.055  & +232.7230 & -237.5150\\
LNBOA & 2014.055  & +232.7230 & -237.5150\\
LNPSA & 2014.055  & +232.7230 & -237.5150\\
LAPTNBOAOF &2014.055  & +223.3488 & -246.8892\\
LAPTNPSAOF &2014.055  & +242.0972 & -228.1408\\
\midrule
\multicolumn{4}{c}{FUV LP1}\\
\midrule
LFMAC      & 2014.055  & +232.7230 & -237.5150\\
LFBOA1     & 2016.151  & +232.7230 & -237.5150\\
LFPSA1     & 2016.151  & +232.7230 & -237.5150\\
LAPTFBOAF1 & 2016.151  & +223.3488 & -246.8892\\
LAPTFPSAF1 & 2016.151  & +242.0972 & -228.1408\\
\midrule
\multicolumn{4}{c}{FUV LP2}\\
\midrule
LFBOA2      & 2016.151  & +235.1580 & -235.0100\\
LFPSA2      & 2016.151  & +235.1580 & -235.0100\\
LAPTFBOAF2  & 2016.151  & +225.7838 & -244.3842\\
LAPTFPSAF2  & 2016.151  & +244.5322 & -225.6358\\
\midrule
\multicolumn{4}{c}{FUV LP3}\\
\midrule
LFBOA3      & 2016.151  & +230.9137 & -239.2749\\
LFPSA3      & 2016.151  & +230.9137 & -239.2749\\
LAPTFBOAF3  & 2016.151  & +221.5395 & -248.6491\\
LAPTFPSAF3  & 2016.151  & +240.2879 & -229.9007\\
\midrule
\multicolumn{4}{c}{FUV LP4}\\
\midrule
LFBOA4      & 2017.031  & +229.1328 & -241.0575\\
LFPSA4      & 2017.031  & +229.1328 & -241.0575\\
LAPTFBOAF4  & 2017.031  & +219.7586 & -250.4317\\
LAPTFPSAF4  & 2017.031  & +238.5070 & -231.6833\\
\bottomrule
\enddata
\tablenotetext{a}{SIAF = Science Instrument Aperture File.}
\tablecomments{COS SIAF ``Aperture Names'' (APERNAME) start with the ``L"", and are
followed by either an ``N'' (NUV), ``F'' (FUV), or
``APT''' indicating that the aperture is only used in APT and not for observations.
``APT'' in the APT entries are immediately followed by an ``N'' or ``F''.
The SA (BOA or PSA) or MAC then follows.
MAC represents the APT/SPSS MACro aperture used for bright object checking.
Finally, FUV entries end in a number giving the LP\#, while the NUV ''offset'' aperturens end with ``OF''.}
\end{deluxetable}
