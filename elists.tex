\subsection{Exposure Lists}\label{subsec:elists}

In Visit 01, we take spectra that meet these requirements with the G130M/1309, G140L/1280, G285M/2676, and G230L/3000, and in Visit 02,
we take spectra with the G160M/1600, G185M/1913, G225M/2306. Table ~\ref{tab:peakxd} the results of these exposures are summarized.
The rightmost column gives the WCA-to-PSA offsets measured in P13972, in arcseconds (\arcsec).
All exposures, except {\sf lcri01h6q}, the G140L/1280 measurement, which showed an offset of 0.15\arcsec\ exceed our $\pm 0.1$\arcsec\ goal.
All exposures exceed our $\pm 0.33$\arcsec\ requirement. The XD profile of G140L spectra is wider that the medium
resolution gratings (G130M and G160M), making in more susceptible to detector `Y-walk' (Penton \& Keyes, 2010).
No action is required at this time as the measured offset is 1/2 of our 0.3\arcsec\ requirement.

The final two exposures of the 02 visit intentionally offset the target by $\pm$ 0.7\arcsec\ to test the effects
of `Y-walk' on G160M \tacq{PEAKXD}s. All three G160M exposures in Visit 02 show offsets from the expected position
of $\le 0.05$\arcsec\ within our 0.1\arcsec\ goal. No action (e.g., updating the \textsc{pcta\_CalTargetOffset} in the FSW)
is required at this time.
\startlongtable
\begin{deluxetable}{rrrrrrrrrrrrr}
\tabcolsep 2pt
\tabletypesize{\tiny}
\tablecolumns{13}
%\tablewidth{0 pt}
\tablecaption{COS/NUV TA Monitoring Imaging Exposures\label{tab:NUVtamonimage}}
\tablehead{
\colhead{ROOTNAME}&\colhead{PROP}&\colhead{TARGNAME}&\colhead{OBS} &\colhead{EXP}   &\colhead{EXP}  &\colhead{PtNe}&\colhead{Lamp}   &\colhead{APERTURE}&\colhead{APERXPOS}&\colhead{APERYPOS}&\colhead{OPT\_}&\colhead{DATE}\\
\colhead{}        &\colhead{ID}  &\colhead{}        &\colhead{MODE}&\colhead{TYPE}&\colhead{TIME(s)}&\colhead{Lamp}&\colhead{Current}&\colhead{}&\colhead{}&\colhead{ }&\colhead{ELEM}&\colhead{OBS}
}
\startdata
\hline
lc6ka1i1q	&	13171	&	427W3	&	ACCUM	&	ACQ/IMAGE	&	60	&	P2	&	Low	&	PSA	&	22.1	&	127.1	&	MIRRORA	&	2013-03-02	\\
lc6ka1i3q	&	13171	&	427W3	&	ACCUM	&	ACQ/IMAGE	&	300	&	P2	&	Low	&	PSA	&	22.1	&	127.1	&	MIRRORB	&	2013-03-02	\\
lc6ka2imq	&	13171	&	206W3	&	ACCUM	&	ACQ/IMAGE	&	60	&	P2	&	Low	&	PSA	&	22.1	&	127.1	&	MIRRORA	&	2013-09-01	\\
lc6ka2ioq	&	13171	&	206W3	&	ACCUM	&	ACQ/IMAGE	&	300	&	P2	&	Low	&	PSA	&	22.1	&	127.1	&	MIRRORB	&	2013-09-01	\\
lcgp01bpq	&	13523	&	WAVE	&	TT	&	WAVECAL	&	40	&	P2	&	Low	&	WCA	&	22.1	&	127.1	&	MIRRORB	&	2013-11-11	\\
lcgp01bsq	&	13523	&	WAVE	&	TT	&	WAVECAL	&	40	&	P1	&	Low	&	WCA	&	22.1	&	127.1	&	MIRRORB	&	2013-11-11	\\
lcgp01byq	&	13523	&	WAVE	&	TT	&	WAVECAL	&	20	&	P2	&	Low	&	WCA	&	22.1	&	127.1	&	MIRRORA	&	2013-11-11	\\
lcgp01c3q	&	13523	&	WAVE	&	TT	&	WAVECAL	&	20	&	P1	&	Low	&	WCA	&	22.1	&	127.1	&	MIRRORA	&	2013-11-11	\\
lci4a1dcq	&	13616	&	427W3	&	ACCUM	&	ACQ/IMAGE	&	60	&	P2	&	Low	&	PSA	&	22.1	&	127.1	&	MIRRORA	&	2014-04-03	\\
lci4a1deq	&	13616	&	427W3	&	ACCUM	&	ACQ/IMAGE	&	300	&	P2	&	Low	&	PSA	&	22.1	&	127.1	&	MIRRORB	&	2014-04-03	\\
lci4a2e3q	&	13616	&	206W3	&	ACCUM	&	ACQ/IMAGE	&	60	&	P2	&	Low	&	PSA	&	22.1	&	127.1	&	MIRRORA	&	2014-10-27	\\
lci4a2e5q	&	13616	&	206W3	&	ACCUM	&	ACQ/IMAGE	&	300	&	P2	&	Med	&	PSA	&	22.1	&	127.1	&	MIRRORB	&	2014-10-27	\\
lcgq01q5q	&	13526	&	WD-1657+343	&	ACCUM	&	ACQ/IMAGE	&	12	&	P2	&	Med	&	PSA	&	22.1	&	127.1	&	MIRRORB	&	2014-11-19	\\
lcgq01q7q	&	13526	&	WD-1657+343	&	TT	&	EXT/SCI	&	16	&	P2	&	Med	&	PSA	&	22.1	&	127.1	&	MIRRORB	&	2014-11-19	\\
lcgq01q9q	&	13526	&	WD-1657+343	&	TT	&	EXT/SCI	&	150	&	P2	&	Med	&	BOA	&	22.1	&	-153.1	&	MIRRORA	&	2014-11-19	\\
lcgq01qbq	&	13526	&	WAVE	&	TT	&	WAVECAL	&	7	&	P2	&	Low	&	WCA	&	22.1	&	126.1	&	MIRRORA	&	2014-11-19	\\
lcgq01qdq	&	13526	&	WD-1657+343	&	ACCUM	&	ACQ/IMAGE	&	150	&	P2	&	Low	&	BOA	&	22.1	&	-153.1	&	MIRRORA	&	2014-11-19	\\
lcgq01qfq	&	13526	&	WAVE	&	TT	&	WAVECAL	&	7	&	P2	&	Low	&	WCA	&	22.1	&	126.1	&	MIRRORA	&	2014-11-19	\\
lcgq01qhq	&	13526	&	WD-1657+343	&	TT	&	EXT/SCI	&	12	&	P2	&	Med	&	PSA	&	22.1	&	126.1	&	MIRRORB	&	2014-11-19	\\
lcgq01qjq	&	13526	&	WD-1657+343	&	ACCUM	&	ACQ/IMAGE	&	12	&	P2	&	Med	&	PSA	&	22.1	&	126.1	&	MIRRORB	&	2014-11-19	\\
lcgq02hmq	&	13526	&	HIP66578	&	ACCUM	&	ACQ/IMAGE	&	12	&	P2	&	Low	&	BOA	&	22.1	&	-153.1	&	MIRRORA	&	2014-11-17	\\
lcgq02hoq	&	13526	&	WAVE	&	TT	&	WAVECAL	&	7	&	P2	&	Low	&	WCA	&	22.1	&	126.1	&	MIRRORA	&	2014-11-17	\\
lcgq02hqq	&	13526	&	HIP66578	&	TT	&	EXT/SCI	&	181	&	P2	&	Low	&	BOA	&	22.1	&	-153.1	&	MIRRORB	&	2014-11-17	\\
lcgq02hsq	&	13526	&	WAVE	&	TT	&	WAVECAL	&	12	&	P2	&	Med	&	WCA	&	22.1	&	126.1	&	MIRRORB	&	2014-11-17	\\
lcgq02huq	&	13526	&	HIP66578	&	ACCUM	&	ACQ/IMAGE	&	181	&	P2	&	Med	&	BOA	&	22.1	&	-153.1	&	MIRRORB	&	2014-11-17	\\
lcgq02hwq	&	13526	&	WAVE	&	TT	&	WAVECAL	&	12	&	P2	&	Med	&	WCA	&	22.1	&	126.1	&	MIRRORB	&	2014-11-17	\\
lcgq02hyq	&	13526	&	WAVE	&	TT	&	WAVECAL	&	10	&	P2	&	Low	&	WCA	&	22.1	&	126.1	&	MIRRORA	&	2014-11-17	\\
lcgq02i0q	&	13526	&	HIP66578	&	ACCUM	&	ACQ/IMAGE	&	12	&	P2	&	Low	&	BOA	&	22.1	&	-153.1	&	MIRRORA	&	2014-11-17	\\
lcgq02icq	&	13526	&	WAVE	&	TT	&	WAVECAL	&	10	&	P1	&	Low	&	WCA	&	22.1	&	127.1	&	MIRRORA	&	2014-11-17	\\
lcgq02ieq	&	13526	&	WAVE	&	TT	&	WAVECAL	&	10	&	P2	&	Low	&	WCA	&	22.1	&	127.1	&	MIRRORA	&	2014-11-17	\\
lcgq02igq	&	13526	&	WAVE	&	TT	&	WAVECAL	&	30	&	P1	&	Low	&	WCA	&	22.1	&	127.1	&	MIRRORB	&	2014-11-17	\\
lcgq02iiq	&	13526	&	WAVE	&	TT	&	WAVECAL	&	20	&	P2	&	Med	&	WCA	&	22.1	&	127.1	&	MIRRORB	&	2014-11-17	\\
lcgq03dbq	&	13526	&	206W3	&	ACCUM	&	ACQ/IMAGE	&	15	&	P2	&	Low	&	PSA	&	22.1	&	127.1	&	MIRRORA	&	2014-10-06	\\
lcgq03ddq	&	13526	&	206W3	&	TT	&	EXT/SCI	&	15	&	P2	&	Low	&	PSA	&	22.1	&	127.1	&	MIRRORA	&	2014-10-06	\\
lcgq03dfq	&	13526	&	206W3	&	TT	&	EXT/SCI	&	160	&	P2	&	Low	&	PSA	&	22.1	&	127.1	&	MIRRORB	&	2014-10-06	\\
lcgq03dhq	&	13526	&	206W3	&	TT	&	EXT/SCI	&	180	&	P2	&	Low	&	PSA	&	22.1	&	127.1	&	MIRRORB	&	2014-10-06	\\
lcgq03djq	&	13526	&	206W3	&	TT	&	EXT/SCI	&	180	&	P2	&	Med	&	PSA	&	22.1	&	127.1	&	MIRRORB	&	2014-10-06	\\
lcgq03dlq	&	13526	&	206W3	&	ACCUM	&	ACQ/IMAGE	&	160	&	P2	&	Med	&	PSA	&	22.1	&	127.1	&	MIRRORB	&	2014-10-06	\\
lcgq03dnq	&	13526	&	206W3	&	TT	&	EXT/SCI	&	180	&	P2	&	Med	&	PSA	&	22.1	&	127.1	&	MIRRORB	&	2014-10-06	\\
lcgq03dpq	&	13526	&	206W3	&	TT	&	EXT/SCI	&	160	&	P2	&	Low	&	PSA	&	22.1	&	127.1	&	MIRRORB	&	2014-10-06	\\
lcgq03drq	&	13526	&	206W3	&	TT	&	EXT/SCI	&	12	&	P2	&	Low	&	PSA	&	22.1	&	127.1	&	MIRRORA	&	2014-10-06	\\
lcgq03dtq	&	13526	&	206W3	&	ACCUM	&	ACQ/IMAGE	&	12	&	P2	&	Low	&	PSA	&	22.1	&	127.1	&	MIRRORA	&	2014-10-06	\\
lcri01fzq	&	13972	&	WD-1657+343	&	ACCUM	&	ACQ/IMAGE	&	12	&	P2	&	Med	&	PSA	&	22.1	&	125.1	&	MIRRORB	&	2015-10-06	\\
lcri01g1q	&	13972	&	WD-1657+343	&	TT	&	EXT/SCI	&	12	&	P2	&	Med	&	PSA	&	22.1	&	125.1	&	MIRRORB	&	2015-10-06	\\
lcri01g3q	&	13972	&	WD-1657+343	&	TT	&	EXT/SCI	&	150	&	P2	&	Med	&	BOA	&	22.1	&	-153.1	&	MIRRORA	&	2015-10-06	\\
lcri01g5q	&	13972	&	WAVE	&	TT	&	WAVECAL	&	10	&	P2	&	Low	&	WCA	&	22.1	&	126.1	&	MIRRORA	&	2015-10-06	\\
lcri01g7q	&	13972	&	WD-1657+343	&	ACCUM	&	ACQ/IMAGE	&	150	&	P2	&	Low	&	BOA	&	22.1	&	-153.1	&	MIRRORA	&	2015-10-06	\\
lcri01g9q	&	13972	&	WAVE	&	TT	&	WAVECAL	&	10	&	P2	&	Low	&	WCA	&	22.1	&	126.1	&	MIRRORA	&	2015-10-06	\\
lcri01gcq	&	13972	&	WD-1657+343	&	TT	&	EXT/SCI	&	14	&	P2	&	Med	&	PSA	&	22.1	&	126.1	&	MIRRORB	&	2015-10-06	\\
lcri01geq	&	13972	&	WD-1657+343	&	ACCUM	&	ACQ/IMAGE	&	12	&	P2	&	Med	&	PSA	&	22.1	&	126.1	&	MIRRORB	&	2015-10-06	\\
lcri02h8q	&	13972	&	HIP66578	&	ACCUM	&	ACQ/IMAGE	&	12	&	P2	&	Low	&	BOA	&	22.1	&	-153.1	&	MIRRORA	&	2015-10-06	\\
lcri02haq	&	13972	&	WAVE	&	TT	&	WAVECAL	&	14	&	P2	&	Low	&	WCA	&	22.1	&	126.1	&	MIRRORA	&	2015-10-06	\\
lcri02hcq	&	13972	&	HIP66578	&	TT	&	EXT/SCI	&	181	&	P2	&	Low	&	BOA	&	22.1	&	-153.1	&	MIRRORB	&	2015-10-06	\\
lcri02heq	&	13972	&	WAVE	&	TT	&	WAVECAL	&	24	&	P2	&	Med	&	WCA	&	22.1	&	126.1	&	MIRRORB	&	2015-10-06	\\
lcri02hgq	&	13972	&	HIP66578	&	ACCUM	&	ACQ/IMAGE	&	181	&	P2	&	Med	&	BOA	&	22.1	&	-153.1	&	MIRRORB	&	2015-10-06	\\
lcri02hiq	&	13972	&	WAVE	&	TT	&	WAVECAL	&	24	&	P2	&	Med	&	WCA	&	22.1	&	126.1	&	MIRRORB	&	2015-10-06	\\
lcri02hkq	&	13972	&	WAVE	&	TT	&	WAVECAL	&	14	&	P2	&	Low	&	WCA	&	22.1	&	126.1	&	MIRRORA	&	2015-10-06	\\
lcri02hmq	&	13972	&	HIP66578	&	ACCUM	&	ACQ/IMAGE	&	12	&	P2	&	Low	&	BOA	&	22.1	&	-153.1	&	MIRRORA	&	2015-10-06	\\
lcri02hyq	&	13972	&	WAVE	&	TT	&	WAVECAL	&	14	&	P1	&	Low	&	WCA	&	22.1	&	125.1	&	MIRRORA	&	2015-10-06	\\
lcri02i0q	&	13972	&	WAVE	&	TT	&	WAVECAL	&	24	&	P2	&	Low	&	WCA	&	22.1	&	125.1	&	MIRRORA	&	2015-10-06	\\
lcri02i2q	&	13972	&	WAVE	&	TT	&	WAVECAL	&	30	&	P1	&	Low	&	WCA	&	22.1	&	125.1	&	MIRRORB	&	2015-10-06	\\
lcri02i4q	&	13972	&	WAVE	&	TT	&	WAVECAL	&	24	&	P2	&	Med	&	WCA	&	22.1	&	125.1	&	MIRRORB	&	2015-10-06	\\
lcsla1i4q	&	14035	&	427W3	&	ACCUM	&	ACQ/IMAGE	&	60	&	P2	&	Low	&	PSA	&	22.1	&	125.1	&	MIRRORA	&	2015-04-14	\\
lcsla1i6q	&	14035	&	427W3	&	ACCUM	&	ACQ/IMAGE	&	300	&	P2	&	Med	&	PSA	&	22.1	&	125.1	&	MIRRORB	&	2015-04-14	\\
lcsla2bhq	&	14035	&	206W3	&	ACCUM	&	ACQ/IMAGE	&	60	&	P2	&	Low	&	PSA	&	22.1	&	125.1	&	MIRRORA	&	2015-10-02	\\
lcsla2bjq	&	14035	&	206W3	&	ACCUM	&	ACQ/IMAGE	&	300	&	P2	&	Med	&	PSA	&	22.1	&	125.1	&	MIRRORB	&	2015-10-02	\\
ld3701gtq	&	14440	&	WD-1657+343	&	ACCUM	&	ACQ/IMAGE	&	13	&	P2	&	Med	&	PSA	&	22.1	&	125.1	&	MIRRORB	&	2016-10-18	\\
ld3701gvq	&	14440	&	WD-1657+343	&	TT	&	EXT/SCI	&	16	&	P2	&	Med	&	PSA	&	22.1	&	125.1	&	MIRRORB	&	2016-10-18	\\
ld3701gxq	&	14440	&	WD-1657+343	&	TT	&	EXT/SCI	&	150	&	P2	&	Med	&	BOA	&	22.1	&	-153.1	&	MIRRORA	&	2016-10-18	\\
ld3701gzq	&	14440	&	WAVE	&	TT	&	WAVECAL	&	9	&	P2	&	Low	&	WCA	&	22.1	&	126.1	&	MIRRORA	&	2016-10-18	\\
ld3701h1q	&	14440	&	WD-1657+343	&	ACCUM	&	ACQ/IMAGE	&	150	&	P2	&	Low	&	BOA	&	22.1	&	-153.1	&	MIRRORA	&	2016-10-18	\\
ld3701h3q	&	14440	&	WAVE	&	TT	&	WAVECAL	&	10	&	P2	&	Low	&	WCA	&	22.1	&	126.1	&	MIRRORA	&	2016-10-18	\\
ld3701h5q	&	14440	&	WD-1657+343	&	TT	&	EXT/SCI	&	16	&	P2	&	Med	&	PSA	&	22.1	&	126.1	&	MIRRORB	&	2016-10-18	\\
ld3701h7q	&	14440	&	WD-1657+343	&	ACCUM	&	ACQ/IMAGE	&	13	&	P2	&	Med	&	PSA	&	22.1	&	126.1	&	MIRRORB	&	2016-10-18	\\
ld3702mzq&	14440	&	HIP66578	&	ACCUM	&	ACQ/IMAGE	&	16	&	P2	&	Low	&	BOA	&	22.1	&	-153.1	&	MIRRORA	&	2016-10-19	\\
ld3702n1q	&	14440	&	WAVE	&	TT	&	WAVECAL	&	14	&	P2	&	Low	&	WCA	&	22.1	&	126.1	&	MIRRORA	&	2016-10-19	\\
ld3702n4q	&	14440	&	HIP66578	&	TT	&	EXT/SCI	&	183	&	P2	&	Low	&	BOA	&	22.1	&	-153.1	&	MIRRORB	&	2016-10-19	\\
ld3702n7q	&	14440	&	WAVE	&	TT	&	WAVECAL	&	24	&	P2	&	Med	&	WCA	&	22.1	&	126.1	&	MIRRORB	&	2016-10-19	\\
ld3702n9q	&	14440	&	HIP66578	&	ACCUM	&	ACQ/IMAGE	&	183	&	P2	&	Med	&	BOA	&	22.1	&	-153.1	&	MIRRORB	&	2016-10-19	\\
ld3702nbq	&	14440	&	WAVE	&	TT	&	WAVECAL	&	24	&	P2	&	Med	&	WCA	&	22.1	&	126.1	&	MIRRORB	&	2016-10-19	\\
ld3702neq	&	14440	&	WAVE	&	TT	&	WAVECAL	&	14	&	P2	&	Low	&	WCA	&	22.1	&	126.1	&	MIRRORA	&	2016-10-19	\\
ld3702nhq	&	14440	&	HIP66578	&	ACCUM	&	ACQ/IMAGE	&	16	&	P2	&	Low	&	BOA	&	22.1	&	-153.1	&	MIRRORA	&	2016-10-19	\\
ld3702o1q	&	14440	&	WAVE	&	TT	&	WAVECAL	&	14	&	P1	&	Low	&	WCA	&	22.1	&	125.1	&	MIRRORA	&	2016-10-19	\\
ld3702o3q	&	14440	&	WAVE	&	TT	&	WAVECAL	&	24	&	P2	&	Low	&	WCA	&	22.1	&	125.1	&	MIRRORA	&	2016-10-19	\\
ld3702o5q	&	14440	&	WAVE	&	TT	&	WAVECAL	&	30	&	P1	&	Low	&	WCA	&	22.1	&	125.1	&	MIRRORB	&	2016-10-19	\\
ld3702o7q	&	14440	&	WAVE	&	TT	&	WAVECAL	&	24	&	P2	&	Med	&	WCA	&	22.1	&	125.1	&	MIRRORB	&	2016-10-19	\\
ldozbadhq	&	14857	&	WD-1657+343	&	ACCUM	&	ACQ/IMAGE	&	13	&	P2	&	Med	&	PSA	&	22.1	&	125.1	&	MIRRORB	&	2017-09-04	\\
ldozbadjs	&	14857	&	WD-1657+343	&	TT	&	EXT/SCI	&	16	&	P2	&	Med	&	PSA	&	22.1	&	125.1	&	MIRRORB	&	2017-09-04	\\
ldozbadlq	&	14857	&	WD-1657+343	&	TT	&	EXT/SCI	&	150	&	P2	&	Med	&	BOA	&	22.1	&	-153.1	&	MIRRORA	&	2017-09-04	\\
ldozbadnq	&	14857	&	WAVE	&	TT	&	WAVECAL	&	9	&	P2	&	Low	&	WCA	&	22.1	&	126.1	&	MIRRORA	&	2017-09-04	\\
ldozbadpq	&	14857	&	WD-1657+343	&	ACCUM	&	ACQ/IMAGE	&	150	&	P2	&	Low	&	BOA	&	22.1	&	-153.1	&	MIRRORA	&	2017-09-04	\\
ldozbadrq	&	14857	&	WAVE	&	TT	&	WAVECAL	&	10	&	P2	&	Low	&	WCA	&	22.1	&	126.1	&	MIRRORA	&	2017-09-04	\\
ldozbadtq	&	14857	&	WD-1657+343	&	TT	&	EXT/SCI	&	16	&	P2	&	Med	&	PSA	&	22.1	&	126.1	&	MIRRORB	&	2017-09-04	\\
ldozbadvq	&	14857	&	WD-1657+343	&	ACCUM	&	ACQ/IMAGE	&	13	&	P2	&	Med	&	PSA	&	22.1	&	126.1	&	MIRRORB	&	2017-09-04	\\
ldozbbleq	&	14857	&	HIP66578	&	ACCUM	&	ACQ/IMAGE	&	16	&	P2	&	Low	&	BOA	&	22.1	&	-153.1	&	MIRRORA	&	2017-09-06	\\
ldozbblgq	&	14857	&	WAVE	&	TT	&	WAVECAL	&	14	&	P2	&	Low	&	WCA	&	22.1	&	126.1	&	MIRRORA	&	2017-09-06	\\
ldozbbliq	&	14857	&	HIP66578	&	TT	&	EXT/SCI	&	183	&	P2	&	Low	&	BOA	&	22.1	&	-153.1	&	MIRRORB	&	2017-09-06	\\
ldozbblkq	&	14857	&	WAVE	&	TT	&	WAVECAL	&	24	&	P2	&	Med	&	WCA	&	22.1	&	126.1	&	MIRRORB	&	2017-09-06	\\
ldozbblmq	&	14857	&	HIP66578	&	ACCUM	&	ACQ/IMAGE	&	183	&	P2	&	Med	&	BOA	&	22.1	&	-153.1	&	MIRRORB	&	2017-09-06	\\
ldozbbloq	&	14857	&	WAVE	&	TT	&	WAVECAL	&	24	&	P2	&	Med	&	WCA	&	22.1	&	126.1	&	MIRRORB	&	2017-09-06	\\
ldozbblqq	&	14857	&	WAVE	&	TT	&	WAVECAL	&	14	&	P2	&	Low	&	WCA	&	22.1	&	126.1	&	MIRRORA	&	2017-09-06	\\
ldozbblsq	&	14857	&	HIP66578	&	ACCUM	&	ACQ/IMAGE	&	16	&	P2	&	Low	&	BOA	&	22.1	&	-153.1	&	MIRRORA	&	2017-09-06	\\
ldozbbm4q&	14857	&	WAVE	&	TT	&	WAVECAL	&	16	&	P1	&	Low	&	WCA	&	22.1	&	125.1	&	MIRRORA	&	2017-09-06	\\
ldozbbm6q&	14857	&	WAVE	&	TT	&	WAVECAL	&	26	&	P2	&	Low	&	WCA	&	22.1	&	125.1	&	MIRRORA	&	2017-09-06	\\
ldozbbm8q&	14857	&	WAVE	&	TT	&	WAVECAL	&	32	&	P1	&	Low	&	WCA	&	22.1	&	125.1	&	MIRRORB	&	2017-09-06	\\
ldozbbmaq&	14857	&	WAVE	&	TT	&	WAVECAL	&	26	&	P2	&	Med	&	WCA	&	22.1	&	125.1	&	MIRRORB	&	2017-09-06	\\
ldozpbf5q	&	14857	&	206W3	&	ACCUM	&	ACQ/IMAGE	&	20	&	P2	&	Low	&	PSA	&	22.1	&	125.1	&	MIRRORA	&	2017-09-10	\\
ldozpbf7q	&	14857	&	206W3	&	TT	&	EXT/SCI	&	20	&	P2	&	Low	&	PSA	&	22.1	&	125.1	&	MIRRORA	&	2017-09-10	\\
ldozpbf9q	&	14857	&	206W3	&	TT	&	EXT/SCI	&	220	&	P2	&	Med	&	PSA	&	22.1	&	125.1	&	MIRRORB	&	2017-09-10	\\
ldozpbfbq	&	14857	&	206W3	&	ACCUM	&	ACQ/IMAGE	&	220	&	P2	&	Med	&	PSA	&	22.1	&	125.1	&	MIRRORB	&	2017-09-10	\\
ldozpbfdq	&	14857	&	206W3	&	TT	&	EXT/SCI	&	220	&	P2	&	Med	&	PSA	&	22.1	&	125.1	&	MIRRORB	&	2017-09-10	\\
ldozpbffq	&	14857	&	206W3	&	TT	&	EXT/SCI	&	20	&	P2	&	Low	&	PSA	&	22.1	&	125.1	&	MIRRORA	&	2017-09-10	\\
ldozpbfhq	&	14857	&	206W3	&	ACCUM	&	ACQ/IMAGE	&	20	&	P2	&	Low	&	PSA	&	22.1	&	125.1	&	MIRRORA	&	2017-09-10	\\
\hline
\enddata
%\end{center}
\tablecomments{Exposures listed as \textsc{EXPTYPE}=EXT/SCI contain coeval target and PtNe lamp (P1 or P2) images taken in time-tag (\textsc{OBSTYPE}=TT) mode.
Exposures listed as \textsc{EXPTYPE}=WAVECAL (target = WAVE) contain only TT PtNe lamp (WCA) images.  \tacq{IMAGE} exposures return before and after target images in \textsc{OBSTYPE}=ACCUM, but do not return  lamp images.}
\end{deluxetable}

% $Id: NUVspectamonfiles.tex,v 1.5 2018/03/30 15:20:58 penton Exp $
\begin{deluxetable}{rrrrrrrrrrr}
\tabcolsep 3pt
\tabletypesize{\scriptsize}
\tablecolumns{11}
%\tablewidth{5.5 in}
\tablecaption{COS/NUV TA Spectroscopic Monitoring Exposures\label{tab:NUVtamonspec}}
\tablehead{
\colhead{\textit{ROOTNAME}}&\colhead{PID}&\colhead{TARGNAME}&
\colhead{\textit{EXPTIME}}&\colhead{LAMP}&\colhead{\textit{OPT\_ELEM}}&\colhead{\cenwave}&
\colhead{LP}&\colhead{\textit{APERXPOS}}&\colhead{\textit{APERYPOS}}&\colhead{\textit{DATE-OBS}}\\
\colhead{}&\colhead{}&\colhead{}&
\colhead{(s)}&\colhead{USED}&\colhead{}&
\colhead{}&\colhead{}&\colhead{}&\colhead{}&\colhead{}
}
\startdata
lcgq01qlq	&	13526	&	WD-1657+343	&	20	&	P2	&	G230L	&	3000	&	1	&	22.1	&	126.1	&	2014-11-19	\\
lcgq01r6q	&	13526	&	WD-1657+343	&	151	&	P2	&	G285M	&	2850	&	1	&	22.1	&	126.1	&	2014-11-19	\\
lcgq02i2q	&	13526	&	HIP66578	&	40	&	P2	&	G185M	&	1890	&	1	&	22.1	&	126.1	&	2014-11-17	\\
lcgq02i4q	&	13526	&	HIP66578	&	52	&	P2	&	G225M	&	2306	&	1	&	22.1	&	126.1	&	2014-11-17	\\
lcri01ggq	&	13972	&	WD-1657+343	&	20	&	P2	&	G230L	&	3000	&	1	&	22.1	&	126.1	&	2015-10-06	\\
lcri01giq	&	13972	&	WD-1657+343	&	151	&	P2	&	G285M	&	2676	&	1	&	22.1	&	126.1	&	2015-10-06	\\
lcri02hoq	&	13972	&	HIP66578	&	52	&	P2	&	G225M	&	2306	&	1	&	22.1	&	126.1	&	2015-10-06	\\
lcri02hqq	&	13972	&	HIP66578	&	40	&	P2	&	G185M	&	1913	&	1	&	22.1	&	126.1	&	2015-10-06	\\
ld3701h9q	&	14440	&	WD-1657+343	&	21	&	P2	&	G230L	&	3000	&	1	&	22.1	&	126.1	&	2016-10-18	\\
ld3701hbq	&	14440	&	WD-1657+343	&	151	&	P2	&	G285M	&	2676	&	1	&	22.1	&	126.1	&	2016-10-18	\\
ld3702nmq	&	14440	&	HIP66578	&	53	&	P2	&	G225M	&	2306	&	1	&	22.1	&	126.1	&	2016-10-19	\\
ld3702noq	&	14440	&	HIP66578	&	40	&	P2	&	G185M	&	1913	&	1	&	22.1	&	126.1	&	2016-10-19	\\
ldozbadxq	&	14857	&	WD-1657+343	&	23	&	P2	&	G230L	&	3000	&	1	&	22.1	&	126.1	&	2017-09-04	\\
ldozbadzq	&	14857	&	WD-1657+343	&	151	&	P2	&	G285M	&	2676	&	1	&	22.1	&	126.1	&	2017-09-04	\\
ldozbbluq	&	14857	&	HIP66578	&	53	&	P2	&	G225M	&	2306	&	1	&	22.1	&	126.1	&	2017-09-06	\\
ldozbblwq	&	14857	&	HIP66578	&	40	&	P2	&	G185M	&	1913	&	1	&	22.1	&	126.1	&	2017-09-06	\\
\hline
\enddata
\tablenotemark{y}{The NUV (LP1) XD location of the aperture (\texttt{APERYPOS}) is 126 in the FSW table \texttt{pcmech\_ApMXDispPosition}.).}
\tablecomments{All exposures were taken with the PSA at \texttt{FP-POS=3}. All exposures executed at the expected aperture position (\texttt{APERXPOS} \& \texttt{APERYPOS}).}
\end{deluxetable}

% $Id: FUVtamonfiles.tex,v 1.6 2018/03/30 22:05:03 penton Exp $
\begin{deluxetable}{rrrrrrrrrrr}
\tabcolsep 3 pt
\tabletypesize{\scriptsize}
\tablecolumns{11}
%\tablewidth{5 in}
\tablecaption{COS/FUV TA Monitoring Exposures\label{tab:FUVtamon}}
\tablehead{
\colhead{\textit{ROOTNAME}}&\colhead{\textit{PROPOSID}}&\colhead{\textit{TARGNAME}}&
\colhead{\textit{EXPTIME}}&\colhead{LAMP}&\colhead{\cenwave}&
\colhead{LP}&\colhead{\textit{APERXPOS}}&\colhead{\textit{APERYPOS}}&\colhead{\textit{OPT\_ELEM}}&\colhead{\textit{DATE-OBS}}\\
\colhead{}&\colhead{}&\colhead{}&\colhead{(s)}&\colhead{USED}&\colhead{}&
\colhead{}&\colhead{}&\colhead{}&\colhead{}&\colhead{}\\
\colhead{(1)}&\colhead{(2)} &
\colhead{(3)}&\colhead{(4)} &
\colhead{(5)}&\colhead{(6)} &
\colhead{(7)}&\colhead{(8)} &
\colhead{(9)}&\colhead{(10)} &
\colhead{(11)}
}
\startdata
lcgq01r8q	&	13526	&	WD-1657+343	&	20	&	P2	&	1309	&	2	&	22.1	&	52.1	&	G130M	&	2014-11-19	\\
lcgq01r8q	&	13526	&	WD-1657+343	&	20	&	P2	&	1309	&	2	&	22.1	&	52.1	&	G130M	&	2014-11-19	\\
lcgq01raq	&	13526	&	WD-1657+343	&	7	&	P2	&	1280	&	2	&	22.1	&	52.1	&	G140L	&	2014-11-19	\\
lcgq01raq	&	13526	&	WD-1657+343	&	7	&	P2	&	1280	&	2	&	22.1	&	52.1	&	G140L	&	2014-11-19	\\
lcgq02i6q	&	13526	&	HIP66578	&	18	&	P2	&	1600	&	2	&	22.1	&	52.1	&	G160M	&	2014-11-17	\\
lcgq02i8q	&	13526	&	HIP66578	&	22	&	P2	&	1600	&	2	&	22.1	&	52.1	&	G160M	&	2014-11-17	\\
lcgq02iaq	&	13526	&	HIP66578	&	22	&	P2	&	1600	&	2	&	22.1	&	52.1	&	G160M	&	2014-11-17	\\
\hline
lcri01gkq	&	13972	&	WD-1657+343	&	20	&	P2	&	1309	&	3	&	22.1	&	182.1	&	G130M	&	2015-10-06	\\
lcri01gkq	&	13972	&	WD-1657+343	&	20	&	P2	&	1309	&	3	&	22.1	&	182.1	&	G130M	&	2015-10-06	\\
lcri01h6q	&	13972	&	WD-1657+343	&	7	&	P2	&	1280	&	3	&	22.1	&	182.1	&	G140L	&	2015-10-06	\\
lcri01h6q	&	13972	&	WD-1657+343	&	7	&	P2	&	1280	&	3	&	22.1	&	182.1	&	G140L	&	2015-10-06	\\
lcri02hsq	&	13972	&	HIP66578	&	22	&	P2	&	1600	&	3	&	22.1	&	182.1	&	G160M	&	2015-10-06	\\
lcri02huq	&	13972	&	HIP66578	&	25	&	P2	&	1600	&	3	&	22.1	&	182.1	&	G160M	&	2015-10-06	\\
lcri02hwq	&	13972	&	HIP66578	&	25	&	P2	&	1600	&	3	&	22.1	&	182.1	&	G160M	&	2015-10-06	\\
ld3701hdq	&	14440	&	WD-1657+343	&	25	&	P2	&	1309	&	3	&	22.1	&	182.1	&	G130M	&	2016-10-18	\\
ld3701hdq	&	14440	&	WD-1657+343	&	25	&	P2	&	1309	&	3	&	22.1	&	182.1	&	G130M	&	2016-10-18	\\
ld3701hfq	&	14440	&	WD-1657+343	&	10	&	P2	&	1280	&	3	&	22.1	&	182.1	&	G140L	&	2016-10-18	\\
ld3701hfq	&	14440	&	WD-1657+343	&	10	&	P2	&	1280	&	3	&	22.1	&	182.1	&	G140L	&	2016-10-18	\\
ld3702nqq	&	14440	&	HIP66578	&	22	&	P2	&	1600	&	3	&	22.1	&	182.1	&	G160M	&	2016-10-19	\\
ld3702nsq	&	14440	&	HIP66578	&	25	&	P2	&	1600	&	3	&	22.1	&	182.1	&	G160M	&	2016-10-19	\\
ld3702nuq	&	14440	&	HIP66578	&	25	&	P2	&	1600	&	3	&	22.1	&	182.1	&	G160M	&	2016-10-19	\\
ldozbae1q	&	14857	&	WD-1657+343	&	25	&	P2	&	1309	&	3	&	22.1	&	182.1	&	G130M	&	2017-09-04	\\
ldozbae1q	&	14857	&	WD-1657+343	&	25	&	P2	&	1309	&	3	&	22.1	&	182.1	&	G130M	&	2017-09-04	\\
ldozbae3q	&	14857	&	WD-1657+343	&	10	&	P2	&	1280	&	3	&	22.1	&	182.1	&	G140L	&	2017-09-04	\\
ldozbae3q	&	14857	&	WD-1657+343	&	10	&	P2	&	1280	&	3	&	22.1	&	182.1	&	G140L	&	2017-09-04	\\
ldozbblyq	&	14857	&	HIP66578	&	22	&	P2	&	1600	&	3	&	22.1	&	182.1	&	G160M	&	2017-09-06	\\
ldozbbm0q	&	14857	&	HIP66578	&	27	&	P2	&	1600	&	3	&	22.1	&	182.1	&	G160M	&	2017-09-06	\\
ldozbbm2q	&	14857	&	HIP66578	&	27	&	P2	&	1600	&	3	&	22.1	&	182.1	&	G160M	&	2017-09-06	\\
\hline
\enddata
\tablecomments{All exposures were taken with the PSA at \textit{FP-POS=3}. All exposures executed at the expected aperture positions (\textit{APERXPOS} \& \textit{APERYPOS}).}
\end{deluxetable}

