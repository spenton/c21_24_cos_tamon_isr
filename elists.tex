% $Id: elists.tex,v 1.5 2018/04/17 18:38:43 penton Exp $
\clearpage
\subsection{Exposure Lists}\label{subsec:elists}

Table~\ref{tab:NUVtamonimagePSA} gives the operational details of all NUV PSA imaging exposures which opened the
external shutter used in this ISR, while Table~\ref{tab:NUVtamonimageBOA} details the NUV BOA imaging exposures.
Table~\ref{tab:NUVtamonimageWCA} gives the operational details of all NUV imaging WAVECAL exposures.
Tables~\ref{tab:NUVtamonspec} and \ref{tab:FUVtamon} give the details of all spectroscopic exposures used in this ISR.
All tables follow the convention that if an entry was extracted from a FITS header, then the column name will appear in \textit{ITALICIZED ALL CAPITALS}.\\

The columns of the Table~\ref{tab:NUVtamonimagePSA} and ~\ref{tab:NUVtamonimageBOA} give:
\footnotesize
\begin{enumerate}
\item \textit{ROOTNAME} gives the IPPPSSOOT of the COS exposure,
\item \textit{PROPOSID} gives the HST program id (PID),
\item \textit{TARGNAME} gives the target name as present in the MAST archive,
\item \textit{OBSMODE} gives the observation mode, where ``TT'' is used for Time-Tag observations,
\item \textit{EXPTYPE} gives the exposure type, which is either \tacq{IMAGE} or EXT/SCI. EXT/SCI images using \textit{APERTURE} = PSA allow co-eval target and lamp images for
direct measurement of their WCA-to-SA offset. \tacq{IMAGE} exposures return before and after target images in \textsc{OBSTYPE}=ACCUM, but do not return lamp images.
\item \textit{EXPTIME} gives the exposure time in seconds. For EXT/SCI PSA images, the lamp time may be different.
%These lamp times are given in Table~\ref{tab:lamptimes}.
\item gives the PtNe Lamp\# gives the wavelength calibration lamp name (\plampone{} or \plamptwo{}) and current setting. The conversion from current setting to current in milli-amps (mA) is given in \S~\ref{subsec:conventions}, and in the table footnotes.
\item gives the configuration (\textit{APERTURE}$\times$\textit{OPT\_ELEM}) of the SA and grating or MIRROR used as the primary optic.
\item \textit{APERXPOS} gives the AD (X$_{USER}$) aperture position. The default position is \textit{APERXPOS}=22 for all FUV and NUV science and TA exposures.\footnote{The trailing "0.1" is a FITS conversion anomaly present in all aperture positions (\textit{APERXPOS} \& \textit{APERYPOS}).}
\item \textit{APERYPOS} gives the XD (Y$_{USER}$) aperture position. It is not uncommon that the XD aperture location (\textit{APERYPOS}) is $\pm 1$ step off from its nominal position during the \fsw{IMCAL} phase. Each \textit{APERYPOS} step is $\approx0.053\arcsec$, or about $\frac{1}{6}$ of
our XD centering requirement, and $\frac{1}{2}$ of our $1\sigma$ XD centering goal. The default NUV and FUV LP1 PSA/BOA positions are \textit{APERYPOS=$126/-153$}, where the WCA has the same XD (\textit{APERYPOS}) position as the PSA.
As shown in Table~\ref{tab:ApMXDispPosition}, the nominal PSA \& WCA \textit{APERYPOS} positions for LP2, LP3, and LP4 are +53, +181, and +234, respectively.\footnote{Due
to the known behavior of the XD aperture mechanism to miss by one step in \textit{APERYPOS}, entries in the \textsc{pcmech\_ApMXDispPosition} FSW table were intentionally offset by $\pm 1$ step, depending on travel direction from NUV/FUV LP1, which
share the common \textsc{pcmech\_ApMXDispPosition} (\textit{APERYPOS}) entry of +126.}
\item \textit{DATE-OBS} gives the date of the observation in YEAR-MOnth-DAy format.
\end{enumerate}
\normalsize

The columns of the NUV imaging WAVECAL table (Table~\ref{tab:NUVtamonimageWCA} are similar to those of Tables~\ref{tab:NUVtamonimagePSA} \& \ref{tab:NUVtamonimageBOA}, with the following exception:
\begin{enumerate}
\item All exposures have \textit{TARGNAME}=WAVE, \textit{OBSMODE}=TT, and \textit{EXPTYPE}=WAVECAL, so these columns have been removed.
\end{enumerate}

The columns of the spectroscopic NUV (Table~\ref{tab:NUVtamonspec}) and FUV (Table~\ref{tab:FUVtamon}) tables are similar to the columns listed above for Tables~\ref{tab:NUVtamonimagePSA} \& \ref{tab:NUVtamonimageBOA}, with the following exceptions:
\begin{itemize}
\item All spectroscopic exposures are time-tag (\textit{OBSMODE}=TT) with the PSA, so the configuration column has been removed.
\item The ``Lamp\#'/Current'' column is not present. In the programs of this ISR, the default \plampone{} lamp usage for spectroscopic observations was overridden with the use of the \textsc{USELAMP=LINE2} and \textsc{CURRENT=MEDIUM} special commanding
in APT to simulate the lamp exposures obtained in \textsc{LTAPKXD} exposures.
The APT optional parameter \texttt{FLASH} was used to set \plamptwo{} exposures times that
provided counts in excess of those expected during the TA exposures.  Since all spectra were taken in TT mode,
if required, an exact replica of the counts received in an actual \textsc{LTAPKXD} WCA spectrum could be re-produced.
The additional counts allow for a better determination of the WCA-to-SA XD offsets discussed in \S~\ref{sec:spVER} and \S~\ref{subsec:FspVER}.
\item The \textit{OPT\_ELEM} column shows the grating in use, as this in now the primary optic.
\item \cenwave{} now follows the \textit{OPT\_ELEM} column, giving the central wavelength setting.
\item The LP gives the lifetime position (\textit{LIFE\_ADJ}) of the observation. Note that there is only one NUV LP, so all NUV observations are LP1.
\end{itemize}
\normalsize

The nominal focus value for \tacq{IMAGE} exposures is \textit{LOMFSTP}=-88. Minor deviations from this value are expected, and should not impact any results presented in this ISR.
For the exposures in this ISR, the $-91\leq \textit{LOMFSTP} \leq-86 $. For all exposures in this ISR, OSM1 reported the nominal \tacq{IMAGE} position of \textit{LOM1STP}, OSM2
also reported all nominal values of \textit{LOM2STP}=8964 for MIRRORA and  9424 for MIRRORB.
% $Id: NUVimagetamonfiles2.tex,v 1.3 2018/04/17 18:38:43 penton Exp $
\begin{center}
\begin{deluxetable}{rcccccccrrr}
\tabcolsep 4 pt
\tabletypesize{\scriptsize}
\tablecolumns{11}
%\tablewidth{0 pt}
\tablecaption{COS/NUV TA Monitoring Imaging Exposures - PSA\label{tab:NUVtamonimagePSA}}
\tablehead{
\colhead{\textit{PROP}}&\colhead{\textit{ROOT}}&\colhead{\textit{TARG}}&
\colhead{\textit{OBS}} &\colhead{\textit{EXPTYPE}} &
\colhead{\textit{EXPTIME}} &\colhead{PtNe (WCA)} &
\colhead{\textit{APERTURE}}&
\colhead{\textit{APER}}&\colhead{\textit{APER}}&\colhead{\textit{DATE-OBS}}\\
\colhead{\textit{OSID}}&\colhead{\textit{NAME}}&
\colhead{\textit{NAME}}&\colhead{\textit{MODE\tablenotemark{t}}}&\colhead{}&
\colhead{(s)}&\colhead{Lamp\#$\times$Current}&
\colhead{$\times$\textit{OPT\_ELEM}}&
\colhead{\textit{XPOS\tablenotemark{x}}}&\colhead{\textit{YPOS\tablenotemark{y}}}&\colhead{}\\
\colhead{(1)}&\colhead{(2)}	&	\colhead{(3)}&\colhead{(4)} &
\colhead{(5)}&\colhead{(6)}	&	\colhead{(7)}&\colhead{(8)} &
\colhead{(9)}&\colhead{(10)}&	\colhead{(11)}
}
\startdata
\toprule
\multicolumn{11}{c}{PSA$\times$MIRA}\\
\midrule
13171	&	lc6ka1i1q	&	427W3	&	ACCUM	&	ACQ/IMAGE	&	60	&	\plamptwo{}$\times$Low	&	PSA$\times$MIRA	&	22.1	&	127.1	&	2013-03-02	\\
13171	&	lc6ka2imq	&	206W3	&	ACCUM	&	ACQ/IMAGE	&	60	&	\plamptwo{}$\times$Low	&	PSA$\times$MIRA	&	22.1	&	127.1	&	2013-09-01	\\
13616	&	lci4a1dcq	&	427W3	&	ACCUM	&	ACQ/IMAGE	&	60	&	\plamptwo{}$\times$Low	&	PSA$\times$MIRA	&	22.1	&	127.1	&	2014-04-03	\\
13616	&	lci4a2e3q	&	206W3	&	ACCUM	&	ACQ/IMAGE	&	60	&	\plamptwo{}$\times$Low	&	PSA$\times$MIRA	&	22.1	&	127.1	&	2014-10-27	\\
13526	&	lcgq03dbq	&	206W3	&	ACCUM	&	ACQ/IMAGE	&	15	&	\plamptwo{}$\times$Low	&	PSA$\times$MIRA	&	22.1	&	127.1	&	2014-10-06	\\
13526	&	lcgq03ddq	&	206W3	&	 TT 	&	EXT/SCI 	&	15	&	\plamptwo{}$\times$Low	&	PSA$\times$MIRA	&	22.1	&	127.1	&	2014-10-06	\\
13526	&	lcgq03drq	&	206W3	&	 TT 	&	EXT/SCI 	&	12	&	\plamptwo{}$\times$Low	&	PSA$\times$MIRA	&	22.1	&	127.1	&	2014-10-06	\\
13526	&	lcgq03dtq	&	206W3	&	ACCUM	&	ACQ/IMAGE	&	12	&	\plamptwo{}$\times$Low	&	PSA$\times$MIRA	&	22.1	&	127.1	&	2014-10-06	\\
14035	&	lcsla1i4q	&	427W3	&	ACCUM	&	ACQ/IMAGE	&	60	&	\plamptwo{}$\times$Low	&	PSA$\times$MIRA	&	22.1	&	125.1	&	2015-04-14	\\
14035	&	lcsla2bhq	&	206W3	&	ACCUM	&	ACQ/IMAGE	&	60	&	\plamptwo{}$\times$Low	&	PSA$\times$MIRA	&	22.1	&	125.1	&	2015-10-02	\\
14452	&	ld3la1coq	&	427W3	&	ACCUM	&	ACQ/IMAGE	&	60	&	\plamptwo{}$\times$Med	&	PSA$\times$MIRA	&	22.1	&	125.1	&	2016-04-01 \\
14452	&	ld3la2ojq	&	206W3	&	ACCUM	&	ACQ/IMAGE	&	60	&	\plamptwo{}$\times$Med	&	PSA$\times$MIRA	&	22.1	&	125.1	&	2016-10-02 \\
14857	&	ldozpbf5q	&	206W3	&	ACCUM	&	ACQ/IMAGE	&	20	&	\plamptwo{}$\times$Low	&	PSA$\times$MIRA	&	22.1	&	125.1	&	2017-09-10	\\
14857	&	ldozpbf7q	&	206W3	&	 TT 	&	EXT/SCI 	&	20	&	\plamptwo{}$\times$Low	&	PSA$\times$MIRA	&	22.1	&	125.1	&	2017-09-10	\\
14857	&	ldozpbffq	&	206W3	&	 TT 	&	EXT/SCI 	&	20	&	\plamptwo{}$\times$Low	&	PSA$\times$MIRA	&	22.1	&	125.1	&	2017-09-10	\\
14857	&	ldozpbfhq	&	206W3	&	ACCUM	&	ACQ/IMAGE	&	20	&	\plamptwo{}$\times$Low	&	PSA$\times$MIRA	&	22.1	&	125.1	&	2017-09-10	\\
\midrule
\multicolumn{11}{c}{PSA$\times$MIRB}\\
\midrule
13124	&	lc6601rrq	&	WD-1657+343	&	ACCUM	&	ACQ/IMAGE	&	12	&	\plamptwo{}$\times$Med	&	PSA$\times$MIRB	&	22.1	&	127.1&	2013-10-24\\
13124	&	lc6601rtq	&	WD-1657+343	&	TT	&	EXT/SCI	&	20	&	\dots\tablenotemark{a}	&	PSA$\times$MIRB	&	22.1	&	127.1&	2013-10-24\\
13124	&	lc6601s0q	&	WD-1657+343	&	TT	&	EXT/SCI	&	20	&	\dots\tablenotemark{a}	&	PSA$\times$MIRB	&	22.1	&	126.1&	2013-10-24\\
13171	&	lc6ka1i3q	&	427W3	&	ACCUM	&	ACQ/IMAGE	&	300	&	\plamptwo{}$\times$Low	&	PSA$\times$MIRB	&22.1	&	127.1	&	2013-03-02	\\
13171	&	lc6ka2ioq	&	206W3	&	ACCUM	&	ACQ/IMAGE	&	300	&	\plamptwo{}$\times$Low	&	PSA$\times$MIRB	&22.1	&	127.1	&	2013-09-01	\\
13616	&	lci4a1deq	&	427W3	&	ACCUM	&	ACQ/IMAGE	&	300	&	\plamptwo{}$\times$Low	&	PSA$\times$MIRB	&22.1	&	127.1	&	2014-04-03	\\
13616	&	lci4a2e5q	&	206W3	&	ACCUM	&	ACQ/IMAGE	&	300	&	\plamptwo{}$\times$Med	&	PSA$\times$MIRB	&22.1	&	127.1	&	2014-10-27	\\
13526	&	lcgq01q5q	&	WD-1657+343	&	ACCUM	&	ACQ/IMAGE	&	12	&	\plamptwo{}$\times$Med	&	PSA$\times$MIRB	&22.1	&	127.1	&	2014-11-19	\\
13526	&	lcgq01q7q	&	WD-1657+343	&	 TT 	&	EXT/SCI 	&	16	&	\plamptwo{}$\times$Med	&	PSA$\times$MIRB	&22.1	&	127.1	&	2014-11-19	\\
13526	&	lcgq01qhq	&	WD-1657+343	&	 TT 	&	EXT/SCI 	&	12	&	\plamptwo{}$\times$Med	&	PSA$\times$MIRB	&22.1	&	126.1	&	2014-11-19	\\
13526	&	lcgq01qjq	&	WD-1657+343	&	ACCUM	&	ACQ/IMAGE	&	12	&	\plamptwo{}$\times$Med	&	PSA$\times$MIRB	&22.1	&	126.1	&	2014-11-19	\\
13526	&	lcgq03dfq	&	206W3	&	 TT 	&	EXT/SCI 	&	160	&	\plamptwo{}$\times$Low	&	PSA$\times$MIRB	&22.1	&	127.1	&	2014-10-06	\\
13526	&	lcgq03dhq	&	206W3	&	 TT 	&	EXT/SCI 	&	180	&	\plamptwo{}$\times$Low	&	PSA$\times$MIRB	&22.1	&	127.1	&	2014-10-06	\\
13526	&	lcgq03djq	&	206W3	&	 TT 	&	EXT/SCI 	&	180	&	\plamptwo{}$\times$Med	&	PSA$\times$MIRB	&22.1	&	127.1	&	2014-10-06	\\
13526	&	lcgq03dlq	&	206W3	&	ACCUM	&	ACQ/IMAGE	&	160	&	\plamptwo{}$\times$Med	&	PSA$\times$MIRB	&22.1	&	127.1	&	2014-10-06	\\
13526	&	lcgq03dnq	&	206W3	&	 TT 	&	EXT/SCI 	&	180	&	\plamptwo{}$\times$Med	&	PSA$\times$MIRB	&22.1	&	127.1	&	2014-10-06	\\
13526	&	lcgq03dpq	&	206W3	&	 TT 	&	EXT/SCI 	&	160	&	\plamptwo{}$\times$Low	&	PSA$\times$MIRB	&22.1	&	127.1	&	2014-10-06	\\
13972	&	lcri01fzq	&	WD-1657+343	&	ACCUM	&	ACQ/IMAGE	&	12	&	\plamptwo{}$\times$Med	&	PSA$\times$MIRB	&22.1	&	125.1	&	2015-10-06	\\
13972	&	lcri01g1q	&	WD-1657+343	&	 TT 	&	EXT/SCI 	&	12	&	\plamptwo{}$\times$Med	&	PSA$\times$MIRB	&22.1	&	125.1	&	2015-10-06	\\
13972	&	lcri01gcq	&	WD-1657+343	&	 TT 	&	EXT/SCI 	&	14	&	\plamptwo{}$\times$Med	&	PSA$\times$MIRB	&22.1	&	126.1	&	2015-10-06	\\
13972	&	lcri01geq	&	WD-1657+343	&	ACCUM	&	ACQ/IMAGE	&	12	&	\plamptwo{}$\times$Med	&	PSA$\times$MIRB	&22.1	&	126.1	&	2015-10-06	\\
14035	&	lcsla1i6q	&	427W3	&	ACCUM	&	ACQ/IMAGE	&	300	&	\plamptwo{}$\times$Med	&	PSA$\times$MIRB	&22.1	&	125.1	&	2015-04-14	\\
14035	&	lcsla2bjq	&	206W3	&	ACCUM	&	ACQ/IMAGE	&	300	&	\plamptwo{}$\times$Med	&	PSA$\times$MIRB	&22.1	&	125.1	&	2015-10-02	\\
14452	&	ld3la1csq	&	427W3	&	ACCUM	&	ACQ/IMAGE	&	300	&	\plamptwo{}$\times$Med	&	PSA$\times$MIRB	&	22.1	&	125.1	&	2016-04-01 \\
14452	&	ld3la2onq	&	206W3	&	ACCUM	&	ACQ/IMAGE	&	300	&	\plamptwo{}$\times$Med	&	PSA$\times$MIRB	&	22.1	&	125.1	&	2016-10-02 \\
14440	&	ld3701gtq	&	WD-1657+343	&	ACCUM	&	ACQ/IMAGE	&	13	&	\plamptwo{}$\times$Med	&	PSA$\times$MIRB	&22.1	&	125.1	&	2016-10-18	\\
14440	&	ld3701gvq	&	WD-1657+343	&	 TT 	&	EXT/SCI 	&	16	&	\plamptwo{}$\times$Med	&	PSA$\times$MIRB	&22.1	&	125.1	&	2016-10-18	\\
14440	&	ld3701h5q	&	WD-1657+343	&	 TT 	&	EXT/SCI 	&	16	&	\plamptwo{}$\times$Med	&	PSA$\times$MIRB	&22.1	&	126.1	&	2016-10-18	\\
14440	&	ld3701h7q	&	WD-1657+343	&	ACCUM	&	ACQ/IMAGE	&	13	&	\plamptwo{}$\times$Med	&	PSA$\times$MIRB	&22.1	&	126.1	&	2016-10-18	\\
14857	&	ldozbadhq	&	WD-1657+343	&	ACCUM	&	ACQ/IMAGE	&	13	&	\plamptwo{}$\times$Med	&	PSA$\times$MIRB	&22.1	&	125.1	&	2017-09-04	\\
14857	&	ldozbadjs	&	WD-1657+343	&	 TT 	&	EXT/SCI 	&	16	&	\plamptwo{}$\times$Med	&	PSA$\times$MIRB	&22.1	&	125.1	&	2017-09-04	\\
14857	&	ldozbadtq	&	WD-1657+343	&	 TT 	&	EXT/SCI 	&	16	&	\plamptwo{}$\times$Med	&	PSA$\times$MIRB	&22.1	&	126.1	&	2017-09-04	\\
14857	&	ldozbadvq	&	WD-1657+343	&	ACCUM	&	ACQ/IMAGE	&	13	&	\plamptwo{}$\times$Med	&	PSA$\times$MIRB	&22.1	&	126.1	&	2017-09-04	\\
14857	&	ldozpbf9q	&	206W3	&	 TT 	&	EXT/SCI 	&	220	&	\plamptwo{}$\times$Med	&	PSA$\times$MIRB	&22.1	&	125.1	&	2017-09-10	\\
14857	&	ldozpbfbq	&	206W3	&	ACCUM	&	ACQ/IMAGE	&	220	&	\plamptwo{}$\times$Med	&	PSA$\times$MIRB	&22.1	&	125.1	&	2017-09-10	\\
14857	&	ldozpbfdq	&	206W3	&	 TT 	&	EXT/SCI 	&	220	&	\plamptwo{}$\times$Med	&	PSA$\times$MIRB	&22.1	&	125.1	&	2017-09-10	\\
\bottomrule
\enddata
\tablenotetext{a}{The intended PtNe lamp flashes of \pid{13124} did not occur as expected. This was corrected in all other programs.}
\tablenotetext{t}{TT = TIME-TAG.}
\tablenotetext{x}{\textit{APERYPOS}, the AD aperture mechanism positions, are stored in the FSW in \textsc{pcmech\_ApMDispPosition}.
The trailing ``0.1'' reported in the FITS headers is a conversion anomaly that is present in all aperture positions (\textit{APERXPOS} \& \textit{APERYPOS}).}
\tablenotetext{y}{It is not uncommon that the XD aperture location (\textit{APERXPOS}) is $\pm 1$ step off from its nominal position.
The XD aperture mechanism positions are stored in the FSW in \textsc{pcmech\_ApMXDispPosition} (see Table~\ref{tab:ApMXDispPosition}).}
\tablecomments{PSA \textsc{EXPTYPE}=EXT/SCI exposures contain coeval target and PtNe lamp TT images.}
\end{deluxetable}
\end{center}

\begin{center}
\begin{deluxetable}{rcccccccrrr}
\tabcolsep 4 pt
\tabletypesize{\scriptsize}
\tablecolumns{11}
%\tablewidth{0 pt}
\tablecaption{COS/NUV TA Monitoring Imaging Exposures - BOA\label{tab:NUVtamonimageBOA}}
\tablehead{
\colhead{\textit{PROP}}&\colhead{\textit{ROOT}}&\colhead{\textit{TARG}}&
\colhead{\textit{OBS}} &\colhead{\textit{EXPTYPE}}  &
\colhead{\textit{EXPTIME}} &\colhead{PtNe (WCA)}  &
\colhead{\textit{APERTURE}}&
\colhead{\textit{APER}}&\colhead{\textit{APER}}&\colhead{\textit{DATE-OBS}}\\
\colhead{\textit{OSID}}&\colhead{\textit{NAME}}&
\colhead{\textit{NAME}}&\colhead{\textit{MODE\tablenotemark{t}}}&\colhead{}&
\colhead{(s)}&\colhead{Lamp\#$\times$Current}&
\colhead{$\times$\textit{OPT\_ELEM}}&
\colhead{\textit{XPOS\tablenotemark{x}}}&\colhead{\textit{YPOS\tablenotemark{y}}}&\colhead{}\\
\colhead{(1)}&\colhead{(2)}	&	\colhead{(3)}&\colhead{(4)} &
\colhead{(5)}&\colhead{(6)}	&	\colhead{(7)}&\colhead{(8)} &
\colhead{(9)}&\colhead{(10)}	&	\colhead{(11)}
}
\startdata
\toprule
\multicolumn{11}{c}{BOA$\times$MIRRORA}\\
\midrule
13124	&	lc6601rvq	&	WD-1657+343	&	ACCUM	&	ACQ/IMAGE	&	150	&	\plamptwo{}$\times$Med	&	BOA$\times$MIRA	&22.1	&	-153.1	&	2013-10-24\\
13124	&	lc6601ryq	&	WD-1657+343	&	TT	&	EXT/SCI	&	150	&	\dots\tablenotemark{a}	&	BOA$\times$MIRA	&	22.1	&	-153.1&	2013-10-24\\
13124	&	lc6602y5q	&	HIP66578	&	ACCUM	&	ACQ/IMAGE	&	12	&	\plamptwo{}$\times$Med	&	BOA$\times$MIRA	&22.1	&	-153.1	&	2013-11-01\\
13124	&	lc6602y7q	&	HIP66578	&	TT	&	EXT/SCI	&	12	&	\dots\tablenotemark{a}	&	BOA$\times$MIRA	&	22.1	&	-153.1&	2013-11-01\\
13526	&	lcgq02hmq	&	HIP66578	&	ACCUM	&	ACQ/IMAGE	&	12	&	\plamptwo{}$\times$Low	&	BOA$\times$MIRA	&22.1	&	-153.1	&	2014-11-17	\\
13526	&	lcgq02i0q	&	HIP66578	&	ACCUM	&	ACQ/IMAGE	&	12	&	\plamptwo{}$\times$Low	&	BOA$\times$MIRA	&22.1	&	-153.1	&	2014-11-17	\\
13526	&	lcgq01q9q	&	WD-1657+343	&	 TT 	&	EXT/SCI 	&	150	&	\plamptwo{}$\times$Med	&	BOA$\times$MIRA	&22.1	&	-153.1	&	2014-11-19	\\
13526	&	lcgq01qdq	&	WD-1657+343	&	ACCUM	&	ACQ/IMAGE	&	150	&	\plamptwo{}$\times$Low	&	BOA$\times$MIRA	&22.1	&	-153.1	&	2014-11-19	\\
13972	&	lcri01g3q	&	WD-1657+343	&	 TT 	&	EXT/SCI 	&	150	&	\plamptwo{}$\times$Med	&	BOA$\times$MIRA	&22.1	&	-153.1	&	2015-10-06	\\
13972	&	lcri01g7q	&	WD-1657+343	&	ACCUM	&	ACQ/IMAGE	&	150	&	\plamptwo{}$\times$Low	&	BOA$\times$MIRA	&22.1	&	-153.1	&	2015-10-06	\\
13972	&	lcri02h8q	&	HIP66578	&	ACCUM	&	ACQ/IMAGE	&	12	&	\plamptwo{}$\times$Low	&	BOA$\times$MIRA	&22.1	&	-153.1	&	2015-10-06	\\
13972	&	lcri02hmq	&	HIP66578	&	ACCUM	&	ACQ/IMAGE	&	12	&	\plamptwo{}$\times$Low	&	BOA$\times$MIRA	&22.1	&	-153.1	&	2015-10-06	\\
14440	&	ld3701gxq	&	WD-1657+343	&	 TT 	&	EXT/SCI 	&	150	&	\plamptwo{}$\times$Med	&	BOA$\times$MIRA	&22.1	&	-153.1	&	2016-10-18	\\
14440	&	ld3701h1q	&	WD-1657+343	&	ACCUM	&	ACQ/IMAGE	&	150	&	\plamptwo{}$\times$Low	&	BOA$\times$MIRA	&22.1	&	-153.1	&	2016-10-18	\\
14440	&	ld3702mzq	&	HIP66578	&	ACCUM	&	ACQ/IMAGE	&	16	&	\plamptwo{}$\times$Low	&	BOA$\times$MIRA	&22.1	&	-153.1	&	2016-10-19	\\
14440	&	ld3702nhq	&	HIP66578	&	ACCUM	&	ACQ/IMAGE	&	16	&	\plamptwo{}$\times$Low	&	BOA$\times$MIRA	&22.1	&	-153.1	&	2016-10-19	\\
14857	&	ldozbadlq	&	WD-1657+343	&	 TT 	&	EXT/SCI 	&	150	&	\plamptwo{}$\times$Med	&	BOA$\times$MIRA	&22.1	&	-153.1	&	2017-09-04	\\
14857	&	ldozbadpq	&	WD-1657+343	&	ACCUM	&	ACQ/IMAGE	&	150	&	\plamptwo{}$\times$Low	&	BOA$\times$MIRA	&22.1	&	-153.1	&	2017-09-04	\\
14857	&	ldozbbleq	&	HIP66578	&	ACCUM	&	ACQ/IMAGE	&	16	&	\plamptwo{}$\times$Low	&	BOA$\times$MIRA	&22.1	&	-153.1	&	2017-09-06	\\
14857	&	ldozbblsq	&	HIP66578	&	ACCUM	&	ACQ/IMAGE	&	16	&	\plamptwo{}$\times$Low	&	BOA$\times$MIRA	&22.1	&	-153.1	&	2017-09-06	\\
\midrule
\multicolumn{11}{c}{BOA$\times$MIRRORA}\\
\midrule
13124	&	lc6602y9q	&	HIP66578	&	ACCUM	&	ACQ/IMAGE	&	175	&	\plamptwo{}$\times$Med	&	BOA$\times$MIRB	&22.1	&	-153.1	&	2013-11-01\\
13124	&	lc6602ycq	&	HIP66578	&	TT	&	EXT/SCI	&	175	&	\dots\tablenotemark{a}	&	BOA$\times$MIRB	&	22.1	&	-153.1&	2013-11-01\\
13124	&	lc6602yxq	&	HIP66578	&	TT	&	EXT/SCI	&	12	&	\dots\tablenotemark{a}	&	BOA$\times$MIRA	&	22.1	&	-153.1&	2013-11-01\\
13526	&	lcgq02hqq	&	HIP66578	&	 TT 	&	EXT/SCI 	&	181	&	\plamptwo{}$\times$Low	&	BOA$\times$MIRB	&22.1	&	-153.1	&	2014-11-17	\\
13526	&	lcgq02huq	&	HIP66578	&	ACCUM	&	ACQ/IMAGE	&	181	&	\plamptwo{}$\times$Med	&	BOA$\times$MIRB	&22.1	&	-153.1	&	2014-11-17	\\
13972	&	lcri02hcq	&	HIP66578	&	 TT 	&	EXT/SCI 	&	181	&	\plamptwo{}$\times$Low	&	BOA$\times$MIRB	&22.1	&	-153.1	&	2015-10-06	\\
13972	&	lcri02hgq	&	HIP66578	&	ACCUM	&	ACQ/IMAGE	&	181	&	\plamptwo{}$\times$Med	&	BOA$\times$MIRB	&22.1	&	-153.1	&	2015-10-06	\\
14440	&	ld3702n4q	&	HIP66578	&	 TT 	&	EXT/SCI 	&	183	&	\plamptwo{}$\times$Low	&	BOA$\times$MIRB	&22.1	&	-153.1	&	2016-10-19	\\
14440	&	ld3702n9q	&	HIP66578	&	ACCUM	&	ACQ/IMAGE	&	183	&	\plamptwo{}$\times$Med	&	BOA$\times$MIRB	&22.1	&	-153.1	&	2016-10-19	\\
14857	&	ldozbbliq	&	HIP66578	&	 TT 	&	EXT/SCI 	&	183	&	\plamptwo{}$\times$Low	&	BOA$\times$MIRB	&22.1	&	-153.1	&	2017-09-06	\\
14857	&	ldozbblmq	&	HIP66578	&	ACCUM	&	ACQ/IMAGE	&	183	&	\plamptwo{}$\times$Med	&	BOA$\times$MIRB	&22.1	&	-153.1	&	2017-09-06	\\
\bottomrule
\enddata
\tablenotetext{a}{The intended PtNe lamp flashes of \pid{13124} did not occur as expected. This was corrected in all other programs.}
%\tablenotetext{c}{The \plamptwo{}wavelength calibration lamp current settings are LOW (3~mA), Med (10~mA) and HIGH (14~mA).}
\tablenotetext{t}{TT = TIME-TAG.}
\tablenotetext{x}{\textit{APERYPOS}, the AD aperture mechanism positions, are stored in the FSW in \textsc{pcmech\_ApMDispPosition}.
The trailing "0.1" reported in the FITS headers is a conversion anomaly that is present in all aperture positions (\textit{APERXPOS} \& \textit{APERYPOS}).}
\tablenotetext{y}{It is not uncommon that the XD aperture location (\textit{APERXPOS}) is $\pm1$ step off from its nominal position.
The XD aperture mechanism positions are stored in the FSW in \textsc{pcmech\_ApMXDispPosition} (see Table~\ref{tab:ApMXDispPosition}).}
\tablecomments{PSA \textsc{EXPTYPE}=EXT/SCI exposures contain coeval target and PtNe lamp TT images.}
\end{deluxetable}
\end{center}
\begin{center}
\begin{deluxetable}{rccccrrr}
%\tabcolsep 8pt
\tabletypesize{\scriptsize}
\tablecolumns{8}
%\tablewidth{0 pt}
\tablecaption{COS/NUV TA Monitoring Imaging Exposures - WCA only\label{tab:NUVtamonimageWCA}}
\tablehead{
\colhead{\textit{PROP}}&\colhead{\textit{ROOT}} &
%\colhead{\textit{TARG}}&
%\colhead{\textit{OBS}} &\colhead{\textit{EXPTYPE}}  &
\colhead{\textit{EXPTIME}} &\colhead{PtNe (WCA)}  &
\colhead{\textit{APERTURE}}&
\colhead{\textit{APER}}&\colhead{\textit{APER}}&\colhead{\textit{DATE-OBS}}\\
\colhead{\textit{OSID}}&\colhead{\textit{NAME}}&
%\colhead{\textit{NAME}}&
%\colhead{\textit{MODE\tablenotemark{t}}}&\colhead{}&
\colhead{(s)}&\colhead{Lamp\#$\times$Current}&
\colhead{$\times$\textit{OPT\_ELEM}}&
\colhead{\textit{XPOS\tablenotemark{x}}}&\colhead{\textit{YPOS\tablenotemark{y}}}&\colhead{}\\
\colhead{(1)}&\colhead{(2)}	&	\colhead{(3)}&\colhead{(4)} &
\colhead{(5)}&\colhead{(6)}	&	\colhead{(7)}&\colhead{(8)}
}
\startdata
\multicolumn{8}{c}{WCA$\times$MIRRORA}\\
\midrule
13523	&	lcgp01byq	&	20	&	\plamptwo{}$\times$Low	&	WCA$\times$MIRA	&	22.1	&	127.1	&	2013-11-11	\\
13523	&	lcgp01c3q	&	20	&	\plampone{}$\times$Low	&	WCA$\times$MIRA	&	22.1	&	127.1	&	2013-11-11	\\
13526	&	lcgq01qbq	&	7	&	\plamptwo{}$\times$Low	&	WCA$\times$MIRA	&	22.1	&	126.1	&	2014-11-19	\\
13526	&	lcgq01qfq	&	7	&	\plamptwo{}$\times$Low	&	WCA$\times$MIRA	&	22.1	&	126.1	&	2014-11-19	\\
13526	&	lcgq02hoq	&	7	&	\plamptwo{}$\times$Low	&	WCA$\times$MIRA	&	22.1	&	126.1	&	2014-11-17	\\
13526	&	lcgq02hyq	&	10	&	\plamptwo{}$\times$Low	&	WCA$\times$MIRA	&	22.1	&	126.1	&	2014-11-17	\\
13526	&	lcgq02icq	&	10	&	\plampone{}$\times$Low	&	WCA$\times$MIRA	&	22.1	&	127.1	&	2014-11-17	\\
13526	&	lcgq02ieq	&	10	&	\plamptwo{}$\times$Low	&	WCA$\times$MIRA	&	22.1	&	127.1	&	2014-11-17	\\
13972	&	lcri01g5q	&	10	&	\plamptwo{}$\times$Low	&	WCA$\times$MIRA	&	22.1	&	126.1	&	2015-10-06	\\
13972	&	lcri01g9q	&	10	&	\plamptwo{}$\times$Low	&	WCA$\times$MIRA	&	22.1	&	126.1	&	2015-10-06	\\
13972	&	lcri02haq	&	14	&	\plamptwo{}$\times$Low	&	WCA$\times$MIRA	&	22.1	&	126.1	&	2015-10-06	\\
13972	&	lcri02hkq	&	14	&	\plamptwo{}$\times$Low	&	WCA$\times$MIRA	&	22.1	&	126.1	&	2015-10-06	\\
13972	&	lcri02hyq	&	14	&	\plampone{}$\times$Low	&	WCA$\times$MIRA	&	22.1	&	125.1	&	2015-10-06	\\
13972	&	lcri02i0q	&	24	&	\plamptwo{}$\times$Low	&	WCA$\times$MIRA	&	22.1	&	125.1	&	2015-10-06	\\
14452	&	ld3la1cqq	&	10	&	\plamptwo{}$\times$Low	&	WCA$\times$MIRA	&	22.1	&	125.1 	&	2016-04-01 \\
14452	&	ld3la2olq	&	10	&	\plamptwo{}$\times$Low	&	WCA$\times$MIRA	&	22.1	&	125.1 	&	2016-10-02 \\
14440	&	ld3701gzq	&	9	&	\plamptwo{}$\times$Low	&	WCA$\times$MIRA	&	22.1	&	126.1	&	2016-10-18	\\
14440	&	ld3701h3q	&	10	&	\plamptwo{}$\times$Low	&	WCA$\times$MIRA	&	22.1	&	126.1	&	2016-10-18	\\
14440	&	ld3702n1q	&	14	&	\plamptwo{}$\times$Low	&	WCA$\times$MIRA	&	22.1	&	126.1	&	2016-10-19	\\
14440	&	ld3702neq	&	14	&	\plamptwo{}$\times$Low	&	WCA$\times$MIRA	&	22.1	&	126.1	&	2016-10-19	\\
14440	&	ld3702o1q	&	14	&	\plampone{}$\times$Low	&	WCA$\times$MIRA	&	22.1	&	125.1	&	2016-10-19	\\
14440	&	ld3702o3q	&	24	&	\plamptwo{}$\times$Low	&	WCA$\times$MIRA	&	22.1	&	125.1	&	2016-10-19	\\
14857	&	ldozbadnq	&	9	&	\plamptwo{}$\times$Low	&	WCA$\times$MIRA	&	22.1	&	126.1	&	2017-09-04	\\
14857	&	ldozbadrq	&	10	&	\plamptwo{}$\times$Low	&	WCA$\times$MIRA	&	22.1	&	126.1	&	2017-09-04	\\
14857	&	ldozbblgq	&	14	&	\plamptwo{}$\times$Low	&	WCA$\times$MIRA	&	22.1	&	126.1	&	2017-09-06	\\
14857	&	ldozbblqq	&	14	&	\plamptwo{}$\times$Low	&	WCA$\times$MIRA	&	22.1	&	126.1	&	2017-09-06	\\
14857	&	ldozbbm4q	&	16	&	\plampone{}$\times$Low	&	WCA$\times$MIRA	&	22.1	&	125.1	&	2017-09-06	\\
14857	&	ldozbbm6q	&	26	&	\plamptwo{}$\times$Low	&	WCA$\times$MIRA	&	22.1	&	125.1	&	2017-09-06	\\
\midrule
\multicolumn{8}{c}{WCA$\times$MIRRORB}\\
\midrule
13523	&	lcgp01bpq	&	40	&	\plamptwo{}$\times$Low	&	WCA$\times$MIRB	&	22.1	&	127.1	&	2013-11-11	\\
13523	&	lcgp01bsq	&	40	&	\plampone{}$\times$Low	&	WCA$\times$MIRB	&	22.1	&	127.1	&	2013-11-11	\\
13526	&	lcgq02hsq	&	12	&	\plamptwo{}$\times$Med	&	WCA$\times$MIRB	&	22.1	&	126.1	&	2014-11-17	\\
13526	&	lcgq02hwq	&	12	&	\plamptwo{}$\times$Med	&	WCA$\times$MIRB	&	22.1	&	126.1	&	2014-11-17	\\
13526	&	lcgq02igq	&	30	&	\plampone{}$\times$Low	&	WCA$\times$MIRB	&	22.1	&	127.1	&	2014-11-17	\\
13526	&	lcgq02iiq	&	20	&	\plamptwo{}$\times$Med	&	WCA$\times$MIRB	&	22.1	&	127.1	&	2014-11-17	\\
13972	&	lcri02heq	&	24	&	\plamptwo{}$\times$Med	&	WCA$\times$MIRB	&	22.1	&	126.1	&	2015-10-06	\\
13972	&	lcri02hiq	&	24	&	\plamptwo{}$\times$Med	&	WCA$\times$MIRB	&	22.1	&	126.1	&	2015-10-06	\\
13972	&	lcri02i2q	&	30	&	\plampone{}$\times$Low	&	WCA$\times$MIRB	&	22.1	&	125.1	&	2015-10-06	\\
13972	&	lcri02i4q	&	24	&	\plamptwo{}$\times$Med	&	WCA$\times$MIRB	&	22.1	&	125.1	&	2015-10-06	\\
14452	&	ld3la1cuq	&	20	&	\plamptwo{}$\times$Med	&	WCA$\times$MIRB	&	22.1	&	125.1	&	2016-04-01	\\
14452	&	ld3la2opq	&	20	&	\plamptwo{}$\times$Med	&	WCA$\times$MIRB	&	22.1	&	125.1	&	2016-10-02	\\
14440	&	ld3702n7q	&	24	&	\plamptwo{}$\times$Med	&	WCA$\times$MIRB	&	22.1	&	126.1	&	2016-10-19	\\
14440	&	ld3702nbq	&	24	&	\plamptwo{}$\times$Med	&	WCA$\times$MIRB	&	22.1	&	126.1	&	2016-10-19	\\
14440	&	ld3702o5q	&	30	&	\plampone{}$\times$Low	&	WCA$\times$MIRB	&	22.1	&	125.1	&	2016-10-19	\\
14440	&	ld3702o7q	&	24	&	\plamptwo{}$\times$Med	&	WCA$\times$MIRB	&	22.1	&	125.1	&	2016-10-19	\\
14857	&	ldozbblkq	&	24	&	\plamptwo{}$\times$Med	&	WCA$\times$MIRB	&	22.1	&	126.1	&	2017-09-06	\\
14857	&	ldozbbloq	&	24	&	\plamptwo{}$\times$Med	&	WCA$\times$MIRB	&	22.1	&	126.1	&	2017-09-06	\\
14857	&	ldozbbm8q	&	32	&	\plampone{}$\times$Low	&	WCA$\times$MIRB	&	22.1	&	125.1	&	2017-09-06	\\
14857	&	ldozbbmaq	&	26	&	\plamptwo{}$\times$Med	&	WCA$\times$MIRB	&	22.1	&	125.1	&	2017-09-06	\\
\bottomrule
\enddata
%\tablenotetext{c}{The \plamptwo{}wavelength calibration lamp current settings are LOW (6~mA), Med (10~mA) and HIGH (18~mA). The \plamptwo{}wavelength calibration lamp current settings are LOW (3~mA), Med (10~mA) and HIGH (14~mA).}
\tablenotetext{x}{\textit{APERYPOS}, the AD aperture mechanism positions are stored in the FSW in \textsc{pcmech\_ApMDispPosition}.
The trailing "0.1" reported in the FITS headers is a conversion anomaly present in all reported aperture positions (\textit{APERXPOS} \& \textit{APERYPOS}).}
\tablenotetext{y}{It is not uncommon that the XD aperture location (\textit{APERXPOS}) is $\pm1$ step off from its nominal position.}
\tablecomments{All exposures in this table are \textit{TARGNAME}=WAVE, time-tag (TT) \textit{EXPTYPE}=WAVECAL (target = WAVE) exposures and contain only PtNe lamp (WCA) images at the indicated MIRROR position (\textit{OPT\_ELEM}).}
\end{deluxetable}
\end{center}

% $Id$
%pcmech_ApMXDispPosition
%% $Id$
%pcmech_ApMXDispPosition
%% $Id$
%pcmech_ApMXDispPosition
%\input{pcmechApMXDispPosition}

\begin{deluxetable}{ccccc}
\tablecolumns{5}
\tablewidth{5 in}
\tablecaption{Cross-Dispersion (XD) Aperture Positions (\textit{APERXPOS})\label{tab:ApMXDispPosition}}
\tablehead{
\colhead{\textit{LIFE\_ADJ}} &    \multicolumn{2}{c}{NUV} & \multicolumn{2}{c}{FUV} \\
\colhead{(LP)} & \colhead{PSA\tablenotemark{a}$/$WCA\tablenotemark{b}} & \colhead{BOA\tablenotemark{c}$/$FCA\tablenotemark{d}} & \colhead{PSA$/$WCA} & \colhead{BOA$/$FCA} \\
\colhead{(1)}&\colhead{(2)} & \colhead{(3)}&\colhead{(4)} & \colhead{(5)}
}
\startdata
\toprule
LP1 &  126	&	-153 	& 126	&	153\\
LP2 &  53	&	-226 	& \dots	&	\dots\\
LP3 &  181	&	 -98	& \dots	&	\dots\\
LP4 &  234	&	 -45 	& \dots	&	\dots\\
\bottomrule
\enddata
\footnotesize
\tablenotetext{a}{PSA=Primary Science Aperture}
\tablenotetext{b}{WCA=Wavelength Calibration Aperture}
\tablenotetext{c}{BOA=Bright Object Aperture}
\tablenotetext{d}{FCA=Flat-field Calibration Aperture}
\tablecomments{COS XD aperture positions (\textit{APERXPOS}) are stored in the \textsc{pcmech\_ApMXDispPosition} FSW table. Although LP1-8 are defined in that table for both NUV and FUV, only the NUV LP1 and FUV LP1--4 entries listed here have been used for science observations.
Values used for FCA calibration observations are different from those listed here, and are commanded via APT special commanding (e.g., during the semi-annual FUV Gain Map programs, {\bf REFERENCE}).
Along-Dispersion (AD) values (\textit{APERYPOS}) are stored in the \textsc{pcmech\_ApMDispPosition} FSW table. All COS apertures and LPs use \textit{APERYPOS=22}. }
\normalsize
\end{deluxetable}


\begin{deluxetable}{ccccc}
\tablecolumns{5}
\tablewidth{5 in}
\tablecaption{Cross-Dispersion (XD) Aperture Positions (\textit{APERXPOS})\label{tab:ApMXDispPosition}}
\tablehead{
\colhead{\textit{LIFE\_ADJ}} &    \multicolumn{2}{c}{NUV} & \multicolumn{2}{c}{FUV} \\
\colhead{(LP)} & \colhead{PSA\tablenotemark{a}$/$WCA\tablenotemark{b}} & \colhead{BOA\tablenotemark{c}$/$FCA\tablenotemark{d}} & \colhead{PSA$/$WCA} & \colhead{BOA$/$FCA} \\
\colhead{(1)}&\colhead{(2)} & \colhead{(3)}&\colhead{(4)} & \colhead{(5)}
}
\startdata
\toprule
LP1 &  126	&	-153 	& 126	&	153\\
LP2 &  53	&	-226 	& \dots	&	\dots\\
LP3 &  181	&	 -98	& \dots	&	\dots\\
LP4 &  234	&	 -45 	& \dots	&	\dots\\
\bottomrule
\enddata
\footnotesize
\tablenotetext{a}{PSA=Primary Science Aperture}
\tablenotetext{b}{WCA=Wavelength Calibration Aperture}
\tablenotetext{c}{BOA=Bright Object Aperture}
\tablenotetext{d}{FCA=Flat-field Calibration Aperture}
\tablecomments{COS XD aperture positions (\textit{APERXPOS}) are stored in the \textsc{pcmech\_ApMXDispPosition} FSW table. Although LP1-8 are defined in that table for both NUV and FUV, only the NUV LP1 and FUV LP1--4 entries listed here have been used for science observations.
Values used for FCA calibration observations are different from those listed here, and are commanded via APT special commanding (e.g., during the semi-annual FUV Gain Map programs, {\bf REFERENCE}).
Along-Dispersion (AD) values (\textit{APERYPOS}) are stored in the \textsc{pcmech\_ApMDispPosition} FSW table. All COS apertures and LPs use \textit{APERYPOS=22}. }
\normalsize
\end{deluxetable}


\begin{deluxetable}{ccccc}
\tablecolumns{5}
\tablewidth{5 in}
\tablecaption{Cross-Dispersion (XD) Aperture Positions (\textit{APERXPOS})\label{tab:ApMXDispPosition}}
\tablehead{
\colhead{\textit{LIFE\_ADJ}} &    \multicolumn{2}{c}{NUV} & \multicolumn{2}{c}{FUV} \\
\colhead{(LP)} & \colhead{PSA\tablenotemark{a}$/$WCA\tablenotemark{b}} & \colhead{BOA\tablenotemark{c}$/$FCA\tablenotemark{d}} & \colhead{PSA$/$WCA} & \colhead{BOA$/$FCA} \\
\colhead{(1)}&\colhead{(2)} & \colhead{(3)}&\colhead{(4)} & \colhead{(5)}
}
\startdata
\toprule
LP1 &  126	&	-153 	& 126	&	153\\
LP2 &  53	&	-226 	& \dots	&	\dots\\
LP3 &  181	&	 -98	& \dots	&	\dots\\
LP4 &  234	&	 -45 	& \dots	&	\dots\\
\bottomrule
\enddata
\footnotesize
\tablenotetext{a}{PSA=Primary Science Aperture}
\tablenotetext{b}{WCA=Wavelength Calibration Aperture}
\tablenotetext{c}{BOA=Bright Object Aperture}
\tablenotetext{d}{FCA=Flat-field Calibration Aperture}
\tablecomments{COS XD aperture positions (\textit{APERXPOS}) are stored in the \textsc{pcmech\_ApMXDispPosition} FSW table. Although LP1-8 are defined in that table for both NUV and FUV, only the NUV LP1 and FUV LP1--4 entries listed here have been used for science observations.
Values used for FCA calibration observations are different from those listed here, and are commanded via APT special commanding (e.g., during the semi-annual FUV Gain Map programs, {\bf REFERENCE}).
Along-Dispersion (AD) values (\textit{APERYPOS}) are stored in the \textsc{pcmech\_ApMDispPosition} FSW table. All COS apertures and LPs use \textit{APERYPOS=22}. }
\normalsize
\end{deluxetable}

\begin{deluxetable}{|r|r|r|r|r|r|r|r|r|r|r|}
\tabcolsep 2pt
\tabletypesize{\tiny}
\tablecolumns{11}
\tablewidth{0 pt}
\tablecaption{COS/NUV TA Spectroscopic Monitoring Exposures\label{table:NUVtamonspec}}
\tablehead{
\colhead{ROOTNAME}&\colhead{PROP}&\colhead{TARGNAME}&
\colhead{EXPTIME}&\colhead{LAMP}&\colhead{CEN}&
\colhead{LP}&\colhead{APER}&\colhead{APER}&\colhead{OPT}&\colhead{DATE}\\
\colhead{}&\colhead{ID}&\colhead{}&
\colhead{(s)}&\colhead{USED}&\colhead{WAVE}&
\colhead{}&\colhead{XPOS}&\colhead{YPOS}&\colhead{ELEM}&\colhead{OBS}\\

}
\startdata
lcgq01qlq	&	13526	&	WD-1657+343	&	20	&	P2	&	3000	&	1	&	22.1	&	126.1	&	G230L	&	2014-11-19	\\
lcgq01r6q	&	13526	&	WD-1657+343	&	151	&	P2	&	2850	&	1	&	22.1	&	126.1	&	G285M	&	2014-11-19	\\
lcgq02i2q	&	13526	&	HIP66578	&	40	&	P2	&	1890	&	1	&	22.1	&	126.1	&	G185M	&	2014-11-17	\\
lcgq02i4q	&	13526	&	HIP66578	&	52	&	P2	&	2306	&	1	&	22.1	&	126.1	&	G225M	&	2014-11-17	\\
lcri01ggq	&	13972	&	WD-1657+343	&	20	&	P2	&	3000	&	1	&	22.1	&	126.1	&	G230L	&	2015-10-06	\\
lcri01giq	&	13972	&	WD-1657+343	&	151	&	P2	&	2676	&	1	&	22.1	&	126.1	&	G285M	&	2015-10-06	\\
lcri02hoq	&	13972	&	HIP66578	&	52	&	P2	&	2306	&	1	&	22.1	&	126.1	&	G225M	&	2015-10-06	\\
lcri02hqq	&	13972	&	HIP66578	&	40	&	P2	&	1913	&	1	&	22.1	&	126.1	&	G185M	&	2015-10-06	\\
ld3701h9q	&	14440	&	WD-1657+343	&	21	&	P2	&	3000	&	1	&	22.1	&	126.1	&	G230L	&	2016-10-18	\\
ld3701hbq	&	14440	&	WD-1657+343	&	151	&	P2	&	2676	&	1	&	22.1	&	126.1	&	G285M	&	2016-10-18	\\
ld3702nmq	&	14440	&	HIP66578	&	53	&	P2	&	2306	&	1	&	22.1	&	126.1	&	G225M	&	2016-10-19	\\
ld3702noq	&	14440	&	HIP66578	&	40	&	P2	&	1913	&	1	&	22.1	&	126.1	&	G185M	&	2016-10-19	\\
ldozbadxq	&	14857	&	WD-1657+343	&	23	&	P2	&	3000	&	1	&	22.1	&	126.1	&	G230L	&	2017-09-04	\\
ldozbadzq	&	14857	&	WD-1657+343	&	151	&	P2	&	2676	&	1	&	22.1	&	126.1	&	G285M	&	2017-09-04	\\
ldozbbluq	&	14857	&	HIP66578	&	53	&	P2	&	2306	&	1	&	22.1	&	126.1	&	G225M	&	2017-09-06	\\
ldozbblwq	&	14857	&	HIP66578	&	40	&	P2	&	1913	&	1	&	22.1	&	126.1	&	G185M	&	2017-09-06	\\
\hline
\enddata
\tablenotetext{a}{All exposures were taken with the PSA at \texttt{FP-POS}=3.}
\end{deluxetable}

% $Id: FUVtamonfiles.tex,v 1.8 2018/04/17 18:38:43 penton Exp $
\begin{deluxetable}{ccrccccccrr}
\tabcolsep 4 pt
\tablewidth{5.7 in}
\tabletypesize{\scriptsize}
\tablecolumns{10}
\tablewidth{0pt}
\tablecaption{FUV TA Monitoring Exposures\label{tab:FUVtamon}}
\tablehead{
\colhead{\textit{PROPOSID}}&\colhead{\textit{ROOTNAME}}&\colhead{\textit{TARGNAME}}&
\colhead{\textit{EXPTIME}}&\colhead{\textit{OPT\_ELEM}}&\colhead{\cenwave{}}&
\colhead{LP}&\colhead{\textit{APER}}&\colhead{\textit{APERY}}&\colhead{\textit{DATE-OBS}}\\
\colhead{}&\colhead{}&\colhead{}&\colhead{(s)}&\colhead{}&\colhead{}&
\colhead{}&\colhead{\textit{XPOS}}&\colhead{\textit{YPOS}}&\colhead{}\\
\colhead{(1)}&\colhead{(2)} &
\colhead{(3)}&\colhead{(4)} &
\colhead{(5)}&\colhead{(6)} &
\colhead{(7)}&\colhead{(8)} &
\colhead{(9)}&\colhead{(10)}
}
\startdata
\toprule
13124	&	lc6601s7q	&	WD-1657+343	&	110	&	G130M	&1309	&	2	&	22.1	&	52.1	&	2013-10-24\\
13124	&	lc6601s9q	&	WD-1657+343	&	30	&	G140L	&1280	&	2	&	22.1	&	52.1	&	2013-10-24\\
13124	&	lc6602z3q	&	HIP66578	&	20	&	G160M	&1623	&	2	&	22.1	&	52.1	&	2013-11-01\\
13124	&	lc6602z9q\tablenotemark{a}	&	HIP66578	&	323	&	G160M	&	1623	&	2	&	22.1	&	-224.1	&	2013-11-01\\
13124	&	lc6602zbq\tablenotemark{b}	&	WAVE	&	12	&	G160M	&	1623	&	2	&	22.1	&	51.1	&	2013-11-01\\
13526	&lcgq01r8q	&	WD-1657+343	&	20	&	G130M	&	1309	&	2	&	22.1	&	52.1	&	2014-11-19	\\
13526	&lcgq01r8q	&	WD-1657+343	&	20	&	G130M	&	1309	&	2	&	22.1	&	52.1	&	2014-11-19	\\
13526	&lcgq01raq	&	WD-1657+343	&	7	&	G140L	&	1280	&	2	&	22.1	&	52.1	&	2014-11-19	\\
13526	&lcgq01raq	&	WD-1657+343	&	7	&	G140L	&	1280	&	2	&	22.1	&	52.1	&	2014-11-19	\\
13526	&lcgq02i6q	&	HIP66578	&	18	&	G160M	&	1600	&	2	&	22.1	&	52.1	&	2014-11-17	\\
13526	&lcgq02i8q	&	HIP66578	&	22	&	G160M	&	1600	&	2	&	22.1	&	52.1	&	2014-11-17	\\
13526	&lcgq02iaq	&	HIP66578	&	22	&	G160M	&	1600	&	2	&	22.1	&	52.1	&	2014-11-17	\\
\midrule
13972	&lcri01gkq	&	WD-1657+343	&	20	&	G130M	&	1309	&	3	&	22.1	&	182.1	&	2015-10-06	\\
13972	&lcri01gkq	&	WD-1657+343	&	20	&	G130M	&	1309	&	3	&	22.1	&	182.1	&	2015-10-06	\\
13972	&lcri01h6q	&	WD-1657+343	&	7	&	G140L	&	1280	&	3	&	22.1	&	182.1	&	2015-10-06	\\
13972	&lcri01h6q	&	WD-1657+343	&	7	&	G140L	&	1280	&	3	&	22.1	&	182.1	&	2015-10-06	\\
13972	&lcri02hsq	&	HIP66578	&	22	&	G160M	&	1600	&	3	&	22.1	&	182.1	&	2015-10-06	\\
13972	&lcri02huq	&	HIP66578	&	25	&	G160M	&	1600	&	3	&	22.1	&	182.1	&	2015-10-06	\\
13972	&lcri02hwq	&	HIP66578	&	25	&	G160M	&	1600	&	3	&	22.1	&	182.1	&	2015-10-06	\\
14440	&ld3701hdq	&	WD-1657+343	&	25	&	G130M	&	1309	&	3	&	22.1	&	182.1	&	2016-10-18	\\
14440	&ld3701hdq	&	WD-1657+343	&	25	&	G130M	&	1309	&	3	&	22.1	&	182.1	&	2016-10-18	\\
14440	&ld3701hfq	&	WD-1657+343	&	10	&	G140L	&	1280	&	3	&	22.1	&	182.1	&	2016-10-18	\\
14440	&ld3701hfq	&	WD-1657+343	&	10	&	G140L	&	1280	&	3	&	22.1	&	182.1	&	2016-10-18	\\
14440	&ld3702nqq	&	HIP66578	&	22	&	G160M	&	1600	&	3	&	22.1	&	182.1	&	2016-10-19	\\
14440	&ld3702nsq	&	HIP66578	&	25	&	G160M	&	1600	&	3	&	22.1	&	182.1	&	2016-10-19	\\
14440	&ld3702nuq	&	HIP66578	&	25	&	G160M	&	1600	&	3	&	22.1	&	182.1	&	2016-10-19	\\
14857	&ldozbae1q	&	WD-1657+343	&	25	&	G130M	&	1309	&	3	&	22.1	&	182.1	&	2017-09-04	\\
14857	&ldozbae1q	&	WD-1657+343	&	25	&	G130M	&	1309	&	3	&	22.1	&	182.1	&	2017-09-04	\\
14857	&ldozbae3q	&	WD-1657+343	&	10	&	G140L	&	1280	&	3	&	22.1	&	182.1	&	2017-09-04	\\
14857	&ldozbae3q	&	WD-1657+343	&	10	&	G140L	&	1280	&	3	&	22.1	&	182.1	&	2017-09-04	\\
14857	&ldozbblyq	&	HIP66578	&	22	&	G160M	&	1600	&	3	&	22.1	&	182.1	&	2017-09-06	\\
14857	&ldozbbm0q	&	HIP66578	&	27	&	G160M	&	1600	&	3	&	22.1	&	182.1	&	2017-09-06	\\
14857	&ldozbbm2q	&	HIP66578	&	27	&	G160M	&	1600	&	3	&	22.1	&	182.1	&	2017-09-06	\\
\bottomrule
\enddata
\tablenotetext{a}{For C20 only (\pid{13124}), a G160M BOA spectrum and WAVECAL were obtained to measure the WCA-to-BOA offset. The BOA was 2 steps off (0.105\arcsec) of its LP2 expected \textit{APERYPOS} position of -226 for this exposure.
This is similar to the $\pm1$ step offset often seen during \tacq{IMAGE}s.}
\tablenotetext{b}{This WAVECAL exposure was used to measure the WCA portion of the WCA-to-BOA offset for the proceeding BOA spectrum, and it off its nominal position of 52.1 by 1 \textit{APERYPOS} step.}
\tablecomments{All exposures taken at \textit{FP-POS=3}. All PSA spectra executed at the expected aperture positions (\textit{APERXPOS} \& \textit{APERYPOS}), while
the indicated BOA spectrum was off by 2 \textit{APERYPOS} steps.}
\end{deluxetable}

