% $Id: elists.tex,v 1.3 2018/03/30 15:20:58 penton Exp $
\subsection{Exposure Lists}\label{subsec:elists}

Table~\ref{tab:NUVtamonimage} gives the operational details of all NUV imaging exposures which opened the
external shutter used in this ISR. Table~\ref{tab:NUVtamonimage} gives the operational details of all NUV imaging WAVECAL
exposures
Tables~\ref{tab:NUVtamonspec} and ~\ref{tab:FUVtamonspec} give the details of all spectroscopic exposures used in this ISR.
All tables follow the convention that if an entry was extracted from a FITS header, then the column name will appear in \textit{ITALICIZED ALL CAPITALS}.

The columns of the Table~\ref{tab:NUVtamonimage} give:
\footnotesize
\begin{enumerate}
\item \textit{ROOTNAME} gives the IPPPSSOOT of the COS exposure,
\item \textit{PROPOSID} gives the HST program id (PID),
\item \textit{TARNMAME} gives the target name as present in the MAST archive,
\item \textit{OBSMODE} gives the observation mode, where ``TT'' is used for Time-Tag observations,
\item \textit{EXPTYPE} gives the exposure type, which is either \tacq{IMAGE} or EXT/SCI. \textit{APERTURE} = PSA EXT/SCI images allow co-eval target and lamp images for
direct measurement of their WCA-to-SA offset. \tacq{IMAGE} exposures return before and after target images in \textsc{OBSTYPE}=ACCUM, but do not return lamp images.
\item \textit{EXPTIME} gives the exposure time in seconds. For EXT/SCI PSA images, the lamp time may be different. These lamp times are given in Table~\ref{tab:lamptimes}.
\item PtNe Lamp\# gives the wavelength calibration lamp name, P1 or P2.
\item Lamp Current gives the lamp current setting. The conversion from current setting to current in milli-amps (mA) is given in \S~\ref{subsec:intro}.
\item \textit{APERTURE} gives the COS SA (PSA or BOA).
\item \textit{APERXPOS} gives the AD (X in detector coordinates) aperture position. The default position is \textit{APERXPOS=$22$} for all FUV and NUV science and TA exposures.\footnote{The trailing "0.1" reported in the FITS headers is a conversion anomaly that is present in all aperture positions (\textit{APERXPOS} \& \textit{APERYPOS}).}
\item \textit{APERYPOS} gives the XD (Y in detector coordinates) aperture position. It is not uncommon that the XD aperture location (\textit{APERYPOS}) is $\pm$ one step off from its nominal position. Each \textit{APERYPOS} step is $\approx0.05\arcsec$, or about 1/6 of
our XD centering requirement, and 1/2 of our $1\sigma$ XD centering goal. The default NUV and FUV LP1 PSA/BOA positions are \textit{APERYPOS=$126/-153$}, where the WCA has the same XD (\textit{APERYPOS}) position as the PSA.
As shown in Table~\ref{tab:ApMXDispPosition} the nominal PSA \& WCA \textit{APERYPOS} position for LP2, LP3, and LP4 are +53, +181, and +234, respectively.\footnote{As explained in the TA enabling ISRs for each FUV LP, due
to known behavior of the COS aperture mechanism to miss by one step in \textit{APERYPOS}, entries in the \textsc{pcmech\_ApMXDispPosition} FSW table were intentionally offset by $\pm$ on step, depending on travel direction from NUV/FUV LP1, which
shared the common \textsc{pcmech\_ApMXDispPosition} (\textit{APERYPOS}) entry of +126.}
\item \textit{OPT\_ELEM} gives the grating or MIRROR used as the primary optic.
\item \textit{DATE-OBS} gives the date of the observation in YEAR-MOnth-DAy format.
\end{enumerate}
\normalsize

The columns of the spectroscopic NUV (Table~\ref{tab:NUVtamonspec}) and FUV (Table~\ref{tab:FUVtamonspec})
tables are identical to the columns listed above for Table~\ref{tab:NUVtamonimage}, with the exception that
the ``Lamp\#''  and ``Lamp Current'' columns are not present. In the programs of this ISR,
the default P1 lamp usage for spectroscopic observations was overridden with the use of the \textsc{USELAMP=LINE2} and \textsc{CURRENT=MEDIUM} special commanding
in APT to simulate the lamp exposures obtained in \numpos1 \tacq{PEAKXD} (\texttt{LTAPKXD}) exposures.
In addition, the APT optional parameter \textsc{FLASH} was used to set P2 exposures times that
provided counts in excess of those expected during the TA exposures.  Since all spectra were taken in TT mode,
if required, an exact replica of the counts received in an actual \texttt{LTAPKXD} WCA spectrum could be re-produced.
The additional counts allow for a better determination of the WCA-to-PSA XD offsets discussed in \S~\ref{sec:sec:spVER} and \S~\ref{subsec:subsec:FspVER}.
\normalsize
% $Id: NUVimagetamonfiles2.tex,v 1.1 2018/03/30 15:20:58 penton Exp $
\begin{deluxetable}{rrrrrrrrrrrrr}
\tabcolsep 4 pt
\tabletypesize{\tiny}
\tablecolumns{13}
%\tablewidth{0 pt}
\tablecaption{COS/NUV TA Monitoring Imaging Exposures - PSA\label{tab:NUVtamonimagePSA}}
\tablehead{
\colhead{\textit{ROOTNAME}}&\colhead{\textit{PROPOSID}}&\colhead{\textit{TARGNAME}}&\colhead{\textit{OBSMODE\tablenotemark{t}}} &\colhead{\textit{EXPTYPE}}   &\colhead{\textit{EXPTIME}}  &\colhead{PtNe}&\colhead{Lamp}   &\colhead{\textit{APERTURE}}&\colhead{\textit{OPT\_ELEM}}&\colhead{\textit{APERXPOS\tablenotemark{x}}}&\colhead{\textit{APERYPOS}\tablenotemark{y}}&\colhead{\textit{DATE-OBS}}\\
\colhead{}                 &\colhead{}  &\colhead{}        &\colhead{}&\colhead{}&\colhead{(s)}&\colhead{Lamp \#}&\colhead{Current\tablenotemark{c}}&\colhead{}&\colhead{}&\colhead{ }&\colhead{}&\colhead{}\\
\colhead{(1)}&\colhead{(2)}	&	\colhead{(3)}&\colhead{(4)} &
\colhead{(5)}&\colhead{(6)}	&	\colhead{(7)}&\colhead{(8)} &
\colhead{(9)}&\colhead{(10)}	&	\colhead{(11)}	&	\colhead{(12)}&\colhead{(13)}
}
\startdata
\toprule
\midrule
\multicolumn{13}{c}{PSA $\times$ MIRA}\\
\midrule
lc6ka1i1q	&	13171	&	427W3	&	ACCUM	&	ACQ/IMAGE	&	60	&	P2	&	Low	&	PSA	&	MIRA	&	22.1	&	127.1	&	2013-03-02	\\
lc6ka2imq	&	13171	&	206W3	&	ACCUM	&	ACQ/IMAGE	&	60	&	P2	&	Low	&	PSA	&	MIRA	&	22.1	&	127.1	&	2013-09-01	\\
lci4a1dcq	&	13616	&	427W3	&	ACCUM	&	ACQ/IMAGE	&	60	&	P2	&	Low	&	PSA	&	MIRA	&	22.1	&	127.1	&	2014-04-03	\\
lci4a2e3q	&	13616	&	206W3	&	ACCUM	&	ACQ/IMAGE	&	60	&	P2	&	Low	&	PSA	&	MIRA	&	22.1	&	127.1	&	2014-10-27	\\
lcgq03dbq	&	13526	&	206W3	&	ACCUM	&	ACQ/IMAGE	&	15	&	P2	&	Low	&	PSA	&	MIRA	&	22.1	&	127.1	&	2014-10-06	\\
lcgq03ddq	&	13526	&	206W3	&	  TT 	&	EXT/SCI 	&	15	&	P2	&	Low	&	PSA	&	MIRA	&	22.1	&	127.1	&	2014-10-06	\\
lcgq03drq	&	13526	&	206W3	&	  TT 	&	EXT/SCI 	&	12	&	P2	&	Low	&	PSA	&	MIRA	&	22.1	&	127.1	&	2014-10-06	\\
lcgq03dtq	&	13526	&	206W3	&	ACCUM	&	ACQ/IMAGE	&	12	&	P2	&	Low	&	PSA	&	MIRA	&	22.1	&	127.1	&	2014-10-06	\\
lcsla1i4q	&	14035	&	427W3	&	ACCUM	&	ACQ/IMAGE	&	60	&	P2	&	Low	&	PSA	&	MIRA	&	22.1	&	125.1	&	2015-04-14	\\
lcsla2bhq	&	14035	&	206W3	&	ACCUM	&	ACQ/IMAGE	&	60	&	P2	&	Low	&	PSA	&	MIRA	&	22.1	&	125.1	&	2015-10-02	\\
ld3la1coq	&	14452	&	427W3	&	ACCUM	&	ACQ/IMAGE	&	60	&	P2	&	Med	&	PSA	&	MIRA	&	22.1	&	125.1	&	2016-04-01 \\
ld3la2ojq	&	14452	&	206W3	&	ACCUM	&	ACQ/IMAGE	&	60	&	P2	&	Med	&	PSA	&	MIRA	&	22.1	&	125.1	&	2016-10-02 \\
ldozpbf5q	&	14857	&	206W3	&	ACCUM	&	ACQ/IMAGE	&	20	&	P2	&	Low	&	PSA	&	MIRA	&	22.1	&	125.1	&	2017-09-10	\\
ldozpbf7q	&	14857	&	206W3	&	  TT 	&	EXT/SCI 	&	20	&	P2	&	Low	&	PSA	&	MIRA	&	22.1	&	125.1	&	2017-09-10	\\
ldozpbffq	&	14857	&	206W3	&	  TT 	&	EXT/SCI 	&	20	&	P2	&	Low	&	PSA	&	MIRA	&	22.1	&	125.1	&	2017-09-10	\\
ldozpbfhq	&	14857	&	206W3	&	ACCUM	&	ACQ/IMAGE	&	20	&	P2	&	Low	&	PSA	&	MIRA	&	22.1	&	125.1	&	2017-09-10	\\
\midrule
\multicolumn{13}{c}{PSA $\times$ MIRB}\\
\midrule
lc6ka1i3q	&	13171	&	427W3	&	ACCUM	&	ACQ/IMAGE	&	300	&	P2	&	Low	&	PSA	&	MIRB	&22.1	&	127.1	&	2013-03-02	\\
lc6ka2ioq	&	13171	&	206W3	&	ACCUM	&	ACQ/IMAGE	&	300	&	P2	&	Low	&	PSA	&	MIRB	&22.1	&	127.1	&	2013-09-01	\\
lci4a1deq	&	13616	&	427W3	&	ACCUM	&	ACQ/IMAGE	&	300	&	P2	&	Low	&	PSA	&	MIRB	&22.1	&	127.1	&	2014-04-03	\\
lci4a2e5q	&	13616	&	206W3	&	ACCUM	&	ACQ/IMAGE	&	300	&	P2	&	Med	&	PSA	&	MIRB	&22.1	&	127.1	&	2014-10-27	\\
lcgq01q5q	&	13526	&	WD-1657+343	&	ACCUM	&	ACQ/IMAGE	&	12	&	P2	&	Med	&	PSA	&	MIRB	&22.1	&	127.1	&	2014-11-19	\\
lcgq01q7q	&	13526	&	WD-1657+343	&	  TT 	&	EXT/SCI 	&	16	&	P2	&	Med	&	PSA	&	MIRB	&22.1	&	127.1	&	2014-11-19	\\
lcgq01qhq	&	13526	&	WD-1657+343	&	  TT 	&	EXT/SCI 	&	12	&	P2	&	Med	&	PSA	&	MIRB	&22.1	&	126.1	&	2014-11-19	\\
lcgq01qjq	&	13526	&	WD-1657+343	&	ACCUM	&	ACQ/IMAGE	&	12	&	P2	&	Med	&	PSA	&	MIRB	&22.1	&	126.1	&	2014-11-19	\\
lcgq03dfq	&	13526	&	206W3	&	  TT 	&	EXT/SCI 	&	160	&	P2	&	Low	&	PSA	&	MIRB	&22.1	&	127.1	&	2014-10-06	\\
lcgq03dhq	&	13526	&	206W3	&	  TT 	&	EXT/SCI 	&	180	&	P2	&	Low	&	PSA	&	MIRB	&22.1	&	127.1	&	2014-10-06	\\
lcgq03djq	&	13526	&	206W3	&	  TT 	&	EXT/SCI 	&	180	&	P2	&	Med	&	PSA	&	MIRB	&22.1	&	127.1	&	2014-10-06	\\
lcgq03dlq	&	13526	&	206W3	&	ACCUM	&	ACQ/IMAGE	&	160	&	P2	&	Med	&	PSA	&	MIRB	&22.1	&	127.1	&	2014-10-06	\\
lcgq03dnq	&	13526	&	206W3	&	  TT 	&	EXT/SCI 	&	180	&	P2	&	Med	&	PSA	&	MIRB	&22.1	&	127.1	&	2014-10-06	\\
lcgq03dpq	&	13526	&	206W3	&	  TT 	&	EXT/SCI 	&	160	&	P2	&	Low	&	PSA	&	MIRB	&22.1	&	127.1	&	2014-10-06	\\
lcri01fzq	&	13972	&	WD-1657+343	&	ACCUM	&	ACQ/IMAGE	&	12	&	P2	&	Med	&	PSA	&	MIRB	&22.1	&	125.1	&	2015-10-06	\\
lcri01g1q	&	13972	&	WD-1657+343	&	  TT 	&	EXT/SCI 	&	12	&	P2	&	Med	&	PSA	&	MIRB	&22.1	&	125.1	&	2015-10-06	\\
lcri01gcq	&	13972	&	WD-1657+343	&	  TT 	&	EXT/SCI 	&	14	&	P2	&	Med	&	PSA	&	MIRB	&22.1	&	126.1	&	2015-10-06	\\
lcri01geq	&	13972	&	WD-1657+343	&	ACCUM	&	ACQ/IMAGE	&	12	&	P2	&	Med	&	PSA	&	MIRB	&22.1	&	126.1	&	2015-10-06	\\
lcsla1i6q	&	14035	&	427W3	&	ACCUM	&	ACQ/IMAGE	&	300	&	P2	&	Med	&	PSA	&	MIRB	&22.1	&	125.1	&	2015-04-14	\\
lcsla2bjq	&	14035	&	206W3	&	ACCUM	&	ACQ/IMAGE	&	300	&	P2	&	Med	&	PSA	&	MIRB	&22.1	&	125.1	&	2015-10-02	\\
ld3la1csq	&	14452	&	427W3	&	ACCUM	&	ACQ/IMAGE	&	300	&	P2	&	Med	&	PSA	&	MIRB	&	22.1	&	125.1	&	2016-04-01 \\
ld3la2onq	&	14452	&	206W3	&	ACCUM	&	ACQ/IMAGE	&	300	&	P2	&	Med	&	PSA	&	MIRB	&	22.1	&	125.1	&	2016-10-02 \\
ld3701gtq	&	14440	&	WD-1657+343	&	ACCUM	&	ACQ/IMAGE	&	13	&	P2	&	Med	&	PSA	&	MIRB	&22.1	&	125.1	&	2016-10-18	\\
ld3701gvq	&	14440	&	WD-1657+343	&	  TT 	&	EXT/SCI 	&	16	&	P2	&	Med	&	PSA	&	MIRB	&22.1	&	125.1	&	2016-10-18	\\
ld3701h5q	&	14440	&	WD-1657+343	&	  TT 	&	EXT/SCI 	&	16	&	P2	&	Med	&	PSA	&	MIRB	&22.1	&	126.1	&	2016-10-18	\\
ld3701h7q	&	14440	&	WD-1657+343	&	ACCUM	&	ACQ/IMAGE	&	13	&	P2	&	Med	&	PSA	&	MIRB	&22.1	&	126.1	&	2016-10-18	\\
ldozbadhq	&	14857	&	WD-1657+343	&	ACCUM	&	ACQ/IMAGE	&	13	&	P2	&	Med	&	PSA	&	MIRB	&22.1	&	125.1	&	2017-09-04	\\
ldozbadjs	&	14857	&	WD-1657+343	&	  TT 	&	EXT/SCI 	&	16	&	P2	&	Med	&	PSA	&	MIRB	&22.1	&	125.1	&	2017-09-04	\\
ldozbadtq	&	14857	&	WD-1657+343	&	  TT 	&	EXT/SCI 	&	16	&	P2	&	Med	&	PSA	&	MIRB	&22.1	&	126.1	&	2017-09-04	\\
ldozbadvq	&	14857	&	WD-1657+343	&	ACCUM	&	ACQ/IMAGE	&	13	&	P2	&	Med	&	PSA	&	MIRB	&22.1	&	126.1	&	2017-09-04	\\
ldozpbf9q	&	14857	&	206W3	&	  TT 	&	EXT/SCI 	&	220	&	P2	&	Med	&	PSA	&	MIRB	&22.1	&	125.1	&	2017-09-10	\\
ldozpbfbq	&	14857	&	206W3	&	ACCUM	&	ACQ/IMAGE	&	220	&	P2	&	Med	&	PSA	&	MIRB	&22.1	&	125.1	&	2017-09-10	\\
ldozpbfdq	&	14857	&	206W3	&	  TT 	&	EXT/SCI 	&	220	&	P2	&	Med	&	PSA	&	MIRB	&22.1	&	125.1	&	2017-09-10	\\
\bottomrule
\enddata
\tablenotetext{c}{The P2 wavelength calibration lamp current settings are LOW (3~mA), MED (10~mA) and HIGH (14~mA).}
\tablenotetext{t}{TT = TIME-TAG.}
\tablenotetext{x}{\textit{APERYPOS}, the AD aperture mechanism positions, are stored in the FSW in \textsc{pcmech\_ApMDispPosition}.
The trailing ``0.1'' reported in the FITS headers is a conversion anomaly that is present in all aperture positions (\textit{APERXPOS} \& \textit{APERYPOS}).}
\tablenotetext{y}{It is not uncommon that the XD aperture location (\textit{APERXPOS}) is $\pm$ one step off from its nominal position.
The XD aperture mechanism positions are stored in the FSW in \textsc{pcmech\_ApMXDispPosition} (see Table~\ref{tab:ApMXDispPosition}).}
\tablecomments{PSA \textsc{EXPTYPE}=EXT/SCI  exposures contain coeval target and PtNe lamp TT images.}
\end{deluxetable}
\begin{deluxetable}{rrrrrrrrrrrrr}
\tabcolsep 4 pt
\tabletypesize{\tiny}
\tablecolumns{13}
%\tablewidth{0 pt}
\tablecaption{COS/NUV TA Monitoring Imaging Exposures - BOA\label{tab:NUVtamonimageBOA}}
\tablehead{
\colhead{\textit{ROOTNAME}}&\colhead{\textit{PROPOSID}}&\colhead{\textit{TARGNAME}}&\colhead{\textit{OBSMODE\tablenotemark{t}}} &\colhead{\textit{EXPTYPE}}   &\colhead{\textit{EXPTIME}}  &\colhead{PtNe}&\colhead{Lamp}   &\colhead{\textit{APERTURE}}&\colhead{\textit{OPT\_ELEM}}&\colhead{\textit{APERXPOS\tablenotemark{x}}}&\colhead{\textit{APERYPOS}\tablenotemark{y}}&\colhead{\textit{DATE-OBS}}\\
\colhead{}                 &\colhead{}  &\colhead{}        &\colhead{}&\colhead{}&\colhead{(s)}&\colhead{Lamp \#}&\colhead{Current\tablenotemark{c}}&\colhead{}&\colhead{}&\colhead{ }&\colhead{}&\colhead{}\\
\colhead{(1)}&\colhead{(2)}	&	\colhead{(3)}&\colhead{(4)} &
\colhead{(5)}&\colhead{(6)}	&	\colhead{(7)}&\colhead{(8)} &
\colhead{(9)}&\colhead{(10)}	&	\colhead{(11)}	&	\colhead{(12)}&\colhead{(13)}
}
\startdata
\toprule
\multicolumn{13}{c}{BOA $\times$ MIRA}\\
\midrule
lcgq01q9q	&	13526	&	WD-1657+343	&	  TT 	&	EXT/SCI 	&	150	&	P2	&	Med	&	BOA	&	MIRA	&22.1	&	-153.1	&	2014-11-19	\\
lcgq01qdq	&	13526	&	WD-1657+343	&	ACCUM	&	ACQ/IMAGE	&	150	&	P2	&	Low	&	BOA	&	MIRA	&22.1	&	-153.1	&	2014-11-19	\\
lcgq02hmq	&	13526	&	HIP66578	&	ACCUM	&	ACQ/IMAGE	&	12	&	P2	&	Low	&	BOA	&	MIRA	&22.1	&	-153.1	&	2014-11-17	\\
lcgq02i0q	&	13526	&	HIP66578	&	ACCUM	&	ACQ/IMAGE	&	12	&	P2	&	Low	&	BOA	&	MIRA	&22.1	&	-153.1	&	2014-11-17	\\
lcri01g3q	&	13972	&	WD-1657+343	&	  TT 	&	EXT/SCI 	&	150	&	P2	&	Med	&	BOA	&	MIRA	&22.1	&	-153.1	&	2015-10-06	\\
lcri01g7q	&	13972	&	WD-1657+343	&	ACCUM	&	ACQ/IMAGE	&	150	&	P2	&	Low	&	BOA	&	MIRA	&22.1	&	-153.1	&	2015-10-06	\\
lcri02h8q	&	13972	&	HIP66578	&	ACCUM	&	ACQ/IMAGE	&	12	&	P2	&	Low	&	BOA	&	MIRA	&22.1	&	-153.1	&	2015-10-06	\\
lcri02hmq	&	13972	&	HIP66578	&	ACCUM	&	ACQ/IMAGE	&	12	&	P2	&	Low	&	BOA	&	MIRA	&22.1	&	-153.1	&	2015-10-06	\\
ld3701gxq	&	14440	&	WD-1657+343	&	  TT 	&	EXT/SCI 	&	150	&	P2	&	Med	&	BOA	&	MIRA	&22.1	&	-153.1	&	2016-10-18	\\
ld3701h1q	&	14440	&	WD-1657+343	&	ACCUM	&	ACQ/IMAGE	&	150	&	P2	&	Low	&	BOA	&	MIRA	&22.1	&	-153.1	&	2016-10-18	\\
ld3702mzq	&	14440	&	HIP66578	&	ACCUM	&	ACQ/IMAGE	&	16	&	P2	&	Low	&	BOA	&	MIRA	&22.1	&	-153.1	&	2016-10-19	\\
ld3702nhq	&	14440	&	HIP66578	&	ACCUM	&	ACQ/IMAGE	&	16	&	P2	&	Low	&	BOA	&	MIRA	&22.1	&	-153.1	&	2016-10-19	\\
ldozbadlq	&	14857	&	WD-1657+343	&	  TT 	&	EXT/SCI 	&	150	&	P2	&	Med	&	BOA	&	MIRA	&22.1	&	-153.1	&	2017-09-04	\\
ldozbadpq	&	14857	&	WD-1657+343	&	ACCUM	&	ACQ/IMAGE	&	150	&	P2	&	Low	&	BOA	&	MIRA	&22.1	&	-153.1	&	2017-09-04	\\
ldozbbleq	&	14857	&	HIP66578	&	ACCUM	&	ACQ/IMAGE	&	16	&	P2	&	Low	&	BOA	&	MIRA	&22.1	&	-153.1	&	2017-09-06	\\
ldozbblsq	&	14857	&	HIP66578	&	ACCUM	&	ACQ/IMAGE	&	16	&	P2	&	Low	&	BOA	&	MIRA	&22.1	&	-153.1	&	2017-09-06	\\
\midrule
\multicolumn{13}{c}{BOA $\times$ MIRA}\\
\midrule
lcgq02hqq	&	13526	&	HIP66578	&	  TT 	&	EXT/SCI 	&	181	&	P2	&	Low	&	BOA	&	MIRB	&22.1	&	-153.1	&	2014-11-17	\\
lcgq02huq	&	13526	&	HIP66578	&	ACCUM	&	ACQ/IMAGE	&	181	&	P2	&	Med	&	BOA	&	MIRB	&22.1	&	-153.1	&	2014-11-17	\\
lcri02hcq	&	13972	&	HIP66578	&	  TT 	&	EXT/SCI 	&	181	&	P2	&	Low	&	BOA	&	MIRB	&22.1	&	-153.1	&	2015-10-06	\\
lcri02hgq	&	13972	&	HIP66578	&	ACCUM	&	ACQ/IMAGE	&	181	&	P2	&	Med	&	BOA	&	MIRB	&22.1	&	-153.1	&	2015-10-06	\\
ld3702n4q	&	14440	&	HIP66578	&	  TT 	&	EXT/SCI 	&	183	&	P2	&	Low	&	BOA	&	MIRB	&22.1	&	-153.1	&	2016-10-19	\\
ld3702n9q	&	14440	&	HIP66578	&	ACCUM	&	ACQ/IMAGE	&	183	&	P2	&	Med	&	BOA	&	MIRB	&22.1	&	-153.1	&	2016-10-19	\\
ldozbbliq	&	14857	&	HIP66578	&	  TT 	&	EXT/SCI 	&	183	&	P2	&	Low	&	BOA	&	MIRB	&22.1	&	-153.1	&	2017-09-06	\\
ldozbblmq	&	14857	&	HIP66578	&	ACCUM	&	ACQ/IMAGE	&	183	&	P2	&	Med	&	BOA	&	MIRB	&22.1	&	-153.1	&	2017-09-06	\\
\bottomrule
\enddata
\tablenotetext{c}{The P2 wavelength calibration lamp current settings are LOW (3~mA), MED (10~mA) and HIGH (14~mA).}
\tablenotetext{t}{TT = TIME-TAG.}
\tablenotetext{x}{\textit{APERYPOS}, the AD aperture mechanism positions, are stored in the FSW in \textsc{pcmech\_ApMDispPosition}.
The trailing "0.1" reported in the FITS headers is a conversion anomaly that is present in all aperture positions (\textit{APERXPOS} \& \textit{APERYPOS}).}
\tablenotetext{y}{It is not uncommon that the XD aperture location (\textit{APERXPOS}) is $\pm$ one step off from its nominal position.
The XD aperture mechanism positions are stored in the FSW in \textsc{pcmech\_ApMXDispPosition} (see Table~\ref{tab:ApMXDispPosition}).}
\tablecomments{PSA \textsc{EXPTYPE}=EXT/SCI  exposures contain coeval target and PtNe lamp TT images.}
\end{deluxetable}

\begin{deluxetable}{rrrcrrrrrrrrr}
\tabcolsep 4 pt
\tabletypesize{\tiny}
\tablecolumns{13}
%\tablewidth{0 pt}
\tablecaption{COS/NUV TA Monitoring Imaging Exposures - WCA only\label{tab:NUVtamonimageWCA}}
\tablehead{
\colhead{\textit{ROOTNAME}}&\colhead{\textit{PROPOSID}}&\colhead{\textit{TARGNAME}}&\colhead{\textit{OBSMODE\tablenotemark{t}}} &\colhead{\textit{EXPTYPE}}   &\colhead{\textit{EXPTIME}}  &\colhead{PtNe}&\colhead{Lamp}   &\colhead{\textit{APERTURE}}&\colhead{\textit{OPT\_ELEM}}&\colhead{\textit{APERXPOS\tablenotemark{x}}}&\colhead{\textit{APERYPOS}\tablenotemark{y}}&\colhead{\textit{DATE-OBS}}\\
\colhead{}                 &\colhead{}  &\colhead{}        &\colhead{}&\colhead{}&\colhead{(s)}&\colhead{Lamp \#}&\colhead{Current\tablenotemark{c}}&\colhead{}&\colhead{}&\colhead{ }&\colhead{}&\colhead{}\\
\colhead{(1)}&\colhead{(2)}	&	\colhead{(3)}&\colhead{(4)} &
\colhead{(5)}&\colhead{(6)}	&	\colhead{(7)}&\colhead{(8)} &
\colhead{(9)}&\colhead{(10)}	&	\colhead{(11)}	&	\colhead{(12)}&\colhead{(13)}
}
\startdata
\multicolumn{13}{c}{WCA$\times$MIRA}\\
\midrule
lcgp01byq	&	13523	&	WAVE	&	  TT 	&	WAVECAL	&	20	&	P2	&	Low	&	WCA&	MIRA	&	22.1	&	127.1	&	2013-11-11	\\
lcgp01c3q	&	13523	&	WAVE	&	  TT 	&	WAVECAL	&	20	&	P1	&	Low	&	WCA&	MIRA	&	22.1	&	127.1	&	2013-11-11	\\
lcgq01qbq	&	13526	&	WAVE	&	  TT 	&	WAVECAL	&	7	&	P2	&	Low	&	WCA&	MIRA	&	22.1	&	126.1	&	2014-11-19	\\
lcgq01qfq	&	13526	&	WAVE	&	  TT 	&	WAVECAL	&	7	&	P2	&	Low	&	WCA&	MIRA	&	22.1	&	126.1	&	2014-11-19	\\
lcgq02hoq	&	13526	&	WAVE	&	  TT 	&	WAVECAL	&	7	&	P2	&	Low	&	WCA&	MIRA	&	22.1	&	126.1	&	2014-11-17	\\
lcgq02hyq	&	13526	&	WAVE	&	  TT 	&	WAVECAL	&	10	&	P2	&	Low	&	WCA&	MIRA	&	22.1	&	126.1	&	2014-11-17	\\
lcgq02icq	&	13526	&	WAVE	&	  TT 	&	WAVECAL	&	10	&	P1	&	Low	&	WCA&	MIRA	&	22.1	&	127.1	&	2014-11-17	\\
lcgq02ieq	&	13526	&	WAVE	&	  TT 	&	WAVECAL	&	10	&	P2	&	Low	&	WCA&	MIRA	&	22.1	&	127.1	&	2014-11-17	\\
lcri01g5q	&	13972	&	WAVE	&	  TT 	&	WAVECAL	&	10	&	P2	&	Low	&	WCA&	MIRA	&	22.1	&	126.1	&	2015-10-06	\\
lcri01g9q	&	13972	&	WAVE	&	  TT 	&	WAVECAL	&	10	&	P2	&	Low	&	WCA&	MIRA	&	22.1	&	126.1	&	2015-10-06	\\
lcri02haq	&	13972	&	WAVE	&	  TT 	&	WAVECAL	&	14	&	P2	&	Low	&	WCA&	MIRA	&	22.1	&	126.1	&	2015-10-06	\\
lcri02hkq	&	13972	&	WAVE	&	  TT 	&	WAVECAL	&	14	&	P2	&	Low	&	WCA&	MIRA	&	22.1	&	126.1	&	2015-10-06	\\
lcri02hyq	&	13972	&	WAVE	&	  TT 	&	WAVECAL	&	14	&	P1	&	Low	&	WCA&	MIRA	&	22.1	&	125.1	&	2015-10-06	\\
lcri02i0q	&	13972	&	WAVE	&	  TT 	&	WAVECAL	&	24	&	P2	&	Low	&	WCA&	MIRA	&	22.1	&	125.1	&	2015-10-06	\\
ld3la1cqq	&	14452	&	WAVE	&	  TT 	&	WAVECAL	&	10	&	P2	&	Low	&	WCA&	MIRA	&	22.1 	&	125.1 	&	2016-04-01 \\
ld3la2olq	&	14452	&	WAVE	&	  TT 	&	WAVECAL	&	10	&	P2	&	Low	&	WCA&	MIRA	&	22.1 	&	125.1 	&	2016-10-02 \\
ld3701gzq	&	14440	&	WAVE	&	  TT 	&	WAVECAL	&	9	&	P2	&	Low	&	WCA&	MIRA	&	22.1	&	126.1	&	2016-10-18	\\
ld3701h3q	&	14440	&	WAVE	&	  TT 	&	WAVECAL	&	10	&	P2	&	Low	&	WCA&	MIRA	&	22.1	&	126.1	&	2016-10-18	\\
ld3702n1q	&	14440	&	WAVE	&	  TT 	&	WAVECAL	&	14	&	P2	&	Low	&	WCA&	MIRA	&	22.1	&	126.1	&	2016-10-19	\\
ld3702neq	&	14440	&	WAVE	&	  TT 	&	WAVECAL	&	14	&	P2	&	Low	&	WCA&	MIRA	&	22.1	&	126.1	&	2016-10-19	\\
ld3702o1q	&	14440	&	WAVE	&	  TT 	&	WAVECAL	&	14	&	P1	&	Low	&	WCA&	MIRA	&	22.1	&	125.1	&	2016-10-19	\\
ld3702o3q	&	14440	&	WAVE	&	  TT 	&	WAVECAL	&	24	&	P2	&	Low	&	WCA&	MIRA	&	22.1	&	125.1	&	2016-10-19	\\
ldozbadnq	&	14857	&	WAVE	&	  TT 	&	WAVECAL	&	9	&	P2	&	Low	&	WCA&	MIRA	&	22.1	&	126.1	&	2017-09-04	\\
ldozbadrq	&	14857	&	WAVE	&	  TT 	&	WAVECAL	&	10	&	P2	&	Low	&	WCA&	MIRA	&	22.1	&	126.1	&	2017-09-04	\\
ldozbblgq	&	14857	&	WAVE	&	  TT 	&	WAVECAL	&	14	&	P2	&	Low	&	WCA&	MIRA	&	22.1	&	126.1	&	2017-09-06	\\
ldozbblqq	&	14857	&	WAVE	&	  TT 	&	WAVECAL	&	14	&	P2	&	Low	&	WCA&	MIRA	&	22.1	&	126.1	&	2017-09-06	\\
ldozbbm4q	&	14857	&	WAVE	&	  TT 	&	WAVECAL	&	16	&	P1	&	Low	&	WCA&	MIRA	&	22.1	&	125.1	&	2017-09-06	\\
ldozbbm6q	&	14857	&	WAVE	&	  TT 	&	WAVECAL	&	26	&	P2	&	Low	&	WCA&	MIRA	&	22.1	&	125.1	&	2017-09-06	\\
\midrule
\multicolumn{13}{c}{WCA$\times$MIRB}\\
\midrule
lcgp01bpq	&	13523	&	WAVE	&	  TT 	&	WAVECAL	&	40	&	P2	&	Low	&	WCA&	MIRB	&	22.1	&	127.1	&	2013-11-11	\\
lcgp01bsq	&	13523	&	WAVE	&	  TT 	&	WAVECAL	&	40	&	P1	&	Low	&	WCA&	MIRB	&	22.1	&	127.1	&	2013-11-11	\\
lcgq02hsq	&	13526	&	WAVE	&	  TT 	&	WAVECAL	&	12	&	P2	&	Med	&	WCA&	MIRB	&	22.1	&	126.1	&	2014-11-17	\\
lcgq02hwq	&	13526	&	WAVE	&	  TT 	&	WAVECAL	&	12	&	P2	&	Med	&	WCA&	MIRB	&	22.1	&	126.1	&	2014-11-17	\\
lcgq02igq	&	13526	&	WAVE	&	  TT 	&	WAVECAL	&	30	&	P1	&	Low	&	WCA&	MIRB	&	22.1	&	127.1	&	2014-11-17	\\
lcgq02iiq	&	13526	&	WAVE	&	  TT 	&	WAVECAL	&	20	&	P2	&	Med	&	WCA&	MIRB	&	22.1	&	127.1	&	2014-11-17	\\
lcri02heq	&	13972	&	WAVE	&	  TT 	&	WAVECAL	&	24	&	P2	&	Med	&	WCA&	MIRB	&	22.1	&	126.1	&	2015-10-06	\\
lcri02hiq	&	13972	&	WAVE	&	  TT 	&	WAVECAL	&	24	&	P2	&	Med	&	WCA&	MIRB	&	22.1	&	126.1	&	2015-10-06	\\
lcri02i2q	&	13972	&	WAVE	&	  TT 	&	WAVECAL	&	30	&	P1	&	Low	&	WCA&	MIRB	&	22.1	&	125.1	&	2015-10-06	\\
lcri02i4q	&	13972	&	WAVE	&	  TT 	&	WAVECAL	&	24	&	P2	&	Med	&	WCA&	MIRB	&	22.1	&	125.1	&	2015-10-06	\\
ld3la1cuq	&	14452	&	WAVE	&	  TT 	&	WAVECAL	&	20	&	P2	&	Med	&	WCA&	MIRB	&	22.1	&	125.1	&	2016-04-01	\\
ld3la2opq	&	14452	&	WAVE	&	  TT 	&	WAVECAL	&	20	&	P2	&	Med	&	WCA&	MIRB	&	22.1	&	125.1	&	2016-10-02	\\
ld3702n7q	&	14440	&	WAVE	&	  TT 	&	WAVECAL	&	24	&	P2	&	Med	&	WCA&	MIRB	&	22.1	&	126.1	&	2016-10-19	\\
ld3702nbq	&	14440	&	WAVE	&	  TT 	&	WAVECAL	&	24	&	P2	&	Med	&	WCA&	MIRB	&	22.1	&	126.1	&	2016-10-19	\\
ld3702o5q	&	14440	&	WAVE	&	  TT 	&	WAVECAL	&	30	&	P1	&	Low	&	WCA&	MIRB	&	22.1	&	125.1	&	2016-10-19	\\
ld3702o7q	&	14440	&	WAVE	&	  TT 	&	WAVECAL	&	24	&	P2	&	Med	&	WCA&	MIRB	&	22.1	&	125.1	&	2016-10-19	\\
ldozbblkq	&	14857	&	WAVE	&	  TT 	&	WAVECAL	&	24	&	P2	&	Med	&	WCA&	MIRB	&	22.1	&	126.1	&	2017-09-06	\\
ldozbbloq	&	14857	&	WAVE	&	  TT 	&	WAVECAL	&	24	&	P2	&	Med	&	WCA&	MIRB	&	22.1	&	126.1	&	2017-09-06	\\
ldozbbm8q	&	14857	&	WAVE	&	  TT 	&	WAVECAL	&	32	&	P1	&	Low	&	WCA&	MIRB	&	22.1	&	125.1	&	2017-09-06	\\
ldozbbmaq	&	14857	&	WAVE	&	  TT 	&	WAVECAL	&	26	&	P2	&	Med	&	WCA&	MIRB	&	22.1	&	125.1	&	2017-09-06	\\
\bottomrule
\enddata
\tablenotetext{c}{The P2 wavelength calibration lamp current settings are LOW (6~mA), MED (10~mA) and HIGH (18~mA). The P2 wavelength calibration lamp current settings are LOW (3~mA), MED (10~mA) and HIGH (14~mA).}
\tablenotetext{t}{TT = TIME-TAG.}
\tablenotetext{x}{\textit{APERYPOS}, the AD aperture mechanism positions are stored in the FSW in \textsc{pcmech\_ApMDispPosition}.
The trailing "0.1" reported in the FITS headers is a conversion anomaly that is present in all aperture positions (\textit{APERXPOS} \& \textit{APERYPOS}).}
\tablenotetext{y}{It is not uncommon that the XD aperture location (\textit{APERXPOS}) is $\pm$ one step off from its nominal position.}
\tablecomments{All exposures in this table are \textit{EXPTYPE}=WAVECAL (target = WAVE) and contain only TT PtNe lamp (WCA) images, and the indicated MIRROR position (\textit{OPT\_ELEM}).}
\end{deluxetable}

% $Id: NUVspectamonfiles.tex,v 1.5 2018/03/30 15:20:58 penton Exp $
\begin{deluxetable}{rrrrrrrrrrr}
\tabcolsep 3pt
\tabletypesize{\scriptsize}
\tablecolumns{11}
%\tablewidth{5.5 in}
\tablecaption{COS/NUV TA Spectroscopic Monitoring Exposures\label{tab:NUVtamonspec}}
\tablehead{
\colhead{\textit{ROOTNAME}}&\colhead{PID}&\colhead{TARGNAME}&
\colhead{\textit{EXPTIME}}&\colhead{LAMP}&\colhead{\textit{OPT\_ELEM}}&\colhead{\cenwave}&
\colhead{LP}&\colhead{\textit{APERXPOS}}&\colhead{\textit{APERYPOS}}&\colhead{\textit{DATE-OBS}}\\
\colhead{}&\colhead{}&\colhead{}&
\colhead{(s)}&\colhead{USED}&\colhead{}&
\colhead{}&\colhead{}&\colhead{}&\colhead{}&\colhead{}
}
\startdata
lcgq01qlq	&	13526	&	WD-1657+343	&	20	&	P2	&	G230L	&	3000	&	1	&	22.1	&	126.1	&	2014-11-19	\\
lcgq01r6q	&	13526	&	WD-1657+343	&	151	&	P2	&	G285M	&	2850	&	1	&	22.1	&	126.1	&	2014-11-19	\\
lcgq02i2q	&	13526	&	HIP66578	&	40	&	P2	&	G185M	&	1890	&	1	&	22.1	&	126.1	&	2014-11-17	\\
lcgq02i4q	&	13526	&	HIP66578	&	52	&	P2	&	G225M	&	2306	&	1	&	22.1	&	126.1	&	2014-11-17	\\
lcri01ggq	&	13972	&	WD-1657+343	&	20	&	P2	&	G230L	&	3000	&	1	&	22.1	&	126.1	&	2015-10-06	\\
lcri01giq	&	13972	&	WD-1657+343	&	151	&	P2	&	G285M	&	2676	&	1	&	22.1	&	126.1	&	2015-10-06	\\
lcri02hoq	&	13972	&	HIP66578	&	52	&	P2	&	G225M	&	2306	&	1	&	22.1	&	126.1	&	2015-10-06	\\
lcri02hqq	&	13972	&	HIP66578	&	40	&	P2	&	G185M	&	1913	&	1	&	22.1	&	126.1	&	2015-10-06	\\
ld3701h9q	&	14440	&	WD-1657+343	&	21	&	P2	&	G230L	&	3000	&	1	&	22.1	&	126.1	&	2016-10-18	\\
ld3701hbq	&	14440	&	WD-1657+343	&	151	&	P2	&	G285M	&	2676	&	1	&	22.1	&	126.1	&	2016-10-18	\\
ld3702nmq	&	14440	&	HIP66578	&	53	&	P2	&	G225M	&	2306	&	1	&	22.1	&	126.1	&	2016-10-19	\\
ld3702noq	&	14440	&	HIP66578	&	40	&	P2	&	G185M	&	1913	&	1	&	22.1	&	126.1	&	2016-10-19	\\
ldozbadxq	&	14857	&	WD-1657+343	&	23	&	P2	&	G230L	&	3000	&	1	&	22.1	&	126.1	&	2017-09-04	\\
ldozbadzq	&	14857	&	WD-1657+343	&	151	&	P2	&	G285M	&	2676	&	1	&	22.1	&	126.1	&	2017-09-04	\\
ldozbbluq	&	14857	&	HIP66578	&	53	&	P2	&	G225M	&	2306	&	1	&	22.1	&	126.1	&	2017-09-06	\\
ldozbblwq	&	14857	&	HIP66578	&	40	&	P2	&	G185M	&	1913	&	1	&	22.1	&	126.1	&	2017-09-06	\\
\hline
\enddata
\tablenotemark{y}{The NUV (LP1) XD location of the aperture (\texttt{APERYPOS}) is 126 in the FSW table \texttt{pcmech\_ApMXDispPosition}.).}
\tablecomments{All exposures were taken with the PSA at \texttt{FP-POS=3}. All exposures executed at the expected aperture position (\texttt{APERXPOS} \& \texttt{APERYPOS}).}
\end{deluxetable}

% $Id: FUVtamonfiles.tex,v 1.6 2018/03/30 22:05:03 penton Exp $
\begin{deluxetable}{rrrrrrrrrrr}
\tabcolsep 3 pt
\tabletypesize{\scriptsize}
\tablecolumns{11}
%\tablewidth{5 in}
\tablecaption{COS/FUV TA Monitoring Exposures\label{tab:FUVtamon}}
\tablehead{
\colhead{\textit{ROOTNAME}}&\colhead{\textit{PROPOSID}}&\colhead{\textit{TARGNAME}}&
\colhead{\textit{EXPTIME}}&\colhead{LAMP}&\colhead{\cenwave}&
\colhead{LP}&\colhead{\textit{APERXPOS}}&\colhead{\textit{APERYPOS}}&\colhead{\textit{OPT\_ELEM}}&\colhead{\textit{DATE-OBS}}\\
\colhead{}&\colhead{}&\colhead{}&\colhead{(s)}&\colhead{USED}&\colhead{}&
\colhead{}&\colhead{}&\colhead{}&\colhead{}&\colhead{}\\
\colhead{(1)}&\colhead{(2)} &
\colhead{(3)}&\colhead{(4)} &
\colhead{(5)}&\colhead{(6)} &
\colhead{(7)}&\colhead{(8)} &
\colhead{(9)}&\colhead{(10)} &
\colhead{(11)}
}
\startdata
lcgq01r8q	&	13526	&	WD-1657+343	&	20	&	P2	&	1309	&	2	&	22.1	&	52.1	&	G130M	&	2014-11-19	\\
lcgq01r8q	&	13526	&	WD-1657+343	&	20	&	P2	&	1309	&	2	&	22.1	&	52.1	&	G130M	&	2014-11-19	\\
lcgq01raq	&	13526	&	WD-1657+343	&	7	&	P2	&	1280	&	2	&	22.1	&	52.1	&	G140L	&	2014-11-19	\\
lcgq01raq	&	13526	&	WD-1657+343	&	7	&	P2	&	1280	&	2	&	22.1	&	52.1	&	G140L	&	2014-11-19	\\
lcgq02i6q	&	13526	&	HIP66578	&	18	&	P2	&	1600	&	2	&	22.1	&	52.1	&	G160M	&	2014-11-17	\\
lcgq02i8q	&	13526	&	HIP66578	&	22	&	P2	&	1600	&	2	&	22.1	&	52.1	&	G160M	&	2014-11-17	\\
lcgq02iaq	&	13526	&	HIP66578	&	22	&	P2	&	1600	&	2	&	22.1	&	52.1	&	G160M	&	2014-11-17	\\
\hline
lcri01gkq	&	13972	&	WD-1657+343	&	20	&	P2	&	1309	&	3	&	22.1	&	182.1	&	G130M	&	2015-10-06	\\
lcri01gkq	&	13972	&	WD-1657+343	&	20	&	P2	&	1309	&	3	&	22.1	&	182.1	&	G130M	&	2015-10-06	\\
lcri01h6q	&	13972	&	WD-1657+343	&	7	&	P2	&	1280	&	3	&	22.1	&	182.1	&	G140L	&	2015-10-06	\\
lcri01h6q	&	13972	&	WD-1657+343	&	7	&	P2	&	1280	&	3	&	22.1	&	182.1	&	G140L	&	2015-10-06	\\
lcri02hsq	&	13972	&	HIP66578	&	22	&	P2	&	1600	&	3	&	22.1	&	182.1	&	G160M	&	2015-10-06	\\
lcri02huq	&	13972	&	HIP66578	&	25	&	P2	&	1600	&	3	&	22.1	&	182.1	&	G160M	&	2015-10-06	\\
lcri02hwq	&	13972	&	HIP66578	&	25	&	P2	&	1600	&	3	&	22.1	&	182.1	&	G160M	&	2015-10-06	\\
ld3701hdq	&	14440	&	WD-1657+343	&	25	&	P2	&	1309	&	3	&	22.1	&	182.1	&	G130M	&	2016-10-18	\\
ld3701hdq	&	14440	&	WD-1657+343	&	25	&	P2	&	1309	&	3	&	22.1	&	182.1	&	G130M	&	2016-10-18	\\
ld3701hfq	&	14440	&	WD-1657+343	&	10	&	P2	&	1280	&	3	&	22.1	&	182.1	&	G140L	&	2016-10-18	\\
ld3701hfq	&	14440	&	WD-1657+343	&	10	&	P2	&	1280	&	3	&	22.1	&	182.1	&	G140L	&	2016-10-18	\\
ld3702nqq	&	14440	&	HIP66578	&	22	&	P2	&	1600	&	3	&	22.1	&	182.1	&	G160M	&	2016-10-19	\\
ld3702nsq	&	14440	&	HIP66578	&	25	&	P2	&	1600	&	3	&	22.1	&	182.1	&	G160M	&	2016-10-19	\\
ld3702nuq	&	14440	&	HIP66578	&	25	&	P2	&	1600	&	3	&	22.1	&	182.1	&	G160M	&	2016-10-19	\\
ldozbae1q	&	14857	&	WD-1657+343	&	25	&	P2	&	1309	&	3	&	22.1	&	182.1	&	G130M	&	2017-09-04	\\
ldozbae1q	&	14857	&	WD-1657+343	&	25	&	P2	&	1309	&	3	&	22.1	&	182.1	&	G130M	&	2017-09-04	\\
ldozbae3q	&	14857	&	WD-1657+343	&	10	&	P2	&	1280	&	3	&	22.1	&	182.1	&	G140L	&	2017-09-04	\\
ldozbae3q	&	14857	&	WD-1657+343	&	10	&	P2	&	1280	&	3	&	22.1	&	182.1	&	G140L	&	2017-09-04	\\
ldozbblyq	&	14857	&	HIP66578	&	22	&	P2	&	1600	&	3	&	22.1	&	182.1	&	G160M	&	2017-09-06	\\
ldozbbm0q	&	14857	&	HIP66578	&	27	&	P2	&	1600	&	3	&	22.1	&	182.1	&	G160M	&	2017-09-06	\\
ldozbbm2q	&	14857	&	HIP66578	&	27	&	P2	&	1600	&	3	&	22.1	&	182.1	&	G160M	&	2017-09-06	\\
\hline
\enddata
\tablecomments{All exposures were taken with the PSA at \textit{FP-POS=3}. All exposures executed at the expected aperture positions (\textit{APERXPOS} \& \textit{APERYPOS}).}
\end{deluxetable}

