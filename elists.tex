\subsection{Exposure Lists}\label{subsec:elists}

In Visit 01, we take spectra that meet these requirements with the G130M/1309, G140L/1280, G285M/2676, and G230L/3000, and in Visit 02,
we take spectra with the G160M/1600, G185M/1913, G225M/2306. Table ~\ref{tab:peakxd} the results of these exposures are summarized.
The rightmost column gives the WCA-to-PSA offsets measured in P13972, in arcseconds (\arcsec).
All exposures, except {\sf lcri01h6q}, the G140L/1280 measurement, which showed an offset of 0.15\arcsec\ exceed our $\pm 0.1$\arcsec\ goal.
All exposures exceed our $\pm 0.33$\arcsec\ requirement. The XD profile of G140L spectra is wider that the medium
resolution gratings (G130M and G160M), making in more susceptible to detector `Y-walk' (Penton \& Keyes, 2010).
No action is required at this time as the measured offset is 1/2 of our 0.3\arcsec\ requirement.

The final two exposures of the 02 visit intentionally offset the target by $\pm$ 0.7\arcsec\ to test the effects
of `Y-walk' on G160M \tacq{PEAKXD}s. All three G160M exposures in Visit 02 show offsets from the expected position
of $\le 0.05$\arcsec\ within our 0.1\arcsec\ goal. No action (e.g., updating the \textsc{pcta\_CalTargetOffset} in the FSW)
is required at this time.
% $Id: NUVimagetamonfiles.tex,v 1.4 2018/03/30 15:20:58 penton Exp $
\begin{deluxetable}{rrrrrrrrrrrrr}
\tabcolsep 4 pt
\tabletypesize{\tiny}
\tablecolumns{13}
%\tablewidth{0 pt}
\tablecaption{COS/NUV TA Monitoring Imaging Exposures\label{tab:NUVtamonimage}}
\tablehead{
\colhead{\textit{ROOTNAME}}&\colhead{\textit{PROPOSID}}&\colhead{\textit{TARGNAME}}&\colhead{\textit{OBSMODE\tablenotemark{t}}} &\colhead{\textit{EXPTYPE}}   &\colhead{\textit{EXPTIME}}  &\colhead{PtNe}&\colhead{Lamp}   &\colhead{\textit{APERTURE}}&\colhead{\textit{APERXPOS\tablenotemark{x}}}&\colhead{\textit{APERYPOS\tablenotemark{y}}}&\colhead{\textit{OPT\_ELEM}}&\colhead{\textit{DATE-OBS}}\\
\colhead{}                 &  &\colhead{}        &\colhead{}&\colhead{}&\colhead{(s)}&\colhead{Lamp \#}&\colhead{Current\tablenotemark{c}}&\colhead{}&\colhead{}&\colhead{ }&\colhead{}&\colhead{} \\
\colhead{(1)}&\colhead{(2)} & \colhead{(3)}&\colhead{(4)} &
\colhead{(5)}&\colhead{(6)} & \colhead{(7)}&\colhead{(8)} &
\colhead{(9)}&\colhead{(10)} & \colhead{(11)} &\colhead{(12)} & \colhead{(13)}
}
\startdata
\toprule
lc6ka1i1q	&	13171	&	427W3	&	ACCUM	&	ACQ/IMAGE	&	60	&	P2	&	Low	&	PSA	&	22.1	&	127.1	&	MIRA	&	2013-03-02	\\
lc6ka1i3q	&	13171	&	427W3	&	ACCUM	&	ACQ/IMAGE	&	300	&	P2	&	Low	&	PSA	&	22.1	&	127.1	&	MIRB	&	2013-03-02	\\
lc6ka2imq	&	13171	&	206W3	&	ACCUM	&	ACQ/IMAGE	&	60	&	P2	&	Low	&	PSA	&	22.1	&	127.1	&	MIRA	&	2013-09-01	\\
lc6ka2ioq	&	13171	&	206W3	&	ACCUM	&	ACQ/IMAGE	&	300	&	P2	&	Low	&	PSA	&	22.1	&	127.1	&	MIRB	&	2013-09-01	\\
lcgp01bpq	&	13523	&	WAVE	&	TT	&	WAVECAL	&	40	&	P2	&	Low	&	WCA	&	22.1	&	127.1	&	MIRB	&	2013-11-11	\\
lcgp01bsq	&	13523	&	WAVE	&	TT	&	WAVECAL	&	40	&	P1	&	Low	&	WCA	&	22.1	&	127.1	&	MIRB	&	2013-11-11	\\
lcgp01byq	&	13523	&	WAVE	&	TT	&	WAVECAL	&	20	&	P2	&	Low	&	WCA	&	22.1	&	127.1	&	MIRA	&	2013-11-11	\\
lcgp01c3q	&	13523	&	WAVE	&	TT	&	WAVECAL	&	20	&	P1	&	Low	&	WCA	&	22.1	&	127.1	&	MIRA	&	2013-11-11	\\
lci4a1dcq	&	13616	&	427W3	&	ACCUM	&	ACQ/IMAGE	&	60	&	P2	&	Low	&	PSA	&	22.1	&	127.1	&	MIRA	&	2014-04-03	\\
lci4a1deq	&	13616	&	427W3	&	ACCUM	&	ACQ/IMAGE	&	300	&	P2	&	Low	&	PSA	&	22.1	&	127.1	&	MIRB	&	2014-04-03	\\
lci4a2e3q	&	13616	&	206W3	&	ACCUM	&	ACQ/IMAGE	&	60	&	P2	&	Low	&	PSA	&	22.1	&	127.1	&	MIRA	&	2014-10-27	\\
lci4a2e5q	&	13616	&	206W3	&	ACCUM	&	ACQ/IMAGE	&	300	&	P2	&	Med	&	PSA	&	22.1	&	127.1	&	MIRB	&	2014-10-27	\\
lcgq01q5q	&	13526	&	WD-1657+343	&	ACCUM	&	ACQ/IMAGE	&	12	&	P2	&	Med	&	PSA	&	22.1	&	127.1	&	MIRB	&	2014-11-19	\\
lcgq01q7q	&	13526	&	WD-1657+343	&	TT	&	EXT/SCI	&	16	&	P2	&	Med	&	PSA	&	22.1	&	127.1	&	MIRB	&	2014-11-19	\\
lcgq01q9q	&	13526	&	WD-1657+343	&	TT	&	EXT/SCI	&	150	&	P2	&	Med	&	BOA	&	22.1	&	-153.1	&	MIRA	&	2014-11-19	\\
lcgq01qbq	&	13526	&	WAVE	&	TT	&	WAVECAL	&	7	&	P2	&	Low	&	WCA	&	22.1	&	126.1	&	MIRA	&	2014-11-19	\\
lcgq01qdq	&	13526	&	WD-1657+343	&	ACCUM	&	ACQ/IMAGE	&	150	&	P2	&	Low	&	BOA	&	22.1	&	-153.1	&	MIRA	&	2014-11-19	\\
lcgq01qfq	&	13526	&	WAVE	&	TT	&	WAVECAL	&	7	&	P2	&	Low	&	WCA	&	22.1	&	126.1	&	MIRA	&	2014-11-19	\\
lcgq01qhq	&	13526	&	WD-1657+343	&	TT	&	EXT/SCI	&	12	&	P2	&	Med	&	PSA	&	22.1	&	126.1	&	MIRB	&	2014-11-19	\\
lcgq01qjq	&	13526	&	WD-1657+343	&	ACCUM	&	ACQ/IMAGE	&	12	&	P2	&	Med	&	PSA	&	22.1	&	126.1	&	MIRB	&	2014-11-19	\\
lcgq02hmq	&	13526	&	HIP66578	&	ACCUM	&	ACQ/IMAGE	&	12	&	P2	&	Low	&	BOA	&	22.1	&	-153.1	&	MIRA	&	2014-11-17	\\
lcgq02hoq	&	13526	&	WAVE	&	TT	&	WAVECAL	&	7	&	P2	&	Low	&	WCA	&	22.1	&	126.1	&	MIRA	&	2014-11-17	\\
lcgq02hqq	&	13526	&	HIP66578	&	TT	&	EXT/SCI	&	181	&	P2	&	Low	&	BOA	&	22.1	&	-153.1	&	MIRB	&	2014-11-17	\\
lcgq02hsq	&	13526	&	WAVE	&	TT	&	WAVECAL	&	12	&	P2	&	Med	&	WCA	&	22.1	&	126.1	&	MIRB	&	2014-11-17	\\
lcgq02huq	&	13526	&	HIP66578	&	ACCUM	&	ACQ/IMAGE	&	181	&	P2	&	Med	&	BOA	&	22.1	&	-153.1	&	MIRB	&	2014-11-17	\\
lcgq02hwq	&	13526	&	WAVE	&	TT	&	WAVECAL	&	12	&	P2	&	Med	&	WCA	&	22.1	&	126.1	&	MIRB	&	2014-11-17	\\
lcgq02hyq	&	13526	&	WAVE	&	TT	&	WAVECAL	&	10	&	P2	&	Low	&	WCA	&	22.1	&	126.1	&	MIRA	&	2014-11-17	\\
lcgq02i0q	&	13526	&	HIP66578	&	ACCUM	&	ACQ/IMAGE	&	12	&	P2	&	Low	&	BOA	&	22.1	&	-153.1	&	MIRA	&	2014-11-17	\\
lcgq02icq	&	13526	&	WAVE	&	TT	&	WAVECAL	&	10	&	P1	&	Low	&	WCA	&	22.1	&	127.1	&	MIRA	&	2014-11-17	\\
lcgq02ieq	&	13526	&	WAVE	&	TT	&	WAVECAL	&	10	&	P2	&	Low	&	WCA	&	22.1	&	127.1	&	MIRA	&	2014-11-17	\\
lcgq02igq	&	13526	&	WAVE	&	TT	&	WAVECAL	&	30	&	P1	&	Low	&	WCA	&	22.1	&	127.1	&	MIRB	&	2014-11-17	\\
lcgq02iiq	&	13526	&	WAVE	&	TT	&	WAVECAL	&	20	&	P2	&	Med	&	WCA	&	22.1	&	127.1	&	MIRB	&	2014-11-17	\\
lcgq03dbq	&	13526	&	206W3	&	ACCUM	&	ACQ/IMAGE	&	15	&	P2	&	Low	&	PSA	&	22.1	&	127.1	&	MIRA	&	2014-10-06	\\
lcgq03ddq	&	13526	&	206W3	&	TT	&	EXT/SCI	&	15	&	P2	&	Low	&	PSA	&	22.1	&	127.1	&	MIRA	&	2014-10-06	\\
lcgq03dfq	&	13526	&	206W3	&	TT	&	EXT/SCI	&	160	&	P2	&	Low	&	PSA	&	22.1	&	127.1	&	MIRB	&	2014-10-06	\\
lcgq03dhq	&	13526	&	206W3	&	TT	&	EXT/SCI	&	180	&	P2	&	Low	&	PSA	&	22.1	&	127.1	&	MIRB	&	2014-10-06	\\
lcgq03djq	&	13526	&	206W3	&	TT	&	EXT/SCI	&	180	&	P2	&	Med	&	PSA	&	22.1	&	127.1	&	MIRB	&	2014-10-06	\\
lcgq03dlq	&	13526	&	206W3	&	ACCUM	&	ACQ/IMAGE	&	160	&	P2	&	Med	&	PSA	&	22.1	&	127.1	&	MIRB	&	2014-10-06	\\
lcgq03dnq	&	13526	&	206W3	&	TT	&	EXT/SCI	&	180	&	P2	&	Med	&	PSA	&	22.1	&	127.1	&	MIRB	&	2014-10-06	\\
lcgq03dpq	&	13526	&	206W3	&	TT	&	EXT/SCI	&	160	&	P2	&	Low	&	PSA	&	22.1	&	127.1	&	MIRB	&	2014-10-06	\\
lcgq03drq	&	13526	&	206W3	&	TT	&	EXT/SCI	&	12	&	P2	&	Low	&	PSA	&	22.1	&	127.1	&	MIRA	&	2014-10-06	\\
lcgq03dtq	&	13526	&	206W3	&	ACCUM	&	ACQ/IMAGE	&	12	&	P2	&	Low	&	PSA	&	22.1	&	127.1	&	MIRA	&	2014-10-06	\\
lcri01fzq	&	13972	&	WD-1657+343	&	ACCUM	&	ACQ/IMAGE	&	12	&	P2	&	Med	&	PSA	&	22.1	&	125.1	&	MIRB	&	2015-10-06	\\
lcri01g1q	&	13972	&	WD-1657+343	&	TT	&	EXT/SCI	&	12	&	P2	&	Med	&	PSA	&	22.1	&	125.1	&	MIRB	&	2015-10-06	\\
lcri01g3q	&	13972	&	WD-1657+343	&	TT	&	EXT/SCI	&	150	&	P2	&	Med	&	BOA	&	22.1	&	-153.1	&	MIRA	&	2015-10-06	\\
lcri01g5q	&	13972	&	WAVE	&	TT	&	WAVECAL	&	10	&	P2	&	Low	&	WCA	&	22.1	&	126.1	&	MIRA	&	2015-10-06	\\
lcri01g7q	&	13972	&	WD-1657+343	&	ACCUM	&	ACQ/IMAGE	&	150	&	P2	&	Low	&	BOA	&	22.1	&	-153.1	&	MIRA	&	2015-10-06	\\
lcri01g9q	&	13972	&	WAVE	&	TT	&	WAVECAL	&	10	&	P2	&	Low	&	WCA	&	22.1	&	126.1	&	MIRA	&	2015-10-06	\\
lcri01gcq	&	13972	&	WD-1657+343	&	TT	&	EXT/SCI	&	14	&	P2	&	Med	&	PSA	&	22.1	&	126.1	&	MIRB	&	2015-10-06	\\
lcri01geq	&	13972	&	WD-1657+343	&	ACCUM	&	ACQ/IMAGE	&	12	&	P2	&	Med	&	PSA	&	22.1	&	126.1	&	MIRB	&	2015-10-06	\\
lcri02h8q	&	13972	&	HIP66578	&	ACCUM	&	ACQ/IMAGE	&	12	&	P2	&	Low	&	BOA	&	22.1	&	-153.1	&	MIRA	&	2015-10-06	\\
lcri02haq	&	13972	&	WAVE	&	TT	&	WAVECAL	&	14	&	P2	&	Low	&	WCA	&	22.1	&	126.1	&	MIRA	&	2015-10-06	\\
lcri02hcq	&	13972	&	HIP66578	&	TT	&	EXT/SCI	&	181	&	P2	&	Low	&	BOA	&	22.1	&	-153.1	&	MIRB	&	2015-10-06	\\
lcri02heq	&	13972	&	WAVE	&	TT	&	WAVECAL	&	24	&	P2	&	Med	&	WCA	&	22.1	&	126.1	&	MIRB	&	2015-10-06	\\
lcri02hgq	&	13972	&	HIP66578	&	ACCUM	&	ACQ/IMAGE	&	181	&	P2	&	Med	&	BOA	&	22.1	&	-153.1	&	MIRB	&	2015-10-06	\\
lcri02hiq	&	13972	&	WAVE	&	TT	&	WAVECAL	&	24	&	P2	&	Med	&	WCA	&	22.1	&	126.1	&	MIRB	&	2015-10-06	\\
lcri02hkq	&	13972	&	WAVE	&	TT	&	WAVECAL	&	14	&	P2	&	Low	&	WCA	&	22.1	&	126.1	&	MIRA	&	2015-10-06	\\
lcri02hmq	&	13972	&	HIP66578	&	ACCUM	&	ACQ/IMAGE	&	12	&	P2	&	Low	&	BOA	&	22.1	&	-153.1	&	MIRA	&	2015-10-06	\\
lcri02hyq	&	13972	&	WAVE	&	TT	&	WAVECAL	&	14	&	P1	&	Low	&	WCA	&	22.1	&	125.1	&	MIRA	&	2015-10-06	\\
lcri02i0q	&	13972	&	WAVE	&	TT	&	WAVECAL	&	24	&	P2	&	Low	&	WCA	&	22.1	&	125.1	&	MIRA	&	2015-10-06	\\
lcri02i2q	&	13972	&	WAVE	&	TT	&	WAVECAL	&	30	&	P1	&	Low	&	WCA	&	22.1	&	125.1	&	MIRB	&	2015-10-06	\\
lcri02i4q	&	13972	&	WAVE	&	TT	&	WAVECAL	&	24	&	P2	&	Med	&	WCA	&	22.1	&	125.1	&	MIRB	&	2015-10-06	\\
lcsla1i4q	&	14035	&	427W3	&	ACCUM	&	ACQ/IMAGE	&	60	&	P2	&	Low	&	PSA	&	22.1	&	125.1	&	MIRA	&	2015-04-14	\\
lcsla1i6q	&	14035	&	427W3	&	ACCUM	&	ACQ/IMAGE	&	300	&	P2	&	Med	&	PSA	&	22.1	&	125.1	&	MIRB	&	2015-04-14	\\
lcsla2bhq	&	14035	&	206W3	&	ACCUM	&	ACQ/IMAGE	&	60	&	P2	&	Low	&	PSA	&	22.1	&	125.1	&	MIRA	&	2015-10-02	\\
lcsla2bjq	&	14035	&	206W3	&	ACCUM	&	ACQ/IMAGE	&	300	&	P2	&	Med	&	PSA	&	22.1	&	125.1	&	MIRB	&	2015-10-02	\\
\midrule
lcq	&	14452	&	427W3	&	ACCUM	&	ACQ/IMAGE	&	60	&	P2	&	Low	&	PSA	&	22.1	&	 	&	MIRA	&	2016	\\
lcq	&	14452	&	427W3	&	ACCUM	&	ACQ/IMAGE	&	300	&	P2	&	Med	&	PSA	&	22.1	&	 	&	MIRB	&	2016	\\
lcq	&	14452	&	206W3	&	ACCUM	&	ACQ/IMAGE	&	60	&	P2	&	Low	&	PSA	&	22.1	&	 	&	MIRA	&	2016	\\
lcq	&	14452	&	206W3	&	ACCUM	&	ACQ/IMAGE	&	300	&	P2	&	Med	&	PSA	&	22.1	&	 	&	MIRB	&	2016	\\
\midrule
ld3701gtq	&	14440	&	WD-1657+343	&	ACCUM	&	ACQ/IMAGE	&	13	&	P2	&	Med	&	PSA	&	22.1	&	125.1	&	MIRB	&	2016-10-18	\\
ld3701gvq	&	14440	&	WD-1657+343	&	TT	&	EXT/SCI	&	16	&	P2	&	Med	&	PSA	&	22.1	&	125.1	&	MIRB	&	2016-10-18	\\
ld3701gxq	&	14440	&	WD-1657+343	&	TT	&	EXT/SCI	&	150	&	P2	&	Med	&	BOA	&	22.1	&	-153.1	&	MIRA	&	2016-10-18	\\
ld3701gzq	&	14440	&	WAVE	&	TT	&	WAVECAL	&	9	&	P2	&	Low	&	WCA	&	22.1	&	126.1	&	MIRA	&	2016-10-18	\\
ld3701h1q	&	14440	&	WD-1657+343	&	ACCUM	&	ACQ/IMAGE	&	150	&	P2	&	Low	&	BOA	&	22.1	&	-153.1	&	MIRA	&	2016-10-18	\\
ld3701h3q	&	14440	&	WAVE	&	TT	&	WAVECAL	&	10	&	P2	&	Low	&	WCA	&	22.1	&	126.1	&	MIRA	&	2016-10-18	\\
ld3701h5q	&	14440	&	WD-1657+343	&	TT	&	EXT/SCI	&	16	&	P2	&	Med	&	PSA	&	22.1	&	126.1	&	MIRB	&	2016-10-18	\\
ld3701h7q	&	14440	&	WD-1657+343	&	ACCUM	&	ACQ/IMAGE	&	13	&	P2	&	Med	&	PSA	&	22.1	&	126.1	&	MIRB	&	2016-10-18	\\
ld3702mzq&	14440	&	HIP66578	&	ACCUM	&	ACQ/IMAGE	&	16	&	P2	&	Low	&	BOA	&	22.1	&	-153.1	&	MIRA	&	2016-10-19	\\
ld3702n1q	&	14440	&	WAVE	&	TT	&	WAVECAL	&	14	&	P2	&	Low	&	WCA	&	22.1	&	126.1	&	MIRA	&	2016-10-19	\\
ld3702n4q	&	14440	&	HIP66578	&	TT	&	EXT/SCI	&	183	&	P2	&	Low	&	BOA	&	22.1	&	-153.1	&	MIRB	&	2016-10-19	\\
ld3702n7q	&	14440	&	WAVE	&	TT	&	WAVECAL	&	24	&	P2	&	Med	&	WCA	&	22.1	&	126.1	&	MIRB	&	2016-10-19	\\
ld3702n9q	&	14440	&	HIP66578	&	ACCUM	&	ACQ/IMAGE	&	183	&	P2	&	Med	&	BOA	&	22.1	&	-153.1	&	MIRB	&	2016-10-19	\\
ld3702nbq	&	14440	&	WAVE	&	TT	&	WAVECAL	&	24	&	P2	&	Med	&	WCA	&	22.1	&	126.1	&	MIRB	&	2016-10-19	\\
ld3702neq	&	14440	&	WAVE	&	TT	&	WAVECAL	&	14	&	P2	&	Low	&	WCA	&	22.1	&	126.1	&	MIRA	&	2016-10-19	\\
ld3702nhq	&	14440	&	HIP66578	&	ACCUM	&	ACQ/IMAGE	&	16	&	P2	&	Low	&	BOA	&	22.1	&	-153.1	&	MIRA	&	2016-10-19	\\
ld3702o1q	&	14440	&	WAVE	&	TT	&	WAVECAL	&	14	&	P1	&	Low	&	WCA	&	22.1	&	125.1	&	MIRA	&	2016-10-19	\\
ld3702o3q	&	14440	&	WAVE	&	TT	&	WAVECAL	&	24	&	P2	&	Low	&	WCA	&	22.1	&	125.1	&	MIRA	&	2016-10-19	\\
ld3702o5q	&	14440	&	WAVE	&	TT	&	WAVECAL	&	30	&	P1	&	Low	&	WCA	&	22.1	&	125.1	&	MIRB	&	2016-10-19	\\
ld3702o7q	&	14440	&	WAVE	&	TT	&	WAVECAL	&	24	&	P2	&	Med	&	WCA	&	22.1	&	125.1	&	MIRB	&	2016-10-19	\\
ldozbadhq	&	14857	&	WD-1657+343	&	ACCUM	&	ACQ/IMAGE	&	13	&	P2	&	Med	&	PSA	&	22.1	&	125.1	&	MIRB	&	2017-09-04	\\
ldozbadjs	&	14857	&	WD-1657+343	&	TT	&	EXT/SCI	&	16	&	P2	&	Med	&	PSA	&	22.1	&	125.1	&	MIRB	&	2017-09-04	\\
ldozbadlq	&	14857	&	WD-1657+343	&	TT	&	EXT/SCI	&	150	&	P2	&	Med	&	BOA	&	22.1	&	-153.1	&	MIRA	&	2017-09-04	\\
ldozbadnq	&	14857	&	WAVE	&	TT	&	WAVECAL	&	9	&	P2	&	Low	&	WCA	&	22.1	&	126.1	&	MIRA	&	2017-09-04	\\
ldozbadpq	&	14857	&	WD-1657+343	&	ACCUM	&	ACQ/IMAGE	&	150	&	P2	&	Low	&	BOA	&	22.1	&	-153.1	&	MIRA	&	2017-09-04	\\
ldozbadrq	&	14857	&	WAVE	&	TT	&	WAVECAL	&	10	&	P2	&	Low	&	WCA	&	22.1	&	126.1	&	MIRA	&	2017-09-04	\\
ldozbadtq	&	14857	&	WD-1657+343	&	TT	&	EXT/SCI	&	16	&	P2	&	Med	&	PSA	&	22.1	&	126.1	&	MIRB	&	2017-09-04	\\
ldozbadvq	&	14857	&	WD-1657+343	&	ACCUM	&	ACQ/IMAGE	&	13	&	P2	&	Med	&	PSA	&	22.1	&	126.1	&	MIRB	&	2017-09-04	\\
ldozbbleq	&	14857	&	HIP66578	&	ACCUM	&	ACQ/IMAGE	&	16	&	P2	&	Low	&	BOA	&	22.1	&	-153.1	&	MIRA	&	2017-09-06	\\
ldozbblgq	&	14857	&	WAVE	&	TT	&	WAVECAL	&	14	&	P2	&	Low	&	WCA	&	22.1	&	126.1	&	MIRA	&	2017-09-06	\\
ldozbbliq	&	14857	&	HIP66578	&	TT	&	EXT/SCI	&	183	&	P2	&	Low	&	BOA	&	22.1	&	-153.1	&	MIRB	&	2017-09-06	\\
ldozbblkq	&	14857	&	WAVE	&	TT	&	WAVECAL	&	24	&	P2	&	Med	&	WCA	&	22.1	&	126.1	&	MIRB	&	2017-09-06	\\
ldozbblmq	&	14857	&	HIP66578	&	ACCUM	&	ACQ/IMAGE	&	183	&	P2	&	Med	&	BOA	&	22.1	&	-153.1	&	MIRB	&	2017-09-06	\\
ldozbbloq	&	14857	&	WAVE	&	TT	&	WAVECAL	&	24	&	P2	&	Med	&	WCA	&	22.1	&	126.1	&	MIRB	&	2017-09-06	\\
ldozbblqq	&	14857	&	WAVE	&	TT	&	WAVECAL	&	14	&	P2	&	Low	&	WCA	&	22.1	&	126.1	&	MIRA	&	2017-09-06	\\
ldozbblsq	&	14857	&	HIP66578	&	ACCUM	&	ACQ/IMAGE	&	16	&	P2	&	Low	&	BOA	&	22.1	&	-153.1	&	MIRA	&	2017-09-06	\\
ldozbbm4q&	14857	&	WAVE	&	TT	&	WAVECAL	&	16	&	P1	&	Low	&	WCA	&	22.1	&	125.1	&	MIRA	&	2017-09-06	\\
ldozbbm6q&	14857	&	WAVE	&	TT	&	WAVECAL	&	26	&	P2	&	Low	&	WCA	&	22.1	&	125.1	&	MIRA	&	2017-09-06	\\
ldozbbm8q&	14857	&	WAVE	&	TT	&	WAVECAL	&	32	&	P1	&	Low	&	WCA	&	22.1	&	125.1	&	MIRB	&	2017-09-06	\\
ldozbbmaq&	14857	&	WAVE	&	TT	&	WAVECAL	&	26	&	P2	&	Med	&	WCA	&	22.1	&	125.1	&	MIRB	&	2017-09-06	\\
ldozpbf5q	&	14857	&	206W3	&	ACCUM	&	ACQ/IMAGE	&	20	&	P2	&	Low	&	PSA	&	22.1	&	125.1	&	MIRA	&	2017-09-10	\\
ldozpbf7q	&	14857	&	206W3	&	TT	&	EXT/SCI	&	20	&	P2	&	Low	&	PSA	&	22.1	&	125.1	&	MIRA	&	2017-09-10	\\
ldozpbf9q	&	14857	&	206W3	&	TT	&	EXT/SCI	&	220	&	P2	&	Med	&	PSA	&	22.1	&	125.1	&	MIRB	&	2017-09-10	\\
ldozpbfbq	&	14857	&	206W3	&	ACCUM	&	ACQ/IMAGE	&	220	&	P2	&	Med	&	PSA	&	22.1	&	125.1	&	MIRB	&	2017-09-10	\\
ldozpbfdq	&	14857	&	206W3	&	TT	&	EXT/SCI	&	220	&	P2	&	Med	&	PSA	&	22.1	&	125.1	&	MIRB	&	2017-09-10	\\
ldozpbffq	&	14857	&	206W3	&	TT	&	EXT/SCI	&	20	&	P2	&	Low	&	PSA	&	22.1	&	125.1	&	MIRA	&	2017-09-10	\\
ldozpbfhq	&	14857	&	206W3	&	ACCUM	&	ACQ/IMAGE	&	20	&	P2	&	Low	&	PSA	&	22.1	&	125.1	&	MIRA	&	2017-09-10
\bottomrule
\enddata
%\end{center}
\tablenotetext{c}{For the P1 lamp, the three current settings are LOW (6mA), MED (10mA) and HIGH (18mA). For the P2 lamp, the current settings are LOW (3mA), MED (10mA) and HIGH (14mA).}
\tablenotetext{t}{TT = TIME-TAG.}
\tablecomments{Exposures listed as \textsc{EXPTYPE}=EXT/SCI contain coeval target and PtNe lamp (P1 or P2) images taken in time-tag (\textsc{OBSTYPE}=TT) mode.
Exposures listed as \textsc{EXPTYPE}=WAVECAL (target = WAVE) contain only TT PtNe lamp (WCA) images.  \tacq{IMAGE} exposures return before and after target images in \textsc{OBSTYPE}=ACCUM, but do not return  lamp images.}
\end{deluxetable}
%   740	const SHORT pcmech_ApMXDispPosition[TA_NUM_APERTURES][MIE_NUM_DETECTORS] =
%   741	{
%   742	   /*  FUV   NUV  */
%   743	   /*  ---   ---  */
%   744	      { 126,  126 }, /* PSA_LP1 */
%   745	      {-153, -153 }, /* BOA_LP1 */
%   746	      {-153, -153 }, /* FCA_LP1 */
%   747	      { 126,  126 }, /* WCA_LP1 */
%   748	      {  53,  126 }, /* PSA_LP2 */
%   749	      {-226, -153 }, /* BOA_LP2 */
%   750	      {-226, -153 }, /* FCA_LP2 */
%   751	      {  53,  126 }, /* WCA_LP2 */
%   752	      { 181,  126 }, /* PSA_LP3 */
%   753	      { -98, -153 }, /* BOA_LP3 */
%   754	      { -98, -153 }, /* FCA_LP3 */
%   755	      { 181,  126 }, /* WCA_LP3 */
%   756	      { 234,  126 }, /* PSA_LP4 */
%   757	      { -45, -153 }, /* BOA_LP4 */
%   758	      { -45, -153 }, /* FCA_LP4 */
%   759	      { 234,  126 }, /* WCA_LP4 */
%   760	      { 181,  126 }, /* PSA_LP5 */
%   761	      { -98, -153 }, /* BOA_LP5 */
%   762	      { -98, -153 }, /* FCA_LP5 */
%   763	      { 181,  126 }, /* WCA_LP5 */
%   764	      { 181,  126 }, /* PSA_LP6 */
%   765	      { -98, -153 }, /* BOA_LP6 */
%   766	      { -98, -153 }, /* FCA_LP6 */
%   767	      { 181,  126 }, /* WCA_LP6 */
%   768	      { 181,  126 }, /* PSA_LP7 */
%   769	      { -98, -153 }, /* BOA_LP7 */
%   770	      { -98, -153 }, /* FCA_LP7 */
%   771	      { 181,  126 }, /* WCA_LP7 */
%   772	      { 181,  126 }, /* PSA_LP8 */
%   773	      { -98, -153 }, /* BOA_LP8 */
%   774	      { -98, -153 }, /* FCA_LP8 */
%   775	      { 181,  126 }  /* WCA_LP8 */
%   776	};

\begin{deluxetable}{|r|r|r|r|r|r|r|r|r|r|r|}
\tabcolsep 2pt
\tabletypesize{\tiny}
\tablecolumns{11}
\tablewidth{0 pt}
\tablecaption{COS/NUV TA Spectroscopic Monitoring Exposures\label{table:NUVtamonspec}}
\tablehead{
\colhead{ROOTNAME}&\colhead{PROP}&\colhead{TARGNAME}&
\colhead{EXPTIME}&\colhead{LAMP}&\colhead{CEN}&
\colhead{LP}&\colhead{APER}&\colhead{APER}&\colhead{OPT}&\colhead{DATE}\\
\colhead{}&\colhead{ID}&\colhead{}&
\colhead{(s)}&\colhead{USED}&\colhead{WAVE}&
\colhead{}&\colhead{XPOS}&\colhead{YPOS}&\colhead{ELEM}&\colhead{OBS}\\

}
\startdata
lcgq01qlq	&	13526	&	WD-1657+343	&	20	&	P2	&	3000	&	1	&	22.1	&	126.1	&	G230L	&	2014-11-19	\\
lcgq01r6q	&	13526	&	WD-1657+343	&	151	&	P2	&	2850	&	1	&	22.1	&	126.1	&	G285M	&	2014-11-19	\\
lcgq02i2q	&	13526	&	HIP66578	&	40	&	P2	&	1890	&	1	&	22.1	&	126.1	&	G185M	&	2014-11-17	\\
lcgq02i4q	&	13526	&	HIP66578	&	52	&	P2	&	2306	&	1	&	22.1	&	126.1	&	G225M	&	2014-11-17	\\
lcri01ggq	&	13972	&	WD-1657+343	&	20	&	P2	&	3000	&	1	&	22.1	&	126.1	&	G230L	&	2015-10-06	\\
lcri01giq	&	13972	&	WD-1657+343	&	151	&	P2	&	2676	&	1	&	22.1	&	126.1	&	G285M	&	2015-10-06	\\
lcri02hoq	&	13972	&	HIP66578	&	52	&	P2	&	2306	&	1	&	22.1	&	126.1	&	G225M	&	2015-10-06	\\
lcri02hqq	&	13972	&	HIP66578	&	40	&	P2	&	1913	&	1	&	22.1	&	126.1	&	G185M	&	2015-10-06	\\
ld3701h9q	&	14440	&	WD-1657+343	&	21	&	P2	&	3000	&	1	&	22.1	&	126.1	&	G230L	&	2016-10-18	\\
ld3701hbq	&	14440	&	WD-1657+343	&	151	&	P2	&	2676	&	1	&	22.1	&	126.1	&	G285M	&	2016-10-18	\\
ld3702nmq	&	14440	&	HIP66578	&	53	&	P2	&	2306	&	1	&	22.1	&	126.1	&	G225M	&	2016-10-19	\\
ld3702noq	&	14440	&	HIP66578	&	40	&	P2	&	1913	&	1	&	22.1	&	126.1	&	G185M	&	2016-10-19	\\
ldozbadxq	&	14857	&	WD-1657+343	&	23	&	P2	&	3000	&	1	&	22.1	&	126.1	&	G230L	&	2017-09-04	\\
ldozbadzq	&	14857	&	WD-1657+343	&	151	&	P2	&	2676	&	1	&	22.1	&	126.1	&	G285M	&	2017-09-04	\\
ldozbbluq	&	14857	&	HIP66578	&	53	&	P2	&	2306	&	1	&	22.1	&	126.1	&	G225M	&	2017-09-06	\\
ldozbblwq	&	14857	&	HIP66578	&	40	&	P2	&	1913	&	1	&	22.1	&	126.1	&	G185M	&	2017-09-06	\\
\hline
\enddata
\tablenotetext{a}{All exposures were taken with the PSA at \texttt{FP-POS}=3.}
\end{deluxetable}

% $Id: FUVtamonfiles.tex,v 1.8 2018/04/17 18:38:43 penton Exp $
\begin{deluxetable}{ccrccccccrr}
\tabcolsep 4 pt
\tablewidth{5.7 in}
\tabletypesize{\scriptsize}
\tablecolumns{10}
\tablewidth{0pt}
\tablecaption{FUV TA Monitoring Exposures\label{tab:FUVtamon}}
\tablehead{
\colhead{\textit{PROPOSID}}&\colhead{\textit{ROOTNAME}}&\colhead{\textit{TARGNAME}}&
\colhead{\textit{EXPTIME}}&\colhead{\textit{OPT\_ELEM}}&\colhead{\cenwave{}}&
\colhead{LP}&\colhead{\textit{APER}}&\colhead{\textit{APERY}}&\colhead{\textit{DATE-OBS}}\\
\colhead{}&\colhead{}&\colhead{}&\colhead{(s)}&\colhead{}&\colhead{}&
\colhead{}&\colhead{\textit{XPOS}}&\colhead{\textit{YPOS}}&\colhead{}\\
\colhead{(1)}&\colhead{(2)} &
\colhead{(3)}&\colhead{(4)} &
\colhead{(5)}&\colhead{(6)} &
\colhead{(7)}&\colhead{(8)} &
\colhead{(9)}&\colhead{(10)}
}
\startdata
\toprule
13124	&	lc6601s7q	&	WD-1657+343	&	110	&	G130M	&1309	&	2	&	22.1	&	52.1	&	2013-10-24\\
13124	&	lc6601s9q	&	WD-1657+343	&	30	&	G140L	&1280	&	2	&	22.1	&	52.1	&	2013-10-24\\
13124	&	lc6602z3q	&	HIP66578	&	20	&	G160M	&1623	&	2	&	22.1	&	52.1	&	2013-11-01\\
13124	&	lc6602z9q\tablenotemark{a}	&	HIP66578	&	323	&	G160M	&	1623	&	2	&	22.1	&	-224.1	&	2013-11-01\\
13124	&	lc6602zbq\tablenotemark{b}	&	WAVE	&	12	&	G160M	&	1623	&	2	&	22.1	&	51.1	&	2013-11-01\\
13526	&lcgq01r8q	&	WD-1657+343	&	20	&	G130M	&	1309	&	2	&	22.1	&	52.1	&	2014-11-19	\\
13526	&lcgq01r8q	&	WD-1657+343	&	20	&	G130M	&	1309	&	2	&	22.1	&	52.1	&	2014-11-19	\\
13526	&lcgq01raq	&	WD-1657+343	&	7	&	G140L	&	1280	&	2	&	22.1	&	52.1	&	2014-11-19	\\
13526	&lcgq01raq	&	WD-1657+343	&	7	&	G140L	&	1280	&	2	&	22.1	&	52.1	&	2014-11-19	\\
13526	&lcgq02i6q	&	HIP66578	&	18	&	G160M	&	1600	&	2	&	22.1	&	52.1	&	2014-11-17	\\
13526	&lcgq02i8q	&	HIP66578	&	22	&	G160M	&	1600	&	2	&	22.1	&	52.1	&	2014-11-17	\\
13526	&lcgq02iaq	&	HIP66578	&	22	&	G160M	&	1600	&	2	&	22.1	&	52.1	&	2014-11-17	\\
\midrule
13972	&lcri01gkq	&	WD-1657+343	&	20	&	G130M	&	1309	&	3	&	22.1	&	182.1	&	2015-10-06	\\
13972	&lcri01gkq	&	WD-1657+343	&	20	&	G130M	&	1309	&	3	&	22.1	&	182.1	&	2015-10-06	\\
13972	&lcri01h6q	&	WD-1657+343	&	7	&	G140L	&	1280	&	3	&	22.1	&	182.1	&	2015-10-06	\\
13972	&lcri01h6q	&	WD-1657+343	&	7	&	G140L	&	1280	&	3	&	22.1	&	182.1	&	2015-10-06	\\
13972	&lcri02hsq	&	HIP66578	&	22	&	G160M	&	1600	&	3	&	22.1	&	182.1	&	2015-10-06	\\
13972	&lcri02huq	&	HIP66578	&	25	&	G160M	&	1600	&	3	&	22.1	&	182.1	&	2015-10-06	\\
13972	&lcri02hwq	&	HIP66578	&	25	&	G160M	&	1600	&	3	&	22.1	&	182.1	&	2015-10-06	\\
14440	&ld3701hdq	&	WD-1657+343	&	25	&	G130M	&	1309	&	3	&	22.1	&	182.1	&	2016-10-18	\\
14440	&ld3701hdq	&	WD-1657+343	&	25	&	G130M	&	1309	&	3	&	22.1	&	182.1	&	2016-10-18	\\
14440	&ld3701hfq	&	WD-1657+343	&	10	&	G140L	&	1280	&	3	&	22.1	&	182.1	&	2016-10-18	\\
14440	&ld3701hfq	&	WD-1657+343	&	10	&	G140L	&	1280	&	3	&	22.1	&	182.1	&	2016-10-18	\\
14440	&ld3702nqq	&	HIP66578	&	22	&	G160M	&	1600	&	3	&	22.1	&	182.1	&	2016-10-19	\\
14440	&ld3702nsq	&	HIP66578	&	25	&	G160M	&	1600	&	3	&	22.1	&	182.1	&	2016-10-19	\\
14440	&ld3702nuq	&	HIP66578	&	25	&	G160M	&	1600	&	3	&	22.1	&	182.1	&	2016-10-19	\\
14857	&ldozbae1q	&	WD-1657+343	&	25	&	G130M	&	1309	&	3	&	22.1	&	182.1	&	2017-09-04	\\
14857	&ldozbae1q	&	WD-1657+343	&	25	&	G130M	&	1309	&	3	&	22.1	&	182.1	&	2017-09-04	\\
14857	&ldozbae3q	&	WD-1657+343	&	10	&	G140L	&	1280	&	3	&	22.1	&	182.1	&	2017-09-04	\\
14857	&ldozbae3q	&	WD-1657+343	&	10	&	G140L	&	1280	&	3	&	22.1	&	182.1	&	2017-09-04	\\
14857	&ldozbblyq	&	HIP66578	&	22	&	G160M	&	1600	&	3	&	22.1	&	182.1	&	2017-09-06	\\
14857	&ldozbbm0q	&	HIP66578	&	27	&	G160M	&	1600	&	3	&	22.1	&	182.1	&	2017-09-06	\\
14857	&ldozbbm2q	&	HIP66578	&	27	&	G160M	&	1600	&	3	&	22.1	&	182.1	&	2017-09-06	\\
\bottomrule
\enddata
\tablenotetext{a}{For C20 only (\pid{13124}), a G160M BOA spectrum and WAVECAL were obtained to measure the WCA-to-BOA offset. The BOA was 2 steps off (0.105\arcsec) of its LP2 expected \textit{APERYPOS} position of -226 for this exposure.
This is similar to the $\pm1$ step offset often seen during \tacq{IMAGE}s.}
\tablenotetext{b}{This WAVECAL exposure was used to measure the WCA portion of the WCA-to-BOA offset for the proceeding BOA spectrum, and it off its nominal position of 52.1 by 1 \textit{APERYPOS} step.}
\tablecomments{All exposures taken at \textit{FP-POS=3}. All PSA spectra executed at the expected aperture positions (\textit{APERXPOS} \& \textit{APERYPOS}), while
the indicated BOA spectrum was off by 2 \textit{APERYPOS} steps.}
\end{deluxetable}

